\begin{abstract}

%Variations in the strength of the Northern Hemisphere winter polar stratospheric vortex can influence surface variability in the Atlantic sector. Disruptions of the vortex, known as sudden stratospheric warmings (SSW), are associated with an equatorward shift and deceleration of the North Atlantic jet stream, negative phases of the North Atlantic Oscillation as well as cold snaps over Eurasia and North America. Despite clear influences at the surface on sub-seasonal timescales, how stratospheric vortex variability interacts with ocean circulation on decadal to multi-decadal timescales is less well understood.


%Sudden Stratospheric Warmings (SSWs) are major disruptions of the Northern Hemisphere (NH)  stratospheric polar vortex and occur on average approximately 6 times per decade in observation based records. However, within these records, intervals of significantly higher and lower SSW rates are observed suggesting the possibility of low frequency variations in event occurrence. A better understanding of factors that influence this decadal variability  may help to improve predictability of NH mid-latitude surface climate, through stratosphere-troposphere coupling. 

%Research during the last two decades has established that variability of the
%winter polar stratospheric vortex can significantly influence the troposphere,
%affecting the likelihood of extreme weather events and the skill of long-range
%weather forecasts. This influence is particularly strong following the rapid
%breakdown of the vortex in events known as sudden stratospheric warmings
%(SSWs). This thesis addresses some outstanding issues in our understanding of
%the dynamics of this stratospheric variability and its influence on the
%troposphere.


Variability in the strength of the Northern Hemisphere winter polar stratospheric vortex is an important climate feature. Strong vortex conditions, as well as disruptions of the vortex, known as sudden stratospheric warmings (SSW), are associated with significant tropospheric and surface variations. This makes them an important feature for improving the skill of seasonal weather forecasts. This thesis addresses some ongoing research questions regarding the nature of variations in the polar vortex and their interactions with other features of the climate system.  

First, multi-decadal variability of SSW events is examined in a 1000-yr pre-industrial simulation of a coupled global climate model. A wavelet spectral decomposition method shows that hiatus events (intervals of a 5 years or more with no SSWs) and consecutive SSW events (extended intervals with at least one SSW in each year) vary on multi-decadal timescales of period between 60 and 90 years. The major influence on long-term SSW variability is associated with similar variations in amplitude and vertical coherence of the stratospheric quasi biennial oscillation (QBO).

Interactions between these multi-decadal signals in vortex strength and modes of surface and ocean variability are subsequently examined. Intervals which exhibit persistent anomalous vortex behaviour lead to oscillatory responses in the Atlantic Meridional Overturning Circulation (AMOC). The origin of these responses appears to be highly non-stationary with spectral power in vortex variability and the AMOC at periods of 30 and 50 years. AMOC variations on longer timescales (near 90-year periods) are found to lead to a vortex response, through a pathway involving the equatorial Pacific and QBO. 

Finally, the role that vertical coherence in the QBO plays in teleconnections with the vortex and the surface is assessed using a set of relaxation experiments which impose a perpetually deep and shallow QBO. Surface and vortex responses to the QBO are significantly enhanced deep QBO experiment. However, responses are of the opposite sign to those shown in previous work. The perpetual deep QBO is shown to induce unrealistically large anomalies in the equatorial semiannual oscillation (SAO) which modulate the vortex and surface response. Further experiments which impose SAO conditions as well as the QBO are recommended to establish the true nature of QBO teleconnections.

%A regression analysis is used to estimate the potential contribution of the 8 year SSW hiatus interval in the 1990s to the recent negative trend in AMOC observations. The result suggests that approximately 30\% of the trend may have been caused by the SSW hiatus. %The results support recent studies that have highlighted the role of vertical coherence in the QBO when considering coupling between the QBO, the polar vortex and tropospheric circulation. 
%We investigate the possible source of these long-term signals and find that the direct impact of variability in tropical sea surface temperatures, as well as the associated Aleutian Low, can account for only a small portion of the SSW variability.

%First, a geometrical method is developed to characterise two-dimensional polar
%vortex variability. This method is also able to identify types of SSW in which
%the vortex is displaced from the pole and those in which it is split in two;
%known as displaced and split vortex events. It shown to capture vortex
%variability at least as well as previous methods, but has the advantage of
%being easily applicable to climate model simulations. 

  % Such an application is desireable because of the relative lack of SSWs in the
  % observational record.

  % The models display a wide range of biases in the simulated frequency of split
  % and displaced vortex events which are shown to be strongly related to biases
  % in the average state of the vortex


 
%predictability of the polar stratosphere and its influence on the
%troposphere is assessed in a stratosphere-resolving seasonal forecast
%system. Little skill is found in the prediction of the strength of the
%Northern Hemisphere vortex at lead times beyond one month. However, much
%greater skill is found for the Southern Hemisphere vortex during austral
%spring. This allows for forecasts of interannual changes in ozone depletion to
%be inferred at lead times much beyond previous forecasts. It is further
%demonstrated that this stratospheric skill descends with time and leads to an
%enhanced surface skill at lead times of more than three months. 
 



\end{abstract}

%%% Local Variables:
%%% mode: latex
%%% TeX-master: "thesis"
%%% End:
