\begin{abstract}
\normalsize{\noindent Variability in the strength of the Northern Hemisphere stratospheric polar vortex is an important climate feature. Strong vortex conditions, as well as disruptions of the vortex, known as sudden stratospheric warmings (SSW), are associated with significant tropospheric and surface variations. This makes them an important feature for improving the skill of seasonal weather forecasts. This thesis addresses some ongoing research questions regarding the nature of variations in the polar vortex and their interactions with the wider climate system.
\vskip 0.06in
\noindent First, multi-decadal variability of SSW events is examined in a 1000-yr pre-industrial simulation of a coupled global climate model. A wavelet spectral decomposition method shows that hiatus events (intervals of 5 years or more with no SSWs) and consecutive SSW events (extended intervals with at least one SSW in each year) vary on multi-decadal timescales with the most persistent spectral feature appearing at a period of 90 years. This long-term SSW variability is found to be associated with similar variations in amplitude and vertical coherence of the stratospheric quasi-biennial oscillation (QBO) likely via the well-known Holton-Tan link where the QBO has an in-season influence over the vortex.
\vskip 0.06in
\noindent Interactions between multi-decadal variability in vortex strength and modes of surface and ocean variability are subsequently examined. Intervals that exhibit persistent anomalous vortex behaviour are found to lead to oscillatory responses in the Atlantic Meridional Overturning Circulation (AMOC). These AMOC responses peak in magnitude at lags of 2-3 years and $\sim$17 years following the vortex anomalies. The vortex/AMOC interaction is characterised by non-stationary variations at periods of 30 and 50 years and involves a corresponding modulation of the North Atlantic Oscillation (NAO). Using the relationship between persistent vortex behaviour and the AMOC response established in the model, the contribution of the observed 8-year SSW hiatus interval in the 1990s to the recent negative trend in AMOC observations is estimated. This analysis suggests that approximately 30\% of the trend may have been caused by the SSW hiatus. The AMOC is also found to vary on longer timescales, similar to the 90-year SSW variability, but this periodicity is not present in the NAO signal. An additional mechanism is proposed where long-term variations in the AMOC influence the vortex via a pathway involving the equatorial Pacific and QBO.
\vskip 0.06in
\noindent Finally, the role that vertical coherence in the QBO plays in its teleconnections is investigated using a set of climate model experiments that impose a perpetually deep and shallow QBO. While the magnitude of the vortex and surface responses to the QBO is significantly larger in the deep QBO experiment compared to the shallow experiment, they are of the opposite sign to those shown in previous work. The origin of this discrepancy is explored. The perpetual deep QBO is shown to induce unrealistically large anomalies in the equatorial semi-annual oscillation (SAO). The unexpected discrepancy is consistent with a modulation of the vortex and surface response from the SAO region, but showing this explicitly is challenging. Further experiments with imposed SAO conditions, as well as the QBO, are proposed to help establish the true nature of these teleconnections.}
\end{abstract}

%%% Local Variables:
%%% mode: latex
%%% TeX-master: "thesis"
%%% End:
