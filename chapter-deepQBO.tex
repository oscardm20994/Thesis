\chapter{The role of vertical structure in QBO teleconnections}
\label{cha:deepQBO}

\section{Introduction}
\label{sec:deepQBO-introduction}
The QBO is typically defined by the equatorial ZMZW at a single level in the mid-stratosphere. The 50 hPa level is usually used for NH observational studies \citep{Baldwin2001, Baldwin98} but some studies have also noted the importance of characterising the vertical structure of the QBO \citep{Fraedrih1993, Wallace1993,  Baldwin98,  Dunkerton2017, Gray2018, Andrews2019}. In an observational-based study \cite{Gray2018} find an enhanced association between the QBO and polar vortex when a metric incorporating the vertical coherence of equatorial winds via empirical orthogonal functions is utilised \citep{verena2016a}. In a model-based study \cite{Andrews2019} introduce a similar but simpler methodology by defining the QBO as the average ZMZW between two vertical levels, which preferentially selects time intervals that display a vertically coherent QBO phase between the specified levels. The same study reports enhanced responses to the QBO from the NAO and Arctic Oscillation (AO) when such vertically integrated metrics are used to define QBO phase compared to a single level definition. While these studies highlight statistical associations between vertical QBO metrics and the vortex, the importance of vertical coherence in HT strength has not been tested explicitly. Furthermore, influence mechanisms are not well understood. 

%%% Local Variables:
%%% mode: latex
%%% TeX-master: "thesis"
%%% End:
