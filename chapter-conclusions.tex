\chapter{Conclusions}
\label{cha:conclusions}

\section{Summary of results}

The main findings of this thesis are summarised as follows: 
%\begin{enumerate}

\paragraph{Application of moment diagnostics.} It has been demonstrated that
vortex moment diagnostics can be successfully applied to the geopotential height
field, giving similar results as when applied to conservative fields such as
PV. This provides a semi-Lagrangian (or vortex-centric) method which can be
readily used to describe the geometry of the stratospheric polar vortex in
climate model simulations.

It has been further shown that a simple threshold-based method can be applied to
the vortex moment diagnostics in order to identify split and displaced vortex
events. The majority of events identified in this way coincide with events
defined by other methods, and capture equally extreme vortex states.

\paragraph{The stratospheric polar vortex in climate models.} The first
multi-model comparison of stratospheric polar vortex geometry, including split
and displaced vortex events, has been carried out using the
stratosphere-resolving CMIP5 models. A wide range of biases have been identified
in the geometry of the stratospheric polar vortex among models. Some models have
a vortex which is on average too equatorward, others too poleward, while the
majority of models have a vortex which is too circularly symmetric. Models also
vary widely in their frequency of split and displaced vortex events. However,
the nature of these events is largely in agreement with observations, in
particular the fact that split vortex events appear more barotropic and
displaced vortex events are more baroclinic in nature. The consistency of this
difference in baroclinicity among models lends weight to the view that split
vortex events are caused by a resonant excitation of the barotropic mode, as
suggested by \citet{Esler2005} and \citet{Matthewman2011}, rather than relying
on strong transient tropospheric forcing. Significantly, the frequency of split
and displaced vortex events has been demonstrated to be highly correlated with
the aspect ratio and centroid latitude of the average vortex state,
respectively. It therefore follows that an improvement in the mean state of the
vortex is likely to lead to a more accurate representation of these extremes.

\paragraph{Stratosphere-troposphere coupling in climate models and
  observations.}  In reanalysis data, using the geopotential height-based vortex
moments method, a stronger tropospheric NAM signal is seen following split
vortex events than displaced vortex events. This is in agreement with the
results of \citet{Mitchell2013}. However, a bootstrap significance test of the
surface NAM over the month following these events cannot exclude the possibility
that this observed difference is due to chance.

In the CMIP5 models, the tropospheric NAM signal following both split and
displaced vortex events is weak on average. There is no consistent difference
between the two apart from close to the onset of events when there is a negative
anomaly for split vortex events which extends barotropically through the depth
of the atmosphere. However, looking at two-dimensional tropospheric anomalies in
mean sea-level pressure following split and displaced vortex events shows some
consistent features. A negative NAO-like signal is seen which is of similar
magnitude following both types of event. The Pacific response is much less
robust, with some models simulating negative pressure anomalies, and others
positive. The discrepancy between the Atlantic and Pacific responses suggests
that the annular mode may not be a good metric for stratosphere-troposphere
coupling in the NH.

Almost all models show more negative sea-level pressure anomalies over Siberia
following displaced vortex events than split vortex events. Overall, the
differences in the surface signals following the two types of events are
approximately co-located with the difference in lower-stratospheric geopotential
height, which in turn follow stratospheric PV anomalies. A similar pattern is
also seen in tropopause height in reanalysis data. This suggests the mechanism
behind the different surface responses to split and displaced vortex events is
one local to lower stratospheric PV anomalies, as proposed by
\citet{Ambaum2002}. However, it should be stressed that the similarities in the
NAO response suggest that other mechanisms more sensitive to zonal-mean
anomalies, such as baroclinic instability or planetary wave reflection, also
play a role.

\paragraph{Predictability of the polar stratosphere.} Using hindcast simulations
produced by a stratosphere-resolving seasonal forecast system, no skill has been
found in the prediction of NH SSWs or split or displaced vortex events at lead
times beyond one month. This suggests that the greater skill in seasonal
prediction of the winter NAO in the same system, reported by \citet{Scaife2013},
cannot be attributed to improvements in the representation of the
stratosphere. It may, however, be attributable to other model improvements such
as increased atmospheric and oceanic horizontal resolution.

On the other hand, skillful prediction of the SH stratospheric polar vortex
during the austral spring at seasonal lead times has been found. This skill is
greater than a persistence forecast; indeed, a strong late-summer polar vortex
is related to a weak spring vortex, indicating the importance of
preconditioning. Using the observed relationship between the strength of the
stratospheric polar vortex and polar ozone, it was possible to produce skillful
forecasts of interannual variations in polar stratospheric ozone depletion. This
prediction is at longer lead times than previous forecasts. Because interannual
variability is significant when compared to the long-term ozone depletion trend,
and has a significant impact on UV radiation reaching the Earth's surface, such
forecasts are likely to be of some interest to populations in the SH.

A further feature of the hindcast simulations is that the year 2002, in which
the only observed SH SSW occurred, is also the most extreme of the hindcasts
with almost all ensemble members simulating negative stratospheric wind
anomalies. It was also one of only 2 out of 210 ensemble members which simulate
SH SSW-like events (although these are displaced vortex events, rather than the
split that occurred). This suggests that an increased likelihood of the 2002
event may have been detectable almost two months in advance.

\paragraph{Stratospheric influence on tropospheric predictability.} The same
seasonal forecast system produces skillful forecasts of the austral spring mean
surface SAM at one month lead times. It also accurately simulates the surface
temperature pattern associated with the SAM, such that the SAM forecast skill
leads directly to skillful surface temperature forecasts over much of Antarctica,
New Zealand, and eastern Australia. Interestingly, these forecasts were found to
be more skillful during October--November (2 month lead time), than September (1
month lead time). The same pattern is replicated in a statistical hindcast which
takes as its only input the polar-cap mean geopotential height at 10~hPa on 1st
August. The pattern cannot, however, be replicated by a statistical forecast
based on the ENSO index. This suggests, therefore, that the tropospheric skill
during October-November is largely attributable to the influence of the
predictable stratosphere during this time. The October--November stratospheric
SAM is, in turn, highly predictable due to a strong negative correlation with
the 1st August stratospheric SAM. The fact that the stratospheric influence is
greatest in October-November is also backed-up by observational evidence which
shows the largest stratosphere-leading correlations with the surface during this
time. These results highlight the importance of including a well-resolved
stratosphere and accurate stratospheric initial conditions in seasonal forecast
systems.

%\end{enumerate}

\section{Limitations and further investigations}

The work presented in this thesis has raised a number of questions, and its
limitations have motivated future investigations. Some of these ideas are
discussed below:

\paragraph{What is required for a realistic stratosphere?} Several studies over
the past decade have demonstrated that a more realistic climate and improved
weather forecasts can be achieved using models which resolve the
stratosphere. This has proved persuasive to modelling centres, leading an ever
increasing number to include a representation of the stratosphere. Much of the
work in this thesis has reaffirmed and provided a more detailed picture of the
important role of the stratosphere in surface weather and climate. However, we
have also clearly seen that a high-top is not a sufficient condition for a
realistic stratosphere. A major challenge for the stratospheric community is to
identify where limited computing resources should be best spent in simulating
the stratosphere.

It was shown in Figure \ref{fig:aspect_vert_res} that there appears to be a
relationship between the average aspect ratio of the stratospheric polar vortex
and vertical resolution among the CMIP5 models. Although this relationship is
backed up by the physical understanding of the influence of fine-scale vertical
structure on planetary wave propagation in this region, it is not highly
statistically significant. Furthermore, the relationship does not hold when
models of the same family but different resolution are compared. This highlights
a general limitation of multi-model, `ensemble of opportunity', studies such as
that in Chapter \ref{cha:models}; so many variables are changed between
different model simulations it is difficult to attribute model differences to
any one factor (also discussed by \citet{Tebaldi2007}).

These issues could be addressed by performing a series of model integrations in
which resolution is systematically varied. This should involve horizontal as
well as vertical resolution, since it is likely that horizontal resolution is
important for resolving steep PV gradients at the vortex edge which affect wave
propagation (although no significant relationships with horizontal resolution
were found in Chapter \ref{cha:models}). Such a study need not be very
computationally expensive, since it was shown in Figure
\ref{fig:cmip5_moments_scatter} that the average state of the vortex is strongly
related to the frequency of extreme events. Hence, it is only necessary to
simulate enough years to determine the average state, which is far fewer than is
necessary to determine a realistic climatology of extremes. If such a study
finds any thresholds in resolution, beyond which stratospheric biases are much
reduced, then this could act as a recommended resolution for modelling centres.


\paragraph{Synchronisation of the stratosphere and troposphere?} A large part of
this thesis has focussed on developing an increased understanding of the spatial
stucture of stratosphere-troposphere coupling. However, the mechanisms discussed
have retained the traditional temporal chain of causation of the form: \emph{A
  causes B; B causes C} etc.. In the real, chaotic atmosphere, it is unlikely
that such a simple mechanism exists. A new approach to understanding
stratosphere-troposphere coupling could focus on the synchronisation of modes of
variability. Indeed, we have seen here that such modes may be important because
of the barotropic nature of split vortex events, suggesting an excitation of the
barotropic mode during these events.

The instantaneous phase of an arbitrary signal can be calculated through the
Hilbert transform \citep{Pikovsky2001}, and several recent studies have applied
this technique to investigate the phase synchronisation of modes of climate
variability. For example, \citet{Maraun2005} found evidence for intermittent
synchronization of ENSO and the Indian Monsoon, which they suggested were
initiated by volcanic eruptions. \citet{Read2012} also found phase
synchronisation between the QBO and the semi-annual oscillation (a oscillation
of upper stratosheric equatorial zonal winds with a period of six months), but
with a non-stationary ratio of frequencies between the two oscillations.

In principle, a similar technique can be applied to study
stratosphere-troposphere coupling. This could look, for instance, at whether
stratospheric and tropospheric modes are synchronised following particular
events, such as SSWs. A difficuly in this case is deciding which are the
relevant modes of variability. We could choose the NAM, although, as discussed
previously, this has different physical interpretations in the troposphere and
stratosphere. \citet{Thompson2014} suggested the existence of barotropic and
baroclinic annular modes; defined as the leading modes of variability of
zonal-mean kinetic energy and eddy kinetic energy respectively. However, they
found that this separation is less easy to perform in the NH than the SH.

Modes of variability can also be separated by their temporal structure. This is
traditionally carried out through a Fourier spectrum analysis, however, the more
modern technique of empirical mode decomposition (EMD) \citep{Huang1998} may be
better suited to studying stratosphere-troposphere coupling. EMD has also been
used in atmospheric science by \citet{Coughlin2004} to study the influence of
solar variability on the stratosphere. The method decomposes a given time series
into a finite number of `modes', each of which have a characteristic
frequency. Unlike Fourier analysis, this frequency is allowed to vary to some
degree, so the modes need not be perfectly periodic. As such, it is more
applicable to time series of finite length and with a pronounced seasonal
variability, such as is seen in the atmosphere. Figure \ref{fig:emd} shows an
example of EMD applied to NH polar-cap average geopotential height. It can be
seen that the technique identifies different time scale, quasi-periodic modes of
variability, and that these modes occasionally appear coherent through the
stratosphere and troposphere. Closer inspection also reveals mode~2 to be more
baroclinic than mode~1, consistent with \citet{Thompson2014} who found their
periodic baroclinic mode to have a longer time scale than the barotropic mode
(although this analysis was for the SH). Further investigations could be carried
out to analyse the physical relevance of the modes and to quantify
synchronisation between the stratosphere and troposphere. Also, given the
results in this thesis which suggest the NAO is a more relevant metric than the
NAM in stratosphere-troposphere coupling, it may be more appropriate to study
local rather than hemispheric modes of variability in the troposphere.

\begin{figure}[t]
  \centering
  \noindent\includegraphics[width=\textwidth,angle=0]{figures/chapter-conclusions/EMD2.pdf}\\
  \caption[EMD timeseries]{Time slice of empirical modes 1 and 2 of NH polar-cap
    averaged (60-90$^{\circ}$N) geopotential height in ERA-Interim data. Yellow
    and green stars represent the onset of displaced and split vortex events
    respectively.}\label{fig:emd}
\end{figure}


\paragraph{What factors influence seasonal forecast skill?} In Chapter
\ref{cha:seas} an increase tropospheric seasonal forecast skill was attributed
to the influence of the stratosphere through analysis of a statistical
forecast. As previously discussed, this method has the disadvantage that it
cannot rule out a third factor which separately influences both the stratosphere
and troposphere (it was shown that ENSO can be ruled out as such a factor, but
it would be impossible to consider all potential influences). A more robust
understanding of the factors influencing seasonal forecast skill can be gained
by performing a series of hindcasts in which these factors are systematically
changed.

An interesting case study for this investigation would be the 2002 austral
spring, since it was shown that the anomalous nature of this season was, to some
degree, captured two months in advance. For instance, the 2002 hindcasts could
be re-run with an opposite phase of the QBO, different tropical Pacific or
Southern Ocean SSTs, or different polar stratospheric initial conditions. The
change in forecasts of the stratospheric polar vortex and the surface SAM could
then be analysed, indicating which factors are most important. The main
difficulty in this investigation would probably come in imposing these different
initial conditions in a physically consistent manner (e.g., conserving angular
momentum).

\paragraph{Would interactive ozone chemistry improve seasonal forecast skill?}
The seasonal forecast system analysed in Chapter \ref{cha:seas} did not include
interactive chemistry, with ozone concentrations set to a climatology. It is
therefore unable to capture the feedback between ozone concentrations and the
stratospheric circulation, or zonal asymmetries in ozone. \citet{Waugh2009}
suggested that such asymmetries could have a significant impact on tropospheric
climate. This motivates an additional investigation as to whether improved
stratospheric or tropospheric forecasts may be achieved by including interactive
chemistry in a seasonal forecast system. Such a chemistry scheme is likely to be
expensive, so the investigation should determine which reactions have the most
impact on forecast skill.


% \paragraph{Is there decadal variability in SSWs?} Looking by eye at the events
% detected in Chapter \ref{cha:moments}, it appears that they cluster in time. For
% instance, there are 10 events in the 1970s, but only 4 in the 1990s. Figure
% \ref{fig:decadal} shows an attempt to analyse whether this decadal variability
% is statistically significant. It shows (solid black line) the frequency of given
% numbers of split/displaced vortex events within a 10-year moving window
% (shifting by one year at a time) in the ERA data set. Also shown (dashed black
% line) is the same calculation applied to randomly shuffled events. Error bars
% are calculated from the distribution of frequencies of the randomly shuffled
% events. It can be seen that the frequencies of 8/9 events and 3 events per
% decade are sightly statistically significant from the random variability, so it
% might be inferred that there is statistically significant decadal variability.

% Figure \ref{fig:decadal} also shows the same calculation applied to a 2 ensemble
% member 1860--2005 historical simulation of the HadGEM2-CC model (a total of 290
% years). It can be seen in this case that the simulation is not distinct from
% random variability. However, \citet{Schimanke2011} did find significant decadal
% variability in a coupled ocean-atmosphere GCM, although their model simulated
% only 2 events per decade. A future investigation could aim to resolve this issue
% by studying decadal variability in a greater number of models (such as the CMIP5
% ensemble). Longer simulations than those studied in Chapter \ref{cha:models}
% would be required in order to achieve statistically signifiant results. It would
% also be interesting to compare historical and control simulations in order to
% determine if any decadal variability is externally driven or internally
% generated.

% \begin{figure}[t]
%   \centering
%   \noindent\includegraphics[width=0.6\textwidth,angle=0]{figures/chapter-conclusions/events_decadal.pdf}\\
%   \caption[Decadal variability of SSWs]{(solid lines) Normalised frequency of
%     the number of SSWs detected in a 10-year moving window. (dashed lines)
%     Average of 10-year moving window frequency of SSWs whose timing is shuffled
%     randomly 1000 times. Error bars depict the 2.5--97.5\%
%     range of the randomly shuffled events.}\label{fig:decadal}
% \end{figure}


%\paragraph{NAM or NAO?}



% \section{Personal outlook}

% The Earth's atmosphere contains complex three-dimensional variability that is
% both difficult to conceptualise and to visualise. Historically, because of a
% relative lack of data and computing power, this has restricted our understanding
% of this variabiliy to simplified metrics such as the zonal mean wind, Fourier
% decomposition, or the annular modes. However, as our resources have increased
% there has been a clear trend towards the development of more complex diagnostic
% tools, such as feature tracking algorithms or empirical mode decomposition. I
% hope that the moment diagnostic tools and their application developed in this
% thesis contributes towards this trend.

% Numerous studies over the past decade have demonstrated a more realistic climate
% and improved weather forecasts can be achieved using models which resolve the
% stratosphere. This has proved persuasive to modelling centres, and although
% there is still some way to go, I believe it is now a matter of time before
% almost all weather and climate prediction models include a stratosphere. Much of
% the work in this thesis has reaffirmed the importance of the stratosphere in
% surface weather and climate (although providing a more detailed
% picture). However, we have also clearly seen that a ``high-top'' is not a
% sufficient condition for a realistic stratosphere. The next major challenge for
% the stratospheric community is to identify where limited computing resources
% should be best spent in simulating the stratosphere; is vertical resolution near
% the tropopause most important? Or horizontal resolution to resolve steep PV
% gradients near the vortex edge? Or are gravity wave parametrisations the largest
% source of error? It is likely that different factors will affect different
% aspects of stratospheric variability, and so we must choose which to
% prioritise. 



%%% Local Variables:
%%% mode: latex
%%% TeX-master: "thesis"
%%% End:
