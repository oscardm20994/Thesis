\chapter{Conclusions}
\label{cha:conclusions}

\section{Summary of results}

The main findings of this thesis are summarised as follows: 
%\begin{enumerate}

\textbf{Multi-decadal variations in polar vortex strength.}
While most works focus on seasonal effects of vortex variability, their non-uniform distribution in time in reanalysis datasets suggests the possibility of longer term variations. We show that in a pi-control integration of a GCM (UKESM), the strength of the vortex exhibits spectral power on timescales between 30 and 100 years. This power is particularly apparent when the vortex index, either in the form of an event based SSW timeseries (as is analysed in chapter 3) or a continuous metric (e.g. the NAM, chapter 4), is smoothed to capture the occurrences of intervals of consecutive NH winters exhibiting the same type of anomalous behaviour. These signals are highly non-stationary with the most persistent spectral feature appearing for approximately 450 years on the $\sim$90 year timescale. 

\paragraph{Drivers of multi-decadal variability in the vortex.}
We identified co-variations between multi-decadal signals in SSWs and the NAM timeseries (which capture with amplitude variations in a deep QBO metric. These amplitude variations appear most pronounced in the westerly phase of the QBO and exhibit most coherent behaviour with the appearance of hiatus intervals of SSWs. This association is consistent with the well studied Holton-Tan relationship in which the presence westerly QBO phase in NH winter leads to greater latitudinal dispersion of vertically propagating planetary wave activity and subsequently a stronger vortex.  

The vortex indices exhibit minimal co-variability with major modes of surface variation such as ENSO and the Aleutian Low and these modes fail to account for large portions of the SSW and NAM power spectrum.



\textbf{Surface influence of signals in the vortex.}

\textbf{Stratospheric role in recent AMOC trends}

\textbf{Importance of QBO vertical structure in teleconnections}

\section{Limitations and further investigations}

The work presented in this thesis has raised a number of questions, and its
limitations have motivated future investigations. Some of these ideas are
discussed below:

\paragraph{Are multi-decadal oscillations in SSWs real?} 

Despite the novel and potentially important results presented in chapter 2, our analysis has a number of limitations which we intend to address in a body of further work. First, while we have identified statistically coherent signals in the QBO and SSWs, this does not confirm causal connections between them. A possible natural approach to determining connections could be to apply multi-linear regression techniques to the SSW and QBO time series while also incorporating the surface metrics mentioned above to evaluate the portion of $SSW_{5yr}$ variance accounted for by other variables. We expect however, that this method may suffer from a number of pitfalls, namely the non-stationary nature of signals we are considering as well as the potential non-orthogonality of explanatory variables (ENSO, QBO, AL). Instead of using such methods, we propose to test the robustness of our result by designing a set of sensitivity experiments in which QBO variability is fixed. These experiments will take the form of a set of GCM simulations in which the amplitude of the deep QBO is either allowed to evolve freely (as is the case in the PI control) or is nudged towards a constant value (i.e. with no amplitude variation). Comparison of multi-decadal signals in SSWs from these runs could confirm the key causal link between deep QBO amplitude and the vortex which is suggested by results in this study. Second, our result is derived from a single simulation of a single GCM - UKESM. The model exhibits biases in its mean SSW rate, which we attempt to allow for by only considering Dec-Mar SSWs, as well as a lengthened period of the QBO compared to ERA-Interim \citep{Bushell2020}. This could lead to an over-inflated HT strength due to longer persistence of QBO phases altering the mid-latitude waveguide through NH winter. As a result, teleconnections found in UKESM may not be fully representative of the real atmosphere. Recent work has also shown a large degree of inter-model variability exists in representations of the QBO as well as SSWs \citep{Bushell2020,ayarzag2020} so it is possible the result from this study is model dependant. Further work could consider a set of CMIP6 models and examine the signals in QBO and SSW time series. 

Recent work has also shown a large degree of inter-model variability exists in representations of the QBO as well as SSWs \citep{Bushell2020,Ayarz2020} so it is possible the result from this study is model dependant. Additional analysis of long simulations from different models is required to verify these results. 
Further exploration of the source of the long-term variability in amplitude of the QBO-W phase is also required. While a direct influence of tropical SSTs on long-term variability in SSW frequency has not been supported by this study there may nevertheless be an important teleconnection via the QBO, in which the SSTs influence the QBO which subsequently influences the SSW frequency via the Holton Tan relationship. Initial investigation through cross spectrum analysis of the deep QBO index with ENSO and selected SST indices shows some contribution from each of the regions (supp figures A6 and A7), not surprisingly because of the tropical source of equatorial waves that are known to drive the QBO. A closer examination of the precise nature of forcing of the QBO-W phase in the model, in terms of Kelvin and gravity wave forcing would be helpful (but outside the scope of this study).

despite the possible shortcomings of this analysis, findings may point to a novel source of understanding and predictability in SSW emergence. This in turn could impact decadal predictability of NH winter surface climate (given SSWs' role in stratosphere-troposphere coupling), an area in which significant improvement can be made on current capabilities \citep{Zhang2019}. This work also suggests the possible importance of a new QBO metric, the deep QBO amplitude modulation, which couples with vortex variability. This points towards the need to include such metrics in future work analysing HT mechanisms in observations and model data.



\paragraph{Are Vortex-AMOC interactions present in the real climate system?} 

Another method for testing this robustness could involve a set of nudging experiments with a GCM in which variability in vortex strength is imposed on decadal timescales to a pre-defined period. Comparisons of the AMOC in such simulations with runs that imposed no decadal variability in the vortex could test explicitly the role of resonant signals in the NAM$_{10}$ and the AMOC. 