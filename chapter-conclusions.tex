\chapter{Conclusions}
\label{cha:conclusions}

\section{Summary of results}

The main findings of this thesis are summarised as follows: 
%\begin{enumerate}

\paragraph{Multi-decadal variations in polar vortex strength:}
While most works focus on seasonal effects of vortex variability, their non-uniform distribution in time in reanalysis datasets suggests the possibility of longer term variations. We show that in a pi-control integration of a GCM (UKESM), the strength of the vortex exhibits spectral power on timescales between 30 and 100 years. This power is particularly apparent when the vortex index, either in the form of an event-based SSW time series (as is analysed in chapter 3) or a continuous metric (e.g. the NAM$_{10}$, chapter 4), is smoothed to capture the occurrences of intervals of consecutive NH winters exhibiting the same type of anomalous behaviour. These signals are highly non-stationary with the most persistent spectral feature appearing for approximately 450 years on the $\sim$90-year timescale. 

\paragraph{Drivers of multi-decadal variability in the vortex:}
We identified co-variations between 90-year period signals in SSWs and the NAM time series with amplitude variations in a deep QBO metric. These amplitude variations appear most pronounced in the westerly phase of the QBO and exhibit the most coherent behaviour with the appearance of hiatus intervals of SSWs. This association is consistent with the well studied Holton-Tan relationship in which the presence of the westerly QBO phase in NH winter leads to greater latitudinal dispersion of vertically propagating planetary wave activity and subsequently a stronger vortex. The vortex indices exhibited minimal co-variability with major modes of surface variation on multi-decadal timescales and Wavelet spectra for ENSO and the AL fail to account for large portions of the SSW and NAM wavelet power spectra.

\paragraph{Surface influence of signals in the vortex:}
We analysed possible influences of multi-decadal signals in the vortex on tropospheric, surface and ocean variability with a particular focus on the Atlantic sector given the timescales involved (similar to the AMOC's characteristic periods) as well as the in-season coupling between the vortex and this region. We found oscillatory responses in Atlantic circulation (specifically the AMOC) to intervals of approximately 6-8 years of persistent anomalous vortex behaviour. The origin of these responses appears to be highly non-stationary with spectral power in the vortex and the AMOC at periods of 30 and 50 years as well as feedback mechanisms involving the stratospheric influence over the NAO. 

\paragraph{AMOC influence on the QBO:}
We subsequently analysed possible sources of QBO amplitude modulation to determine whether variations were internally generated in the atmosphere or driven by low-frequency surface modes. While indices of major surface variation such as ENSO and the AL failed to account for large parts of the QBO and vortex wavelet spectra, significant co-variations between the QBO amplitude and deep convection indices over the eastern Pacific region were found. This co-variation supports the hypothesis that tropical upwelling and deep convection alter the vertical structure of the QBO and hence the amplitude of a deep QBO index, providing a source of 90-year oscillations in the vortex from the surface. Co-variability between the AMOC and East Pacific deep convection was also shown at 90-year periods and AMOC variability on this timescale was distinct to that at 50 and 30-year periods as it is not accompanied by similar NAO variations. This result suggests a physical pathway by which 90-year timescale variations in the AMOC influence the vortex via modulation of tropical east pacific convection and modulation of the deep QBO amplitude. 

\paragraph{Stratospheric role in recent AMOC trends:}
Using the lagged response of the AMOC to anomalous vortex intervals in UKESM, we estimated the contribution to recent negative trends in the observed AMOC from the interval of anomalously strong vortex winters in the 1990s. We found that the magnitude of the filtered NAM$_{10}$ extreme, which captures both the strength and persistence of the vortex anomaly, was directly proportional to the AMOC anomaly 17 years later with a remarkable degree of correlation (r = -0.908). Furthermore, this analysis yielded an estimate of the stratospheric contributions to the AMOC downturn between 2004 and 2012 of $-0.89Sv$ ($-0.5Sv$ at 30$\degree$N). This contribution represents approximately 30\% (17\% at 30$\degree$N) of the total negative trend and suggests a key role for the vortex, a mode of internal variability, in driving observed AMOC variations. 

\paragraph{Importance of QBO vertical structure in teleconnections:}
We designed a set of nudging experiments to explicitly test the influence of vertical coherence in the QBO (i.e. agreement in phase across pressure levels) on the strength of teleconnections with the vortex and the surface (the AO and NAO). We found that, in early winter, the HT link in which QBOE conditions are associated with a weakened vortex \citep{HoltonJamesRTan1980} is marginally enhanced in an experiment that imposed a perpetually deep QBO compared to one which imposed a shallow QBO as well as a free-running model (the pi-clim cntrl). However, the enhanced early winter vortex response in the deep experiment was not reflected in the MSLP response, a finding not consistent with \cite{andrewsObserved2019d} who showed an enhanced MSLP response using a deep QBO metric. 

\paragraph{Opposite responses to previous work:}
The QBO responses of the deep experiment in mid-late winter were dominated by large amplitude anomalies in the vortex and MSLP with QBOE(W) phases associated with a stronger (weaker) vortex and positive NAO and AO patterns. These are of the opposite sign to an expected HT link as well as the results of similar previous works \citep{graySurface2018b, andrewsObserved2019d}. We found that these unexpected responses were associated with large amplitude anomalies in the SAO region likely caused by preferential filtering of equatorial waves by the deep QBO phase below. These SAO winds exhibited significant correlations with the vortex and were shown to modulate the high latitude ZMZW response via alterations in meridional propagation of EP flux away from the vortex under QBOE conditions. We concluded that the deep QBO experiment induces unrealistically large SAO responses which, in late winter, obscure signals from all other possible pathways involved in QBO teleconnections. 

\section{Limitations and further investigations}
\label{sec:limitations}
The work presented in this thesis has raised a number of questions, and its limitations motivate further investigation. Some of these ideas are discussed below:

\paragraph{Determining causality in teleconnections:}
While time series analyses such as those presented in chapters 3 and 4 can highlight associations between key modes of variability, they are less able to determine causality. Where possible, we have selected indices that confirm well-known in-season causality, such as an early winter QBO index compared with a mid-winter SSW index (as in chapter 3) or the vortex impact on Atlantic MSLP and subsequent links to the AMOC (chapter 4). Nevertheless, determining causality on longer timescales is difficult as the climate system is extremely complex, with many different interactions between modes of variability. These complexities can hamper simple statistical analysis such as multi-linear regression (as used in chapter 3 on SSW, QBO and surface time series)it fails to capture the non-stationary nature of signals and suffers from the potential non-orthogonality of explanatory variables (ENSO, QBO and the AL). While we employed more sophisticated techniques such as wavelet analysis to account for non-stationarity, establishing causal connections remains an issue.

Further work into determining causality could involve tailor-made statistical methods such as the Causal Effect Network (CEN) framework \citep{Kretschmer2016} which has been developed for the analysis of causal connections in complex systems of time series data. This method identifies connections between pairs of time-series selected from a large group of indices while filtering out spurious correlations that arise between series which are not causally linked. For example, a correlation may arise between two variables that have no direct physical link if they are both driven by the same underlying process or via autocorrelation effects \citep{rungeQuantifying2014}. Implementation of the CEN technique on indices such as those analysed in chapter 4 (the NAM$_{10}$, NAO, sub-polar North Atlantic heat flux and the AMOC) may aid in improving understanding of the physics pathways involved in interactions further. For example, it may establish whether the NAO-AMOC interaction, which was shown to arise following a filtered NAM$_{10}$ extreme, is prominent in the absence of a stratospheric anomaly or if the extreme vortex behaviour is a necessary triggering mechanism. 

\paragraph{Targeted experiments:}
To overcome these issues with demonstrating causality in the teleconnections considered in chapters 3 and 4, we also recommend a set of targeted sensitivity experiments using GCM simulations whose design is informed by our results. These experiments impose different multi-decadal signals on key climate indices using a similar nudging method as utilised in chapter 5 \citep{telfordTechnical2008}.

\begin{itemize}
    \item To test the connection between multi-decadal signals in QBO amplitude modulation and SSWs suggested in chapter 3, a set of simulations which impose variability in QBO amplitude: For example, one simulation could prescribe oscillating QBO amplitude on the 90-year timescales while another imposes a constant QBO amplitude. Comparison of multi-decadal signals in SSW frequency in these experiments could demonstrate a causal connection between QBO amplitude and vortex variations.  
    
    \item Similarly, Another method for testing this robustness of vortex-AMOC interactions could involve a set of nudging experiments in which variability in vortex strength is imposed on decadal timescales to a pre-defined period. Comparisons of the AMOC in such simulations with runs that imposed no decadal variability in the vortex could explicitly test the role of NAM$_{10}$ variations in multi-decadal AMOC variability. 
\end{itemize}

\paragraph{Limitations of a single model approach:}
The results from chapters 3 and 4 regarding the interaction between multi-decadal variations in the polar vortex and other parts of the climate system is derived from a single simulation of a single GCM - the UKESM pi-control. This poses a potentially significant issue as the model exhibits biases in its representation of some key modes of climate variability. For example, November SSWs are significantly over-represented (figure \ref{fig:SSW_histogram}) which may affect the progression of its strength through the winter as the vortex recovers from a November disruption which precludes further significant wave driving. The mean period of the QBO in the model is also longer than that exhibited by ERA-Interim \citep{bushellEvaluation2020b}. This could lead to an over-inflated HT strength due to the longer persistence of QBO phases altering the mid-latitude waveguide through, on average, a larger portion of NH winter. Finally, the model under-represents decadal-scale variation in AMOC over the historical period \citep{robsonEvaluation2020d}. 

These biases may result in teleconnections in the model that are not be fully representative of the real atmosphere. Recent work has also shown a large degree of inter-model variability exists in representations of the QBO \citep{bushellEvaluation2020b} as well as SSWs \citep{ayarzaguenaUncertainty2020b} suggesting the possibility that results from our studies are model dependant. Further work could consider a suite of CMIP6 models and examine the multi-decadal signals in the vortex, QBO, NAO and AMOC to test the robustness of the various teleconnections established in the analysis.   

\paragraph{More effective QBO nudging:}
In chapter 5, we designed a set of nudging experiments which relaxed the QBO winds towards idealised fields (the deep and shallow QBO). However, a comparison of these fields with the model winds (figures \ref{fig:winds_on_levs_deep} and \ref{fig:winds_on_levs_shallow}) revealed the westerly phase of the QBO is under-represented in the model output despite the applied relaxation. This problem was particularly prevalent when nudging towards a deep QBO and poses potential issues with future QBO studies as it leads to significant asymmetries in nudged QBO phases.

Further analysis into rectifying this bias could involve a set of sensitivity experiments that implement nudging of the QBO towards the same idealised fields using smaller values for nudging timescales ($\tau$ in equation \ref{eq:nudging}) as we used in our experiments ($\tau$ = 6 hours). $\tau$ is generally set to 6 hours in line with the temporal resolution of reanalysis files commonly used for nudging (and is similar to that used in other families of models) \citep{telfordTechnical2008} however, using idealised fields, which can be generated at any temporal resolutions, removes this constraint. Assessing the reduction in QBO bias achieved by reducing $\tau$ as well as the possible model instability introduced by using a short nudging timescale, could aid in implementing more effective QBO nudging in future studies.

\paragraph{Preventing unrealistic SAO anomalies:}
While our QBO experiments successfully imposed a deep and shallow QBO (figure \ref{fig:experiment_QBOs}), the deep experiment induced unrealistically large anomalies in the SAO region (figure \ref{fig:HT_deep}). These anomalies were shown to lead to large vortex and surface responses of the opposite sign to an expected HT link as well as the results of \cite{andrewsObserved2019d}. This SAO signal obscured other possible physical pathways associated with the QBO and prevented a definitive evaluation of the role of vertical coherence in QBO teleconnections. 

To analyse other possible pathways closer, we recommend further simulations which impose a deep QBO and also nudges the SAO winds. In the SAO region, winds could be nudged towards climatological values calculated from another simulation (the pi-clim cntrl, for example) in order to avoid the large anomalies induced by the deep QBO in our experiment. Eradicating the influence of these SAO responses may aid in isolating the effect of other physical mechanisms involved in QBO teleconnections. For example, via modulation of the vortex (i.e. the HT link) or via the sub-tropical pathway suggested in \cite{graySurface2018b} (and evident from composite analysis of our deep experiment) in which QBO phases lead to NH MSLP responses via perturbations in the sub-tropical jet strength. Isolation of these other effects may develop a better understanding of the mechanisms at play in the presence of a vertically coherent QBO.  
















