\chapter{Conclusions}
\label{cha:conclusions}

\section{Summary of results}

The main findings of this thesis are summarised as follows: 
%\begin{enumerate}

\textbf{Multi-decadal variations in polar vortex strength.}
While most works focus on seasonal effects of vortex variability, their non-uniform distribution in time in reanalysis datasets suggests the possibility of longer term variations. We show that in a pi-control integration of a GCM (UKESM), the strength of the vortex exhibits spectral power on timescales between 30 and 100 years. This power is particularly apparent when the vortex index, either in the form of an event based SSW timeseries (as is analysed in chapter 3) or a continuous metric (e.g. the NAM, chapter 4), is smoothed to capture the occurrences of intervals of consecutive NH winters exhibiting the same type of anomalous behaviour. These signals are highly non-stationary with the most persistent spectral feature appearing for approximately 450 years on the $\sim$90 year timescale. 

\paragraph{Drivers of multi-decadal variability in the vortex.}
We identified co-variations between 90 year period signals in SSWs and the NAM timeseries with amplitude variations in a deep QBO metric. These amplitude variations appear most pronounced in the westerly phase of the QBO and exhibit most coherent behaviour with the appearance of hiatus intervals of SSWs. This association is consistent with the well studied Holton-Tan relationship in which the presence westerly QBO phase in NH winter leads to greater latitudinal dispersion of vertically propagating planetary wave activity and subsequently a stronger vortex. The vortex indices exhibited minimal co-variability with major modes of surface variation on multi-decadal timescales and Wavelet spectra for ENSO and the AL fail to account for large portions of the SSW and NAM wavelet power spectra.

\textbf{Surface influence of signals in the vortex.}
We analysed possible influences of multi-decadal signals in the vortex on tropospheric, surface and ocean variability with a particular focus on the Atlantic sector given the timescales involved (similar to the AMOC's characteristic periods) as well as the in-season coupling between the vortex and this region. We found oscillatory responses in Atlantic circulation (specifically the AMOC) to intervals of approximately 6-8 years of persistent anomalous vortex behaviour. The origin of these responses appears to be highly non-stationary with spectral power in the vortex and the AMOC at periods of 30 and 50 years as well as feedback mechanisms involving the stratospheric influence over the NAO. 

\textbf{AMOC influence on the QBO}
We subsequently analysed possible sources of QBO amplitude modulation to determine whether variations were internally generated in the atmosphere or driven by low frequency surface modes. While indices of major surface variation such as the Nino3.4 and AL failed to account for large parts of the QBO and vortex wavelet spectra, significant co-variations between the QBO amplitude and deep convection indices over the east pacific region were found. This co-variation supports the hypothesis that tropical upwelling and deep convection alter the vertical structure of the QBO and hence the amplitude of a deep QBO index, providing a source of 90 year oscillations in the vortex from the surface. Co-variability between the AMOC and East Pacific deep convection was also shown at 90 year periods and AMOC variability on this timescale was distinct to that at 50 and 30 year periods as it is not accompanied by similar NAO variations. This results suggests a physical pathway by which 90 year timescale variations in the AMOC influence the vortex via modulation of tropical east pacific convection and modulation of the deep QBO amplitude. 

\textbf{Stratospheric role in recent AMOC trends}
Using the lagged response of the AMOC to anomalous vortex intervals in UKESM, we estimated the contribution to recent negative trends in the observed AMOC from the interval of anomalously strong vortex winters in the 1990s. We found that the magnitude of the filtered NAM$_{10}$ extreme, which captures both the strength and persistence of the vortex anomaly, was directly proportional to the AMOC anomaly appearing 17 years later with a remarkable degree of correlation (r = -0.908). Furthermore this analysis yielded an estimate to the stratospheric contributions to the AMOC at 50N downturn between xxx and xxx of -0.89Sv (-0.5Sv at 30N). This contribution represents approximately 30\% (17\% at 30N) of the total negative trend and suggests a key role for the vortex, a mode of internal variability, in driving observed AMOC variations. 

\textbf{Importance of QBO vertical structure in teleconnections}

\section{Limitations and further investigations}
\label{sec:limitations}
The work presented in this thesis has raised a number of questions, and its limitations motivate further investigation. Some of these ideas are discussed below:

\paragraph{Determining causality in teleconnections:}
While time series analyses such as those presented in chapters 3 and 4 can highlight associations between key modes of variability, they are less able to determine causality. Where possible we have selected indices that confirm well-known in-season causality, such as an early winter QBO index compared with a mid-winter SSW index or the vortex impact on Atlantic MSLP, but determining causality on longer timescales is difficult. The climate system is extremely complex, with many different interactions between modes of variability. 

In chapter 3, we applied multi-linear regression techniques to the SSW and QBO time series while also incorporating the surface metrics mentioned above to evaluate the portion of $SSW_{5yr}$ variance accounted for by other variables (tables 1-3). This method suffers from a number of pitfalls, namely the non-stationary nature of signals we are considering as well as the potential non-orthogonality of explanatory variables (ENSO, QBO, AL). Further work into determining causality could involve more sophisticated statistical methods such as the Causal Effect framework introduced in xxxx which identifies connections between pairs of time-series selected from a large group of indices while filtering out spurious correlations that may arise due to the influence of other indices. For example, a correlation may arise due to 

Instead of using such methods, we propose to test the robustness of our result by designing a set of sensitivity

\paragraph{Targeted experiments:}
To overcome the issue with demonstrating causality in teleconnections, we recommend a set of targeted sensitivity experiments whose design is informed by the results from this thesis.

-  experiments in which QBO variability is fixed. These experiments will take the form of a set of GCM simulations in which the amplitude of the deep QBO is either allowed to evolve freely (as is the case in the pi-control) or is nudged towards a constant value (i.e. with no amplitude variation). Comparison of multi-decadal signals in SSWs from these runs could confirm the key causal link between deep QBO amplitude and the vortex which is suggested by results in this study.

- AMOC-QBO testing. Nudged AMOC run which imposes some variability. Measure the possible QBO response. 














\paragraph{Limitations of a single model approach:}
%Second, our result is derived from a single simulation of a single GCM - UKESM. The model exhibits biases in its mean SSW rate, which we attempt to allow for by only considering Dec-Mar SSWs, as well as a lengthened period of the QBO compared to ERA-Interim \citep{bushellEvaluation2020}. This could lead to an over-inflated HT strength due to longer persistence of QBO phases altering the mid-latitude waveguide through NH winter. As a result, teleconnections found in UKESM may not be fully representative of the real atmosphere. Recent work has also shown a large degree of inter-model variability exists in representations of the QBO as well as SSWs \citep{bushellEvaluation2020,ayarzaguenaUncertainty2020} so it is possible the result from this study is model dependant. Further work could consider a set of CMIP6 models and examine the signals in QBO and SSW time series. 




\paragraph{Considering non-stationarities in climate variability}
The system is also clearly non-stationary, as evident in our simulation where the QBO-SSW interaction shows power at periodicities of 60-90 years for 450 years but is absent in the early half of the simulation. While this complexity means that it is extremely challenging to disentangle the influences or to attribute causality, improved understanding of individual links in this complex system, such as the relationship between the QBO and SSWs, will nevertheless contribute to improved understanding of the whole complex system.  

Composite and wavelet analysis of time-series data presented here is effective for identifying co-variability between the stratosphere and ocean. However, these techniques are less capable of demonstrating causality within modes of variability and so additional targeted experiments are required to establish this. Nevertheless, we have suggested physical pathways for vortex-ocean interactions on multi-decadal timescales which rely on well established teleconnections (e.g. the in-season vortex-NAO connection) and demonstrate a possible key role of these interactions in recent AMOC behaviour. Our results stress the role and importance of non-stationary signals for understanding long term variability of the climate system. This complexity can be detrimental to analysis which relies on common stationary methods such as Fourier analysis. As a result, improved understanding and diagnosis of non-stationary climate variations as well as their underlying mechanisms is key to overcome these difficulties. 






\paragraph{Are Vortex-AMOC interactions present in the real climate system?} 

Another method for testing this robustness could involve a set of nudging experiments with a GCM in which variability in vortex strength is imposed on decadal timescales to a pre-defined period. Comparisons of the AMOC in such simulations with runs that imposed no decadal variability in the vortex could test explicitly the role of resonant signals in the NAM$_{10}$ and the AMOC. 