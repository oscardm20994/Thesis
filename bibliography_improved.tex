@article{akinwandeVariance2015,
  title = {Variance {{Inflation Factor}}: {{As}} a {{Condition}} for the {{Inclusion}} of {{Suppressor Variable}}(s) in {{Regression Analysis}}},
  shorttitle = {Variance {{Inflation Factor}}},
  author = {Akinwande, Olusegun and Dikko, H.G and Agboola, Samson},
  year = {2015},
  month = jan,
  volume = {05},
  pages = {754--767},
  doi = {10.4236/ojs.2015.57075},
  abstract = {Suppression effect in multiple regression analysis may be more common in research than what is currently recognized. We have reviewed several literatures of interest which treats the concept and types of suppressor variables. Also, we have highlighted systematic ways to identify suppression effect in multiple regressions using statistics such as: R 2 , sum of squares, regression weight and comparing zero-order correlations with Variance Inflation Factor (VIF) respectively. We also establish that suppression effect is a function of multicollinearity; however, a suppressor variable should only be allowed in a regression analysis if its VIF is less than five (5).},
  file = {/Users/oscardimdore-miles/Zotero/storage/HJFG6S7Q/Akinwande et al. - 2015 - Variance Inflation Factor As a Condition for the .pdf},
  journal = {Open Journal of Statistics}
}

@article{albersVortex2014,
  title = {Vortex {{Preconditioning}} Due to {{Planetary}} and {{Gravity Waves}} Prior to {{Sudden Stratospheric Warmings}}},
  author = {Albers, John R. and Birner, Thomas},
  year = {2014},
  month = nov,
  volume = {71},
  pages = {4028--4054},
  publisher = {{American Meteorological Society}},
  issn = {0022-4928, 1520-0469},
  doi = {10.1175/JAS-D-14-0026.1},
  abstract = {{$<$}section class="abstract"{$><$}h2 class="abstractTitle text-title my-1" id="d544e2"{$>$}Abstract{$<$}/h2{$><$}p{$>$}Reanalysis data are used to evaluate the evolution of polar vortex geometry, planetary wave drag, and gravity wave drag prior to split versus displacement sudden stratospheric warmings (SSWs). A composite analysis that extends upward to the lower mesosphere reveals that split SSWs are characterized by a transition from a wide, funnel-shaped vortex that is anomalously strong to a vortex that is constrained about the pole and has little vertical tilt. In contrast, displacement SSWs are characterized by a wide, funnel-shaped vortex that is anomalously weak throughout the prewarming period. Moreover, during split SSWs, gravity wave drag is enhanced in the polar night jet, while planetary wave drag is enhanced within the extratropical surf zone. During displacement SSWs, gravity wave drag is anomalously weak throughout the extratropical stratosphere.Using the composite analysis as a guide, a case study of the 2009 SSW is conducted in order to evaluate the roles of planetary and gravity waves for preconditioning the polar vortex in terms of two SSW-triggering scenarios: anomalous planetary wave forcing from the troposphere and resonance due to either internal or external Rossby waves. The results support the view that split SSWs are caused by resonance rather than anomalously large wave forcing. Given these findings, it is suggested that vortex preconditioning, which is traditionally defined in terms of vortex geometries that increase poleward wave focusing, may be better described by wave events (planetary and/or gravity) that ``tune'' the geometry of the vortex toward its resonant excitation points.{$<$}/p{$><$}/section{$>$}},
  chapter = {Journal of the Atmospheric Sciences},
  file = {/Users/oscardimdore-miles/Zotero/storage/EXTND5U6/Albers and Birner - 2014 - Vortex Preconditioning due to Planetary and Gravit.pdf;/Users/oscardimdore-miles/Zotero/storage/7M3MNC3Z/jas-d-14-0026.1.html},
  journal = {Journal of the Atmospheric Sciences},
  language = {English},
  number = {11}
}

@article{alleyWally2007a,
  title = {Wally {{Was Right}}: {{Predictive Ability}} of the {{North Atlantic}} ``{{Conveyor Belt}}'' {{Hypothesis}} for {{Abrupt Climate Change}}},
  shorttitle = {Wally {{Was Right}}},
  author = {Alley, Richard B.},
  year = {2007},
  volume = {35},
  pages = {241--272},
  doi = {10.1146/annurev.earth.35.081006.131524},
  abstract = {AbstractLinked, abrupt changes of North Atlantic deep water formation, North Atlantic sea ice extent, and widespread climate occurred repeatedly during the last ice age cycle and beyond in response to changing freshwater fluxes and perhaps other causes. This paradigm, developed and championed especially by W.S. Broecker, has repeatedly proven to be successfully predictive as well as explanatory with high confidence. Much work remains to fully understand what happened and to assess possible implications for the future, but the foundations for this work are remarkably solid.},
  file = {/Users/oscardimdore-miles/Zotero/storage/3E3S2HY5/Alley - 2007 - Wally Was Right Predictive Ability of the North A.pdf},
  journal = {Annual Review of Earth and Planetary Sciences},
  number = {1}
}

@article{Andrews2019,
  title = {Observed and Simulated Teleconnections between the Stratospheric Quasi-Biennial Oscillation and Northern Hemisphere Winter Atmospheric Circulation},
  author = {Andrews, Martin B. and Knight, Jeff R. and Scaife, Adam A. and Lu, Yixiong and Wu, Tongwen and Gray, Lesley J. and Schenzinger, Verena},
  year = {2019},
  volume = {124},
  pages = {1219--1232},
  doi = {10.1029/2018JD029368},
  journal = {Journal of Geophysical Research: Atmospheres},
  number = {3}
}

@article{Andrews2020,
  title = {Historical Simulations with {{HadGEM3}}-{{GC3}}.1 for {{CMIP6}}},
  author = {Andrews, Martin B. and Ridley, Jeff K. and Wood, Richard A. and Andrews, Timothy and Blockley, Edward W. and Booth, Ben and Burke, Eleanor and Dittus, Andrea J. and Florek, Piotr and Gray, Lesley J. and Haddad, Stephen and Hardiman, Steven C. and Hermanson, Leon and Hodson, Dan and Hogan, Emma and Jones, Gareth S. and Knight, Jeff R. and Kuhlbrodt, Till and Misios, Stergios and Mizielinski, Matthew S. and Ringer, Mark A. and Robson, Jon and Sutton, Rowan T.},
  year = {2020},
  volume = {12},
  doi = {10.1029/2019MS001995},
  journal = {Journal of Advances in Modeling Earth Systems},
  number = {6}
}

@article{andrewsObserved2019,
  title = {Observed and {{Simulated Teleconnections Between}} the {{Stratospheric Quasi}}-{{Biennial Oscillation}} and {{Northern Hemisphere Winter Atmospheric Circulation}}},
  author = {Andrews, Martin B. and Knight, Jeff R. and Scaife, Adam A. and Lu, Yixiong and Wu, Tongwen and Gray, Lesley J. and Schenzinger, Verena},
  year = {2019},
  volume = {124},
  pages = {1219--1232},
  issn = {2169-8996},
  doi = {10.1029/2018JD029368},
  abstract = {The Quasi-Biennial Oscillation (QBO) is the dominant mode of interannual variability in the tropical stratosphere, with easterly and westerly zonal wind regimes alternating over a period of about 28 months. It appears to influence the Northern Hemisphere winter stratospheric polar vortex and atmospheric circulation near the Earth's surface. However, the short observational record makes unequivocal identification of these surface connections challenging. To overcome this, we use a multicentury control simulation of a climate model with a realistic, spontaneously generated QBO to examine teleconnections with extratropical winter surface pressure patterns. Using a 30-hPa index of the QBO, we demonstrate that the observed teleconnection with the Arctic Oscillation (AO) is likely to be real, and a teleconnection with the North Atlantic Oscillation (NAO) is probable, but not certain. Simulated QBO-AO teleconnections are robust, but appear weaker than in observations. Despite this, inconsistency with the observational record cannot be formally demonstrated. To assess the robustness of our results, we use an alternative measure of the QBO, which selects QBO phases with westerly or easterly winds extending over a wider range of altitudes than phases selected by the single-level index. We find increased strength and significance for both the AO and NAO responses, and better reproduction of the observed surface teleconnection patterns. Further, this QBO metric reveals that the simulated AO response is indeed likely to be weaker than observed. We conclude that the QBO can potentially provide another source of skill for Northern Hemisphere winter prediction, if its surface teleconnections can be accurately simulated.},
  copyright = {\textcopyright 2019 Crown copyright. This article is published with the permission of the Controller of HMSO and the Queen's Printer for Scotland.},
  file = {/Users/oscardimdore-miles/Zotero/storage/E9GKK798/Andrews et al. - 2019 - Observed and Simulated Teleconnections Between the.pdf;/Users/oscardimdore-miles/Zotero/storage/HHWVR7H3/2018JD029368.html},
  journal = {Journal of Geophysical Research: Atmospheres},
  keywords = {Arctic Oscillation,North Atlantic Oscillation,QBO,teleconnections},
  language = {English},
  number = {3}
}

@article{andrewsObserved2019a,
  title = {Observed and {{Simulated Teleconnections Between}} the {{Stratospheric Quasi}}-{{Biennial Oscillation}} and {{Northern Hemisphere Winter Atmospheric Circulation}}},
  author = {Andrews, Martin B. and Knight, Jeff R. and Scaife, Adam A. and Lu, Yixiong and Wu, Tongwen and Gray, Lesley J. and Schenzinger, Verena},
  year = {2019},
  volume = {124},
  pages = {1219--1232},
  issn = {2169-8996},
  doi = {10.1029/2018JD029368},
  abstract = {The Quasi-Biennial Oscillation (QBO) is the dominant mode of interannual variability in the tropical stratosphere, with easterly and westerly zonal wind regimes alternating over a period of about 28 months. It appears to influence the Northern Hemisphere winter stratospheric polar vortex and atmospheric circulation near the Earth's surface. However, the short observational record makes unequivocal identification of these surface connections challenging. To overcome this, we use a multicentury control simulation of a climate model with a realistic, spontaneously generated QBO to examine teleconnections with extratropical winter surface pressure patterns. Using a 30-hPa index of the QBO, we demonstrate that the observed teleconnection with the Arctic Oscillation (AO) is likely to be real, and a teleconnection with the North Atlantic Oscillation (NAO) is probable, but not certain. Simulated QBO-AO teleconnections are robust, but appear weaker than in observations. Despite this, inconsistency with the observational record cannot be formally demonstrated. To assess the robustness of our results, we use an alternative measure of the QBO, which selects QBO phases with westerly or easterly winds extending over a wider range of altitudes than phases selected by the single-level index. We find increased strength and significance for both the AO and NAO responses, and better reproduction of the observed surface teleconnection patterns. Further, this QBO metric reveals that the simulated AO response is indeed likely to be weaker than observed. We conclude that the QBO can potentially provide another source of skill for Northern Hemisphere winter prediction, if its surface teleconnections can be accurately simulated.},
  copyright = {\textcopyright 2019 Crown copyright. This article is published with the permission of the Controller of HMSO and the Queen's Printer for Scotland.},
  file = {/Users/oscardimdore-miles/Zotero/storage/4BXNP7VK/Andrews et al. - 2019 - Observed and Simulated Teleconnections Between the.pdf;/Users/oscardimdore-miles/Zotero/storage/KKLUC7B4/2018JD029368.html},
  journal = {Journal of Geophysical Research: Atmospheres},
  keywords = {Arctic Oscillation,North Atlantic Oscillation,QBO,teleconnections},
  language = {English},
  number = {3}
}

@article{Anstey20,
  title = {The {{SPARC}} Quasi-Biennial Oscillation Initiative},
  author = {Anstey, James A. and Butchart, Neal and Hamilton, Kevin and Osprey, Scott M.},
  year = {2020},
  pages = {1--4},
  doi = {10.1002/qj.3820},
  journal = {Quarterly Journal of the Royal Meteorological Society},
  keywords = {climate modelling,gravity waves,quasi-biennial oscillation,stratosphere,WCRP–SPARC}
}

@article{Anstey2008,
  title = {Response of the Northern Stratospheric Polar Vortex to the Seasonal Alignment of {{QBO}} Phase Transitions},
  author = {Anstey, J. A. and Shepherd, T. G.},
  year = {2008},
  volume = {35},
  doi = {10.1029/2008GL035721},
  journal = {Geophysical Research Letters},
  number = {22}
}

@article{Anstey2014,
  title = {High-Latitude Influence of the Quasi-Biennial Oscillation},
  author = {Anstey, James A. and Shepherd, Theodore G.},
  year = {2014},
  volume = {140},
  pages = {1--21},
  doi = {10.1002/qj.2132},
  journal = {Quarterly Journal of the Royal Meteorological Society},
  number = {678}
}

@article{ansteyHighlatitude2014,
  title = {High-Latitude Influence of the Quasi-Biennial Oscillation},
  author = {Anstey, James A. and Shepherd, Theodore G.},
  year = {2014},
  volume = {140},
  pages = {1--21},
  issn = {1477-870X},
  doi = {10.1002/qj.2132},
  abstract = {The interannual variability of the stratospheric winter polar vortex is correlated with the phase of the quasi-biennial oscillation (QBO) of tropical stratospheric winds. This dynamical coupling between high and low latitudes, often referred to as the Holton\textendash Tan effect, has been the subject of numerous observational and modelling studies, yet important questions regarding its mechanism remain unanswered. In particular it remains unclear which vertical levels of the QBO exert the strongest influence on the winter polar vortex, and how QBO\textendash vortex coupling interacts with the effects of other sources of atmospheric interannual variability such as the 11-year solar cycle or the El Ni\~no Southern Oscillation. As stratosphere-resolving general circulation models begin to resolve the QBO and represent its teleconnections with other parts of the climate system, it seems timely to summarize what is currently known about the QBO's high-latitude influence. In this review article, we offer a synthesis of the modelling and observational analyses of QBO\textendash vortex coupling that have appeared in the literature, and update the observational record.},
  copyright = {\textcopyright{} 2013 Royal Meteorological Society},
  file = {/Users/oscardimdore-miles/Zotero/storage/H6UX587S/qj.html},
  journal = {Quarterly Journal of the Royal Meteorological Society},
  keywords = {annular modes,decadal variability,Holton–Tan effect,interannual variability,middle atmosphere,polar vortex,seasonal predictability,stratosphere–troposphere coupling},
  language = {English},
  number = {678}
}

@article{ansteyResponse2008,
  title = {Response of the Northern Stratospheric Polar Vortex to the Seasonal Alignment of {{QBO}} Phase Transitions},
  author = {Anstey, J. A. and Shepherd, T. G.},
  year = {2008},
  volume = {35},
  issn = {1944-8007},
  doi = {10.1029/2008GL035721},
  abstract = {This study considers the strength of the Northern Hemisphere Holton-Tan effect (HTE) in terms of the phase alignment of the quasi-biennial oscillation (QBO) with respect to the annual cycle. Using the ERA-40 Reanalysis, it is found that the early winter (Nov\textendash Dec) and late winter (Feb\textendash Mar) relation between QBO phase and the strength of the stratospheric polar vortex is optimized for subsets of the 44-year record that are chosen on the basis of the seasonality of QBO phase transitions at the 30 hPa level. The timing of phase transitions serves as a proxy for changes in the vertical structure of the QBO over the whole depth of the tropical stratosphere. The statistical significance of the Nov\textendash Dec (Feb\textendash Mar) HTE is greatest when 30 hPa QBO phase transitions occur 9\textendash 14 (4\textendash 9) months prior to the January of the NH winter in question. This suggests that there exists for both early and late winter a vertical structure of tropical stratospheric winds that is most effective at influencing the interannual variability of the polar vortex, and that an early (late) winter HTE is associated with an early (late) progression of QBO phase towards that structure. It is also shown that the seasonality of QBO phase transitions at 30 hPa varies on a decadal timescale, with transitions during the first half of the calendar year being relatively more common during the first half of the tropical radiosonde wind record. Combining these two results suggests that decadal changes in HTE strength could result from the changing seasonality of QBO phase transitions.},
  copyright = {Copyright 2008 by the American Geophysical Union.},
  file = {/Users/oscardimdore-miles/Zotero/storage/MPKWHT36/Anstey and Shepherd - 2008 - Response of the northern stratospheric polar vorte.pdf;/Users/oscardimdore-miles/Zotero/storage/YUMQBD6T/2008GL035721.html},
  journal = {Geophysical Research Letters},
  keywords = {Holton-Tan effect,interannual variability,quasi-biennial oscillation},
  language = {English},
  number = {22}
}

@article{Ayarz2020,
  title = {Uncertainty in the Response of Sudden Stratospheric Warmings and Stratosphere-Troposphere Coupling to Quadrupled {{CO2}} Concentrations in {{CMIP6}} Models},
  author = {Ayarzag{\"u}ena, B. and {Charlton-Perez}, A. J. and Butler, A. H. and Hitchcock, P. and Simpson, I. R. and Polvani, L. M. and Butchart, N. and Gerber, E. P. and Gray, L. and Hassler, B. and Lin, P. and Lott, F. and Manzini, E. and Mizuta, R. and Orbe, C. and Osprey, S. and {Saint-Martin}, D. and Sigmond, M. and Taguchi, M. and Volodin, E. M. and Watanabe, S.},
  year = {2020},
  volume = {125},
  journal = {Journal of Geophysical Research: Atmospheres},
  keywords = {climate change,CMIP6,stratosphere-troposphere coupling,sudden stratospheric warming},
  number = {6}
}

@article{ayarzag2020,
  title = {Uncertainty in the Response of Sudden Stratospheric Warmings and Stratosphere-Troposphere Coupling to Quadrupled {{CO2}} Concentrations in {{CMIP6}} Models},
  author = {Ayarzag{\"u}ena, Blanca and {Charlton-Perez}, Andrew and Butler, Amy and Hitchcock, Peter and Simpson, Isla and Polvani, Lorenzo and Butchart, Neal and Gerber, Edwin and Gray, Lesley and Hassler, Birgit},
  year = {2020},
  volume = {125},
  pages = {103--121},
  number = {6}
}

@article{ayarzaguena2018,
  title = {Stratospheric Role in Interdecadal Changes of {{El Ni\~no}} Impacts over {{Europe}}},
  author = {Osprey, Scott M and Gray, Lesley J and Hardiman, Steven C and Butchart, Neal and Bushell, Andrew C and Hinton, Tim J},
  year = {2019},
  volume = {52},
  pages = {1173---1186},
  journal = {Clim Dyn}
}

@article{ayarzaguenaIntraseasonal2018a,
  title = {Intraseasonal {{Effects}} of {{El Ni\~no}}\textendash{{Southern Oscillation}} on {{North Atlantic Climate}}},
  author = {Ayarzag{\"u}ena, Blanca and Ineson, Sarah and Dunstone, Nick J. and Baldwin, Mark P. and Scaife, Adam A.},
  year = {2018},
  month = nov,
  volume = {31},
  pages = {8861--8873},
  publisher = {{American Meteorological Society}},
  issn = {0894-8755, 1520-0442},
  doi = {10.1175/JCLI-D-18-0097.1},
  abstract = {{$<$}section class="abstract"{$><$}h2 class="abstractTitle text-title my-1" id="d572e2"{$>$}Abstract{$<$}/h2{$><$}p{$>$}It is well established that El Ni\~no\textendash Southern Oscillation (ENSO) impacts the North Atlantic\textendash European (NAE) climate, with the strongest influence in winter. In late winter, the ENSO signal travels via both tropospheric and stratospheric pathways to the NAE sector and often projects onto the North Atlantic Oscillation. However, this signal does not strengthen gradually during winter, and some studies have suggested that the ENSO signal is different between early and late winter and that the teleconnections involved in the early winter subperiod are not well understood. In this study, we investigate the ENSO teleconnection to NAE in early winter (November\textendash December) and characterize the possible mechanisms involved in that teleconnection. To do so, observations, reanalysis data and the output of different types of model simulations have been used. We show that the intraseasonal winter shift of the NAE response to ENSO is detected for both El Ni\~no and La Ni\~na and is significant in both observations and initialized predictions, but it is not reproduced by free-running Coupled Model Intercomparison Project phase 5 (CMIP5) models. The teleconnection is established through the troposphere in early winter and is related to ENSO effects over the Gulf of Mexico and Caribbean Sea that appear in rainfall and reach the NAE region. CMIP5 model biases in equatorial Pacific ENSO sea surface temperature patterns and strength appear to explain the lack of signal in the Gulf of Mexico and Caribbean Sea and, hence, their inability to reproduce the intraseasonal shift of the ENSO signal over Europe.{$<$}/p{$><$}/section{$>$}},
  chapter = {Journal of Climate},
  file = {/Users/oscardimdore-miles/Zotero/storage/DC3Z7AFL/Ayarzagüena et al. - 2018 - Intraseasonal Effects of El Niño–Southern Oscillat.pdf;/Users/oscardimdore-miles/Zotero/storage/6PWVGGN3/jcli-d-18-0097.1.html},
  journal = {Journal of Climate},
  language = {English},
  number = {21}
}

@article{ayarzaguenaTropospheric2011,
  title = {Tropospheric Forcing of the Stratosphere: {{A}} Comparative Study of the Two Different Major Stratospheric Warmings in 2009 and 2010},
  shorttitle = {Tropospheric Forcing of the Stratosphere},
  author = {Ayarzag{\"u}ena, Blanca and Langematz, Ulrike and Serrano, Encarna},
  year = {2011},
  volume = {116},
  issn = {2156-2202},
  doi = {10.1029/2010JD015023},
  abstract = {In January 2009 and 2010, two major stratospheric warmings (MSWs) took place in the boreal polar stratosphere. Both MSWs were preceded by nearly the strongest injection of tropospheric wave activity on record since 1958 and their central date was almost coincident. However, the typical external factors that influence the occurrence of MSWs (the Quasi-Biennial Oscillation, sunspot cycle, or El Ni\~no) were dissimilar in the two midwinters: favorable in 2010 but unfavorable in 2009. In this study, the driving mechanisms of these two different MSWs were investigated focusing on the amplification of upward wave activity injection into the stratosphere before the MSW onset. By decomposing the total wave flux injection into contributions from the climatological planetary waves and from deviations from the latter we found clear differences in this amplification between both MSWs. The pre-MSW period in 2009 was characterized by a peak in the 100 hPa eddy heat flux with a predominance of wave number 2 activity. This was due to strong anomalies associated with Rossby wave packets originating from a deep ridge over the eastern Pacific. In contrast, the amplification of the upward wave propagation prior to the 2010 MSW was equally due to Rossby wave packets and to the interaction between the latter and the climatological waves. This amplification enhanced wave number 1 stationary waves in January 2010, which seemed at least partially due to the 2009/2010 El Ni\~no event. Our results show the relevance of the internal tropospheric variability in generating MSWs, particularly when the external factors do not play any role.},
  copyright = {Copyright 2011 by the American Geophysical Union.},
  file = {/Users/oscardimdore-miles/Zotero/storage/27SRDYKH/Ayarzagüena et al. - 2011 - Tropospheric forcing of the stratosphere A compar.pdf},
  journal = {Journal of Geophysical Research: Atmospheres},
  keywords = {boreal polar stratosphere,major stratospheric warming,troposphere-stratosphere interactions},
  language = {English},
  number = {D18}
}

@article{ayarzaguenaUncertainty2020,
  title = {Uncertainty in the {{Response}} of {{Sudden Stratospheric Warmings}} and {{Stratosphere}}-{{Troposphere Coupling}} to {{Quadrupled CO}} 2 {{Concentrations}} in {{CMIP6 Models}}},
  author = {Ayarzag{\"u}ena, Blanca and {Charlton-Perez}, A.J. and Butler, Amy and Hitchcock, Peter and Simpson, Isla and Polvani, Lorenzo and Butchart, N. and Gerber, Edwin and Gray, L. and Hassler, Birgit and Lin, P. and Lott, F. and Manzini, Elisa and Mizuta, R. and Orbe, Clara and Osprey, Scott and {Saint-Martin}, David and Sigmond, Michael and Taguchi, M. and Watanabe, Shingo},
  year = {2020},
  month = feb,
  volume = {125},
  pages = {e2019JD032345},
  doi = {10.1029/2019JD032345},
  abstract = {Major sudden stratospheric warmings (SSWs), vortex formation, and final breakdown dates are key highlight points of the stratospheric polar vortex. These phenomena are relevant for stratosphere-troposphere coupling, which explains the interest in understanding their future changes. However, up to now, there is not a clear consensus on which projected changes to the polar vortex are robust, particularly in the Northern Hemisphere, possibly due to short data record or relatively moderate CO2 forcing. The new simulations performed under the Coupled Model Intercomparison Project, Phase 6, together with the long daily data requirements of the DynVarMIP project in preindustrial and quadrupled CO2 (4xCO2) forcing simulations provide a new opportunity to revisit this topic by overcoming the limitations mentioned above. In this study, we analyze this new model output to document the change, if any, in the frequency of SSWs under 4xCO2 forcing. Our analysis reveals a large disagreement across the models as to the sign of this change, even though most models show a statistically significant change. As for the near-surface response to SSWs, the models, however, are in good agreement as to this signal over the North Atlantic: There is no indication of a change under 4xCO2 forcing. Over the Pacific, however, the change is more uncertain, with some indication that there will be a larger mean response. Finally, the models show robust changes to the seasonal cycle in the stratosphere. Specifically, we find a longer duration of the stratospheric polar vortex and thus a longer season of stratosphere-troposphere coupling.},
  file = {/Users/oscardimdore-miles/Zotero/storage/RIMD5SGT/Ayarzagüena et al. - 2020 - Uncertainty in the Response of Sudden Stratospheri.pdf},
  journal = {Journal of Geophysical Research: Atmospheres}
}

@article{bakkerFate2016,
  title = {Fate of the {{Atlantic Meridional Overturning Circulation}}: {{Strong}} Decline under Continued Warming and {{Greenland}} Melting},
  shorttitle = {Fate of the {{Atlantic Meridional Overturning Circulation}}},
  author = {Bakker, P. and Schmittner, A. and Lenaerts, J. T. M. and {Abe-Ouchi}, A. and Bi, D. and {van den Broeke}, M. R. and Chan, W.-L. and Hu, A. and Beadling, R. L. and Marsland, S. J. and Mernild, S. H. and Saenko, O. A. and Swingedouw, D. and Sullivan, A. and Yin, J.},
  year = {2016},
  volume = {43},
  pages = {12,252--12,260},
  issn = {1944-8007},
  doi = {10.1002/2016GL070457},
  abstract = {The most recent Intergovernmental Panel on Climate Change assessment report concludes that the Atlantic Meridional Overturning Circulation (AMOC) could weaken substantially but is very unlikely to collapse in the 21st century. However, the assessment largely neglected Greenland Ice Sheet (GrIS) mass loss, lacked a comprehensive uncertainty analysis, and was limited to the 21st century. Here in a community effort, improved estimates of GrIS mass loss are included in multicentennial projections using eight state-of-the-science climate models, and an AMOC emulator is used to provide a probabilistic uncertainty assessment. We find that GrIS melting affects AMOC projections, even though it is of secondary importance. By years 2090\textendash 2100, the AMOC weakens by 18\% [-3\%, -34\%; 90\% probability] in an intermediate greenhouse-gas mitigation scenario and by 37\% [-15\%, -65\%] under continued high emissions. Afterward, it stabilizes in the former but continues to decline in the latter to -74\% [+4\%, -100\%] by 2290\textendash 2300, with a 44\% likelihood of an AMOC collapse. This result suggests that an AMOC collapse can be avoided by CO2 mitigation.},
  copyright = {\textcopyright 2016. American Geophysical Union and Her Majesty The Queen in Right of Canada. Reproduced with the permission of the Minister of the Environment, Canada.},
  file = {/Users/oscardimdore-miles/Zotero/storage/D6FXVCIC/Bakker et al. - 2016 - Fate of the Atlantic Meridional Overturning Circul.pdf;/Users/oscardimdore-miles/Zotero/storage/YE5ALCCV/2016GL070457.html},
  journal = {Geophysical Research Letters},
  keywords = {Atlantic Meridional Overturning Circulation,climate change,general circulation model},
  language = {English},
  number = {23}
}

@article{Baldwin_harbingers,
  title = {Stratospheric Harbingers of Anomalous Weather Regimes},
  author = {Baldwin, Mark P. and Dunkerton, Timothy J.},
  year = {2001},
  volume = {294},
  pages = {581--584},
  publisher = {{American Association for the Advancement of Science}},
  issn = {0036-8075},
  doi = {10.1126/science.1063315},
  journal = {Science},
  number = {5542}
}

@article{Baldwin1991,
  title = {The Quasi-Biennial Oscillations above 10mb},
  author = {Baldwin, Mark P and Dunkerton, Timothy J},
  year = {1991},
  volume = {18},
  pages = {1205--1208},
  doi = {10.1029/91GL01333},
  journal = {Geophysical Research Letters},
  number = {7}
}

@article{Baldwin2001,
  title = {The Quasi-Biennial Oscillation},
  author = {Baldwin, M. P. and Gray, L. J. and Dunkerton, T. J. and Hamilton, K. and Haynes, P. H. and Randel, W. J. and Holton, J. R. and Alexander, M. J. and Hirota, I. and Horinouchi, T. and Jones, D. B. A. and Kinnersley, J. S. and Marquardt, C. and Sato, K. and Takahashi, M.},
  year = {2001},
  volume = {39},
  pages = {179--229},
  doi = {10.1029/1999RG000073},
  journal = {Reviews of Geophysics},
  number = {2}
}

@article{baldwin2020,
  title = {Sudden Stratospheric Warmings},
  author = {Baldwin, Mark and Ayarzag{\"u}ena, Blanca and Birner, Thomas and Butchart, Neal and {Charlton-Perez}, Andrew and Butler, Amy and Domeisen, Daniela and Garfinkel, Chaim and Garny, Hella and Gerber, Edwin and Hegglin, Michaela and Langematz, U. and Pedatella, Nicholas},
  year = {2021},
  volume = {59},
  doi = {10.1029/2020RG000708},
  journal = {Reviews of Geophysics},
  number = {1}
}

@article{Baldwin98,
  title = {Quasi-Biennial Modulation of the Southern Hemisphere Stratospheric Polar Vortex},
  author = {Baldwin, Mark P. and Dunkerton, Timothy J.},
  year = {1998},
  volume = {25},
  pages = {3343--3346},
  doi = {10.1029/98GL02445},
  journal = {Geophysical Research Letters},
  number = {17}
}

@article{baldwinStratospheric2001a,
  title = {Stratospheric {{Harbingers}} of {{Anomalous Weather Regimes}}},
  author = {Baldwin, M. P.},
  year = {2001},
  month = oct,
  volume = {294},
  pages = {581--584},
  issn = {00368075, 10959203},
  doi = {10.1126/science.1063315},
  file = {/Users/oscardimdore-miles/Zotero/storage/L23IDKGQ/Baldwin - 2001 - Stratospheric Harbingers of Anomalous Weather Regi.pdf},
  journal = {Science},
  language = {English},
  number = {5542}
}

@article{baldwinSudden2021,
  title = {Sudden {{Stratospheric Warmings}}},
  author = {Baldwin, Mark P. and Ayarzag{\"u}ena, Blanca and Birner, Thomas and Butchart, Neal and Butler, Amy H. and {Charlton-Perez}, Andrew J. and Domeisen, Daniela I. V. and Garfinkel, Chaim I. and Garny, Hella and Gerber, Edwin P. and Hegglin, Michaela I. and Langematz, Ulrike and Pedatella, Nicholas M.},
  year = {2021},
  volume = {59},
  pages = {e2020RG000708},
  issn = {1944-9208},
  doi = {10.1029/2020RG000708},
  abstract = {Sudden stratospheric warmings (SSWs) are impressive fluid dynamical events in which large and rapid temperature increases in the winter polar stratosphere ({$\sim$}10\textendash 50 km) are associated with a complete reversal of the climatological wintertime westerly winds. SSWs are caused by the breaking of planetary-scale waves that propagate upwards from the troposphere. During an SSW, the polar vortex breaks down, accompanied by rapid descent and warming of air in polar latitudes, mirrored by ascent and cooling above the warming. The rapid warming and descent of the polar air column affect tropospheric weather, shifting jet streams, storm tracks, and the Northern Annular Mode, making cold air outbreaks over North America and Eurasia more likely. SSWs affect the atmosphere above the stratosphere, producing widespread effects on atmospheric chemistry, temperatures, winds, neutral (nonionized) particles and electron densities, and electric fields. These effects span both hemispheres. Given their crucial role in the whole atmosphere, SSWs are also seen as a key process to analyze in climate change studies and subseasonal to seasonal prediction. This work reviews the current knowledge on the most important aspects of SSWs, from the historical background to dynamical processes, modeling, chemistry, and impact on other atmospheric layers.},
  copyright = {\textcopyright 2020. American Geophysical Union. All Rights Reserved.},
  file = {/Users/oscardimdore-miles/Zotero/storage/K3FZ4GQJ/2020RG000708.html},
  journal = {Reviews of Geophysics},
  keywords = {middle atmosphere,QBO,stratosphere,upper atmosphere,weather forecasts},
  language = {English},
  number = {1}
}

@article{bancalaPreconditioning2012,
  title = {The Preconditioning of Major Sudden Stratospheric Warmings},
  author = {Bancal{\'a}, S. and Kr{\"u}ger, K. and Giorgetta, Marco},
  year = {2012},
  month = feb,
  volume = {117},
  pages = {4101},
  doi = {10.1029/2011JD016769},
  abstract = {The preconditioning of major sudden stratospheric warmings (SSWs) is investigated with two long time series using reanalysis (ERA-40) and model (MAECHAM5/MPI-OM) data. Applying planetary wave analysis, we distinguish between wavenumber-1 and wavenumber-2 major SSWs based on the wave activity of zonal wavenumbers 1 and 2 during the prewarming phase. For this analysis an objective criterion to identify and classify the preconditioning of major SSWs is developed. Major SSWs are found to occur with a frequency of six and seven events per decade in the reanalysis and in the model, respectively, thus highlighting the ability of MAECHAM5/MPI-OM to simulate the frequency of major SSWs realistically. However, from these events only one quarter are wavenumber-2 major warmings, representing a low (\texttildelow 0.25) wavenumber-2 to wavenumber-1 major SSW ratio. Composite analyses for both data sets reveal that the two warming types have different dynamics; while wavenumber-1 major warmings are preceded only by an enhanced activity of the zonal wavenumber-1, wavenumber-2 events are either characterized by only the amplification of zonal wavenumber-2 or by both zonal wavenumber-1 and zonal wavenumber-2, albeit at different time intervals. The role of tropospheric blocking events influencing these two categories of major SSWs is evaluated in the next step. Here, the composite analyses of both reanalysis and model data reveal that blocking events in the Euro-Atlantic sector mostly lead to the development of wavenumber-1 major warmings. The blocking-wavenumber-2 major warming connection can only be statistical reliable analyzed with the model time series, demonstrating that blocking events in the Pacific region mostly precede wavenumber-2 major SSWs.},
  file = {/Users/oscardimdore-miles/Zotero/storage/K6DZNKSN/Bancalá et al. - 2012 - The preconditioning of major sudden stratospheric .pdf},
  journal = {Journal of Geophysical Research (Atmospheres)}
}

@article{Bell2009,
  title = {Stratospheric Communication of El Ni\~no Teleconnections to European Winter},
  author = {Bell, C. J. and Gray, L. J. and {Charlton-Perez}, A. J. and Joshi, M. M. and Scaife, A. A.},
  year = {2009},
  volume = {22},
  pages = {4083--4096},
  doi = {10.1175/2009JCLI2717.1},
  journal = {Journal of Climate},
  number = {15}
}

@article{bellStratospheric2009,
  title = {Stratospheric {{Communication}} of {{El Ni\~no Teleconnections}} to {{European Winter}}},
  author = {Bell, C. J. and Gray, L. J. and {Charlton-Perez}, A. J. and Joshi, M. M. and Scaife, A. A.},
  year = {2009},
  month = aug,
  volume = {22},
  pages = {4083--4096},
  publisher = {{American Meteorological Society}},
  issn = {0894-8755, 1520-0442},
  doi = {10.1175/2009JCLI2717.1},
  abstract = {{$<$}section class="abstract"{$><$}h2 class="abstractTitle text-title my-1" id="d406e2"{$>$}Abstract{$<$}/h2{$><$}p{$>$}The stratospheric role in the European winter surface climate response to El Ni\~no\textendash Southern Oscillation sea surface temperature forcing is investigated using an intermediate general circulation model with a well-resolved stratosphere. Under El Ni\~no conditions, both the modeled tropospheric and stratospheric mean-state circulation changes correspond well to the observed ``canonical'' responses of a late winter negative North Atlantic Oscillation and a strongly weakened polar vortex, respectively. The variability of the polar vortex is modulated by an increase in frequency of stratospheric sudden warming events throughout all winter months. The potential role of this stratospheric response in the tropical Pacific\textendash European teleconnection is investigated by sensitivity experiments in which the mean state and variability of the stratosphere are degraded. As a result, the observed stratospheric response to El Ni\~no is suppressed and the mean sea level pressure response fails to resemble the temporal and spatial evolution of the observations. The results suggest that the stratosphere plays an active role in the European response to El Ni\~no. A saturation mechanism whereby for the strongest El Ni\~no events tropospheric forcing dominates the European response is suggested. This is examined by means of a sensitivity test and it is shown that under large El Ni\~no forcing the European response is insensitive to stratospheric representation.{$<$}/p{$><$}/section{$>$}},
  chapter = {Journal of Climate},
  file = {/Users/oscardimdore-miles/Zotero/storage/N5ILXPL6/Bell et al. - 2009 - Stratospheric Communication of El Niño Teleconnect.pdf;/Users/oscardimdore-miles/Zotero/storage/6TV6Z7C9/2009jcli2717.1.html},
  journal = {Journal of Climate},
  language = {English},
  number = {15}
}

@article{biastochCauses2008a,
  title = {Causes of {{Interannual}}\textendash{{Decadal Variability}} in the {{Meridional Overturning Circulation}} of the {{Midlatitude North Atlantic Ocean}}},
  author = {Biastoch, Arne and B{\"o}ning, Claus W. and Getzlaff, Julia and Molines, Jean-Marc and Madec, Gurvan},
  year = {2008},
  month = dec,
  volume = {21},
  pages = {6599--6615},
  publisher = {{American Meteorological Society}},
  issn = {0894-8755, 1520-0442},
  doi = {10.1175/2008JCLI2404.1},
  abstract = {{$<$}section class="abstract"{$><$}h2 class="abstractTitle text-title my-1" id="d1401e2"{$>$}Abstract{$<$}/h2{$><$}p{$>$}The causes and characteristics of interannual\textendash decadal variability of the meridional overturning circulation (MOC) in the North Atlantic are investigated with a suite of basin-scale ocean models [the Family of Linked Atlantic Model Experiments (FLAME)] and global ocean\textendash ice models (ORCA), varying in resolution from medium to eddy resolving ({$\frac{1}{2}^\circ$}\textendash 1/12\textdegree ), using various forcing configurations built on bulk formulations invoking atmospheric reanalysis products. Comparison of the model hindcasts indicates similar MOC variability characteristics on time scales up to a decade; both model architectures also simulate an upward trend in MOC strength between the early 1970s and mid-1990s. The causes of the MOC changes are examined by perturbation experiments aimed selectively at the response to individual forcing components. The solutions emphasize an inherently linear character of the midlatitude MOC variability by demonstrating that the anomalies of a (non\textendash eddy resolving) hindcast simulation can be understood as a superposition of decadal and longer-term signals originating from thermohaline forcing variability, and a higher-frequency wind-driven variability. The thermohaline MOC signal is linked to the variability in subarctic deep-water formation, and rapidly progressing to the tropical Atlantic. However, throughout the subtropical and midlatitude North Atlantic, this signal is effectively masked by stronger MOC variability related to wind forcing and, especially north of 30\textdegree\textendash 35\textdegree N, by internally induced (eddy) fluctuations.{$<$}/p{$><$}/section{$>$}},
  chapter = {Journal of Climate},
  file = {/Users/oscardimdore-miles/Zotero/storage/NXRDXNI9/Biastoch et al. - 2008 - Causes of Interannual–Decadal Variability in the M.pdf;/Users/oscardimdore-miles/Zotero/storage/QLTCUDH6/2008jcli2404.1.html},
  journal = {Journal of Climate},
  language = {English},
  number = {24}
}

@article{boningDecadal2006a,
  title = {Decadal Variability of Subpolar Gyre Transport and Its Reverberation in the {{North Atlantic}} Overturning},
  author = {B{\"o}ning, C. W. and Scheinert, M. and Dengg, J. and Biastoch, A. and Funk, A.},
  year = {2006},
  volume = {33},
  issn = {1944-8007},
  doi = {10.1029/2006GL026906},
  abstract = {Analyses of sea surface height (SSH) records based on satellite altimeter data and hydrographic properties have suggested a considerable weakening of the North Atlantic subpolar gyre during the 1990s. Here we report hindcast simulations with high-resolution ocean circulation models that demonstrate a close correspondence of the SSH changes with the volume transport of the boundary current system in the Labrador Sea. The 1990s-decline, of about 15\% of the long-term mean, appears as part of a decadal variability of the gyre transport driven by changes in both heat flux and wind stress associated with the North Atlantic Oscillation (NAO). The changes in the subpolar gyre, as manifested in the deep western boundary current off Labrador, reverberate in the strength of the meridional overturning circulation (MOC) in the subtropical North Atlantic, suggesting the potential of a subpolar transport index as an element of a MOC monitoring system.},
  copyright = {Copyright 2006 by the American Geophysical Union.},
  file = {/Users/oscardimdore-miles/Zotero/storage/A9XYTFHC/Böning et al. - 2006 - Decadal variability of subpolar gyre transport and.pdf;/Users/oscardimdore-miles/Zotero/storage/N72SFFNZ/2006GL026906.html},
  journal = {Geophysical Research Letters},
  language = {English},
  number = {21}
}

@article{buckleyObservations2016,
  title = {Observations, Inferences, and Mechanisms of the {{Atlantic Meridional Overturning Circulation}}: {{A}} Review},
  shorttitle = {Observations, Inferences, and Mechanisms of the {{Atlantic Meridional Overturning Circulation}}},
  author = {Buckley, Martha W. and Marshall, John},
  year = {2016},
  volume = {54},
  pages = {5--63},
  issn = {1944-9208},
  doi = {10.1002/2015RG000493},
  abstract = {This is a review about the Atlantic Meridional Overturning Circulation (AMOC), its mean structure, temporal variability, controlling mechanisms, and role in the coupled climate system. The AMOC plays a central role in climate through its heat and freshwater transports. Northward ocean heat transport achieved by the AMOC is responsible for the relative warmth of the Northern Hemisphere compared to the Southern Hemisphere and is thought to play a role in setting the mean position of the Intertropical Convergence Zone north of the equator. The AMOC is a key means by which heat anomalies are sequestered into the ocean's interior and thus modulates the trajectory of climate change. Fluctuations in the AMOC have been linked to low-frequency variability of Atlantic sea surface temperatures with a host of implications for climate variability over surrounding landmasses. On intra-annual timescales, variability in AMOC is large and primarily reflects the response to local wind forcing; meridional coherence of anomalies is limited to that of the wind field. On interannual to decadal timescales, AMOC changes are primarily geostrophic and related to buoyancy anomalies on the western boundary. A pacemaker region for decadal AMOC changes is located in a western ``transition zone'' along the boundary between the subtropical and subpolar gyres. Decadal AMOC anomalies are communicated meridionally from this region. AMOC observations, as well as the expanded ocean observational network provided by the Argo array and satellite altimetry, are inspiring efforts to develop decadal predictability systems using coupled atmosphere-ocean models initialized by ocean data.},
  copyright = {\textcopyright 2015. The Authors.},
  file = {/Users/oscardimdore-miles/Zotero/storage/ZJCQJCFV/Buckley and Marshall - 2016 - Observations, inferences, and mechanisms of the At.pdf;/Users/oscardimdore-miles/Zotero/storage/SQIJBID9/2015RG000493.html},
  journal = {Reviews of Geophysics},
  keywords = {Atlantic Meridional Overturning Circulation,climate variability},
  language = {English},
  number = {1}
}

@article{Bushell2020,
  title = {Evaluation of the {{Quasi}}-{{Biennial Oscillation}} in Global Climate Models for the {{SPARC QBO}}-Initiative},
  author = {Bushell, A. C. and Anstey, J. A. and Butchart, N. and Kawatani, Y. and Osprey, S. M. and Richter, J. H. and Serva, F. and Braesicke, P. and Cagnazzo, C. and Chen, C.-C. and Chun, H.-Y. and Garcia, R. R. and Gray, L. J. and Hamilton, K. and Kerzenmacher, T. and Kim, Y.-H. and Lott, F. and McLandress, C. and Naoe, H. and Scinocca, J. and Smith, A. K. and Stockdale, T. N. and Versick, S. and Watanabe, S. and Yoshida, K. and Yukimoto, S.},
  year = {2020},
  pages = {1--31},
  doi = {10.1002/qj.3765},
  journal = {Quarterly Journal of the Royal Meteorological Society},
  keywords = {general circulation models,gravity waves,Quasi-Biennial Oscillation,stratosphere,tropical variability}
}

@article{bushellEvaluation2020,
  title = {Evaluation of the {{Quasi}}-{{Biennial Oscillation}} in Global Climate Models for the {{SPARC QBO}}-{{Initiative}}},
  author = {Bushell, A. C. and Anstey, J. A. and Butchart, N. and Kawatani, Y. and Osprey, S. M. and Richter, J. H. and Serva, F. and Braesicke, P. and Cagnazzo, C. and Chen, C.-C. and Chun, H.-Y. and Garcia, R. R. and Gray, L. J. and Hamilton, K. and Kerzenmacher, T. and Kim, Y.-H. and Lott, F. and McLandress, C. and Naoe, H. and Scinocca, J. and Smith, A. K. and Stockdale, T. N. and Versick, S. and Watanabe, S. and Yoshida, K. and Yukimoto, S.},
  year = {2020},
  volume = {1},
  pages = {1--31},
  issn = {1477-870X},
  doi = {10.1002/qj.3765},
  abstract = {Quasi-biennial oscillations (QBOs) in thirteen atmospheric general circulation models forced with both observed and annually repeating sea surface temperatures (SSTs) are evaluated. In most models the QBO period is close to, but shorter than, the observed period of 28 months. Amplitudes are within {$\pm$}20\% of the observed QBO amplitude at 10 hPa, but typically about half of that observed at lower altitudes (50 and 70 hPa). For almost all models, the oscillation's amplitude profile shows an overall upward shift compared to reanalysis and its meridional extent is too narrow. Asymmetry in the duration of eastward and westward phases is reasonably well captured, though not all models replicate the observed slowing of the descending westward shear. Westward phases are generally too weak, and most models have an eastward time mean wind bias throughout the depth of the QBO. The intercycle period variability is realistic and in some models is enhanced in the experiment with observed SSTs compared to the experiment with repeated annual cycle SSTs. Mean periods are also sensitive to this difference between SSTs, but only when parametrized non-orographic gravity wave (NOGW) sources are coupled to tropospheric parameters and not prescribed with a fixed value. Overall, however, modelled QBOs are very similar whether or not the prescribed SSTs vary interannually. A portrait of the overall ensemble performance is provided by a normalized grading of QBO metrics. To simulate a QBO, all but one model used parametrized NOGWs, which provided the majority of the total wave forcing at altitudes above 70 hPa in most models. Hence the representation of NOGWs either explicitly or through parametrization is still a major uncertainty underlying QBO simulation in these present-day experiments.},
  copyright = {\textcopyright{} 2020 The Authors. Quarterly Journal of the Royal Meteorological Society published by John Wiley \& Sons Ltd on behalf of the Royal Meteorological Society.},
  file = {/Users/oscardimdore-miles/Zotero/storage/B5D68RJY/Bushell et al. - Evaluation of the Quasi-Biennial Oscillation in gl.pdf;/Users/oscardimdore-miles/Zotero/storage/4RIXY4LD/qj.html},
  journal = {Quarterly Journal of the Royal Meteorological Society},
  keywords = {general circulation models,gravity waves,Quasi-Biennial Oscillation,stratosphere,tropical variability},
  language = {English}
}

@article{Butchart2000,
  title = {The Response of the Stratospheric Climate to Projected Changes in the Concentrations of Well-Mixed Greenhouse Gases from 1992 to 2051},
  author = {Butchart, Neal and Austin, John and Knight, Jeffrey R. and Scaife, Adam A. and Gallani, Mark L.},
  year = {2000},
  volume = {13},
  pages = {2142--2159},
  doi = {10.1175/1520-0442},
  journal = {Journal of Climate},
  number = {13}
}

@article{Butler2015,
  title = {Defining Sudden Stratospheric Warmings},
  author = {Butler, Amy H. and Seidel, Dian J. and Hardiman, Steven C. and Butchart, Neal and Birner, Thomas and Match, Aaron},
  year = {2015},
  volume = {96},
  pages = {1913--1928},
  doi = {10.1175/BAMS-D-13-00173.1},
  journal = {Bulletin of the American Meteorological Society},
  number = {11}
}

@article{Butler2017,
  title = {A Sudden Stratospheric Warming Compendium},
  author = {Butler, Amy H. and Sjoberg, Jeremiah P. and Seidel, Dian J. and Rosenlof, Karen H.},
  year = {2017},
  volume = {9},
  pages = {63--76},
  doi = {10.5194/essd-9-63-2017},
  journal = {Earth System Science Data}
}

@article{butlerDefining2015,
  title = {Defining {{Sudden Stratospheric Warmings}}},
  author = {Butler, Amy H. and Seidel, Dian J. and Hardiman, Steven C. and Butchart, Neal and Birner, Thomas and Match, Aaron},
  year = {2015},
  month = nov,
  volume = {96},
  pages = {1913--1928},
  publisher = {{American Meteorological Society}},
  issn = {0003-0007, 1520-0477},
  doi = {10.1175/BAMS-D-13-00173.1},
  abstract = {{$<$}section class="abstract"{$><$}h2 class="abstractTitle text-title my-1" id="d888e2"{$>$}Abstract{$<$}/h2{$><$}p{$>$}Sudden stratospheric warmings (SSWs) are large, rapid temperature rises in the winter polar stratosphere, occurring predominantly in the Northern Hemisphere. Major SSWs are also associated with a reversal of the climatological westerly zonal-mean zonal winds. Circulation anomalies associated with SSWs can descend into the troposphere with substantial surface weather impacts, such as wintertime extreme cold air outbreaks. After their discovery in 1952, SSWs were classified by the World Meteorological Organization. An examination of literature suggests that a single, original reference for an exact definition of SSWs is elusive, but in many references a definition involves the reversal of the meridional temperature gradient and, for major warmings, the reversal of the zonal circulation poleward of 60\textdegree{} latitude at 10 hPa.Though versions of this definition are still commonly used to detect SSWs, the details of the definition and its implementation remain ambiguous. In addition, other SSW definitions have been used in the last few decades, resulting in inconsistent classification of SSW events. We seek to answer the questions: How has the SSW definition changed, and how sensitive is the detection of SSWs to the definition used? For what kind of analysis is a ``standard'' definition useful? We argue that a standard SSW definition is necessary for maintaining a consistent and robust metric to assess polar stratospheric wintertime variability in climate models and other statistical applications. To provide a basis for, and to encourage participation in, a communitywide discussion currently underway, we explore what criteria are important for a standard definition and propose possible ways to update the definition.{$<$}/p{$><$}/section{$>$}},
  chapter = {Bulletin of the American Meteorological Society},
  file = {/Users/oscardimdore-miles/Zotero/storage/5PN6K5UP/Butler et al. - 2015 - Defining Sudden Stratospheric Warmings.pdf;/Users/oscardimdore-miles/Zotero/storage/G8RB4P9S/bams-d-13-00173.1.html},
  journal = {Bulletin of the American Meteorological Society},
  language = {English},
  number = {11}
}

@article{butlerNino2011,
  title = {El {{Ni\~no}}, {{La Ni\~na}}, and Stratospheric Sudden Warmings: {{A}} Reevaluation in Light of the Observational Record},
  shorttitle = {El {{Ni\~no}}, {{La Ni\~na}}, and Stratospheric Sudden Warmings},
  author = {Butler, Amy H. and Polvani, Lorenzo M.},
  year = {2011},
  volume = {38},
  issn = {1944-8007},
  doi = {10.1029/2011GL048084},
  abstract = {Recent studies have suggested that El Ni\~no-Southern Oscillation (ENSO) may have a considerable impact on Northern Hemisphere wintertime stratospheric conditions. Notably, during El Ni\~no the stratosphere is warmer than during ENSO-neutral winters, and the polar vortex is weaker. Opposite-signed anomalies have been reported during La Ni\~na, but are considerably smaller in amplitude than during El Ni\~no. This has led to the perception that El Ni\~no is able to substantially affect stratospheric conditions, but La Ni\~na is of secondary importance. Here we revisit this issue, but focus on the extreme events that couple the troposphere to the stratosphere: major, mid-winter stratospheric sudden warmings (SSWs). We examine 53 years of reanalysis data and find, as expected, that SSWs are nearly twice as frequent during ENSO winters as during non-ENSO winters. Surprisingly, however, we also find that SSWs occur with equal probability during El Ni\~no and La Ni\~na winters. These findings corroborate the impact of ENSO on stratospheric variability, and highlight that both phases of ENSO are important in enhancing stratosphere-troposphere dynamical coupling via an increased frequency of SSWs.},
  copyright = {Copyright 2011 by the American Geophysical Union.},
  file = {/Users/oscardimdore-miles/Zotero/storage/JZ67E4BM/Butler and Polvani - 2011 - El Niño, La Niña, and stratospheric sudden warming.pdf;/Users/oscardimdore-miles/Zotero/storage/SPE6XKRF/2011GL048084.html},
  journal = {Geophysical Research Letters},
  keywords = {ENSO,stratosphere,sudden warmings},
  language = {English},
  number = {13}
}

@article{charlton-perezInfluence2018a,
  title = {The Influence of the Stratospheric State on {{North Atlantic}} Weather Regimes},
  author = {{Charlton-Perez}, Andrew J. and Ferranti, Laura and Lee, Robert W.},
  year = {2018},
  volume = {144},
  pages = {1140--1151},
  issn = {1477-870X},
  doi = {10.1002/qj.3280},
  file = {/Users/oscardimdore-miles/Zotero/storage/W4WCEVCN/Charlton‐Perez et al. - 2018 - The influence of the stratospheric state on North .pdf;/Users/oscardimdore-miles/Zotero/storage/SU6THAW5/qj.html},
  journal = {Quarterly Journal of the Royal Meteorological Society},
  keywords = {stratosphere–troposphere coupling,weather regimes},
  language = {English},
  number = {713}
}

@article{Charlton2007,
  title = {A New Look at Stratospheric Sudden Warmings. {{Part II}}: {{Evaluation}} of Numerical Model Simulations},
  author = {Charlton, A.J. and Polvani, L.M. and Perlwitz, J. and Sassi, F. and Manzini, E. and Shibata, K. and Pawson, S. and Nielsen, J.E. and Rind, D.},
  year = {2007},
  volume = {10},
  pages = {470--488},
  doi = {doi:10.1175/JCLI3994.1.},
  journal = {Journal of Climate}
}

@article{charltonNew2007a,
  title = {A {{New Look}} at {{Stratospheric Sudden Warmings}}. {{Part I}}: {{Climatology}} and {{Modeling Benchmarks}}},
  shorttitle = {A {{New Look}} at {{Stratospheric Sudden Warmings}}. {{Part I}}},
  author = {Charlton, Andrew J. and Polvani, Lorenzo M.},
  year = {2007},
  month = feb,
  volume = {20},
  pages = {449--469},
  publisher = {{American Meteorological Society}},
  issn = {0894-8755, 1520-0442},
  doi = {10.1175/JCLI3996.1},
  abstract = {{$<$}section class="abstract"{$><$}h2 class="abstractTitle text-title my-1" id="d340e2"{$>$}Abstract{$<$}/h2{$><$}p{$>$}Stratospheric sudden warmings are the clearest and strongest manifestation of dynamical coupling in the stratosphere\textendash troposphere system. While many sudden warmings have been individually documented in the literature, this study aims at constructing a comprehensive climatology: all major midwinter warming events are identified and classified, in both the NCEP\textendash NCAR and 40-yr ECMWF Re-Analysis (ERA-40) datasets. To accomplish this a new, objective identification algorithm is developed. This algorithm identifies sudden warmings based on the zonal mean zonal wind at 60\textdegree N and 10 hPa, and classifies them into events that do and do not split the stratospheric polar vortex.Major midwinter stratospheric sudden warmings are found to occur with a frequency of approximately six events per decade, and 46\% of warming events lead to a splitting of the stratospheric polar vortex. The dynamics of vortex splitting events is contrasted to that of events where the vortex is merely displaced off the pole. In the stratosphere, the two types of events are found to be dynamically distinct: vortex splitting events occur after a clear preconditioning of the polar vortex, and their influence on middle-stratospheric temperatures lasts for up to 20 days longer than vortex displacement events. In contrast, the influence of sudden warmings on the tropospheric state is found to be largely insensitive to the event type.Finally, a table of dynamical benchmarks for major stratospheric sudden warming events is compiled. These benchmarks are used in a companion study to evaluate current numerical model simulations of the stratosphere.{$<$}/p{$><$}/section{$>$}},
  chapter = {Journal of Climate},
  file = {/Users/oscardimdore-miles/Zotero/storage/5A3JGE7X/Charlton and Polvani - 2007 - A New Look at Stratospheric Sudden Warmings. Part .pdf;/Users/oscardimdore-miles/Zotero/storage/57GK67XC/jcli3996.1.html},
  journal = {Journal of Climate},
  language = {English},
  number = {3}
}

@article{Charney1961,
  title = {Propagation of Planetary-Scale Disturbances from the Lower into the Upper Atmosphere},
  author = {Charney, J G and Drazin, P G I},
  year = {1961},
  volume = {66},
  pages = {83--109},
  doi = {doi:10.1029/JZ066i001p00083},
  journal = {Journal of Geophysical Research},
  number = {1}
}

@article{Chen2020,
  title = {Potential Impact of Preceding {{Aleutian Low}} Variation on the {{El Ni\~no}}-{{Southern Oscillation}} during the Following Winter},
  author = {Chen, Shangfeng and Chen, Wen and Wu, Renguang and Yu, Bin and Graf, Hans-F},
  year = {2020},
  volume = {33},
  doi = {10.1175/JCLI-D-19-0717.1},
  journal = {Journal of Climate}
}

@article{chengIce2009a,
  title = {Ice {{Age Terminations}}},
  author = {Cheng, Hai and Edwards, R. Lawrence and Broecker, Wallace S. and Denton, George H. and Kong, Xinggong and Wang, Yongjin and Zhang, Rong and Wang, Xianfeng},
  year = {2009},
  month = oct,
  volume = {326},
  pages = {248--252},
  publisher = {{American Association for the Advancement of Science}},
  issn = {0036-8075, 1095-9203},
  doi = {10.1126/science.1177840},
  abstract = {230Th-dated oxygen isotope records of stalagmites from Sanbao Cave, China, characterize Asian Monsoon (AM) precipitation through the ends of the third- and fourthmost recent ice ages. As a result, AM records for the past four glacial terminations can now be precisely correlated with those from ice cores and marine sediments, establishing the timing and sequence of major events. In all four cases, observations are consistent with a classic Northern Hemisphere summer insolation intensity trigger for an initial retreat of northern ice sheets. Meltwater and icebergs entering the North Atlantic alter oceanic and atmospheric circulation and associated fluxes of heat and carbon, causing increases in atmospheric CO2 and Antarctic temperatures that drive the termination in the Southern Hemisphere. Increasing CO2 and summer insolation drive recession of northern ice sheets, with probable positive feedbacks between sea level and CO2. Variability of the Asian Monsoon over the past 400,000 years correlates with the ends of glacial periods. Variability of the Asian Monsoon over the past 400,000 years correlates with the ends of glacial periods.},
  chapter = {Research Article},
  copyright = {Copyright \textcopyright{} 2009, American Association for the Advancement of Science},
  file = {/Users/oscardimdore-miles/Zotero/storage/XMPYUL6L/Cheng et al. - 2009 - Ice Age Terminations.pdf;/Users/oscardimdore-miles/Zotero/storage/EKIP3ZHV/248.html},
  journal = {Science},
  language = {English},
  number = {5950},
  pmid = {19815769}
}

@article{chenPotential2020,
  title = {Potential {{Impact}} of {{Preceding Aleutian Low Variation}} on {{El Ni\~no}}\textendash{{Southern Oscillation}} during the {{Following Winter}}},
  author = {Chen, Shangfeng and Chen, Wen and Wu, Renguang and Yu, Bin and Graf, Hans-F.},
  year = {2020},
  month = mar,
  volume = {33},
  pages = {3061--3077},
  publisher = {{American Meteorological Society}},
  issn = {0894-8755, 1520-0442},
  doi = {10.1175/JCLI-D-19-0717.1},
  abstract = {{$<$}section class="abstract"{$><$}h2 class="abstractTitle text-title my-1" id="d639e2"{$>$}Abstract{$<$}/h2{$><$}p{$>$}The present study reveals a close relation between the interannual variation of Aleutian low intensity (ALI) in March and the subsequent winter El Ni\~no\textendash Southern Oscillation (ENSO). When March ALI is weaker (stronger) than normal, an El Ni\~no (a La Ni\~na)\textendash like sea surface temperature (SST) warming (cooling) tends to appear in the equatorial central-eastern Pacific during the subsequent winter. The physical process linking March ALI to the following winter ENSO is as follows. When March ALI is below normal, a notable atmospheric dipole pattern develops over the North Pacific, with an anticyclonic anomaly over the Aleutian region and a cyclonic anomaly over the subtropical west-central Pacific. The formation of the anomalous cyclone is attributed to feedback of the synoptic-scale eddy-to-mean-flow energy flux and associated vorticity transportation. Specifically, easterly wind anomalies over the midlatitudes related to the weakened ALI are accompanied by a decrease in synoptic-scale eddy activity, which forces an anomalous cyclone to its southern flank. The accompanying westerly wind anomalies over the tropical west-central Pacific induce SST warming in the equatorial central-eastern Pacific during the following summer\textendash autumn via triggering eastward-propagating warm Kelvin waves, which may sustain and develop into an El Ni\~no event during the following winter via positive air\textendash sea feedback. The relation of March ALI with the following winter ENSO is independent of the preceding tropical Pacific SST, the preceding-winter North Pacific Oscillation, and the spring Arctic Oscillation. The results of this analysis may provide an additional source for the prediction of ENSO.{$<$}/p{$><$}/section{$>$}},
  chapter = {Journal of Climate},
  file = {/Users/oscardimdore-miles/Zotero/storage/VS5II8GW/Chen et al. - 2020 - Potential Impact of Preceding Aleutian Low Variati.pdf;/Users/oscardimdore-miles/Zotero/storage/7NL67SFD/jcli-d-19-0717.1.html},
  journal = {Journal of Climate},
  language = {English},
  number = {8}
}

@article{Cohen2007,
  title = {Stratosphere Troposphere Coupling and Links with Eurasian Land Surface Variability},
  author = {Cohen, Judah and Barlow, Mathew and Kushner, Paul and Saito, Kazuyuki},
  year = {2007},
  volume = {20},
  doi = {10.1175/2007JCLI1725.1},
  journal = {Journal of Climate - J CLIMATE}
}

@article{cohen2009,
  title = {Decadal Fluctuations in Planetary Wave Forcing Modulate Global Warming in Late Boreal Winter},
  author = {Cohen, J. and Barlow, M. and Saito, K.},
  year = {2009},
  volume = {22},
  pages = {4418--4426},
  publisher = {{American Meteorological Society}},
  doi = {10.1175/2009JCLI2931.1},
  journal = {Journal of Climate},
  number = {16}
}

@article{cohenDecadal2009,
  title = {Decadal {{Fluctuations}} in {{Planetary Wave Forcing Modulate Global Warming}} in {{Late Boreal Winter}}},
  author = {Cohen, Judah and Barlow, Mathew and Saito, Kazuyuki},
  year = {2009},
  month = aug,
  volume = {22},
  pages = {4418--4426},
  publisher = {{American Meteorological Society}},
  issn = {0894-8755, 1520-0442},
  doi = {10.1175/2009JCLI2931.1},
  abstract = {{$<$}section class="abstract"{$><$}h2 class="abstractTitle text-title my-1" id="d39294025e76"{$>$}Abstract{$<$}/h2{$><$}p{$>$}The warming trend in global surface temperatures over the last 40 yr is clear and consistent with anthropogenic increases in greenhouse gases. Over the last 2 decades, this trend appears to have accelerated. In contrast to this general behavior, however, here it is shown that trends during the boreal cold months in the recent period have developed a marked asymmetry between early winter and late winter for the Northern Hemisphere, with vigorous warming in October\textendash December followed by a reversal to a neutral/cold trend in January\textendash March. This observed asymmetry in the cold half of the boreal year is linked to a two-way stratosphere\textendash troposphere interaction, which is strongest in the Northern Hemisphere during late winter and is related to variability in Eurasian land surface conditions during autumn. This link has been demonstrated for year-to-year variability and used to improve seasonal time-scale winter forecasts; here, this coupling is shown to strongly modulate the warming trend, with implications for decadal-scale temperature projections.{$<$}/p{$><$}/section{$>$}},
  chapter = {Journal of Climate},
  file = {/Users/oscardimdore-miles/Zotero/storage/DEMBISWW/Cohen et al. - 2009 - Decadal Fluctuations in Planetary Wave Forcing Mod.pdf},
  journal = {Journal of Climate},
  language = {English},
  number = {16}
}

@article{Daubechies,
  title = {The Wavelet Transform, Time-Frequency Localization and Signal Analysis},
  author = {Daubechies, I.},
  year = {1990},
  volume = {36},
  pages = {961--1005},
  doi = {10.1109/18.57199},
  journal = {IEEE Transactions on Information Theory},
  number = {5}
}

@article{daviniBlocking2014,
  title = {A Blocking View of the Stratosphere-Troposphere Coupling},
  author = {Davini, P. and Cagnazzo, C. and Anstey, J. A.},
  year = {2014},
  volume = {119},
  pages = {11,100--11,115},
  issn = {2169-8996},
  doi = {10.1002/2014JD021703},
  abstract = {AbstractDynamical influence from the stratosphere is known to play a role in shaping the wintertime tropospheric circulation patterns. Observations suggest that this influence is strongest following weak and strong polar vortex events, termed sudden stratospheric warmings (SSWs) and vortex intensification (VI) events, respectively. In this work, stratosphere-troposphere coupling is studied through the modulation by extreme vortex events of the Northern Hemispheric tropospheric blocking frequency and eddy-driven jet displacements. This is done using three reanalysis data sets and the Centro Euro-Mediterraneo sui Cambiamenti Climatici Climate Model with a resolved Stratosphere (CMCC-CMS) coupled model control run. Reanalysis results suggest the existence of distinct patterns of blocking activity following extreme vortex events. Over the Atlantic basin, SSWs are shown to lead by about 20\textendash 50 days the occurrence of increased/reduced blocking frequency on the poleward/equatorward side of the Atlantic jet stream. Anomalies of the opposite sense, with poleward reduction and equatorward enhancement of the blocking frequency, occur following VI events. The response over the Pacific sector is less clear. Compared to reanalyses, CMCC-CMS shows a similar but weaker response, especially over the Atlantic: a possible explanation is identified in the different structure of the polar vortex and weaker wind shear anomalies with respect to reanalysis. We finally highlight that patterns identified following vortex extremes show similarities with the Northern Annular Mode over the Atlantic but not over the Pacific. This suggests that the stratosphere-troposphere coupling is more a regional than annular feature.},
  copyright = {\textcopyright 2014. American Geophysical Union. All Rights Reserved.},
  file = {/Users/oscardimdore-miles/Zotero/storage/XBQ9BX87/Davini et al. - 2014 - A blocking view of the stratosphere-troposphere co.pdf;/Users/oscardimdore-miles/Zotero/storage/263KKD64/2014JD021703.html},
  journal = {Journal of Geophysical Research: Atmospheres},
  keywords = {blocking,jet stream,stratosphere-troposphere coupling},
  language = {English},
  number = {19}
}

@article{Dee2011,
  title = {The {{ERA}}-{{Interim}} Reanalysis: {{Configuration}} and Performance of the Data Assimilation System},
  author = {Dee, D. P. and Uppala, S. M. and Simmons, A. J. and Berrisford, P. and Poli, P. and Kobayashi, S. and Andrae, U. and Balmaseda, M. A. and Balsamo, G. and Bauer, P. and Bechtold, P. and Beljaars, A. C.M. and {van de Berg}, L. and Bidlot, J. and Bormann, N. and Delsol, C. and Dragani, R. and Fuentes, M. and Geer, A. J. and Haimberger, L. and Healy, S. B. and Hersbach, H. and H{\'o}lm, E. V. and Isaksen, L. and K{\aa}llberg, P. and K{\"o}hler, M. and Matricardi, M. and Mcnally, A. P. and {Monge-Sanz}, B. M. and Morcrette, J. J. and Park, B. K. and Peubey, C. and {de Rosnay}, P. and Tavolato, C. and Th{\'e}paut, J. N. and Vitart, F.},
  year = {2011},
  volume = {137},
  pages = {553--597},
  doi = {10.1002/qj.828},
  journal = {Quarterly Journal of the Royal Meteorological Society},
  number = {656}
}

@misc{Defining,
  title = {The {{Defining Characteristics}} of {{ENSO Extremes}} and the {{Strong}} 2015/2016 {{El Ni\~no}} - {{Santoso}} - 2017 - {{Reviews}} of {{Geophysics}} - {{Wiley Online Library}}},
  file = {/Users/oscardimdore-miles/Zotero/storage/KWW2BCQI/2017RG000560.html}
}

@article{delworthImpact2016,
  title = {The {{Impact}} of the {{North Atlantic Oscillation}} on {{Climate}} through {{Its Influence}} on the {{Atlantic Meridional Overturning Circulation}}},
  author = {Delworth, Thomas L. and Zeng, Fanrong},
  year = {2016},
  month = feb,
  volume = {29},
  pages = {941--962},
  publisher = {{American Meteorological Society}},
  issn = {0894-8755, 1520-0442},
  doi = {10.1175/JCLI-D-15-0396.1},
  abstract = {{$<$}section class="abstract"{$><$}h2 class="abstractTitle text-title my-1" id="d7e2"{$>$}Abstract{$<$}/h2{$><$}p{$>$}The impact of the North Atlantic Oscillation (NAO) on the Atlantic meridional overturning circulation (AMOC) and large-scale climate is assessed using simulations with three different climate models. Perturbation experiments are conducted in which a pattern of anomalous heat flux corresponding to the NAO is added to the model ocean. Differences between the perturbation experiments and a control illustrate how the model ocean and climate system respond to the NAO. A positive phase of the NAO strengthens the AMOC by extracting heat from the subpolar gyre, thereby increasing deep-water formation, horizontal density gradients, and the AMOC. The flux forcings have the spatial structure of the observed NAO, but the amplitude of the forcing varies in time with distinct periods varying from 2 to 100 yr. The response of the AMOC to NAO variations is small at short time scales but increases up to the dominant time scale of internal AMOC variability (20\textendash 30 yr for the models used). The amplitude of the AMOC response, as well as associated oceanic heat transport, is approximately constant as the time scale of the forcing is increased further. In contrast, the response of other properties, such as hemispheric temperature or Arctic sea ice, continues to increase as the time scale of the forcing becomes progressively longer. The larger response is associated with the time integral of the anomalous oceanic heat transport at longer time scales, combined with an increased impact of radiative feedback processes. It is shown that NAO fluctuations, similar in amplitude to those observed over the last century, can modulate hemispheric temperature by several tenths of a degree.{$<$}/p{$><$}/section{$>$}},
  chapter = {Journal of Climate},
  file = {/Users/oscardimdore-miles/Zotero/storage/83I9JAHW/Delworth and Zeng - 2016 - The Impact of the North Atlantic Oscillation on Cl.pdf},
  journal = {Journal of Climate},
  language = {English},
  number = {3}
}

@article{delworthImplications2000,
  title = {Implications of the {{Recent Trend}} in the {{Arctic}}/{{North Atlantic Oscillation}} for the {{North Atlantic Thermohaline Circulation}}},
  author = {Delworth, Thomas L. and Dixon, Keith W.},
  year = {2000},
  month = nov,
  volume = {13},
  pages = {3721--3727},
  publisher = {{American Meteorological Society}},
  issn = {0894-8755, 1520-0442},
  doi = {10.1175/1520-0442(2000)013<3721:IOTRTI>2.0.CO;2},
  abstract = {{$<$}section class="abstract"{$><$}h2 class="abstractTitle text-title my-1" id="d5e2"{$>$}Abstract{$<$}/h2{$><$}p{$>$}Most projections of greenhouse gas\textendash induced climate change indicate a weakening of the thermohaline circulation (THC) in the North Atlantic in response to increased freshening and warming in the subpolar region. These changes reduce high-latitude upper-ocean density and therefore weaken the THC. Using ensembles of numerical experiments with a coupled ocean\textendash atmosphere model, it is found that this weakening could be delayed by several decades in response to a sustained upward trend in the Arctic/North Atlantic oscillation during winter, such as has been observed over the last 30 years. The stronger winds over the North Atlantic associated with this trend extract more heat from the ocean, thereby cooling and increasing the density of the upper ocean and thus opposing the previously described weakening of the THC. This result is of particular importance if the positive trend in the Arctic/North Atlantic oscillation is a response to increasing greenhouse gases, as has been recently suggested.{$<$}/p{$><$}/section{$>$}},
  chapter = {Journal of Climate},
  file = {/Users/oscardimdore-miles/Zotero/storage/KQKCX4LL/Delworth and Dixon - 2000 - Implications of the Recent Trend in the ArcticNor.pdf;/Users/oscardimdore-miles/Zotero/storage/JXWPIVRW/1520-0442_2000_013_3721_iotrti_2.0.co_2.html},
  journal = {Journal of Climate},
  language = {English},
  number = {21}
}

@article{delworthInterdecadal1993,
  title = {Interdecadal {{Variations}} of the {{Thermohaline Circulation}} in a {{Coupled Ocean}}-{{Atmosphere Model}}},
  author = {Delworth, T. and Manabe, S. and Stouffer, R. J.},
  year = {1993},
  month = nov,
  volume = {6},
  pages = {1993--2011},
  publisher = {{American Meteorological Society}},
  issn = {0894-8755, 1520-0442},
  doi = {10.1175/1520-0442(1993)006<1993:IVOTTC>2.0.CO;2},
  abstract = {{$<$}section class="abstract"{$><$}h2 class="abstractTitle text-title my-1" id="d947e2"{$>$}Abstract{$<$}/h2{$><$}p{$>$}A fully coupled ocean-atmosphere model is shown to have irregular oscillations of the thermohaline circulation in the North Atlantic Ocean with a time scale of approximately 50 years. The irregular oscillation appears to be driven by density anomalies in the sinking region of the thermohaline circulation (approximately 52\textdegree N to 72\textdegree N) combined with much smaller density anomalies of opposite sign in the broad, rising region. The spatial pattern of see surface temperature anomalies associated with this irregular oscillation bears an encouraging resemblance to a pattern of observed interdecadal variability in the North Atlantic. The anomalies of sea surface temperature induce model surface air temperature anomalies over the northern North Atlantic, Arctic, and northwestern Europe.{$<$}/p{$><$}/section{$>$}},
  chapter = {Journal of Climate},
  file = {/Users/oscardimdore-miles/Zotero/storage/WIVBG4F7/Delworth et al. - 1993 - Interdecadal Variations of the Thermohaline Circul.pdf},
  journal = {Journal of Climate},
  language = {English},
  number = {11}
}

@article{delworthMultidecadal2000,
  title = {Multidecadal {{Thermohaline Circulation Variability Driven}} by {{Atmospheric Surface Flux Forcing}}},
  author = {Delworth, Thomas L. and Greatbatch, Richard J.},
  year = {2000},
  month = may,
  volume = {13},
  pages = {1481--1495},
  publisher = {{American Meteorological Society}},
  issn = {0894-8755, 1520-0442},
  doi = {10.1175/1520-0442(2000)013<1481:MTCVDB>2.0.CO;2},
  abstract = {{$<$}section class="abstract"{$><$}h2 class="abstractTitle text-title my-1" id="d938e2"{$>$}Abstract{$<$}/h2{$><$}p{$>$}Previous analyses of an extended integration of the Geophysical Fluid Dynamics Laboratory coupled climate model have revealed pronounced multidecadal variations of the thermohaline circulation (THC) in the North Atlantic. The purpose of the current work is to assess whether those fluctuations can be viewed as a coupled air\textendash sea mode (in the sense of ENSO), or as an oceanic response to forcing from the atmosphere model, in which large-scale feedbacks from the ocean to the atmospheric circulation are not critical.A series of integrations using the ocean component of the coupled model are performed to address the above question. The ocean model is forced by suitably chosen time series of surface fluxes from either the coupled model or a companion integration of an atmosphere-only model run with a prescribed seasonal cycle of SSTs and sea-ice thickness. These experiments reveal that 1) the previously identified multidecadal THC variations can be largely viewed as an oceanic response to surface flux forcing from the atmosphere model, although air\textendash sea coupling through the thermodynamics appears to modify the amplitude of the variability, and 2) variations in heat flux are the dominant term (relative to the freshwater and momentum fluxes) in driving the THC variability. Experiments driving the ocean model using either high- or low-pass-filtered heat fluxes, with a cutoff period of 20 yr, show that the multidecadal THC variability is driven by the low-frequency portion of the spectrum of atmospheric flux forcing. Analyses have also revealed that the multidecadal THC fluctuations are driven by a spatial pattern of surface heat flux variations that bears a strong resemblance to the North Atlantic oscillation. No conclusive evidence is found that the THC variability is part of a dynamically coupled mode of the atmosphere and ocean models.{$<$}/p{$><$}/section{$>$}},
  chapter = {Journal of Climate},
  file = {/Users/oscardimdore-miles/Zotero/storage/YGVVSZXP/Delworth and Greatbatch - 2000 - Multidecadal Thermohaline Circulation Variability .pdf;/Users/oscardimdore-miles/Zotero/storage/FE7TT2ME/1520-0442_2000_013_1481_mtcvdb_2.0.co_2.html},
  journal = {Journal of Climate},
  language = {English},
  number = {9}
}

@article{delworthObserved2000,
  title = {Observed and Simulated Multidecadal Variability in the {{Northern Hemisphere}}},
  author = {Delworth, T. L. and Mann, M. E.},
  year = {2000},
  month = sep,
  volume = {16},
  pages = {661--676},
  issn = {1432-0894},
  doi = {10.1007/s003820000075},
  abstract = {Analyses of proxy based reconstructions of surface temperatures during the past 330 years show the existence of a distinct oscillatory mode of variability with an approximate time scale of 70 years. This variability is also seen in instrumental records, although the oscillatory nature of the variability is difficult to assess due to the short length of the instrumental record. The spatial pattern of this variability is hemispheric or perhaps even global in scale, but with particular emphasis on the Atlantic region. Independent analyses of multicentury integrations of two versions of the GFDL coupled atmosphere-ocean model also show the existence of distinct multidecadal variability in the North Atlantic region which resembles the observed pattern. The model variability involves fluctuations in the intensity of the thermohaline circulation in the North Atlantic. It is our intent here to provide a direct comparison of the observed variability to that simulated in a coupled ocean-atmosphere model, making use of both existing instrumental analyses and newly available proxy based multi-century surface temperature estimates. The analyses demonstrate a substantial agreement between the simulated and observed patterns of multidecadal variability in sea surface temperature (SST) over the North Atlantic. There is much less agreement between the model and observations for sea level pressure. Seasonal analyses of the variability demonstrate that for both the model and observations SST appears to be the primary carrier of the multidecadal signal.},
  file = {/Users/oscardimdore-miles/Zotero/storage/A5SB6I48/Delworth and Mann - 2000 - Observed and simulated multidecadal variability in.pdf},
  journal = {Climate Dynamics},
  language = {English},
  number = {9}
}

@article{deser2017,
  title = {The Northern Hemisphere Extratropical Atmospheric Circulation Response to {{ENSO}}: {{How}} Well Do We Know It and How Do We Evaluate Models Accordingly?},
  author = {Deser, Clara and Simpson, Isla R. and McKinnon, Karen A. and Phillips, Adam S.},
  year = {2017},
  volume = {30},
  pages = {5059--5082},
  doi = {10.1175/JCLI-D-16-0844.1},
  journal = {Journal of Climate},
  number = {13}
}

@article{deserNorthern2017,
  title = {The {{Northern Hemisphere Extratropical Atmospheric Circulation Response}} to {{ENSO}}: {{How Well Do We Know It}} and {{How Do We Evaluate Models Accordingly}}?},
  shorttitle = {The {{Northern Hemisphere Extratropical Atmospheric Circulation Response}} to {{ENSO}}},
  author = {Deser, Clara and Simpson, Isla R. and McKinnon, Karen A. and Phillips, Adam S.},
  year = {2017},
  month = jul,
  volume = {30},
  pages = {5059--5082},
  publisher = {{American Meteorological Society}},
  issn = {0894-8755, 1520-0442},
  doi = {10.1175/JCLI-D-16-0844.1},
  abstract = {{$<$}section class="abstract"{$><$}h2 class="abstractTitle text-title my-1" id="d3010e2"{$>$}Abstract{$<$}/h2{$><$}p{$>$}Application of random sampling techniques to composite differences between 18 El Ni\~no and 14 La Ni\~na events observed since 1920 reveals considerable uncertainty in both the pattern and amplitude of the Northern Hemisphere extratropical winter sea level pressure (SLP) response to ENSO. While the SLP responses over the North Pacific and North America are robust to sampling variability, their magnitudes can vary by a factor of 2; other regions, such as the Arctic, North Atlantic, and Europe are less robust in their SLP patterns, amplitudes, and statistical significance. The uncertainties on the observed ENSO composite are shown to arise mainly from atmospheric internal variability as opposed to ENSO diversity. These observational findings pose considerable challenges for the evaluation of ENSO teleconnections in models. An approach is proposed that incorporates both pattern and amplitude uncertainty in the observational target, allowing for discrimination between true model biases in the forced ENSO response and apparent model biases that arise from limited sampling of non-ENSO-related internal variability. Large initial-condition coupled model ensembles with realistic tropical Pacific sea surface temperature anomaly evolution during 1920\textendash 2013 show similar levels of uncertainty in their ENSO teleconnections as found in observations. Because the set of ENSO events in each of the model composites is the same (and identical to that in observations), these uncertainties are entirely attributable to sampling fluctuations arising from internal variability, which is shown to originate from atmospheric processes. The initial-condition model ensembles thus inform the interpretation of the single observed ENSO composite and vice versa.{$<$}/p{$><$}/section{$>$}},
  chapter = {Journal of Climate},
  file = {/Users/oscardimdore-miles/Zotero/storage/4KBZ4CPE/Deser et al. - 2017 - The Northern Hemisphere Extratropical Atmospheric .pdf;/Users/oscardimdore-miles/Zotero/storage/LRAXM4R8/jcli-d-16-0844.1.html},
  journal = {Journal of Climate},
  language = {English},
  number = {13}
}

@article{dimdore-milesOrigins2020,
  title = {Origins of {{Multi}}-{{Decadal Variability}} in {{Sudden Stratospheric Warmings}}},
  author = {{Dimdore-Miles}, Oscar and Gray, Lesley and Osprey, Scott},
  year = {2020},
  month = nov,
  pages = {1--36},
  publisher = {{Copernicus GmbH}},
  doi = {10.5194/wcd-2020-56},
  abstract = {{$<$}p{$><$}strong class="journal-contentHeaderColor"{$>$}Abstract.{$<$}/strong{$>$} Sudden Stratospheric Warmings (SSWs) are major disruptions of the Northern Hemisphere (NH) stratospheric polar vortex and occur on average approximately 6 times per decade in observation based records. However, within these records, intervals of significantly higher and lower SSW rates are observed suggesting the possibility of low frequency variations in event occurrence. A better understanding of factors that influence this decadal variability may help to improve predictability of NH mid-latitude surface climate, through stratosphere-troposphere coupling. In this work, multi-decadal variability of SSW events is examined in a 1000-yr pre-industrial simulation of a coupled Atmosphere-Ocean-Land-Sea ice model. Using a wavelet spectral decomposition method, we show that hiatus events (intervals of a decade or more with no SSWs) and consecutive SSW events (extended intervals with at least one SSW in each year) vary on multi-decadal timescales of period between 60 and 90 years. Signals on these timescales are present for approximately 450 years of the simulation. We investigate the possible source of these long-term signals and find that the direct impact of variability in tropical sea surface temperatures, as well as the associated Aleutian Low, can account for only a small portion of the SSW variability. Instead, the major influence on long-term SSW variability is associated with long-term variability in amplitude of the stratospheric quasi biennial oscillation (QBO). The QBO influence is consistent with the well known Holton-Tan relationship, with SSW hiatus intervals associated with extended periods of particularly strong, deep QBO westerly phases. The results support recent studies that have highlighted the role of vertical coherence in the QBO when considering coupling between the QBO, the polar vortex and tropospheric circulation.{$<$}/p{$>$}},
  file = {/Users/oscardimdore-miles/Zotero/storage/38YCFSTX/Dimdore-Miles et al. - 2020 - Origins of Multi-decadal Variability in Sudden Str.pdf;/Users/oscardimdore-miles/Zotero/storage/8WD7MRFX/wcd-2020-56.html},
  journal = {Weather and Climate Dynamics Discussions},
  language = {English}
}

@article{doi:10.1029/2020JD033271,
  title = {The Remarkably Strong Arctic Stratospheric Polar Vortex of Winter 2020: {{Links}} to Record-Breaking Arctic Oscillation and Ozone Loss},
  author = {Lawrence, Zachary D. and Perlwitz, Judith and Butler, Amy H. and Manney, Gloria L. and Newman, Paul A. and Lee, Simon H. and Nash, Eric R.},
  volume = {n/a},
  pages = {e2020JD033271},
  doi = {10.1029/2020JD033271},
  journal = {Journal of Geophysical Research: Atmospheres},
  keywords = {Arctic Oscillation,downward wave coupling,planetary waves,stratospheric ozone,stratospheric polar vortex},
  number = {n/a}
}

@article{domeisenEstimating2019,
  title = {Estimating the {{Frequency}} of {{Sudden Stratospheric Warming Events From Surface Observations}} of the {{North Atlantic Oscillation}}},
  author = {Domeisen, Daniela I. V.},
  year = {2019},
  volume = {124},
  pages = {3180--3194},
  issn = {2169-8996},
  doi = {10.1029/2018JD030077},
  abstract = {Sudden stratospheric warming (SSW) events can exhibit long-lasting surface impacts that promise improvements in medium-range to seasonal predictability. Their surface impact is dominated by the negative phase of the North Atlantic Oscillation (NAO). Hence, the question arises if stratospheric variability, and in particular the frequency of SSW events, can in turn be estimated from surface NAO conditions. This is especially relevant for the period before frequent upper air observations became available, while daily surface observations of the NAO date back to 1850. The surface impact is here quantified by NAO characteristics that are commonly observed after SSW events: a switch from a positive to a negative NAO and an extended persistence of the negative NAO, termed NAO events. Two thirds of SSW events are found to be followed by either a persistence or switch NAO event, and a quarter of SSW events are followed by both. On the other hand, less than 25\% of winter surface NAO events are preceded by a SSW event. Based on these findings, an index purely based on surface NAO observations is derived that estimates SSW frequency for the satellite era and extends it back to 1850, indicating that decadal stratospheric variability was present for the entire time series, with no significant trend. The minimum in SSW frequency in the 1990s is found to be coincident with the longest absence of NAO events since 1850, indicating that the early 1990s may constitute the longest absence of SSW events for the 150-year record.},
  file = {/Users/oscardimdore-miles/Zotero/storage/2UAK7EH9/Domeisen - 2019 - Estimating the Frequency of Sudden Stratospheric W.pdf;/Users/oscardimdore-miles/Zotero/storage/TPPTBA75/2018JD030077.html},
  journal = {Journal of Geophysical Research: Atmospheres},
  keywords = {North Atlantic Oscillation,reconstruction,stratosphere-troposphere coupling,sudden stratospheric warming,upper atmosphere},
  language = {English},
  number = {6}
}

@article{domeisenRole2020,
  title = {The {{Role}} of the {{Stratosphere}} in {{Subseasonal}} to {{Seasonal Prediction}}: 1. {{Predictability}} of the {{Stratosphere}}},
  shorttitle = {The {{Role}} of the {{Stratosphere}} in {{Subseasonal}} to {{Seasonal Prediction}}},
  author = {Domeisen, Daniela I. V. and Butler, Amy H. and {Charlton-Perez}, Andrew J. and Ayarzag{\"u}ena, Blanca and Baldwin, Mark P. and {Dunn-Sigouin}, Etienne and Furtado, Jason C. and Garfinkel, Chaim I. and Hitchcock, Peter and Karpechko, Alexey Yu and Kim, Hera and Knight, Jeff and Lang, Andrea L. and Lim, Eun-Pa and Marshall, Andrew and Roff, Greg and Schwartz, Chen and Simpson, Isla R. and Son, Seok-Woo and Taguchi, Masakazu},
  year = {2020},
  volume = {125},
  pages = {e2019JD030920},
  issn = {2169-8996},
  doi = {10.1029/2019JD030920},
  abstract = {The stratosphere has been identified as an important source of predictability for a range of processes on subseasonal to seasonal (S2S) time scales. Knowledge about S2S predictability within the stratosphere is however still limited. This study evaluates to what extent predictability in the extratropical stratosphere exists in hindcasts of operational prediction systems in the S2S database. The stratosphere is found to exhibit extended predictability as compared to the troposphere. Prediction systems with higher stratospheric skill tend to also exhibit higher skill in the troposphere. The analysis also includes an assessment of the predictability for stratospheric events, including early and midwinter sudden stratospheric warming events, strong vortex events, and extreme heat flux events for the Northern Hemisphere and final warming events for both hemispheres. Strong vortex events and final warming events exhibit higher levels of predictability as compared to sudden stratospheric warming events. In general, skill is limited to the deterministic range of 1 to 2 weeks. High-top prediction systems overall exhibit higher stratospheric prediction skill as compared to their low-top counterparts, pointing to the important role of stratospheric representation in S2S prediction models.},
  copyright = {\textcopyright 2019. American Geophysical Union. All Rights Reserved.},
  file = {/Users/oscardimdore-miles/Zotero/storage/6MV8XKW5/Domeisen et al. - 2020 - The Role of the Stratosphere in Subseasonal to Sea.pdf;/Users/oscardimdore-miles/Zotero/storage/NUCJCMA6/2019JD030920.html},
  journal = {Journal of Geophysical Research: Atmospheres},
  keywords = {S2S database,stratosphere,sub-seasonal predictability,sudden stratospheric warming},
  language = {English},
  number = {2}
}

@article{domeisenRole2020a,
  title = {The {{Role}} of the {{Stratosphere}} in {{Subseasonal}} to {{Seasonal Prediction}}: 2. {{Predictability Arising From Stratosphere}}-{{Troposphere Coupling}}},
  shorttitle = {The {{Role}} of the {{Stratosphere}} in {{Subseasonal}} to {{Seasonal Prediction}}},
  author = {Domeisen, Daniela I. V. and Butler, Amy H. and {Charlton-Perez}, Andrew J. and Ayarzag{\"u}ena, Blanca and Baldwin, Mark P. and {Dunn-Sigouin}, Etienne and Furtado, Jason C. and Garfinkel, Chaim I. and Hitchcock, Peter and Karpechko, Alexey Yu and Kim, Hera and Knight, Jeff and Lang, Andrea L. and Lim, Eun-Pa and Marshall, Andrew and Roff, Greg and Schwartz, Chen and Simpson, Isla R. and Son, Seok-Woo and Taguchi, Masakazu},
  year = {2020},
  volume = {125},
  pages = {e2019JD030923},
  issn = {2169-8996},
  doi = {10.1029/2019JD030923},
  abstract = {The stratosphere can have a significant impact on winter surface weather on subseasonal to seasonal (S2S) timescales. This study evaluates the ability of current operational S2S prediction systems to capture two important links between the stratosphere and troposphere: (1) changes in probabilistic prediction skill in the extratropical stratosphere by precursors in the tropics and the extratropical troposphere and (2) changes in surface predictability in the extratropics after stratospheric weak and strong vortex events. Probabilistic skill exists for stratospheric events when including extratropical tropospheric precursors over the North Pacific and Eurasia, though only a limited set of models captures the Eurasian precursors. Tropical teleconnections such as the Madden-Julian Oscillation, the Quasi-Biennial Oscillation, and El Ni\~no\textendash Southern Oscillation increase the probabilistic skill of the polar vortex strength, though these are only captured by a limited set of models. At the surface, predictability is increased over the United States, Russia, and the Middle East for weak vortex events, but not for Europe, and the change in predictability is smaller for strong vortex events for all prediction systems. Prediction systems with poorly resolved stratospheric processes represent this skill to a lesser degree. Altogether, the analyses indicate that correctly simulating stratospheric variability and stratosphere-troposphere dynamical coupling are critical elements for skillful S2S wintertime predictions.},
  copyright = {\textcopyright 2019. American Geophysical Union. All Rights Reserved.},
  file = {/Users/oscardimdore-miles/Zotero/storage/D36EQD8G/Domeisen et al. - 2020 - The Role of the Stratosphere in Subseasonal to Sea.pdf;/Users/oscardimdore-miles/Zotero/storage/27KD2Y59/2019JD030923.html},
  journal = {Journal of Geophysical Research: Atmospheres},
  keywords = {North Atlantic Oscillation,S2S database,stratosphere,stratosphere - troposphere coupling,sub-seasonal predictability,sudden stratospheric warming},
  language = {English},
  number = {2}
}

@article{domeisenRole2020b,
  title = {The {{Role}} of the {{Stratosphere}} in {{Subseasonal}} to {{Seasonal Prediction}}: 2. {{Predictability Arising From Stratosphere}}-{{Troposphere Coupling}}},
  shorttitle = {The {{Role}} of the {{Stratosphere}} in {{Subseasonal}} to {{Seasonal Prediction}}},
  author = {Domeisen, Daniela I. V. and Butler, Amy H. and {Charlton-Perez}, Andrew J. and Ayarzag{\"u}ena, Blanca and Baldwin, Mark P. and {Dunn-Sigouin}, Etienne and Furtado, Jason C. and Garfinkel, Chaim I. and Hitchcock, Peter and Karpechko, Alexey Yu and Kim, Hera and Knight, Jeff and Lang, Andrea L. and Lim, Eun-Pa and Marshall, Andrew and Roff, Greg and Schwartz, Chen and Simpson, Isla R. and Son, Seok-Woo and Taguchi, Masakazu},
  year = {2020},
  volume = {125},
  pages = {e2019JD030923},
  issn = {2169-8996},
  doi = {10.1029/2019JD030923},
  abstract = {The stratosphere can have a significant impact on winter surface weather on subseasonal to seasonal (S2S) timescales. This study evaluates the ability of current operational S2S prediction systems to capture two important links between the stratosphere and troposphere: (1) changes in probabilistic prediction skill in the extratropical stratosphere by precursors in the tropics and the extratropical troposphere and (2) changes in surface predictability in the extratropics after stratospheric weak and strong vortex events. Probabilistic skill exists for stratospheric events when including extratropical tropospheric precursors over the North Pacific and Eurasia, though only a limited set of models captures the Eurasian precursors. Tropical teleconnections such as the Madden-Julian Oscillation, the Quasi-Biennial Oscillation, and El Ni\~no\textendash Southern Oscillation increase the probabilistic skill of the polar vortex strength, though these are only captured by a limited set of models. At the surface, predictability is increased over the United States, Russia, and the Middle East for weak vortex events, but not for Europe, and the change in predictability is smaller for strong vortex events for all prediction systems. Prediction systems with poorly resolved stratospheric processes represent this skill to a lesser degree. Altogether, the analyses indicate that correctly simulating stratospheric variability and stratosphere-troposphere dynamical coupling are critical elements for skillful S2S wintertime predictions.},
  copyright = {\textcopyright 2019. American Geophysical Union. All Rights Reserved.},
  file = {/Users/oscardimdore-miles/Zotero/storage/3WDWJMCQ/Domeisen et al. - 2020 - The Role of the Stratosphere in Subseasonal to Sea.pdf;/Users/oscardimdore-miles/Zotero/storage/7NHK3D65/2019JD030923.html},
  journal = {Journal of Geophysical Research: Atmospheres},
  keywords = {North Atlantic Oscillation,S2S database,stratosphere,stratosphere - troposphere coupling,sub-seasonal predictability,sudden stratospheric warming},
  language = {English},
  number = {2}
}

@article{domeisenSeasonal2015,
  title = {Seasonal {{Predictability}} over {{Europe Arising}} from {{El Ni\~no}} and {{Stratospheric Variability}} in the {{MPI}}-{{ESM Seasonal Prediction System}}},
  author = {Domeisen, Daniela I. V. and Butler, Amy H. and Fr{\"o}hlich, Kristina and Bittner, Matthias and M{\"u}ller, Wolfgang A. and Baehr, Johanna},
  year = {2015},
  month = jan,
  volume = {28},
  pages = {256--271},
  publisher = {{American Meteorological Society}},
  issn = {0894-8755, 1520-0442},
  doi = {10.1175/JCLI-D-14-00207.1},
  abstract = {{$<$}section class="abstract"{$><$}h2 class="abstractTitle text-title my-1" id="d9e2"{$>$}Abstract{$<$}/h2{$><$}p{$>$}Predictability on seasonal time scales over the North Atlantic\textendash Europe region is assessed using a seasonal prediction system based on an initialized version of the Max Planck Institute Earth System Model (MPI-ESM). For this region, two of the dominant predictors on seasonal time scales are El Ni\~no\textendash Southern Oscillation (ENSO) and sudden stratospheric warming (SSW) events. Multiple studies have shown a potential for improved North Atlantic predictability for either predictor. Their respective influences are however difficult to disentangle, since the stratosphere is itself impacted by ENSO. Both El Ni\~no and SSW events correspond to a negative signature of the North Atlantic Oscillation (NAO), which has a major influence on European weather.This study explores the impact on Europe by separating the stratospheric pathway of the El Ni\~no teleconnection. In the seasonal prediction system, the evolution of El Ni\~no events is well captured for lead times of up to 6 months, and stratospheric variability is reproduced with a realistic frequency of SSW events. The model reproduces the El Ni\~no teleconnection through the stratosphere, involving a deepened Aleutian low connected to a warm anomaly in the northern winter stratosphere. The stratospheric anomaly signal then propagates downward into the troposphere through the winter season. Predictability of 500-hPa geopotential height over Europe at lead times of up to 4 months is shown to be increased only for El Ni\~no events that exhibit SSW events, and it is shown that the characteristic negative NAO signal is only obtained for winters also containing major SSW events for both the model and the reanalysis data.{$<$}/p{$><$}/section{$>$}},
  chapter = {Journal of Climate},
  file = {/Users/oscardimdore-miles/Zotero/storage/J3EE5ZZY/Domeisen et al. - 2015 - Seasonal Predictability over Europe Arising from E.pdf;/Users/oscardimdore-miles/Zotero/storage/5QS93PJ2/jcli-d-14-00207.1.html},
  journal = {Journal of Climate},
  language = {English},
  number = {1}
}

@article{domeisenTeleconnection2019,
  title = {The {{Teleconnection}} of {{El Ni\~no Southern Oscillation}} to the {{Stratosphere}}},
  author = {Domeisen, Daniela I. V. and Garfinkel, Chaim I. and Butler, Amy H.},
  year = {2019},
  volume = {57},
  pages = {5--47},
  issn = {1944-9208},
  doi = {10.1029/2018RG000596},
  abstract = {El Ni\~no and La Ni\~na events in the tropical Pacific have significant and disrupting impacts on the global atmospheric and oceanic circulation. El Ni\~no Southern Oscillation (ENSO) impacts also extend above the troposphere, affecting the strength and variability of the stratospheric polar vortex in the high latitudes of both hemispheres, as well as the composition and circulation of the tropical stratosphere. El Ni\~no events are associated with a warming and weakening of the polar vortex in the polar stratosphere of both hemispheres, while a cooling can be observed in the tropical lower stratosphere. These impacts are linked by a strengthened Brewer-Dobson circulation. Anomalous upward wave propagation is observed in the extratropics of both hemispheres. For La Ni\~na, these anomalies are often opposite. The stratosphere in turn affects surface weather and climate over large areas of the globe. Since these surface impacts are long-lived, the changes in the stratosphere can lead to improved surface predictions on time scales of weeks to months. Over the past decade, our understanding of the mechanisms through which ENSO can drive impacts remote from the tropical Pacific has improved. This study reviews the possible mechanisms connecting ENSO to the stratosphere in the tropics and the extratropics of both hemispheres while also considering open questions, including nonlinearities in the teleconnections, the role of ENSO diversity, and the impacts of climate change and variability.},
  copyright = {\textcopyright 2018. American Geophysical Union. All Rights Reserved.},
  file = {/Users/oscardimdore-miles/Zotero/storage/B58YNUNX/Domeisen et al. - 2019 - The Teleconnection of El Niño Southern Oscillation.pdf;/Users/oscardimdore-miles/Zotero/storage/4PSM6JYD/2018RG000596.html},
  journal = {Reviews of Geophysics},
  keywords = {El Nino Southern Oscillation,stratosphere,teleconnection},
  language = {English},
  number = {1}
}

@article{domeisenTeleconnection2019a,
  title = {The {{Teleconnection}} of {{El Ni\~no Southern Oscillation}} to the {{Stratosphere}}},
  author = {Domeisen, Daniela I. V. and Garfinkel, Chaim I. and Butler, Amy H.},
  year = {2019},
  volume = {57},
  pages = {5--47},
  issn = {1944-9208},
  doi = {10.1029/2018RG000596},
  abstract = {El Ni\~no and La Ni\~na events in the tropical Pacific have significant and disrupting impacts on the global atmospheric and oceanic circulation. El Ni\~no Southern Oscillation (ENSO) impacts also extend above the troposphere, affecting the strength and variability of the stratospheric polar vortex in the high latitudes of both hemispheres, as well as the composition and circulation of the tropical stratosphere. El Ni\~no events are associated with a warming and weakening of the polar vortex in the polar stratosphere of both hemispheres, while a cooling can be observed in the tropical lower stratosphere. These impacts are linked by a strengthened Brewer-Dobson circulation. Anomalous upward wave propagation is observed in the extratropics of both hemispheres. For La Ni\~na, these anomalies are often opposite. The stratosphere in turn affects surface weather and climate over large areas of the globe. Since these surface impacts are long-lived, the changes in the stratosphere can lead to improved surface predictions on time scales of weeks to months. Over the past decade, our understanding of the mechanisms through which ENSO can drive impacts remote from the tropical Pacific has improved. This study reviews the possible mechanisms connecting ENSO to the stratosphere in the tropics and the extratropics of both hemispheres while also considering open questions, including nonlinearities in the teleconnections, the role of ENSO diversity, and the impacts of climate change and variability.},
  copyright = {\textcopyright 2018. American Geophysical Union. All Rights Reserved.},
  file = {/Users/oscardimdore-miles/Zotero/storage/QJE8T9NA/Domeisen et al. - 2019 - The Teleconnection of El Niño Southern Oscillation.pdf;/Users/oscardimdore-miles/Zotero/storage/WKWC5MDT/2018RG000596.html},
  journal = {Reviews of Geophysics},
  keywords = {El Nino Southern Oscillation,stratosphere,teleconnection},
  language = {English},
  number = {1}
}

@article{Domeison2015,
  title = {Seasonal Predictability over Europe Arising from El Ni\~no and Stratospheric Variability in the {{MPI}}-{{ESM}} Seasonal Prediction System},
  author = {Domeisen, Daniela I. V. and Butler, Amy H. and Fr{\"o}hlich, Kristina and Bittner, Matthias and M{\"u}ller, Wolfgang A. and Baehr, Johanna},
  year = {2014},
  volume = {28},
  pages = {256--271},
  doi = {10.1175/JCLI-D-14-00207.1},
  journal = {Journal of Climate},
  number = {1}
}

@article{Domeison2019,
  title = {The Teleconnection of El Ni\~no Southern Oscillation to the Stratosphere},
  author = {Domeisen, D. and Garfinkel, C.I. and Butler, A.H.},
  year = {2019},
  volume = {57},
  pages = {5--47},
  doi = {10.1029/2018RG000596},
  journal = {Reviews of Geophysics},
  keywords = {El Nino Southern Oscillation,stratosphere,teleconnection},
  number = {1}
}

@article{Domeison2019_hiatus,
  title = {Estimating the Frequency of Sudden Stratospheric Warming Events from Surface Observations of the North Atlantic Oscillation},
  author = {Domeisen, Daniela I.V.},
  year = {2019},
  volume = {124},
  pages = {3180--3194},
  doi = {10.1029/2018JD030077},
  journal = {Journal of Geophysical Research: Atmospheres},
  keywords = {North Atlantic Oscillation,reconstruction,stratosphere-troposphere coupling,sudden stratospheric warming,upper atmosphere},
  number = {6}
}

@article{Domeison2019-1,
  title = {The Role of the Stratosphere in Subseasonal to Seasonal Prediction: 1. {{Predictability}} of the Stratosphere},
  author = {Domeisen, Daniela I.V. and Butler, Amy H. and {Charlton-Perez}, Andrew J. and Ayarzag{\"u}ena, Blanca and Baldwin, Mark P. and {Dunn-Sigouin}, Etienne and Furtado, Jason C. and Garfinkel, Chaim I. and Hitchcock, Peter and Karpechko, Alexey Yu. and Kim, Hera and Knight, Jeff and Lang, Andrea L. and Lim, Eun-Pa and Marshall, Andrew and Roff, Greg and Schwartz, Chen and Simpson, Isla R. and Son, Seok-Woo and Taguchi, Masakazu},
  year = {2020},
  volume = {125},
  pages = {e2019JD030920},
  doi = {10.1029/2019JD030920},
  journal = {Journal of Geophysical Research: Atmospheres},
  keywords = {S2S database,stratosphere,sub-seasonal predictability,sudden stratospheric warming},
  number = {2}
}

@article{Domeison2019-2,
  title = {The Role of the Stratosphere in Subseasonal to Seasonal Prediction: 2. {{Predictability}} Arising from Stratosphere-Troposphere Coupling},
  author = {Domeisen, Daniela I. V. and Butler, Amy H. and {Charlton-Perez}, Andrew J. and Ayarzag{\"u}ena, Blanca and Baldwin, Mark P. and {Dunn-Sigouin}, Etienne and Furtado, Jason C. and Garfinkel, Chaim I. and Hitchcock, Peter and Karpechko, Alexey Yu. and Kim, Hera and Knight, Jeff and Lang, Andrea L. and Lim, Eun-Pa and Marshall, Andrew and Roff, Greg and Schwartz, Chen and Simpson, Isla R. and Son, Seok-Woo and Taguchi, Masakazu},
  year = {2020},
  volume = {125},
  pages = {e2019JD030923},
  doi = {10.1029/2019JD030923},
  journal = {Journal of Geophysical Research: Atmospheres},
  keywords = {North Atlantic Oscillation,S2S database,stratosphere,stratosphere - troposphere coupling,sub-seasonal predictability,sudden stratospheric warming},
  number = {2}
}

@article{Dunkerton2017,
  title = {Nearly Identical Cycles of the Quasi-Biennial Oscillation in the Equatorial Lower Stratosphere},
  author = {Dunkerton, T. J.},
  year = {2017},
  volume = {122},
  pages = {8467--8493},
  doi = {10.1002/2017JD026542},
  journal = {Journal of Geophysical Research: Atmospheres},
  number = {16}
}

@article{ebisuzakiMethod1997,
  title = {A {{Method}} to {{Estimate}} the {{Statistical Significance}} of a {{Correlation When}} the {{Data Are Serially Correlated}}},
  author = {Ebisuzaki, Wesley},
  year = {1997},
  month = sep,
  volume = {10},
  pages = {2147--2153},
  publisher = {{American Meteorological Society}},
  issn = {0894-8755, 1520-0442},
  doi = {10.1175/1520-0442(1997)010<2147:AMTETS>2.0.CO;2},
  abstract = {{$<$}section class="abstract"{$><$}h2 class="abstractTitle text-title my-1" id="d1471e2"{$>$}Abstract{$<$}/h2{$><$}p{$>$}When analyzing pairs of time series, one often needs to know whether a correlation is statistically significant. If the data are Gaussian distributed and not serially correlated, one can use the results of classical statistics to estimate the significance. While some techniques can handle non-Gaussian distributions, few methods are available for data with nonzero autocorrelation (i.e., serially correlated). In this paper, a nonparametric method is suggested to estimate the statistical significance of a computed correlation coefficient when serial correlation is a concern. This method compares favorably with conventional methods.{$<$}/p{$><$}/section{$>$}},
  chapter = {Journal of Climate},
  file = {/Users/oscardimdore-miles/Zotero/storage/2QTR5U6Y/Ebisuzaki - 1997 - A Method to Estimate the Statistical Significance .pdf;/Users/oscardimdore-miles/Zotero/storage/6XSMSBS8/1520-0442_1997_010_2147_amtets_2.0.co_2.html},
  journal = {Journal of Climate},
  language = {English},
  number = {9}
}

@article{edenMechanism2001,
  title = {Mechanism of {{Interannual}} to {{Decadal Variability}} of the {{North Atlantic Circulation}}},
  author = {Eden, Carsten and Willebrand, J{\"u}rgen},
  year = {2001},
  month = may,
  volume = {14},
  pages = {2266--2280},
  publisher = {{American Meteorological Society}},
  issn = {0894-8755, 1520-0442},
  doi = {10.1175/1520-0442(2001)014<2266:MOITDV>2.0.CO;2},
  abstract = {{$<$}section class="abstract"{$><$}h2 class="abstractTitle text-title my-1" id="d492e2"{$>$}Abstract{$<$}/h2{$><$}p{$>$}A model of the Atlantic Ocean was forced with decadal-scale time series of surface fluxes taken from the National Centers for Environmental Prediction\textendash National Center for Atmospheric Research reanalysis. The bulk of the variability of the oceanic circulation is found to be related to the North Atlantic oscillation (NAO). Both realistic experiments and idealized sensitivity studies with the model show a fast (intraseasonal timescale) barotropic response and a delayed (timescale about 6\textendash 8 yr) baroclinic oceanic response to the NAO. The fast response to a high NAO constitutes a barotropic anticyclonic circulation anomaly near the subpolar front with a substantial decrease of the northward heat transport and an increase of northward heat transport in the subtropics due to changes in Ekman transport. The delayed response is an increase in subpolar heat transport due to enhanced meridional overturning and due to a spinup of the subpolar gyre. The corresponding subpolar and subtropical heat content changes could in principle act as an immediate positive feedback and a delayed negative feedback to the NAO.{$<$}/p{$><$}/section{$>$}},
  chapter = {Journal of Climate},
  file = {/Users/oscardimdore-miles/Zotero/storage/9M9WFX2Y/Eden and Willebrand - 2001 - Mechanism of Interannual to Decadal Variability of.pdf;/Users/oscardimdore-miles/Zotero/storage/8I6RWWH6/1520-0442_2001_014_2266_moitdv_2.0.co_2.html},
  journal = {Journal of Climate},
  language = {English},
  number = {10}
}

@article{edenNorth2001,
  title = {North {{Atlantic Interdecadal Variability}}: {{Oceanic Response}} to the {{North Atlantic Oscillation}} (1865\textendash 1997)},
  shorttitle = {North {{Atlantic Interdecadal Variability}}},
  author = {Eden, Carsten and Jung, Thomas},
  year = {2001},
  month = mar,
  volume = {14},
  pages = {676--691},
  publisher = {{American Meteorological Society}},
  issn = {0894-8755, 1520-0442},
  doi = {10.1175/1520-0442(2001)014<0676:NAIVOR>2.0.CO;2},
  abstract = {{$<$}section class="abstract"{$><$}h2 class="abstractTitle text-title my-1" id="d604e2"{$>$}Abstract{$<$}/h2{$><$}p{$>$}In contrast to the atmosphere, knowledge about interdecadal variability of the North Atlantic circulation is relatively restricted. It is the objective of this study to contribute to understanding how the North Atlantic circulation responds to a forcing by the North Atlantic oscillation (NAO) on interdecadal timescales. For this purpose, the authors analyze observed atmospheric and sea surface temperature (SST) data along with the response of an ocean general circulation model to a realistic monthly surface flux forcing that is solely associated with the NAO for the period 1865\textendash 1997.In agreement with previous studies, it is shown that the relationship between the local forcing by the NAO and observed SST anomalies on interdecadal timescales points toward the importance of oceanic dynamics in generating SST anomalies. A comparison between observed and modeled SST anomalies reveals that the model results can be used to assess interdecadal variability of the North Atlantic circulation.The observed/modeled developments of interdecadal SST anomalies during the periods 1915\textendash 39 and 1960\textendash 84 against the local damping influence from the NAO can be traced back to the lagged response (10\textendash 20 yr) of the North Atlantic thermohaline circulation and the subpolar gyre strength to interdecadal variability of the NAO. Additional sensitivity experiments suggest that primarily interdecadal variability in the surface net heat flux forcing associated with the NAO governs interdecadal changes of the North Atlantic circulation.{$<$}/p{$><$}/section{$>$}},
  chapter = {Journal of Climate},
  file = {/Users/oscardimdore-miles/Zotero/storage/53WG45IW/Eden and Jung - 2001 - North Atlantic Interdecadal Variability Oceanic R.pdf;/Users/oscardimdore-miles/Zotero/storage/2QKLCI7V/1520-0442_2001_014_0676_naivor_2.0.co_2.html},
  journal = {Journal of Climate},
  language = {English},
  number = {5}
}

@article{eslerExcitation2005,
  title = {Excitation of {{Transient Rossby Waves}} on the {{Stratospheric Polar Vortex}} and the {{Barotropic Sudden Warming}}},
  author = {Esler, J. G. and Scott, R. K.},
  year = {2005},
  month = oct,
  volume = {62},
  pages = {3661--3682},
  publisher = {{American Meteorological Society}},
  issn = {0022-4928, 1520-0469},
  doi = {10.1175/JAS3557.1},
  abstract = {{$<$}section class="abstract"{$><$}h2 class="abstractTitle text-title my-1" id="d899e2"{$>$}Abstract{$<$}/h2{$><$}p{$>$}The excitation of Rossby waves on the edge of the stratospheric polar vortex, due to time-dependent topographic forcing, is studied analytically and numerically in a simple quasigeostrophic {$<$}em{$>$}f{$<$}/em{$>$}-plane model. When the atmosphere is compressible, the linear response of the vortex is found to have two distinct components. The first is a spectrum of upward-propagating waves that are excited by forcing with temporal frequencies within a fixed ``Charney\textendash Drazin'' range that depends on the angular velocity at the vortex edge and the vortex Burger number. The second component of the response is a barotropic mode, which is excited by forcing with a fixed temporal frequency outside the Charney\textendash Drazin range. The relative magnitude of the two responses, in terms of total angular pseudomomentum, depends on the ratio of the horizontal scale of the forcing to the Rossby radius. Under typical stratospheric conditions the barotropic response is found to dominate. Nonlinear simulations confirm that the linear results remain relevant to understanding the response in cases when strongly nonlinear Rossby wave breaking ensues. It is shown that a sudden warming, or rapid increase in vortex angular pseudomomentum, can be generated at much lower forcing amplitudes when the barotropic mode is resonantly excited compared to when the upward-propagating waves are excited. A numerical simulation of a ``barotropic sudden warming'' due to excitation of the barotropic mode by a relatively weak topographic forcing is described.{$<$}/p{$><$}/section{$>$}},
  chapter = {Journal of the Atmospheric Sciences},
  file = {/Users/oscardimdore-miles/Zotero/storage/Q2XTV5NA/Esler and Scott - 2005 - Excitation of Transient Rossby Waves on the Strato.pdf;/Users/oscardimdore-miles/Zotero/storage/D4YN672Z/jas3557.1.html},
  journal = {Journal of the Atmospheric Sciences},
  language = {English},
  number = {10}
}

@article{eslerStratospheric2011a,
  title = {Stratospheric {{Sudden Warmings}} as {{Self}}-{{Tuning Resonances}}. {{Part II}}: {{Vortex Displacement Events}}},
  shorttitle = {Stratospheric {{Sudden Warmings}} as {{Self}}-{{Tuning Resonances}}. {{Part II}}},
  author = {Esler, J. G. and Matthewman, N. Joss},
  year = {2011},
  month = nov,
  volume = {68},
  pages = {2505--2523},
  publisher = {{American Meteorological Society}},
  issn = {0022-4928, 1520-0469},
  doi = {10.1175/JAS-D-11-08.1},
  abstract = {{$<$}section class="abstract"{$><$}h2 class="abstractTitle text-title my-1" id="d886e2"{$>$}Abstract{$<$}/h2{$><$}p{$>$}Vortex displacement stratospheric sudden warmings (SSWs) are studied in an idealized model of a quasigeostrophic columnar vortex in an anelastic atmosphere. Motivated by the fact that observed events occur at a fixed orientation to the earth's surface and have a strongly baroclinic vertical structure, vortex Rossby waves are forced by a stationary topographic forcing designed to minimize excursions of the vortex from its initial position. Variations in the background stratospheric ``climate'' are represented by means of an additional flow in solid body rotation. The vortex response is determined numerically as a function of the forcing strength {$<$}em{$>$}M{$<$}/em{$>$} and the background flow strength {$\Omega$}.At moderate {$<$}em{$>$}M{$<$}/em{$>$} it is found that a large response, with many features resembling observed displacement SSWs, occurs only for a narrow range of {$\Omega$}. Linear analysis reveals that for this range of {$\Omega$} the first baroclinic azimuthal wave-1 Rossby wave mode is close to being resonantly excited. A forced nonlinear oscillator equation is proposed to describe the nonlinear behavior, and a method for determining the relevant coefficients numerically, using unforced calculations of steadily propagating vortex ``V states,'' is adopted. The nonlinear equation predicts some qualitative details of the variation in the response at finite {$<$}em{$>$}M{$<$}/em{$>$}. However, it is concluded that strongly nonlinear processes, such as wave breaking and filament formation, are necessarily quantitatively important in determining the amplitude of the near-resonant response at finite {$<$}em{$>$}M{$<$}/em{$>$}.{$<$}/p{$><$}/section{$>$}},
  chapter = {Journal of the Atmospheric Sciences},
  file = {/Users/oscardimdore-miles/Zotero/storage/8KB22JNW/Esler and Matthewman - 2011 - Stratospheric Sudden Warmings as Self-Tuning Reson.pdf;/Users/oscardimdore-miles/Zotero/storage/IREHQ2Y3/jas-d-11-08.1.html},
  journal = {Journal of the Atmospheric Sciences},
  language = {English},
  number = {11}
}

@article{Fletcher,
  title = {The Role of Linear Interference in the Annular Mode Response to Tropical {{SST}} Forcing},
  author = {Fletcher, Christopher G. and Kushner, Paul J.},
  year = {2011},
  volume = {24},
  pages = {778--794},
  publisher = {{American Meteorological Society}},
  journal = {Journal of Climate},
  number = {3}
}

@article{Fletcher_model,
  title = {Linear Interference and the {{Northern Annular Mode}} Response to Tropical {{SST}} Forcing: {{Sensitivity}} to Model Configuration},
  author = {Fletcher, Christopher G. and Kushner, Paul J.},
  year = {2013},
  volume = {118},
  pages = {4267--4279},
  doi = {10.1002/jgrd.50385},
  journal = {Journal of Geophysical Research: Atmospheres},
  keywords = {el nino,northern annular mode,planetary wave,teleconnections,tropical sea surface temperature,winter},
  number = {10}
}

@article{fletcherLinear2013,
  title = {Linear Interference and the {{Northern Annular Mode}} Response to Tropical {{SST}} Forcing: {{Sensitivity}} to Model Configuration},
  shorttitle = {Linear Interference and the {{Northern Annular Mode}} Response to Tropical {{SST}} Forcing},
  author = {Fletcher, Christopher G. and Kushner, Paul J.},
  year = {2013},
  volume = {118},
  pages = {4267--4279},
  issn = {2169-8996},
  doi = {10.1002/jgrd.50385},
  abstract = {AbstractInterannual variability in tropical sea surface temperatures (SSTs) associated with the El Ni\~no\textendash Southern Oscillation is linked to teleconnections with the Northern Annular Mode (NAM). Previous work highlighted that the sign and amplitude of the NAM response to tropical SSTs are controlled by the total wave activity entering the subpolar stratosphere, which depends on the linear interference of planetary wave anomalies with the climatological stationary wave field. This study uses multiple configurations of atmospheric general circulation models to assess the robustness of these linkages to details of the tropical SST forcing and model configuration. Across 23 cases with idealized SST forcing, the amplitudes of the tropical and extratropical wave responses are found to scale approximately linearly with forcing strength. But wave amplitude alone is not sufficient to predict the NAM response. Instead, the spatial structure of the wave response (and hence the linear interference) provides the best explanation of the NAM response in all cases. Linear interference explains most of the total wave activity response even in cases with stronger nonlinear contributions, due to consistent cancellation between quasi-stationary wave nonlinearity and nonlinearity arising from transient waves. Within this limited set of experiments, there is no evidence for a consistent sensitivity of the NAM response to horizontal resolution or to vertical resolution in the stratosphere. These findings reveal that linear interference provides a robust and reproducible mechanism linking midlatitude wave responses to zonal mean circulation (NAM) responses across a wide variety of forcing cases.},
  copyright = {\textcopyright 2013. American Geophysical Union. All Rights Reserved.},
  file = {/Users/oscardimdore-miles/Zotero/storage/WBN7VRRC/Fletcher and Kushner - 2013 - Linear interference and the Northern Annular Mode .pdf;/Users/oscardimdore-miles/Zotero/storage/U2A7SCVT/jgrd.html},
  journal = {Journal of Geophysical Research: Atmospheres},
  keywords = {el nino,northern annular mode,planetary wave,teleconnections,tropical sea surface temperature,winter},
  language = {English},
  number = {10}
}

@article{fletcherRole2011,
  title = {The {{Role}} of {{Linear Interference}} in the {{Annular Mode Response}} to {{Tropical SST Forcing}}},
  author = {Fletcher, Christopher G. and Kushner, Paul J.},
  year = {2011},
  month = feb,
  volume = {24},
  pages = {778--794},
  publisher = {{American Meteorological Society}},
  issn = {0894-8755, 1520-0442},
  doi = {10.1175/2010JCLI3735.1},
  abstract = {{$<$}section class="abstract"{$><$}h2 class="abstractTitle text-title my-1" id="d5e2"{$>$}Abstract{$<$}/h2{$><$}p{$>$}Recent observational and modeling studies have demonstrated a link between eastern tropical Pacific Ocean (TPO) warming associated with the El Ni\~no\textendash Southern Oscillation (ENSO) and the negative phase of the wintertime northern annular mode (NAM). The TPO\textendash NAM link involves a Rossby wave teleconnection from the tropics to the extratropics, and an increase in polar stratospheric wave driving that in turn induces a negative NAM anomaly in the stratosphere and troposphere. Previous work further suggests that tropical Indian Ocean (TIO) warming is associated with a positive NAM anomaly, which is of opposite sign to the TPO case. The TIO case is, however, difficult to interpret because the TPO and TIO warmings are not independent. To better understand the dynamics of tropical influences on the NAM, the current study investigates the NAM response to imposed TPO and TIO warmings in a general circulation model. The NAM responses to the two warmings have opposite sign and can be of surprisingly similar amplitude even though the TIO forcing is relatively weak. It is shown that the sign and strength of the NAM response is often simply related to the phasing, and hence the linear interference, between the Rossby wave response and the climatological stationary wave. The TPO (TIO) wave response reinforces (attenuates) the climatological wave and therefore weakens (strengthens) the stratospheric jet and leads to a negative (positive) NAM response. In additional simulations, it is shown that decreasing the strength of the climatological stationary wave reduces the importance of linear interference and increases the importance of nonlinearity. This work demonstrates that the simulated extratropical annular mode response to climate forcings can depend sensitively on the amplitude and phase of the climatological stationary wave and the wave response.{$<$}/p{$><$}/section{$>$}},
  chapter = {Journal of Climate},
  file = {/Users/oscardimdore-miles/Zotero/storage/M53KPEMX/Fletcher and Kushner - 2011 - The Role of Linear Interference in the Annular Mod.pdf;/Users/oscardimdore-miles/Zotero/storage/7LSW3X8Y/2010jcli3735.1.html},
  journal = {Journal of Climate},
  language = {English},
  number = {3}
}

@article{Fraedrih1993,
  title = {An {{EOF}} Analysis of the Vertical-Time Delay Structure of the Quasi-Biennial Oscillation.},
  author = {Fraedrich, Klaus and Pawson, Steven and Wang, Risheng},
  year = {1993},
  volume = {50},
  pages = {3357--3365},
  doi = {10.1175/1520-0469},
  journal = {Journal of Atmospheric Sciences},
  number = {20}
}

@article{frankcombeNorth2010,
  title = {North {{Atlantic Multidecadal Climate Variability}}: {{An Investigation}} of {{Dominant Time Scales}} and {{Processes}}},
  shorttitle = {North {{Atlantic Multidecadal Climate Variability}}},
  author = {Frankcombe, Leela M. and {von der Heydt}, Anna and Dijkstra, Henk A.},
  year = {2010},
  month = jul,
  volume = {23},
  pages = {3626--3638},
  publisher = {{American Meteorological Society}},
  issn = {0894-8755, 1520-0442},
  doi = {10.1175/2010JCLI3471.1},
  abstract = {{$<$}section class="abstract"{$><$}h2 class="abstractTitle text-title my-1" id="d2272e2"{$>$}Abstract{$<$}/h2{$><$}p{$>$}The issue of multidecadal variability in the North Atlantic has been an important topic of late. It is clear that there are multidecadal variations in several climate variables in the North Atlantic, such as sea surface temperature and sea level height. The details of this variability, in particular the dominant patterns and time scales, are confusing from both an observational as well as a theoretical point of view. After analyzing results from observational datasets and a 500-yr simulation of an Intergovernmental Panel on Climate Change (IPCC) Fourth Assessment Report (AR4) climate model, two dominant time scales (20\textendash 30 and 50\textendash 70 yr) of multidecadal variability in the North Atlantic are proposed. The 20\textendash 30-yr variability is characterized by the westward propagation of subsurface temperature anomalies. The hypothesis is that the 20\textendash 30-yr variability is caused by internal variability of the Atlantic Meridional Overturning Circulation (MOC) while the 50\textendash 70-yr variability is related to atmospheric forcing over the Atlantic Ocean and exchange processes between the Atlantic and Arctic Oceans.{$<$}/p{$><$}/section{$>$}},
  chapter = {Journal of Climate},
  file = {/Users/oscardimdore-miles/Zotero/storage/QTU2GXKT/Frankcombe et al. - 2010 - North Atlantic Multidecadal Climate Variability A.pdf;/Users/oscardimdore-miles/Zotero/storage/V9AZMFDU/2010jcli3471.1.html},
  journal = {Journal of Climate},
  language = {English},
  number = {13}
}

@article{frankignoulInfluence2013,
  title = {The {{Influence}} of the {{AMOC Variability}} on the {{Atmosphere}} in {{CCSM3}}},
  author = {Frankignoul, Claude and Gastineau, Guillaume and Kwon, Young-Oh},
  year = {2013},
  month = dec,
  volume = {26},
  pages = {9774--9790},
  publisher = {{American Meteorological Society}},
  issn = {0894-8755, 1520-0442},
  doi = {10.1175/JCLI-D-12-00862.1},
  abstract = {{$<$}section class="abstract"{$><$}h2 class="abstractTitle text-title my-1" id="d306e2"{$>$}Abstract{$<$}/h2{$><$}p{$>$}The influence of the Atlantic meridional overturning circulation (AMOC) variability on the atmospheric circulation is investigated in a control simulation of the NCAR Community Climate System Model, version 3 (CCSM3), where the AMOC evolves from an oscillatory regime into a red noise regime. In the latter, an AMOC intensification is followed during winter by a positive North Atlantic Oscillation (NAO). The atmospheric response is robust and controlled by AMOC-driven SST anomalies, which shift the heat release to the atmosphere northward near the Gulf Stream/North Atlantic Current. This alters the low-level atmospheric baroclinicity and shifts the maximum eddy growth northward, affecting the storm track and favoring a positive NAO. The AMOC influence is detected in the relation between seasonal upper-ocean heat content or SST anomalies and winter sea level pressure. In the oscillatory regime, no direct AMOC influence is detected in winter. However, an upper-ocean heat content anomaly resembling the AMOC footprint precedes a negative NAO. This opposite NAO polarity seems due to the southward shift of the Gulf Stream during AMOC intensification, displacing the maximum baroclinicity southward near the jet exit. As the mode has somewhat different patterns when using SST, the wintertime impact of the AMOC lacks robustness in this regime. However, none of the signals compares well with the observed influence of North Atlantic SST anomalies on the NAO because SST is dominated in CCSM3 by the meridional shifts of the Gulf Stream/North Atlantic Current that covary with the AMOC. Hence, although there is some potential climate predictability in CCSM3, it is not realistic.{$<$}/p{$><$}/section{$>$}},
  chapter = {Journal of Climate},
  file = {/Users/oscardimdore-miles/Zotero/storage/BE56G3AN/Frankignoul et al. - 2013 - The Influence of the AMOC Variability on the Atmos.pdf;/Users/oscardimdore-miles/Zotero/storage/5PGRXXE2/jcli-d-12-00862.1.html},
  journal = {Journal of Climate},
  language = {English},
  number = {24}
}

@article{friersonContribution2013,
  title = {Contribution of Ocean Overturning Circulation to Tropical Rainfall Peak in the {{Northern Hemisphere}}},
  author = {Frierson, Dargan M. W. and Hwang, Yen-Ting and Fu{\v c}kar, Neven S. and Seager, Richard and Kang, Sarah M. and Donohoe, Aaron and Maroon, Elizabeth A. and Liu, Xiaojuan and Battisti, David S.},
  year = {2013},
  month = nov,
  volume = {6},
  pages = {940--944},
  publisher = {{Nature Publishing Group}},
  issn = {1752-0908},
  doi = {10.1038/ngeo1987},
  abstract = {In the tropics, substantially more rain falls just north of the Equator. An analysis of satellite observations, reanalysis data and model simulations suggests that the meridional ocean overturning circulation contributes significantly to the tropical rainfall peak north of the Equator.},
  copyright = {2013 Nature Publishing Group},
  file = {/Users/oscardimdore-miles/Zotero/storage/DNUC6GHA/Frierson et al. - 2013 - Contribution of ocean overturning circulation to t.pdf;/Users/oscardimdore-miles/Zotero/storage/89R7UBY6/ngeo1987.html},
  journal = {Nature Geoscience},
  language = {English},
  number = {11}
}

@article{garcia-herreraPropagation2006,
  title = {Propagation of {{ENSO}} Temperature Signals into the Middle Atmosphere: {{A}} Comparison of Two General Circulation Models and {{ERA}}-40 Reanalysis Data},
  shorttitle = {Propagation of {{ENSO}} Temperature Signals into the Middle Atmosphere},
  author = {{Garc{\'i}a-Herrera}, R. and Calvo, N. and Garcia, R. R. and Giorgetta, M. A.},
  year = {2006},
  volume = {111},
  issn = {2156-2202},
  doi = {10.1029/2005JD006061},
  abstract = {The vertical propagation of the El Ni\~no\textendash Southern Oscillation (ENSO) temperature signal has been analyzed in two general circulation models, the Whole Atmosphere Community Climate Model and the Middle Atmosphere European Center\textendash Hamburg Model, and in the ERA-40 reanalysis data set. Monthly mean data have been used, and composite differences (El Ni\~no\textendash La Ni\~na) have been computed. Our results show that the ENSO signal propagates into the middle atmosphere by means of planetary Rossby waves. Significant wave-like anomalies are observed up to around 40 km. This propagation is strongly influenced by the zonal mean zonal winds, being most effective in midlatitudes of the Northern Hemisphere because ENSO events tend to peak in northern winter, when stratospheric winds are westerly in the Northern Hemisphere, and allow vertical propagation of Rossby waves. In addition, zonal mean temperature anomalies are observed in the middle atmosphere in the tropics and at polar latitudes of the Northern Hemisphere. These anomalies are the result of changes in the residual mean meridional circulation: Our analysis reveals that during an El Ni\~no event, vertical wave propagation and divergence of Eliassen-Palm flux are enhanced, forcing a stronger residual circulation in the stratosphere, which cools the tropics and warms the higher latitudes. This pattern is highly significant in the models during certain months but much less in the ERA-40 data, where other sources of variability (in particular the quasi-biennial oscillation) also influence the residual circulation.},
  copyright = {Copyright 2006 by the American Geophysical Union.},
  file = {/Users/oscardimdore-miles/Zotero/storage/ARNVRND9/García‐Herrera et al. - 2006 - Propagation of ENSO temperature signals into the m.pdf;/Users/oscardimdore-miles/Zotero/storage/YQMHB7KR/2005JD006061.html},
  journal = {Journal of Geophysical Research: Atmospheres},
  keywords = {ENSO,general circulation models,middle atmosphere},
  language = {English},
  number = {D6}
}

@article{garfinkel20,
  title = {Predictability of the Early Winter {{Arctic}} Oscillation from Autumn {{Eurasian}} Snowcover in Subseasonal Forecast Models},
  author = {Garfinkel, Chaim and Schwartz, Chen and White, Ian and Rao, Jian},
  year = {2020},
  month = aug,
  volume = {55},
  doi = {10.1007/s00382-020-05305-3},
  journal = {Climate Dynamics}
}

@article{Garfinkel2008,
  title = {Different {{ENSO}} Teleconnections and Their Effects on the Stratospheric Polar Vortex},
  author = {Garfinkel, C. I. and Hartmann, D. L.},
  year = {2008},
  volume = {113},
  doi = {10.1029/2008JD009920},
  journal = {Journal of Geophysical Research: Atmospheres},
  keywords = {ENSO,polar vortex,troposphere-stratsophere interactions},
  number = {D18}
}

@article{Garfinkel2010,
  title = {Tropospheric Precursors of Anomalous Northern Hemisphere Stratospheric Polar Vortices},
  author = {Garfinkel, Chaim and Hartmann, Dennis and Sassi, Fabrizio},
  year = {2010},
  volume = {23},
  pages = {3282--3299},
  doi = {10.1175/2010JCLI3010.1},
  journal = {Journal of Climate - J CLIMATE},
  number = {12}
}

@article{Garfinkel2012whymight,
  title = {Why Might Stratospheric Sudden Warmings Occur with Similar Frequency in {{El Ni\~no}} and {{La Ni\~na}} Winters?},
  author = {Garfinkel, Chaim and Butler, Amy and Waugh, D. and Hurwitz, M. and Polvani, Lorenzo},
  year = {2012},
  month = oct,
  volume = {117},
  pages = {19106-},
  doi = {10.1029/2012JD017777},
  journal = {Journal of Geophysical Research (Atmospheres)}
}

@article{Garfinkel2015,
  title = {Effect of Recent Sea Surface Temperature Trends on the {{Arctic}} Stratospheric Vortex},
  author = {Garfinkel, C. I. and Hurwitz, M. M. and Oman, L. D.},
  year = {2015},
  volume = {120},
  pages = {5404--5416},
  journal = {Journal of Geophysical Research: Atmospheres},
  number = {11}
}

@article{Garfinkel2017,
  title = {Stratospheric Variability Contributed to and Sustained the Recent Hiatus in {{Eurasian}} Winter Warming},
  author = {Garfinkel, Chaim I. and Son, Seok-Woo and Song, Kanghyun and Aquila, Valentina and Oman, Luke D.},
  year = {2017},
  volume = {44},
  pages = {374--382},
  doi = {10.1002/2016GL072035},
  journal = {Geophysical Research Letters},
  keywords = {Eurasian temperature,hiatus,stratosphere-troposphere coupling},
  number = {1}
}

@article{Garfinkel2018,
  title = {Extratropical Atmospheric Predictability from the Quasi-Biennial Oscillation in Subseasonal Forecast Models},
  author = {Garfinkel, Chaim I. and Schwartz, Chen and Domeisen, Daniela I. V. and Son, Seok-Woo and Butler, Amy H. and White, Ian P.},
  year = {2018},
  volume = {123},
  pages = {7855--7866},
  journal = {Journal of Geophysical Research: Atmospheres},
  keywords = {annular modes,month-ahead prediction,polar vortex,Quasi-Biennial Oscillation},
  number = {15}
}

@article{garfinkelDifferent2008,
  title = {Different {{ENSO}} Teleconnections and Their Effects on the Stratospheric Polar Vortex},
  author = {Garfinkel, C. I. and Hartmann, D. L.},
  year = {2008},
  volume = {113},
  issn = {2156-2202},
  doi = {10.1029/2008JD009920},
  abstract = {Reanalysis data are used to study the El Ni\~no\textendash Southern Oscillation (ENSO) signal in the troposphere and stratosphere during the late fall to midwinter period. Warm ENSO events have extratropical tropospheric teleconnections that increase the wave 1 eddies and reduce the wave 2 eddies, as compared to cold ENSO. The increase in wave 1 overwhelms the decrease in wave 2, so the net effect is a weakened vortex. This modification in tropospheric wave forcing is induced by a deepening of the wintertime Aleutian low via the Pacific\textendash North America pattern (PNA). Model results are also used to verify that the PNA is the primary mechanism through which ENSO modulates the vortex. During easterly Quasi-Biennial Oscillation (EQBO), warm ENSO does not show a PNA response in the observational record. Consequently, the polar vortex does not show a strong response to the different phases of ENSO under EQBO, nor to the different phases of QBO under WENSO. It is not clear whether the lack of a PNA response to warm ENSO during EQBO is a real physical phenomenon or a feature of the limited data record we have.},
  copyright = {Copyright 2008 by the American Geophysical Union.},
  file = {/Users/oscardimdore-miles/Zotero/storage/DAVZ3YIT/Garfinkel and Hartmann - 2008 - Different ENSO teleconnections and their effects o.pdf;/Users/oscardimdore-miles/Zotero/storage/A59CUU47/2008JD009920.html},
  journal = {Journal of Geophysical Research: Atmospheres},
  keywords = {ENSO,polar vortex,troposphere-stratsophere interactions},
  language = {English},
  number = {D18}
}

@article{garfinkelEffect2015,
  title = {Effect of Recent Sea Surface Temperature Trends on the {{Arctic}} Stratospheric Vortex},
  author = {Garfinkel, C. I. and Hurwitz, M. M. and Oman, L. D.},
  year = {2015},
  volume = {120},
  pages = {5404--5416},
  issn = {2169-8996},
  doi = {10.1002/2015JD023284},
  abstract = {Comprehensive chemistry-climate model experiments and observational data are used to show that up to half of the satellite era early springtime cooling trend in the Arctic lower stratosphere was caused by changing sea surface temperatures (SSTs). An ensemble of experiments forced only by changing SSTs is compared to an ensemble of experiments in which both the observed SSTs and chemically and radiatively active trace species are changing. By comparing the two ensembles, it is shown that warming of Indian Ocean, North Pacific, and North Atlantic SSTs and cooling of the tropical Pacific have strongly contributed to recent polar stratospheric cooling in late winter and early spring. When concentrations of ozone-depleting substances and greenhouse gases are fixed, polar ozone concentrations show a small but robust decline due to changing SSTs. Ozone loss is larger in the presence of changing concentrations of ozone-depleting substances and greenhouse gases. The stratospheric changes can be understood by examining the tropospheric height and heat flux anomalies generated by the anomalous SSTs. Finally, recent SST changes have contributed to a decrease in the frequency of late winter stratospheric sudden warmings.},
  copyright = {\textcopyright 2015. American Geophysical Union. All Rights Reserved.},
  file = {/Users/oscardimdore-miles/Zotero/storage/BLF26MZX/Garfinkel et al. - 2015 - Effect of recent sea surface temperature trends on.pdf;/Users/oscardimdore-miles/Zotero/storage/3D4UQFWN/2015JD023284.html},
  journal = {Journal of Geophysical Research: Atmospheres},
  keywords = {Arctic stratosphere,ozone loss,ozone trends,SST trends},
  language = {English},
  number = {11}
}

@article{garfinkelStratospheric2017,
  title = {Stratospheric Variability Contributed to and Sustained the Recent Hiatus in {{Eurasian}} Winter Warming},
  author = {Garfinkel, Chaim I. and Son, Seok-Woo and Song, Kanghyun and Aquila, Valentina and Oman, Luke D.},
  year = {2017},
  volume = {44},
  pages = {374--382},
  issn = {1944-8007},
  doi = {10.1002/2016GL072035},
  abstract = {The recent hiatus in global-mean surface temperature warming was characterized by a Eurasian winter cooling trend, and the cause(s) for this cooling is unclear. Here we show that the observed hiatus in Eurasian warming was associated with a recent trend toward weakened stratospheric polar vortices. Specifically, by calculating the change in Eurasian surface air temperature associated with a given vortex weakening, we demonstrate that the recent trend toward weakened polar vortices reduced the anticipated Eurasian warming due to increasing greenhouse gas concentrations. Those model integrations whose stratospheric vortex evolution most closely matches that in reanalysis data also simulate a hiatus. While it is unclear whether the recent weakening of the midwinter stratospheric polar vortex was forced, a properly configured model can simulate substantial deviations of the polar vortex on decadal timescales and hence such hiatus events, implying that similar hiatus events may recur even as greenhouse gas concentrations rise.},
  copyright = {\textcopyright 2017. The Authors.},
  file = {/Users/oscardimdore-miles/Zotero/storage/WKG8LHC9/Garfinkel et al. - 2017 - Stratospheric variability contributed to and susta.pdf;/Users/oscardimdore-miles/Zotero/storage/7DSXVMQW/2016GL072035.html},
  journal = {Geophysical Research Letters},
  keywords = {Eurasian temperature,hiatus,stratosphere-troposphere coupling},
  language = {English},
  number = {1}
}

@article{garfinkelTropospheric2010,
  title = {Tropospheric {{Precursors}} of {{Anomalous Northern Hemisphere Stratospheric Polar Vortices}}},
  author = {Garfinkel, Chaim I. and Hartmann, Dennis L. and Sassi, Fabrizio},
  year = {2010},
  month = jun,
  volume = {23},
  pages = {3282--3299},
  publisher = {{American Meteorological Society}},
  issn = {0894-8755, 1520-0442},
  doi = {10.1175/2010JCLI3010.1},
  abstract = {{$<$}section class="abstract"{$><$}h2 class="abstractTitle text-title my-1" id="d62652984e80"{$>$}Abstract{$<$}/h2{$><$}p{$>$}Regional extratropical tropospheric variability in the North Pacific and eastern Europe is well correlated with variability in the Northern Hemisphere wintertime stratospheric polar vortex in both the ECMWF reanalysis record and in the Whole Atmosphere Community Climate Model. To explain this correlation, the link between stratospheric vertical Eliassen\textendash Palm flux variability and tropospheric variability is analyzed. Simple reasoning shows that variability in the North Pacific and eastern Europe can deepen or flatten the wintertime tropospheric stationary waves, and in particular its wavenumber-1 and -2 components, thus providing a physical explanation for the correlation between these regions and vortex weakening. These two pathways begin to weaken the upper stratospheric vortex nearly immediately, with a peak influence apparent after a lag of some 20 days. The influence then appears to propagate downward in time, as expected from wave\textendash mean flow interaction theory. These patterns are influenced by ENSO and October Eurasian snow cover. Perturbations in the vortex induced by the two regions add linearly. These two patterns and the quasi-biennial oscillation (QBO) are linearly related to 40\% of polar vortex variability during winter in the reanalysis record.{$<$}/p{$><$}/section{$>$}},
  chapter = {Journal of Climate},
  file = {/Users/oscardimdore-miles/Zotero/storage/2YNH2C38/Garfinkel et al. - 2010 - Tropospheric Precursors of Anomalous Northern Hemi.pdf;/Users/oscardimdore-miles/Zotero/storage/G7A7LB3S/2010jcli3010.1.html},
  journal = {Journal of Climate},
  language = {English},
  number = {12}
}

@article{garfinkelWhy2012,
  title = {Why Might Stratospheric Sudden Warmings Occur with Similar Frequency in {{El Ni\~no}} and {{La Ni\~na}} Winters?},
  author = {Garfinkel, C. I. and Butler, A. H. and Waugh, D. W. and Hurwitz, M. M. and Polvani, L. M.},
  year = {2012},
  volume = {117},
  issn = {2156-2202},
  doi = {10.1029/2012JD017777},
  abstract = {The effect of El Ni\~no-Southern Oscillation (ENSO) on the frequency and character of Northern Hemisphere major mid-winter stratospheric sudden warmings (SSWs) is evaluated using a meteorological reanalysis data set and comprehensive chemistry-climate models. There is an apparent inconsistency between the impact of opposite phases of ENSO on the seasonal mean vortex and on SSWs: El Ni\~no leads to an anomalously warm, and La Ni\~na leads to an anomalously cool, seasonal mean polar stratospheric state, but both phases of ENSO lead to an increased SSW frequency. A resolution to this apparent paradox is here proposed: the region in the North Pacific most strongly associated with precursors of SSWs is not strongly influenced by El Ni\~no and La Ni\~na teleconnections. In the observational record, both La Ni\~na and El Ni\~no lead to similar anomalies in the region associated with precursors of SSWs and, consistent with this, there is a similar SSW frequency in La Ni\~na and El Ni\~no winters. A similar correspondence between the penetration of ENSO teleconnections into the SSW precursor region and SSW frequency is found in the comprehensive chemistry-climate models. The inability of some of the models to capture the observed relationship between La Ni\~na and SSW frequency appears related to whether the modeled ENSO teleconnections result in extreme anomalies in the region most closely associated with SSWs. Finally, it is confirmed that the seasonal mean polar vortex response to ENSO is only weakly related to the relative frequency of SSWs during El Ni\~no and La Ni\~na.},
  copyright = {\textcopyright 2012. American Geophysical Union. All Rights Reserved.},
  file = {/Users/oscardimdore-miles/Zotero/storage/8JEHJWWG/Garfinkel et al. - 2012 - Why might stratospheric sudden warmings occur with.pdf;/Users/oscardimdore-miles/Zotero/storage/N2JY8L2S/2012JD017777.html},
  journal = {Journal of Geophysical Research: Atmospheres},
  keywords = {ENSO,middle atmosphere,SSW,teleconnections},
  language = {English},
  number = {D19}
}

@article{gerberAnnular2008,
  title = {Annular Mode Time Scales in the {{Intergovernmental Panel}} on {{Climate Change Fourth Assessment Report}} Models},
  author = {Gerber, E. P. and Polvani, L. M. and Ancukiewicz, D.},
  year = {2008},
  volume = {35},
  issn = {1944-8007},
  doi = {10.1029/2008GL035712},
  abstract = {The ability of climate models in the Intergovernmental Panel on Climate Change Fourth Assessment Report to capture the temporal structure of the annular modes is evaluated. The vertical structure and annual cycle of the variability is quantified by the e-folding time scale of the annular mode autocorrelation function. Models vaguely capture the qualitative features of the Northern and Southern Annular Modes: Northern Hemisphere time scales are shorter than those of the Southern Hemisphere and peak in boreal winter, while Southern Hemisphere time scales peak in austral spring and summer. Models, however, systematically overestimate the time scales, particularly in the Southern Hemisphere summer, where the multimodel ensemble average is twice that of reanalyses. Fluctuation-dissipation theory suggests that long time scales in models could be associated with increased sensitivity to anthropogenic forcing. Comparison of model pairs with similar forcings but different annular mode time scales provides a hint of a fluctuation-dissipation relationship.},
  copyright = {Copyright 2008 by the American Geophysical Union.},
  file = {/Users/oscardimdore-miles/Zotero/storage/3GQ9BYJS/Gerber et al. - 2008 - Annular mode time scales in the Intergovernmental .pdf;/Users/oscardimdore-miles/Zotero/storage/ESYAHIVE/2008GL035712.html},
  journal = {Geophysical Research Letters},
  keywords = {annular modes,coupled climate modes,time scales},
  language = {English},
  number = {22}
}

@article{gerberStratospheric2009,
  title = {Stratospheric Influence on the Tropospheric Circulation Revealed by Idealized Ensemble Forecasts},
  author = {Gerber, E. P. and Orbe, C. and Polvani, L. M.},
  year = {2009},
  volume = {36},
  issn = {1944-8007},
  doi = {10.1029/2009GL040913},
  abstract = {The coupling between the stratosphere and troposphere following Stratospheric Sudden Warming (SSW) events is investigated in an idealized atmospheric General Circulation Model, with focus on the influence of stratospheric memory on the troposphere. Ensemble forecasts are performed to confirm the role of the stratosphere in the observed equatorward shift of the tropospheric midlatitude jet following an SSW. It is demonstrated that the tropospheric response to the weakening of the lower stratospheric vortex is robust, but weak in amplitude and thus easily masked by tropospheric variability. The amplitude of the response in the troposphere is crucially sensitive to the depth of the SSW. The persistence of the response in the troposphere is attributed to both the increased predictability of the stratosphere following an SSW, and the dynamical coupling between the tropospheric jet and lower stratosphere. These results suggest value in resolving the stratosphere and assimilating upper atmospheric data in forecast models.},
  copyright = {Copyright 2009 by the American Geophysical Union.},
  file = {/Users/oscardimdore-miles/Zotero/storage/WY5QZ5KY/Gerber et al. - 2009 - Stratospheric influence on the tropospheric circul.pdf;/Users/oscardimdore-miles/Zotero/storage/5Y64KIYF/2009GL040913.html},
  journal = {Geophysical Research Letters},
  keywords = {Northern Annular Mode,stratosphere-troposphere coupling,stratospheric sudden warming},
  language = {English},
  number = {24}
}

@article{gerberTesting2008,
  title = {Testing the {{Annular Mode Autocorrelation Time Scale}} in {{Simple Atmospheric General Circulation Models}}},
  author = {Gerber, Edwin P. and Voronin, Sergey and Polvani, Lorenzo M.},
  year = {2008},
  month = apr,
  volume = {136},
  pages = {1523--1536},
  publisher = {{American Meteorological Society}},
  issn = {1520-0493, 0027-0644},
  doi = {10.1175/2007MWR2211.1},
  abstract = {{$<$}section class="abstract"{$><$}h2 class="abstractTitle text-title my-1" id="d307e2"{$>$}Abstract{$<$}/h2{$><$}p{$>$}A new diagnostic for measuring the ability of atmospheric models to reproduce realistic low-frequency variability is introduced in the context of Held and Suarez's 1994 proposal for comparing the dynamics of different general circulation models. A simple procedure to compute {$<$}em{$>\tau<$}/em{$>$}, the {$<$}em{$>$}e{$<$}/em{$>$}-folding time scale of the annular mode autocorrelation function, is presented. This quantity concisely quantifies the strength of low-frequency variability in a model and is easy to compute in practice. The sensitivity of {$<$}em{$>\tau<$}/em{$>$} to model numerics is then studied for two dry primitive equation models driven with the Held\textendash Suarez forcings: one pseudospectral and the other finite volume. For both models, {$<$}em{$>\tau<$}/em{$>$} is found to be unrealistically large when the horizontal resolutions are low, such as those that are often used in studies in which long integrations are needed to analyze model variability on low frequencies. More surprising is that it is found that, for the pseudospectral model, {$<$}em{$>\tau<$}/em{$>$} is particularly sensitive to vertical resolution, especially with a triangular truncation at wavenumber 42 (a very common resolution choice). At sufficiently high resolution, the annular mode autocorrelation time scale {$<$}em{$>\tau<$}/em{$>$} in both models appears to converge around values of 20\textendash 25 days, suggesting the existence of an intrinsic time scale at which the extratropical jet vacillates in the Held and Suarez system. The importance of {$<$}em{$>\tau<$}/em{$>$} for computing the correct response of a model to climate change is explicitly demonstrated by perturbing the pseudospectral model with simple torques. The amplitude of the model's response to external forcing increases as {$<$}em{$>\tau<$}/em{$>$} increases, as suggested by the fluctuation\textendash dissipation theorem.{$<$}/p{$><$}/section{$>$}},
  chapter = {Monthly Weather Review},
  file = {/Users/oscardimdore-miles/Zotero/storage/TKX7UIVJ/Gerber et al. - 2008 - Testing the Annular Mode Autocorrelation Time Scal.pdf;/Users/oscardimdore-miles/Zotero/storage/C7Y8A4IR/2007mwr2211.1.html},
  journal = {Monthly Weather Review},
  language = {English},
  number = {4}
}

@article{Gray2018,
  title = {Surface Impacts of the Quasi Biennial Oscillation},
  author = {Gray, L. J. and Anstey, J. A. and Kawatani, Y. and Lu, H. and Osprey, S. and Schenzinger, V.},
  year = {2018},
  volume = {18},
  pages = {8227--8247},
  doi = {10.5194/acp-18-8227-2018},
  journal = {Atmospheric Chemistry and Physics},
  number = {11}
}

@article{gray2020,
  title = {Forecasting Extreme Stratospheric Polar Vortex Events},
  author = {Gray, L and Brown, M and Knight, J and Andrews, M and Lu, H and O'Reilly, C and Anstey, J},
  year = {2020},
  volume = {11},
  pages = {4630},
  doi = {10.1038/s41467-020-18299-7},
  journal = {Nature communications}
}

@article{graySurface2018,
  title = {Surface Impacts of the {{Quasi Biennial Oscillation}}},
  author = {Gray, Lesley J. and Anstey, James A. and Kawatani, Yoshio and Lu, Hua and Osprey, Scott and Schenzinger, Verena},
  year = {2018},
  month = jun,
  volume = {18},
  pages = {8227--8247},
  publisher = {{Copernicus GmbH}},
  issn = {1680-7316},
  doi = {10.5194/acp-18-8227-2018},
  abstract = {{$<$}p{$><$}strong class="journal-contentHeaderColor"{$>$}Abstract.{$<$}/strong{$>$} Teleconnections between the Quasi Biennial Oscillation (QBO) and the Northern Hemisphere zonally averaged zonal winds, mean sea level pressure (mslp) and tropical precipitation are explored. The standard approach that defines the QBO using the equatorial zonal winds at a single pressure level is compared with the empirical orthogonal function approach that characterizes the vertical profile of the equatorial winds. Results are interpreted in terms of three potential routes of influence, referred to as the tropical, subtropical and polar routes. A novel technique is introduced to separate responses via the polar route that are associated with the stratospheric polar vortex, from the other two routes. A previously reported mslp response in January, with a pattern that resembles the positive phase of the North Atlantic Oscillation under QBO westerly conditions, is confirmed and found to be primarily associated with a QBO modulation of the stratospheric polar vortex. This mid-winter response is relatively insensitive to the exact height of the maximum QBO westerlies and a maximum positive response occurs with westerlies over a relatively deep range between 10 and 70\&thinsp;hPa. Two additional mslp responses are reported, in early winter (December) and late winter (February/March). In contrast to the January response the early and late winter responses show maximum sensitivity to the QBO winds at {$\sim\&$}thinsp;20 and {$\sim\&$}thinsp;70\&thinsp;hPa respectively, but are relatively insensitive to the QBO winds in between ({$\sim\&$}thinsp;50\&thinsp;hPa). The late winter response is centred over the North Pacific and is associated with QBO influence from the lowermost stratosphere at tropical/subtropical latitudes in the Pacific sector. The early winter response consists of anomalies over both the North Pacific and Europe, but the mechanism for this response is unclear. Increased precipitation occurs over the tropical western Pacific under westerly QBO conditions, particularly during boreal summer, with maximum sensitivity to the QBO winds at 70\&thinsp;hPa. The band of precipitation across the Pacific associated with the Inter-tropical Convergence Zone (ITCZ) shifts southward under QBO westerly conditions. The empirical orthogonal function (EOF)-based analysis suggests that this ITCZ precipitation response may be particularly sensitive to the vertical wind shear in the vicinity of 70\&thinsp;hPa and hence the tropical tropopause temperatures.{$<$}/p{$>$}},
  file = {/Users/oscardimdore-miles/Zotero/storage/FSJ94NB6/Gray et al. - 2018 - Surface impacts of the Quasi Biennial Oscillation.pdf;/Users/oscardimdore-miles/Zotero/storage/QSKMU58P/2018.html},
  journal = {Atmospheric Chemistry and Physics},
  language = {English},
  number = {11}
}

@article{Grinstead2004,
  title = {Application of the Cross Wavelet Transform and Wavelet Coherence to Geophysical Time Series},
  author = {Grinsted, A. and Moore, J. C. and Jevrejeva, S.},
  year = {2004},
  volume = {11},
  pages = {561--566},
  doi = {10.5194/npg-11-561-2004},
  journal = {Nonlinear Processes in Geophysics},
  number = {5/6}
}

@article{grinstedApplication2004,
  title = {Application of the Cross Wavelet Transform and Wavelet Coherence to Geophysical Time Series},
  author = {Grinsted, A. and Moore, J. C. and Jevrejeva, S.},
  year = {2004},
  month = nov,
  volume = {11},
  pages = {561--566},
  publisher = {{Copernicus GmbH}},
  issn = {1023-5809},
  doi = {10.5194/npg-11-561-2004},
  abstract = {{$<$}p{$><$}strong class="journal-contentHeaderColor"{$>$}Abstract.{$<$}/strong{$>$} Many scientists have made use of the wavelet method in analyzing time series, often using popular free software. However, at present there are no similar easy to use wavelet packages for analyzing two time series together. We discuss the cross wavelet transform and wavelet coherence for examining relationships in time frequency space between two time series. We demonstrate how phase angle statistics can be used to gain confidence in causal relationships and test mechanistic models of physical relationships between the time series. As an example of typical data where such analyses have proven useful, we apply the methods to the Arctic Oscillation index and the Baltic maximum sea ice extent record. Monte Carlo methods are used to assess the statistical significance against red noise backgrounds. A software package has been developed that allows users to perform the cross wavelet transform and wavelet coherence ({$<$}a href="http://www.pol.ac.uk/home/research/waveletcoherence/" target="\_blank"{$>$}www.pol.ac.uk/home/research/waveletcoherence/{$<$}/a{$>$}).{$<$}/p{$>$}},
  file = {/Users/oscardimdore-miles/Zotero/storage/6XLTYTSC/Grinsted et al. - 2004 - Application of the cross wavelet transform and wav.pdf;/Users/oscardimdore-miles/Zotero/storage/UZ744T4V/2004.html},
  journal = {Nonlinear Processes in Geophysics},
  language = {English},
  number = {5/6}
}

@article{gruzdevTwo2000,
  title = {Two Regimes of the Quasi-Biennial Oscillation in the Equatorial Stratospheric Wind},
  author = {Gruzdev, Aleksandr N. and Bezverkhny, Vyacheslav A.},
  year = {2000},
  volume = {105},
  pages = {29435--29443},
  issn = {2156-2202},
  doi = {10.1029/2000JD900495},
  abstract = {An analysis of changes in the period of the quasi-biennial oscillation (QBO) in the zonal velocity of the equatorial stratospheric wind on isobaric surfaces 70, 50, 40, 30, 20, 15, and 10 hPa for 1953\textendash 1997 is made. In particular, wavelet analysis, high-resolution spectral analysis, and circle map analysis have been made. Different methods reveal that the QBO in the equatorial wind velocity at the noted stratospheric levels manifests itself mainly as two prevailing regimes with periods of about 2 and 2.5 years alternating with each other. At middle-stratosphere levels the QBO amplitude changes as well with the changing period (increases with the period increase). The two periods are in a rational ratio to the annual period (with the ratios of corresponding frequencies to the annual frequency equal to one half and two fifths), thus pointing out the phase locking of the QBO to the annual cycle. Spectra of wind velocity exhibit, in addition to the main QBO periods, the annual and semiannual oscillations as well as 20-, 8-, and 8.6-month oscillations corresponding to oscillations with combination frequencies equal to the difference and sum of the annual frequency and the main QBO frequencies. Hypothetical conceptual mechanisms of the two regimes of the QBO are discussed.},
  copyright = {Copyright 2000 by the American Geophysical Union.},
  file = {/Users/oscardimdore-miles/Zotero/storage/BZC7AUDM/Gruzdev and Bezverkhny - 2000 - Two regimes of the quasi-biennial oscillation in t.pdf;/Users/oscardimdore-miles/Zotero/storage/AV79W44Y/2000JD900495.html},
  journal = {Journal of Geophysical Research: Atmospheres},
  language = {English},
  number = {D24}
}

@article{Gruzdez2000,
  title = {Two Regimes of the Quasi-Biennial Oscillation in the Equatorial Stratospheric Wind},
  author = {Gruzdev, Aleksandr N. and Bezverkhny, Vyacheslav A.},
  year = {2000},
  volume = {105},
  pages = {29435--29444},
  doi = {10.1029/2000JD900495},
  number = {D24}
}

@article{haaseImportance2018,
  title = {The {{Importance}} of a {{Properly Represented Stratosphere}} for {{Northern Hemisphere Surface Variability}} in the {{Atmosphere}} and the {{Ocean}}},
  author = {Haase, Sabine and Matthes, Katja and Latif, Mojib and Omrani, Nour-Eddine},
  year = {2018},
  month = oct,
  volume = {31},
  pages = {8481--8497},
  publisher = {{American Meteorological Society}},
  issn = {0894-8755, 1520-0442},
  doi = {10.1175/JCLI-D-17-0520.1},
  abstract = {{$<$}section class="abstract"{$><$}h2 class="abstractTitle text-title my-1" id="d614e2"{$>$}Abstract{$<$}/h2{$><$}p{$>$}Major sudden stratospheric warmings (SSWs) are extreme events during boreal winter, which not only impact tropospheric weather up to three months but also can influence oceanic variability through wind stress and heat flux anomalies. In the North Atlantic region, SSWs have the potential to modulate deep convection in the Labrador Sea and thereby the strength of the Atlantic meridional overturning circulation. The impact of SSWs on the Northern Hemisphere surface climate is investigated in two coupled climate models: a stratosphere-resolving (high top) and a non-stratosphere-resolving (low top) model. In both configurations, a robust link between SSWs and a negative NAO is detected, which leads to shallower-than-normal North Atlantic mixed layer depth. The frequency of SSWs and the persistence of this link is better captured in the high-top model. Significant differences occur over the Pacific region, where an unrealistically persistent Aleutian low is observed in the low-top configuration. An overrepresentation of SSWs during El Ni\~no conditions in the low-top model is the main cause for this artifact. Our results underline the importance of a proper representation of the stratosphere in a coupled climate model for a consistent surface response in both the atmosphere and the ocean, which, among others, may have implications for oceanic deep convection in the subpolar North Atlantic.{$<$}/p{$><$}/section{$>$}},
  chapter = {Journal of Climate},
  file = {/Users/oscardimdore-miles/Zotero/storage/6532UXVE/Haase et al. - 2018 - The Importance of a Properly Represented Stratosph.pdf;/Users/oscardimdore-miles/Zotero/storage/IDX7TZIW/jcli-d-17-0520.1.html},
  journal = {Journal of Climate},
  language = {English},
  number = {20}
}

@article{hakkinenVariability1999,
  title = {Variability of the Simulated Meridional Heat Transport in the {{North Atlantic}} for the Period 1951\textendash 1993},
  author = {H{\"a}kkinen, Sirpa},
  year = {1999},
  volume = {104},
  pages = {10991--11007},
  issn = {2156-2202},
  doi = {10.1029/1999JC900034},
  abstract = {A 43-year ocean model simulation for the period 1951\textendash 1993 is analyzed, with a focus on the meridional heat transport (MHT) as a proxy for the strength of meridional overturning cell (MOC) at 25\textdegree N. The surface heat flux forcing associated with the North Atlantic Oscillation (NAO) pattern is related to the variability in MHT both on interannual and longer timescales. The manifestation of the interannual and decadal oceanic response to NAO is nonlocal, as evidenced by the concentrated heat content and sea level variability at the Gulf Stream and North Atlantic Current regions and shown by the comparison of the model sea level variability to the altimeter data between low MHT years in mid-1980s and high MHT years in early 1990s. The same area is singled out by empirical orthogonal function analysis in the leading mode of the sea level. MHT and time series of the leading sea level mode are highly correlated, which reflects the importance of the MOC in determining the variability of the heat content of the whole basin. Also, the model suggests that the MHT/MOC has entered a very strong phase since the mid-1980s and this trend has continued up to the end of 1993; this behavior follows the general trend in the NAO index.},
  copyright = {This paper is not subject to U.S. copyright. Published in 1999 by the American Geophysical Union.},
  file = {/Users/oscardimdore-miles/Zotero/storage/GXI79V5C/Häkkinen - 1999 - Variability of the simulated meridional heat trans.pdf;/Users/oscardimdore-miles/Zotero/storage/IDKFG4MD/1999JC900034.html},
  journal = {Journal of Geophysical Research: Oceans},
  language = {English},
  number = {C5}
}

@article{halpertClimate1997,
  title = {Climate {{Assessment}} for 1996},
  author = {Halpert, Michael S. and Bell, Gerald D.},
  year = {1997},
  month = may,
  volume = {78},
  pages = {S1-S50},
  publisher = {{American Meteorological Society}},
  issn = {0003-0007, 1520-0477},
  doi = {10.1175/1520-0477-78.5s.S1},
  abstract = {{$<$}section class="abstract"{$><$}p{$>$}The climate of 1996 can be characterized by several phenomena that reflect substantial deviations from the mean state of the atmosphere persisting from months to seasons. First, mature cold-episode conditions persisted across the tropical Pacific from November 1995 through May 1996 and contributed to large-scale anomalies of atmospheric circulation, temperature, and precipitation across the Tropics, the North Pacific and North America. These anomalies were in many respects opposite to those that had prevailed during the past several years in association with a prolonged period of tropical Pacific warm-episode conditions (ENSO). Second, strong tropical intraseasonal (Madden\textendash Julian oscillations) activity was observed during most of the year. The impact of these oscillations on extratropical circulation variability was most evident late in the year in association with strong variations in the eastward extent of the East Asian jet and in the attendant downstream circulation, temperature, and precipitation patterns over the eastern North Pacific and central North America. Third, a return to the strong negative phase of the atmospheric North Atlantic oscillation (NAO) during November 1995\textendash February 1996, following a nearly continuous 15-yr period of positive-phase NAO conditions, played a critical role in affecting temperature and precipitation patterns across the North Atlantic, Eurasia, and northern Africa. The NAO also contributed to a significant decrease in wintertime temperatures across large portions of Siberia and northern Russia from those that had prevailed during much of the 1980s and early 1990s.Other regional aspects of the short-term climate during 1996 included severe drought across the southwestern United States and southern plains states during October 1995\textendash May 1996, flooding in the Pacific Northwest region of the United States during the 1995/96 and 1996/97 winters, a cold and extremely snowy 1995/96 winter in the eastern United States, a second consecutive year of above-normal North Atlantic hurricane activity, near-normal rains in the African Sahel, above-normal rainfall across southeastern Africa during October 1995\textendash April 1996, above-normal precipitation for most of the year across eastern and southeastern Australia following severe drought in these areas during 1995, and generally nearnormal monsoonal rains in India with significantly below-normal rainfall in Bangladesh and western Burma.The global annual mean surface temperature for land and marine areas during 1996 averaged 0.21\textdegree C above the 1961\textendash 90 base period means. This is a decrease of 0.19\textdegree C from the record warm year of 1995 but was still among the 10 highest values observed since 1860. The global land-only temperature for 1996 was 0.06\textdegree C above normal and was the lowest anomaly observed since 1985 (-0.11\textdegree C). Much of this relative decrease in global temperatures occurred in the Northern Hemisphere extratropics, where land-only temperatures dropped from 0.42\textdegree C above normal in 1995 to 0.04\textdegree C below normal in 1996.The year also witnessed a continuation of near-record low ozone amounts in the Southern Hemisphere stratosphere, along with an abnormally prolonged appearance of the ``ozone hole'' into early December. The areal extent of the ozone hole in November and early December exceeded that previously observed for any such period on record. However, its areal extent at peak amplitude during late September\textendash early October was near that observed during the past several years.{$<$}/p{$><$}/section{$>$}},
  chapter = {Bulletin of the American Meteorological Society},
  file = {/Users/oscardimdore-miles/Zotero/storage/6B8LXQWT/Halpert and Bell - 1997 - Climate Assessment for 1996.pdf;/Users/oscardimdore-miles/Zotero/storage/MA26N59H/1520-0477-78_5s_s1.html},
  journal = {Bulletin of the American Meteorological Society},
  language = {English},
  number = {5s}
}

@article{hegyiDynamical2011,
  title = {A Dynamical Fingerprint of Tropical {{Pacific}} Sea Surface Temperatures on the Decadal-Scale Variability of Cool-Season {{Arctic}} Precipitation},
  author = {Hegyi, Bradley M. and Deng, Yi},
  year = {2011},
  volume = {116},
  issn = {2156-2202},
  doi = {10.1029/2011JD016001},
  abstract = {The temporal and spatial characteristics of decadal-scale variability in the Northern Hemisphere (NH) cool-season (October\textendash March) Arctic precipitation are identified from both the Climate Prediction Center (CPC) Merged Analysis of Precipitation (CMAP) and the Global Precipitation Climatology Project (GPCP) precipitation data sets. This decadal variability is shown to be partly connected to the decadal-scale variations in tropical central Pacific sea surface temperatures (SSTs) that are primarily associated with a decadal modulation of the El Ni\~no\textendash Southern Oscillation (ENSO), i.e., transitions between periods favoring typical eastern Pacific warming (EPW) events and periods favoring central Pacific warming (CPW) events. Regression and composite analyses reveal that increases of central Pacific SSTs drive a stationary Rossby wave train that destructively interferes with the wave number-1 component of the extratropical planetary wave. This destructive interference is opposite to the mean effect of typical EPW on the extratropical planetary wave. It leads to suppressed upward propagation of wave energy into the polar stratosphere, a stronger stratospheric polar vortex, and a tendency toward a positive phase of the Arctic Oscillation (AO). The positive AO tendency is synchronized on the decadal scale with a poleward shift of the NH storm tracks, particularly in the North Atlantic. Storm track variations further induce changes in the amount of moisture transported into the Arctic by synoptic eddies. The fluctuations in the eddy moisture transport ultimately contribute to the observed decadal-scale variations in the total Arctic precipitation in the NH cool season.},
  copyright = {Copyright 2011 by the American Geophysical Union.},
  file = {/Users/oscardimdore-miles/Zotero/storage/PQAF9YQJ/Hegyi and Deng - 2011 - A dynamical fingerprint of tropical Pacific sea su.pdf;/Users/oscardimdore-miles/Zotero/storage/WRGR7BCD/2011JD016001.html},
  journal = {Journal of Geophysical Research: Atmospheres},
  keywords = {Arctic Oscillation,Arctic precipitation,decadal variability,Modoki,tropical Pacific},
  language = {English},
  number = {D20}
}

@article{Henderson2018,
  title = {Snow\textendash Atmosphere Coupling in the {{Northern Hemisphere}}},
  author = {Henderson, Gina and Peings, Yannick and Furtado, Jason and Kushner, Paul},
  year = {2018},
  volume = {8},
  doi = {10.1038/s41558-018-0295-6},
  journal = {Nature Climate Change}
}

@article{henleyTripole2015,
  title = {A {{Tripole Index}} for the {{Interdecadal Pacific Oscillation}}},
  author = {Henley, Benjamin J. and Gergis, Joelle and Karoly, David J. and Power, Scott and Kennedy, John and Folland, Chris K.},
  year = {2015},
  month = dec,
  volume = {45},
  pages = {3077--3090},
  issn = {1432-0894},
  doi = {10.1007/s00382-015-2525-1},
  abstract = {A new index is developed for the Interdecadal Pacific Oscillation, termed the IPO Tripole Index (TPI). The IPO is associated with a distinct `tripole' pattern of sea surface temperature anomalies (SSTA), with three large centres of action and variations on decadal timescales, evident in the second principal component (PC) of low-pass filtered global SST. The new index is based on the difference between the SSTA averaged over the central equatorial Pacific and the average of the SSTA in the Northwest and Southwest Pacific. The TPI is an easily calculated, non-PC-based index for tracking decadal SST variability associated with the IPO. The TPI time series bears a close resemblance to previously published PC-based indices and has the advantages of being simpler to compute and more consistent with indices used to track the El Ni\~no\textendash Southern Oscillation (ENSO), such as Ni\~no 3.4. The TPI also provides a simple metric in physical units of \textdegree C for evaluating decadal and interdecadal variability of SST fields in a straightforward manner, and can be used to evaluate the skill of dynamical decadal prediction systems. Composites of SST and mean sea level pressure anomalies reveal that the IPO has maintained a broadly stable structure across the seven most recent positive and negative epochs that occurred during 1870\textendash 2013. The TPI is shown to be a robust and stable representation of the IPO phenomenon in instrumental records, with relatively more variance in decadal than shorter timescales compared to Ni\~no 3.4, due to the explicit inclusion of off-equatorial SST variability associated with the IPO.},
  file = {/Users/oscardimdore-miles/Zotero/storage/7GYGK8IJ/Henley et al. - 2015 - A Tripole Index for the Interdecadal Pacific Oscil.pdf},
  journal = {Climate Dynamics},
  language = {English},
  number = {11}
}

@article{hersbachERA52020,
  title = {The {{ERA5}} Global Reanalysis},
  author = {Hersbach, Hans and Bell, Bill and Berrisford, Paul and Hirahara, Shoji and Hor{\'a}nyi, Andr{\'a}s and {Mu{\~n}oz-Sabater}, Joaqu{\'i}n and Nicolas, Julien and Peubey, Carole and Radu, Raluca and Schepers, Dinand and Simmons, Adrian and Soci, Cornel and Abdalla, Saleh and Abellan, Xavier and Balsamo, Gianpaolo and Bechtold, Peter and Biavati, Gionata and Bidlot, Jean and Bonavita, Massimo and Chiara, Giovanna De and Dahlgren, Per and Dee, Dick and Diamantakis, Michail and Dragani, Rossana and Flemming, Johannes and Forbes, Richard and Fuentes, Manuel and Geer, Alan and Haimberger, Leo and Healy, Sean and Hogan, Robin J. and H{\'o}lm, El{\'i}as and Janiskov{\'a}, Marta and Keeley, Sarah and Laloyaux, Patrick and Lopez, Philippe and Lupu, Cristina and Radnoti, Gabor and {de Rosnay}, Patricia and Rozum, Iryna and Vamborg, Freja and Villaume, Sebastien and Th{\'e}paut, Jean-No{\"e}l},
  year = {2020},
  volume = {146},
  pages = {1999--2049},
  issn = {1477-870X},
  doi = {10.1002/qj.3803},
  abstract = {Within the Copernicus Climate Change Service (C3S), ECMWF is producing the ERA5 reanalysis which, once completed, will embody a detailed record of the global atmosphere, land surface and ocean waves from 1950 onwards. This new reanalysis replaces the ERA-Interim reanalysis (spanning 1979 onwards) which was started in 2006. ERA5 is based on the Integrated Forecasting System (IFS) Cy41r2 which was operational in 2016. ERA5 thus benefits from a decade of developments in model physics, core dynamics and data assimilation. In addition to a significantly enhanced horizontal resolution of 31 km, compared to 80 km for ERA-Interim, ERA5 has hourly output throughout, and an uncertainty estimate from an ensemble (3-hourly at half the horizontal resolution). This paper describes the general set-up of ERA5, as well as a basic evaluation of characteristics and performance, with a focus on the dataset from 1979 onwards which is currently publicly available. Re-forecasts from ERA5 analyses show a gain of up to one day in skill with respect to ERA-Interim. Comparison with radiosonde and PILOT data prior to assimilation shows an improved fit for temperature, wind and humidity in the troposphere, but not the stratosphere. A comparison with independent buoy data shows a much improved fit for ocean wave height. The uncertainty estimate reflects the evolution of the observing systems used in ERA5. The enhanced temporal and spatial resolution allows for a detailed evolution of weather systems. For precipitation, global-mean correlation with monthly-mean GPCP data is increased from 67\% to 77\%. In general, low-frequency variability is found to be well represented and from 10 hPa downwards general patterns of anomalies in temperature match those from the ERA-Interim, MERRA-2 and JRA-55 reanalyses.},
  copyright = {\textcopyright{} 2020 The Authors. Quarterly Journal of the Royal Meteorological Society published by John Wiley \& Sons Ltd on behalf of the Royal Meteorological Society.},
  file = {/Users/oscardimdore-miles/Zotero/storage/B9NP7FT4/Hersbach et al. - 2020 - The ERA5 global reanalysis.pdf},
  journal = {Quarterly Journal of the Royal Meteorological Society},
  keywords = {climate reanalysis,Copernicus Climate Change Service,data assimilation,ERA5,historical observations},
  language = {English},
  number = {730}
}

@article{hersbachERA52020b,
  title = {The {{ERA5}} Global Reanalysis},
  author = {Hersbach, Hans and Bell, Bill and Berrisford, Paul and Hirahara, Shoji and Hor{\'a}nyi, Andr{\'a}s and {Mu{\~n}oz-Sabater}, Joaqu{\'i}n and Nicolas, Julien and Peubey, Carole and Radu, Raluca and Schepers, Dinand and Simmons, Adrian and Soci, Cornel and Abdalla, Saleh and Abellan, Xavier and Balsamo, Gianpaolo and Bechtold, Peter and Biavati, Gionata and Bidlot, Jean and Bonavita, Massimo and Chiara, Giovanna De and Dahlgren, Per and Dee, Dick and Diamantakis, Michail and Dragani, Rossana and Flemming, Johannes and Forbes, Richard and Fuentes, Manuel and Geer, Alan and Haimberger, Leo and Healy, Sean and Hogan, Robin J. and H{\'o}lm, El{\'i}as and Janiskov{\'a}, Marta and Keeley, Sarah and Laloyaux, Patrick and Lopez, Philippe and Lupu, Cristina and Radnoti, Gabor and de Rosnay, Patricia and Rozum, Iryna and Vamborg, Freja and Villaume, Sebastien and Th{\'e}paut, Jean-No{\"e}l},
  year = {2020},
  volume = {146},
  pages = {1999--2049},
  issn = {1477-870X},
  doi = {10.1002/qj.3803},
  abstract = {Within the Copernicus Climate Change Service (C3S), ECMWF is producing the ERA5 reanalysis which, once completed, will embody a detailed record of the global atmosphere, land surface and ocean waves from 1950 onwards. This new reanalysis replaces the ERA-Interim reanalysis (spanning 1979 onwards) which was started in 2006. ERA5 is based on the Integrated Forecasting System (IFS) Cy41r2 which was operational in 2016. ERA5 thus benefits from a decade of developments in model physics, core dynamics and data assimilation. In addition to a significantly enhanced horizontal resolution of 31 km, compared to 80 km for ERA-Interim, ERA5 has hourly output throughout, and an uncertainty estimate from an ensemble (3-hourly at half the horizontal resolution). This paper describes the general set-up of ERA5, as well as a basic evaluation of characteristics and performance, with a focus on the dataset from 1979 onwards which is currently publicly available. Re-forecasts from ERA5 analyses show a gain of up to one day in skill with respect to ERA-Interim. Comparison with radiosonde and PILOT data prior to assimilation shows an improved fit for temperature, wind and humidity in the troposphere, but not the stratosphere. A comparison with independent buoy data shows a much improved fit for ocean wave height. The uncertainty estimate reflects the evolution of the observing systems used in ERA5. The enhanced temporal and spatial resolution allows for a detailed evolution of weather systems. For precipitation, global-mean correlation with monthly-mean GPCP data is increased from 67\% to 77\%. In general, low-frequency variability is found to be well represented and from 10 hPa downwards general patterns of anomalies in temperature match those from the ERA-Interim, MERRA-2 and JRA-55 reanalyses.},
  annotation = {\_eprint: https://rmets.onlinelibrary.wiley.com/doi/pdf/10.1002/qj.3803},
  copyright = {\textcopyright{} 2020 The Authors. Quarterly Journal of the Royal Meteorological Society published by John Wiley \& Sons Ltd on behalf of the Royal Meteorological Society.},
  file = {/Users/oscardimdore-miles/Zotero/storage/BRKE8PNA/Hersbach et al. - 2020 - The ERA5 global reanalysis.pdf},
  journal = {Quarterly Journal of the Royal Meteorological Society},
  keywords = {climate reanalysis,Copernicus Climate Change Service,data assimilation,ERA5,historical observations},
  language = {en},
  number = {730}
}

@article{Hirota2018,
  title = {The Influences of {{El Nino}} and {{Arctic}} Sea-Ice on the {{QBO}} Disruption in {{February}} 2016},
  author = {Hirota, Nagio and Shiogama, Hideo and Akiyoshi, Hideharu and Ogura, T. and Takahashi, M. and Kawatani, Y. and Kimoto, M. and Mori, Masato},
  year = {2018},
  volume = {1},
  doi = {10.1038/s41612-018-0020-1},
  journal = {npj Climate and Atmospheric Science}
}

@article{hitchcockDownward2014,
  title = {The {{Downward Influence}} of {{Stratospheric Sudden Warmings}}},
  author = {Hitchcock, Peter and Simpson, Isla R.},
  year = {2014},
  month = oct,
  volume = {71},
  pages = {3856--3876},
  publisher = {{American Meteorological Society}},
  issn = {0022-4928, 1520-0469},
  doi = {10.1175/JAS-D-14-0012.1},
  abstract = {{$<$}section class="abstract"{$><$}h2 class="abstractTitle text-title my-1" id="d1897e2"{$>$}Abstract{$<$}/h2{$><$}p{$>$}The coupling between the stratosphere and the troposphere following two major stratospheric sudden warmings is studied in the Canadian Middle Atmosphere Model using a nudging technique by which the zonal-mean evolution of the reference sudden warmings are artificially induced in an \textasciitilde 100-member ensemble spun off from a control simulation. Both reference warmings are taken from a freely running integration of the model. One event is a displacement, the other is a split, and both are followed by extended recoveries in the lower stratosphere. The methodology permits a statistically robust study of their influence on the troposphere below.The nudged ensembles exhibit a tropospheric annular mode response closely analogous to that seen in observations, confirming the downward influence of sudden warmings on the troposphere in a comprehensive model. This tropospheric response coincides more closely with the lower-stratospheric annular mode anomalies than with the midstratospheric wind reversal. In addition to the expected synoptic-scale eddy feedback, the planetary-scale eddies also reinforce the tropospheric wind changes, apparently responding directly to the stratospheric anomalies.Furthermore, despite the zonal symmetry of the stratospheric perturbation, a highly zonally asymmetric near-surface response is produced, corresponding to a strongly negative phase of the North Atlantic Oscillation with a much weaker response over the Pacific basin that matches composites of sudden warmings from the Interim ECMWF Re-Analysis (ERA-Interim). Phase 5 of the Coupled Model Intercomparison Project models exhibit a similar response, though in most models the response's magnitude is underrepresented.{$<$}/p{$><$}/section{$>$}},
  chapter = {Journal of the Atmospheric Sciences},
  file = {/Users/oscardimdore-miles/Zotero/storage/5AZXZXI3/Hitchcock and Simpson - 2014 - The Downward Influence of Stratospheric Sudden War.pdf;/Users/oscardimdore-miles/Zotero/storage/N94VM7IU/jas-d-14-0012.1.html},
  journal = {Journal of the Atmospheric Sciences},
  language = {English},
  number = {10}
}

@article{Holton1982,
  title = {The Quasi-Biennial Oscillation in the Northern Hemisphere Lower Stratosphere},
  author = {Holton, James R and Tan, Hsiu-Chi},
  year = {1982},
  volume = {60},
  pages = {140--148},
  doi = {10.2151/jmsj1965.60.1_140},
  journal = {Journal of the Meteorological Society of Japan},
  number = {1}
}

@article{HoltonJamesRTan1980,
  title = {The Influence of the Equatorial Quasi-Biennial Oscillation on the Global Circulation at 50 Mb},
  author = {Holton, J. R. and Tan, H. C.},
  year = {1980},
  volume = {37},
  pages = {2200--2208},
  doi = {10.1175/1520-0469},
  journal = {Journal of Atmospheric Sciences},
  number = {10}
}

@article{Horan2017,
  title = {Modeling Seasonal Sudden Stratospheric Warming Climatology Based on Polar Vortex Statistics},
  author = {Horan, Matthew F. and Reichler, Thomas},
  year = {2017},
  volume = {30},
  pages = {10101--10116},
  publisher = {{American Meteorological Society}},
  journal = {Journal of Climate},
  number = {24}
}

@article{hu2018,
  title = {Decadal Relationship between the Stratospheric Arctic Vortex and Pacific Decadal Oscillation},
  author = {Hu, Dingzhu and Guan, Zhaoyong},
  year = {2018},
  volume = {31},
  pages = {3371--3386},
  doi = {10.1175/JCLI-D-17-0266.1},
  journal = {Journal of Climate},
  number = {9}
}

@article{huDecadal2018,
  title = {Decadal {{Relationship}} between the {{Stratospheric Arctic Vortex}} and {{Pacific Decadal Oscillation}}},
  author = {Hu, Dingzhu and Guan, Zhaoyong},
  year = {2018},
  month = may,
  volume = {31},
  pages = {3371--3386},
  publisher = {{American Meteorological Society}},
  issn = {0894-8755, 1520-0442},
  doi = {10.1175/JCLI-D-17-0266.1},
  abstract = {{$<$}section class="abstract"{$><$}h2 class="abstractTitle text-title my-1" id="d192e2"{$>$}Abstract{$<$}/h2{$><$}p{$>$}Using reanalysis datasets and numerical simulations, the relationship between the stratospheric Arctic vortex (SAV) and the Pacific decadal oscillation (PDO) on decadal time scales was investigated. A significant in-phase relationship between the PDO and SAV on decadal time scales during 1950\textendash 2014 is found, that is, the North Pacific sea surface temperature (SST) cooling (warming) associated with the positive (negative) PDO phases is closely related to the strengthening (weakening) of the SAV. This decadal relationship between the North Pacific SST and SAV is different from their relationship on subdecadal time scales. Observational and modeling results both demonstrate that the decadal variation in the SAV is strongly affected by the North Pacific SSTs related to the PDO via modifying the upward propagation of planetary wavenumber-1 waves from the troposphere to the stratosphere. The decreased SSTs over the North Pacific tend to result in a deepened Aleutian low along with a strengthened jet stream over the North Pacific, which excites a weakened western Pacific pattern and a strengthened Pacific\textendash North American pattern. These tropospheric circulation anomalies are in accordance with the decreased refractive index (RI) at middle and high latitudes in the northern stratosphere during the positive PDO phase. The increased RI at high latitudes in the upper troposphere impedes the planetary wavenumber-1 wave from propagating into the stratosphere, and in turn strengthens the SAV. The responses of the RI to the PDO are mainly contributions of the changes in the meridional gradient of the zonal-mean potential vorticity via alteration of the baroclinic term {$<$}inline-formula xmlns:ifp="http://www.ifactory.com/press"{$><$}img src="jcli-d-17-0266.1-inf1.gif"{$><$}/img{$><$}/inline-formula{$>$}.{$<$}/p{$><$}/section{$>$}},
  chapter = {Journal of Climate},
  file = {/Users/oscardimdore-miles/Zotero/storage/MSWQSTZL/Hu and Guan - 2018 - Decadal Relationship between the Stratospheric Arc.pdf;/Users/oscardimdore-miles/Zotero/storage/THGYF57P/jcli-d-17-0266.1.html},
  journal = {Journal of Climate},
  language = {English},
  number = {9}
}

@article{hurrellNorth2003,
  title = {The {{North Atlantic Oscillation}}: {{Climatic Significance}} and {{Environmental Impact}}},
  shorttitle = {The {{North Atlantic Oscillation}}},
  author = {Hurrell, J.W and Kushnir, Yochanan and Ottersen, Geir and Visbeck, Martin},
  year = {2003},
  month = jan,
  volume = {134},
  doi = {10.1029/GM134},
  abstract = {Over the middle and high latitudes of the Northern Hemisphere the most prominent and recurrent pattern of atmospheric variability is the North Atlantic Oscillation (NAO). The NAO refers to swings in the atmospheric sea level pressure difference between the Arctic and the subtropical Atlantic that are most noticeable during the boreal cold season (November-April) and are associated with changes in the mean wind speed and direction. Such changes alter the seasonal mean heat and moisture transport between the Atlantic and the neighboring continents, as well as the intensity and number of storms, their paths, and their weather. Significant changes in ocean surface temperature and heat content, ocean currents and their related heat transport, and sea ice cover in the Arctic and sub-Arctic regions are also induced by changes in the NAO. Such climatic fluctuations affect agricultural harvests, water management, energy supply and demand, and fisheries yields. All these effects have led to many studies of the phenomenon; yet, despite this interest, unanswered questions remain regarding the climatic processes that govern NAO variability, how the phenomenon has varied in the past or will vary in the future, and whether it is at all predictable.},
  file = {/Users/oscardimdore-miles/Zotero/storage/MJ5TNL25/Hurrell et al. - 2003 - The North Atlantic Oscillation Climatic Significa.pdf},
  journal = {Volcanism and Subduction: The Kamchatka Region}
}

@book{hurrellNorth2003a,
  title = {The {{North Atlantic Oscillation}}: {{Climatic Significance}} and {{Environmental Impact}}},
  shorttitle = {The {{North Atlantic Oscillation}}},
  author = {Hurrell, J.W and Kushnir, Yochanan and Ottersen, Geir and Visbeck, Martin},
  year = {2003},
  month = jan,
  volume = {134},
  doi = {10.1029/GM134},
  abstract = {Over the middle and high latitudes of the Northern Hemisphere the most prominent and recurrent pattern of atmospheric variability is the North Atlantic Oscillation (NAO). The NAO refers to swings in the atmospheric sea level pressure difference between the Arctic and the subtropical Atlantic that are most noticeable during the boreal cold season (November-April) and are associated with changes in the mean wind speed and direction. Such changes alter the seasonal mean heat and moisture transport between the Atlantic and the neighboring continents, as well as the intensity and number of storms, their paths, and their weather. Significant changes in ocean surface temperature and heat content, ocean currents and their related heat transport, and sea ice cover in the Arctic and sub-Arctic regions are also induced by changes in the NAO. Such climatic fluctuations affect agricultural harvests, water management, energy supply and demand, and fisheries yields. All these effects have led to many studies of the phenomenon; yet, despite this interest, unanswered questions remain regarding the climatic processes that govern NAO variability, how the phenomenon has varied in the past or will vary in the future, and whether it is at all predictable.},
  file = {/Users/oscardimdore-miles/Zotero/storage/LDUZFP5G/Hurrell et al. - 2003 - The North Atlantic Oscillation Climatic Significa.pdf}
}

@incollection{hurrellNorth2005,
  title = {North {{Atlantic Oscillation}}},
  booktitle = {Encyclopedia of {{World Climatology}}},
  author = {Hurrell, James W.},
  editor = {Oliver, John E.},
  year = {2005},
  pages = {536--539},
  publisher = {{Springer Netherlands}},
  address = {{Dordrecht}},
  doi = {10.1007/1-4020-3266-8_150},
  file = {/Users/oscardimdore-miles/Zotero/storage/8QAJJ7QY/Hurrell - 2005 - North Atlantic Oscillation.pdf},
  isbn = {978-1-4020-3266-0},
  language = {English}
}

@article{Ineson2009,
  title = {The Role of the Stratosphere in the {{European}} Climate Response to {{El Ni\~no}}},
  author = {Ineson, S. and Scaife, A. A.},
  year = {2009},
  volume = {2},
  pages = {32--36},
  doi = {10.1038/ngeo381},
  journal = {Nature Geoscience},
  number = {1}
}

@article{inesonRole2009,
  title = {The Role of the Stratosphere in the {{European}} Climate Response to {{El Ni\~no}}},
  author = {Ineson, S. and Scaife, A. A.},
  year = {2009},
  month = jan,
  volume = {2},
  pages = {32--36},
  publisher = {{Nature Publishing Group}},
  issn = {1752-0908},
  doi = {10.1038/ngeo381},
  abstract = {Observational studies show a clear response in European climate to El Ni\~no/Southern Oscillation in late winter. Simulations with an atmospheric general circulation model identify a long-distance pathway connecting climate variability in the Pacific region and Europe via the stratosphere, the upper layer of the atmosphere.},
  copyright = {2009 Nature Publishing Group},
  file = {/Users/oscardimdore-miles/Zotero/storage/FYLAX3WX/Ineson and Scaife - 2009 - The role of the stratosphere in the European clima.pdf;/Users/oscardimdore-miles/Zotero/storage/LMB9AIEN/ngeo381.html},
  journal = {Nature Geoscience},
  language = {English},
  number = {1}
}

@article{Iza2016,
  title = {The Stratospheric Pathway of La Ni\~na},
  author = {Iza, Maddalen and Calvo, Natalia and Manzini, Elisa},
  year = {2016},
  volume = {29},
  pages = {8899--8914},
  doi = {10.1175/JCLI-D-16-0230.1},
  journal = {Journal of Climate},
  number = {24}
}

@article{izaRole2015,
  title = {Role of {{Stratospheric Sudden Warmings}} on the Response to {{Central Pacific El Ni\~no}}},
  author = {Iza, Maddalen and Calvo, Natalia},
  year = {2015},
  volume = {42},
  pages = {2482--2489},
  issn = {1944-8007},
  doi = {10.1002/2014GL062935},
  abstract = {The Northern Hemisphere (NH) polar stratospheric response to Central Pacific El Ni\~no (CP-El Ni\~no) events remains unclear. Contradictory results have been reported depending on the definition and events considered. We show that this is due to the prominent role of Stratospheric Sudden Warmings (SSWs), whose signal dominates the NH winter polar stratospheric response to CP-El Ni\~no. In fact, the CP-El Ni\~no signal is robust when the events are classified according to the occurrence of SSWs and displays opposite response in winters with and without SSWs. In the absence of SSWs, polar stratospheric responses to Central Pacific and Eastern Pacific El Ni\~no are clearly distinguishable in early winter, in relation to differences in the Pacific-North American pattern. Our results demonstrate that the occurrence of SSWs needs to be taken into account when studying the stratospheric response to CP-El Ni\~no and explain why different responses to CP-El Ni\~no have been reported previously.},
  copyright = {\textcopyright 2015. The Authors.},
  file = {/Users/oscardimdore-miles/Zotero/storage/BUL69Z9U/Iza and Calvo - 2015 - Role of Stratospheric Sudden Warmings on the respo.pdf;/Users/oscardimdore-miles/Zotero/storage/JUDPK2GR/2014GL062935.html},
  journal = {Geophysical Research Letters},
  keywords = {Central Pacific ENSO,Eastern Pacific ENSO,flavors of ENSO,polar vortex,SSWs,stratospheric dynamics},
  language = {English},
  number = {7}
}

@article{izaStratospheric2016,
  title = {The {{Stratospheric Pathway}} of {{La Ni\~na}}},
  author = {Iza, Maddalen and Calvo, Natalia and Manzini, Elisa},
  year = {2016},
  month = dec,
  volume = {29},
  pages = {8899--8914},
  publisher = {{American Meteorological Society}},
  issn = {0894-8755, 1520-0442},
  doi = {10.1175/JCLI-D-16-0230.1},
  abstract = {{$<$}section class="abstract"{$><$}h2 class="abstractTitle text-title my-1" id="d261e2"{$>$}Abstract{$<$}/h2{$><$}p{$>$}A Northern Hemisphere (NH) polar stratospheric pathway for La Ni\~na events is established during wintertime based on reanalysis data for the 1958\textendash 2012 period. A robust polar stratospheric response is observed in the NH during strong La Ni\~na events, characterized by a significantly stronger and cooler polar vortex. Significant wind anomalies reach the surface, and a robust impact on the North Atlantic\textendash European (NAE) region is observed. A dynamical analysis reveals that the stronger polar stratospheric winds during La Ni\~na winters are due to reduced upward planetary wave activity into the stratosphere. This finding is the result of destructive interference between the climatological and the anomalous La Ni\~na tropospheric stationary eddies over the Pacific\textendash North American region.In addition, the lack of a robust stratospheric signature during La Ni\~na winters reported in previous studies is investigated. It is found that this is related to the lower threshold used to detect the events, which signature is consequently more prone to be obscured by the influence of other sources of variability. In particular, the occurrence of stratospheric sudden warmings (SSWs), partly linked to the phase of the quasi-biennial oscillation, modulates the observed stratospheric signal. In the case of La Ni\~na winters defined by a lower threshold, a robust stratospheric cooling is found only in the absence of SSWs. Therefore, these results highlight the importance of using a relatively restrictive threshold to define La Ni\~na events in order to obtain a robust surface response in the NAE region through the stratosphere.{$<$}/p{$><$}/section{$>$}},
  chapter = {Journal of Climate},
  file = {/Users/oscardimdore-miles/Zotero/storage/FTTYE6F8/Iza et al. - 2016 - The Stratospheric Pathway of La Niña.pdf;/Users/oscardimdore-miles/Zotero/storage/N29AS7UT/jcli-d-16-0230.1.html},
  journal = {Journal of Climate},
  language = {English},
  number = {24}
}

@article{johnsonHow2013,
  title = {How {{Many ENSO Flavors Can We Distinguish}}?},
  author = {Johnson, Nathaniel C.},
  year = {2013},
  month = jul,
  volume = {26},
  pages = {4816--4827},
  publisher = {{American Meteorological Society}},
  issn = {0894-8755, 1520-0442},
  doi = {10.1175/JCLI-D-12-00649.1},
  abstract = {{$<$}section class="abstract"{$><$}h2 class="abstractTitle text-title my-1" id="d6e2"{$>$}Abstract{$<$}/h2{$><$}p{$>$}It is now widely recognized that El Ni\~no\textendash Southern Oscillation (ENSO) occurs in more than one form, with the canonical eastern Pacific (EP) and more recently recognized central Pacific (CP) ENSO types receiving the most focus. Given that these various ENSO ``flavors'' may contribute to climate variability and long-term trends in unique ways, and that ENSO variability is not limited to these two types, this study presents a framework that treats ENSO as a continuum but determines a finite maximum number of statistically distinguishable representative ENSO patterns. A neural network\textendash based cluster analysis called self-organizing map (SOM) analysis paired with a statistical distinguishability test determines nine unique patterns that characterize the September\textendash February tropical Pacific SST anomaly fields for the period from 1950 through 2011. These nine patterns represent the flavors of ENSO, which include EP, CP, and mixed ENSO patterns. Over the 1950\textendash 2011 period, the most significant trends reflect changes in La Ni\~na patterns, with a shift in dominance of La Ni\~na\textendash like patterns with weak or negative western Pacific warm pool SST anomalies until the mid-1970s, followed by a dominance of La Ni\~na\textendash like patterns with positive western Pacific warm pool SST anomalies, particularly after the mid-1990s. Both an EP and especially a CP El Ni\~no pattern experienced positive frequency trends, but these trends are indistinguishable from natural variability. Overall, changes in frequency within the ENSO continuum contributed to the pattern of tropical Pacific warming, particularly in the equatorial eastern Pacific and especially in relation to changes of La Ni\~na\textendash like rather than El Ni\~no\textendash like patterns.{$<$}/p{$><$}/section{$>$}},
  chapter = {Journal of Climate},
  file = {/Users/oscardimdore-miles/Zotero/storage/9CS662C2/Johnson - 2013 - How Many ENSO Flavors Can We Distinguish.pdf;/Users/oscardimdore-miles/Zotero/storage/7SX4DP9P/jcli-d-12-00649.1.html},
  journal = {Journal of Climate},
  language = {English},
  number = {13}
}

@article{johnsonHow2013b,
  title = {How {{Many ENSO Flavors Can We Distinguish}}?},
  author = {Johnson, Nathaniel C.},
  year = {2013},
  month = jul,
  volume = {26},
  pages = {4816--4827},
  publisher = {{American Meteorological Society}},
  issn = {0894-8755, 1520-0442},
  doi = {10.1175/JCLI-D-12-00649.1},
  abstract = {{$<$}section class="abstract"{$><$}h2 class="abstractTitle text-title my-1" id="d68342764e63"{$>$}Abstract{$<$}/h2{$><$}p{$>$}It is now widely recognized that El Ni\~no\textendash Southern Oscillation (ENSO) occurs in more than one form, with the canonical eastern Pacific (EP) and more recently recognized central Pacific (CP) ENSO types receiving the most focus. Given that these various ENSO ``flavors'' may contribute to climate variability and long-term trends in unique ways, and that ENSO variability is not limited to these two types, this study presents a framework that treats ENSO as a continuum but determines a finite maximum number of statistically distinguishable representative ENSO patterns. A neural network\textendash based cluster analysis called self-organizing map (SOM) analysis paired with a statistical distinguishability test determines nine unique patterns that characterize the September\textendash February tropical Pacific SST anomaly fields for the period from 1950 through 2011. These nine patterns represent the flavors of ENSO, which include EP, CP, and mixed ENSO patterns. Over the 1950\textendash 2011 period, the most significant trends reflect changes in La Ni\~na patterns, with a shift in dominance of La Ni\~na\textendash like patterns with weak or negative western Pacific warm pool SST anomalies until the mid-1970s, followed by a dominance of La Ni\~na\textendash like patterns with positive western Pacific warm pool SST anomalies, particularly after the mid-1990s. Both an EP and especially a CP El Ni\~no pattern experienced positive frequency trends, but these trends are indistinguishable from natural variability. Overall, changes in frequency within the ENSO continuum contributed to the pattern of tropical Pacific warming, particularly in the equatorial eastern Pacific and especially in relation to changes of La Ni\~na\textendash like rather than El Ni\~no\textendash like patterns.{$<$}/p{$><$}/section{$>$}},
  chapter = {Journal of Climate},
  file = {/Users/oscardimdore-miles/Zotero/storage/TSHX9V3G/Johnson - 2013 - How Many ENSO Flavors Can We Distinguish.pdf;/Users/oscardimdore-miles/Zotero/storage/Q7BV32TJ/jcli-d-12-00649.1.html},
  journal = {Journal of Climate},
  language = {English},
  number = {13}
}

@article{Kang2017,
  title = {More Frequent Sudden Stratospheric Warming Events Due to Enhanced {{MJO}} Forcing Expected in a Warmer Climate},
  author = {Kang, Wanying and Tziperman, Eli},
  year = {2017},
  volume = {30},
  pages = {8727--8743},
  doi = {10.1175/JCLI-D-17-0044.1},
  journal = {Journal of Climate},
  number = {21}
}

@article{kaoContrasting2009,
  title = {Contrasting {{Eastern}}-{{Pacific}} and {{Central}}-{{Pacific Types}} of {{ENSO}}},
  author = {Kao, Hsun-Ying and Yu, Jin-Yi},
  year = {2009},
  month = feb,
  volume = {22},
  pages = {615--632},
  publisher = {{American Meteorological Society}},
  issn = {0894-8755, 1520-0442},
  doi = {10.1175/2008JCLI2309.1},
  abstract = {{$<$}section class="abstract"{$><$}h2 class="abstractTitle text-title my-1" id="d699e2"{$>$}Abstract{$<$}/h2{$><$}p{$>$}Surface observations and subsurface ocean assimilation datasets are examined to contrast two distinct types of El Ni\~no\textendash Southern Oscillation (ENSO) in the tropical Pacific: an eastern-Pacific (EP) type and a central-Pacific (CP) type. An analysis method combining empirical orthogonal function (EOF) analysis and linear regression is used to separate these two types. Correlation and composite analyses based on the principal components of the EOF were performed to examine the structure, evolution, and teleconnection of these two ENSO types. The EP type of ENSO is found to have its SST anomaly center located in the eastern equatorial Pacific attached to the coast of South America. This type of ENSO is associated with basinwide thermocline and surface wind variations and shows a strong teleconnection with the tropical Indian Ocean. In contrast, the CP type of ENSO has most of its surface wind, SST, and subsurface anomalies confined in the central Pacific and tends to onset, develop, and decay in situ. This type of ENSO appears less related to the thermocline variations and may be influenced more by atmospheric forcing. It has a stronger teleconnection with the southern Indian Ocean. Phase-reversal signatures can be identified in the anomaly evolutions of the EP-ENSO but not for the CP-ENSO. This implies that the CP-ENSO may occur more as events or epochs than as a cycle. The EP-ENSO has experienced a stronger interdecadal change with the dominant period of its SST anomalies shifted from 2 to 4 yr near 1976/77, while the dominant period for the CP-ENSO stayed near the 2-yr band. The different onset times of these two types of ENSO imply that the difference between the EP and CP types of ENSO could be caused by the timing of the mechanisms that trigger the ENSO events.{$<$}/p{$><$}/section{$>$}},
  chapter = {Journal of Climate},
  file = {/Users/oscardimdore-miles/Zotero/storage/FE4LR4RB/Kao and Yu - 2009 - Contrasting Eastern-Pacific and Central-Pacific Ty.pdf;/Users/oscardimdore-miles/Zotero/storage/SDHF8MBC/2008jcli2309.1.html},
  journal = {Journal of Climate},
  language = {English},
  number = {3}
}

@article{Kawatani2016,
  title = {Representation of the Tropical Stratospheric Zonal Wind in Global Atmospheric Reanalyses},
  author = {Kawatani, Y.},
  year = {2016},
  volume = {16},
  pages = {6681--6699},
  doi = {10.5194/acp-16-6681-2016},
  journal = {Atmospheric Chemistry and Physics},
  number = {11}
}

@article{kawataniRepresentation2016,
  title = {Representation of the Tropical Stratospheric Zonal Wind in Global Atmospheric Reanalyses},
  author = {Kawatani, Yoshio and Hamilton, Kevin and Miyazaki, Kazuyuki and Fujiwara, Masatomo and Anstey, James A.},
  year = {2016},
  month = jun,
  volume = {16},
  pages = {6681--6699},
  publisher = {{Copernicus GmbH}},
  issn = {1680-7316},
  doi = {10.5194/acp-16-6681-2016},
  abstract = {{$<$}p{$><$}strong class="journal-contentHeaderColor"{$>$}Abstract.{$<$}/strong{$>$} This paper reports on a project to compare the representation of the monthly-mean zonal wind in the equatorial stratosphere among major global atmospheric reanalysis data sets. The degree of disagreement among the reanalyses is characterized by the standard deviation (SD) of the monthly-mean zonal wind and this depends on latitude, longitude, height, and the phase of the quasi-biennial oscillation (QBO). At each height the SD displays a prominent equatorial maximum, indicating the particularly challenging nature of the reanalysis problem in the low-latitude stratosphere. At 50\&ndash;70 hPa the geographical distributions of SD are closely related to the density of radiosonde observations. The largest SD values are over the central Pacific, where few in situ observations are available. At 10\&ndash;20 hPa the spread among the reanalyses and differences with in situ observations both depend significantly on the QBO phase. Notably the easterly-to-westerly phase transitions in all the reanalyses except MERRA are delayed relative to those directly observed in Singapore. In addition, the timing of the easterly-to-westerly phase transitions displays considerable variability among the different reanalyses and this spread is much larger than for the timing of the westerly-to-easterly phase changes. The eddy component in the monthly-mean zonal wind near the Equator is dominated by zonal wavenumber 1 and 2 quasi-stationary planetary waves propagating from midlatitudes in the westerly phase of the QBO. There generally is considerable disagreement among the reanalyses in the details of the quasi-stationary waves near the Equator. At each level, there is a tendency for the agreement to be best near the longitude of Singapore, suggesting that the Singapore observations act as a strong constraint on all the reanalyses. Our measures of the quality of the reanalysis clearly show systematic improvement over the period considered (1979\textendash 2012). The SD among the reanalysis declines significantly over the record, although the geographical pattern of SD remains nearly constant.{$<$}/p{$>$}},
  file = {/Users/oscardimdore-miles/Zotero/storage/N3BPBIBS/Kawatani et al. - 2016 - Representation of the tropical stratospheric zonal.pdf;/Users/oscardimdore-miles/Zotero/storage/PI92HVBD/2016.html},
  journal = {Atmospheric Chemistry and Physics},
  language = {English},
  number = {11}
}

@article{Kidston2015,
  title = {Stratospheric Influence on Tropospheric Jet Streams, Storm Tracks and Surface Weather},
  author = {Kidston, Joseph and Scaife, Adam and Hardiman, Steven and Mitchell, Daniel and Butchart, Neal and Baldwin, Mark and Gray, L.},
  year = {2015},
  volume = {8},
  pages = {433--440},
  doi = {10.1038/ngeo2424},
  journal = {Nature Geoscience}
}

@article{Kim2020,
  title = {Characteristics of Stratospheric Polar Vortex Fluctuations Associated with Sea Ice Variability in the {{Arctic}} Winter},
  author = {Kim, Jinju and Kim, Kwang-Yul},
  year = {2020},
  volume = {54},
  doi = {10.1007/s00382-020-05191-9},
  journal = {Climate Dynamics}
}

@article{kingObserved2019,
  title = {Observed {{Relationships Between Sudden Stratospheric Warmings}} and {{European Climate Extremes}}},
  author = {King, Andrew D. and Butler, Amy H. and Jucker, Martin and Earl, Nick O. and Rudeva, Irina},
  year = {2019},
  volume = {124},
  pages = {13943--13961},
  issn = {2169-8996},
  doi = {10.1029/2019JD030480},
  abstract = {Sudden stratospheric warmings (SSWs) have been linked with anomalously cold temperatures at the surface in the middle to high latitudes of the Northern Hemisphere as climatological westerly winds in the stratosphere tend to weaken and turn easterly. However, previous studies have largely relied on reanalyses and model simulations to infer the role of SSWs on surface climate and SSW relationships with extremes have not been fully analyzed. Here, we use observed daily gridded temperature and precipitation data over Europe to comprehensively examine the response of climate extremes to the occurrence of SSWs. We show that for much of Scandinavia, winters with SSWs are on average at least 1 \textdegree C cooler, but the coldest day and night of winter is on average at least 2 \textdegree C colder than in non-SSW winters. Anomalously high pressure over Scandinavia reduces precipitation on the northern Atlantic coast but increases overall rainfall and the number of wet days in southern Europe. In the 60 days after SSWs, cold extremes are more intense over Scandinavia with anomalously high pressure and drier conditions prevailing. Over southern Europe there is a tendency toward lower pressure, increased precipitation and more wet days. The surface response in cold temperature extremes over northwest Europe to the 2018 SSW was stronger than observed for any SSW during 1979\textendash 2016. Our analysis shows that SSWs have an effect not only on mean climate but also extremes over much of Europe. Only with carefully designed analyses are the relationships between SSWs and climate means and extremes detectable above synoptic-scale variability.},
  file = {/Users/oscardimdore-miles/Zotero/storage/Y7TZMF77/King et al. - 2019 - Observed Relationships Between Sudden Stratospheri.pdf;/Users/oscardimdore-miles/Zotero/storage/BPYH2XYQ/2019JD030480.html},
  journal = {Journal of Geophysical Research: Atmospheres},
  keywords = {Cold extremes,Europe,Rainfall extremes,Snowfall,SSWs,Temperature extremes},
  language = {English},
  number = {24}
}

@article{knightSignature2005,
  title = {A Signature of Persistent Natural Thermohaline Circulation Cycles in Observed Climate},
  author = {Knight, Jeff R. and Allan, Robert J. and Folland, Chris K. and Vellinga, Michael and Mann, Michael E.},
  year = {2005},
  volume = {32},
  issn = {1944-8007},
  doi = {10.1029/2005GL024233},
  abstract = {Analyses of global climate from measurements dating back to the nineteenth century show an `Atlantic Multidecadal Oscillation' (AMO) as a leading large-scale pattern of multidecadal variability in surface temperature. Yet it is not possible to determine whether these fluctuations are genuinely oscillatory from the relatively short observational record alone. Using a 1400 year climate model calculation, we are able to simulate the observed pattern and amplitude of the AMO. The results imply the AMO is a genuine quasi-periodic cycle of internal climate variability persisting for many centuries, and is related to variability in the oceanic thermohaline circulation (THC). This relationship suggests we can attempt to reconstruct past THC changes, and we infer an increase in THC strength over the last 25 years. Potential predictability associated with the mode implies natural THC and AMO decreases over the next few decades independent of anthropogenic climate change.},
  copyright = {Copyright 2005 by the American Geophysical Union.},
  file = {/Users/oscardimdore-miles/Zotero/storage/YF2HMN8S/Knight et al. - 2005 - A signature of persistent natural thermohaline cir.pdf},
  journal = {Geophysical Research Letters},
  language = {English},
  number = {20}
}

@article{kolstadAssociation2010,
  title = {The Association between Stratospheric Weak Polar Vortex Events and Cold Air Outbreaks in the {{Northern Hemisphere}}},
  author = {Kolstad, Erik W. and Breiteig, Tarjei and Scaife, Adam A.},
  year = {2010},
  volume = {136},
  pages = {886--893},
  issn = {1477-870X},
  doi = {10.1002/qj.620},
  abstract = {Previous studies have identified an association between temperature anomalies in the Northern Hemisphere and the strength of stratospheric polar westerlies. Large regions in northern Asia, Europe and North America have been found to cool during the mature and late stages of weak vortex events in the stratosphere. A substantial part of the temperature changes are associated with changes in the Northern Annular Mode (NAM) and North Atlantic Oscillation (NAO) pressure patterns in the troposphere. The apparent coupling between the stratosphere and the troposphere may be of relevance for weather forecasting, but only if the temporal and spatial nature of the coupling is known. Using 51 winters of re-analysis data, we show that the development of the lower-tropospheric temperature relative to stratospheric weak polar vortex events goes through a series of well-defined stages, including the formation of geographically distinct cold air outbreaks. At the inception of weak vortex events, a precursor signal in the form of a strong high-pressure anomaly over northwest Eurasia is associated with long-lived and robust cold anomalies over Asia and Europe. A few weeks later, near the mature stage of the weak vortex events, a shorter-lived cold anomaly emerges off the east coast of North America. The probability of cold air outbreaks increases by more than 50\% in one or more of these regions during all phases of the weak vortex events. This shows that the stratospheric polar vortex contains information that can be used to enhance forecasts of cold air outbreaks. As large changes in the frequency of extremes are involved, this process is important for the medium-range and seasonal prediction of extreme cold winter days. Three-hundred-year pre-industrial control simulations by 13 coupled climate models corroborate our results. Copyright \textcopyright{} 2010 Royal Meteorological Society and Crown Copyright.},
  file = {/Users/oscardimdore-miles/Zotero/storage/HET74Y7F/Kolstad et al. - 2010 - The association between stratospheric weak polar v.pdf;/Users/oscardimdore-miles/Zotero/storage/E5CC58SL/qj.html},
  journal = {Quarterly Journal of the Royal Meteorological Society},
  keywords = {Arctic Oscillation,climate models,natural variability,North Atlantic Oscillation,stratosphere-troposphere interactions},
  language = {English},
  number = {649}
}

@article{kren2016,
  title = {Wintertime Northern Hemisphere Response in the Stratosphere to the Pacific Decadal Oscillation Using the Whole Atmosphere Community Climate Model},
  author = {Kren, A. C. and Marsh, D. R. and Smith, A. K. and Pilewskie, P.},
  year = {2016},
  volume = {29},
  pages = {1031--1049},
  doi = {10.1175/JCLI-D-15-0176.1},
  journal = {Journal of Climate},
  number = {3}
}

@article{krenWintertime2016,
  title = {Wintertime {{Northern Hemisphere Response}} in the {{Stratosphere}} to the {{Pacific Decadal Oscillation Using}} the {{Whole Atmosphere Community Climate Model}}},
  author = {Kren, A. C. and Marsh, D. R. and Smith, A. K. and Pilewskie, P.},
  year = {2016},
  month = feb,
  volume = {29},
  pages = {1031--1049},
  publisher = {{American Meteorological Society}},
  issn = {0894-8755, 1520-0442},
  doi = {10.1175/JCLI-D-15-0176.1},
  abstract = {{$<$}section class="abstract"{$><$}h2 class="abstractTitle text-title my-1" id="d346e2"{$>$}Abstract{$<$}/h2{$><$}p{$>$}The response of the Northern Hemisphere winter stratosphere to the Pacific decadal oscillation (PDO) is examined using the Whole Atmosphere Community Climate Model. A 200-yr preindustrial control simulation that includes fully interactive chemistry, ocean and sea ice, constant solar forcing, and greenhouse gases fixed to 1850 levels is analyzed. Based on principal component analysis, the PDO spatial pattern, frequency, and amplitude agree well with the observed PDO over the period 1900\textendash 2014. Consistent with previous studies, the positive phase of the PDO is marked by a strengthened Aleutian low and a wave train of geopotential height anomalies reminiscent of the Pacific\textendash North American pattern in the troposphere. In addition to a tropospheric signal, a zonal-mean warming of about 2 K in the northern polar stratosphere and a zonal-mean zonal wind decrease of about 4 m s{$^{-1}$} in the PDO positive phase are found. When compositing PDO positive or negative winters during neutral El Ni\~no years, the magnitude is reduced and depicts an early winter forcing of the stratosphere compared to a late winter response from El Ni\~no. Contamination between PDO and ENSO signals is also discussed. Stratospheric sudden warmings occur 63\% of the time in the PDO positive phase compared to 40\% in the negative phase. Although this sudden warming frequency is not statistically significant, it is quantitatively consistent with NCEP\textendash NCAR reanalysis data and recent observational evidence linking the PDO positive phase to weak stratospheric vortex events.{$<$}/p{$><$}/section{$>$}},
  chapter = {Journal of Climate},
  file = {/Users/oscardimdore-miles/Zotero/storage/MPH3QV36/Kren et al. - 2016 - Wintertime Northern Hemisphere Response in the Str.pdf;/Users/oscardimdore-miles/Zotero/storage/LDDHIS3D/jcli-d-15-0176.1.html},
  journal = {Journal of Climate},
  language = {English},
  number = {3}
}

@article{Kretschmer2018,
  title = {Stratospheric Influence on Tropospheric Jet Streams, Storm Tracks and Surface Weather},
  author = {Kretschmer, M. and Cohen, J. and Matthias, V. and Runge, J.and Coumou, D.},
  year = {2018},
  volume = {1},
  pages = {433--440},
  doi = {10.1038/s41612-018-0054-4},
  journal = {npj Climate and Atmospheric Science},
  number = {44}
}

@article{Krzywinski,
  title = {Multiple Linear Regression},
  author = {Krzywinski, M \& Altman, N},
  year = {2015},
  volume = {12},
  pages = {1103--1104},
  journal = {Nature Methods}
}

@article{kuhlbrodtDriving2007,
  title = {On the Driving Processes of the {{Atlantic}} Meridional Overturning Circulation},
  author = {Kuhlbrodt, T. and Griesel, A. and Montoya, M. and Levermann, A. and Hofmann, M. and Rahmstorf, S.},
  year = {2007},
  volume = {45},
  issn = {1944-9208},
  doi = {10.1029/2004RG000166},
  abstract = {Because of its relevance for the global climate the Atlantic meridional overturning circulation (AMOC) has been a major research focus for many years. Yet the question of which physical mechanisms ultimately drive the AMOC, in the sense of providing its energy supply, remains a matter of controversy. Here we review both observational data and model results concerning the two main candidates: vertical mixing processes in the ocean's interior and wind-induced Ekman upwelling in the Southern Ocean. In distinction to the energy source we also discuss the role of surface heat and freshwater fluxes, which influence the volume transport of the meridional overturning circulation and shape its spatial circulation pattern without actually supplying energy to the overturning itself in steady state. We conclude that both wind-driven upwelling and vertical mixing are likely contributing to driving the observed circulation. To quantify their respective contributions, future research needs to address some open questions, which we outline.},
  copyright = {Copyright 2007 by the American Geophysical Union.},
  file = {/Users/oscardimdore-miles/Zotero/storage/JEJB2VFY/Kuhlbrodt et al. - 2007 - On the driving processes of the Atlantic meridiona.pdf;/Users/oscardimdore-miles/Zotero/storage/MX7KI4UB/2004RG000166.html},
  journal = {Reviews of Geophysics},
  keywords = {diapycnal mixing,meridional overturning circulation,thermohaline circulation,wind-driven upwelling},
  language = {English},
  number = {2}
}

@article{kuttippurathComparative2012,
  title = {A Comparative Study of the Major Sudden Stratospheric Warmings in the {{Arctic}} Winters 2003/2004-2009/2010},
  author = {Kuttippurath, Jayanarayanan and Nikulin, Grigory},
  year = {2012},
  month = sep,
  volume = {12},
  pages = {8115--8129},
  doi = {10.5194/acp-12-8115-2012},
  abstract = {We present an analysis of the major sudden stratospheric warmings (SSWs) in the Arctic winters 2003/04-2009/10. There were 6 major SSWs (major warmings [MWs]) in 6 out of the 7 winters, in which the MWs of 2003/04, 2005/06, and 2008/09 were in January and those of 2006/07, 2007/08, and 2009/10 were in February. Although the winter 2009/10 was relatively cold from mid-December to mid-January, strong wave 1 activity led to a MW in early February, for which the largest momentum flux among the winters was estimated at 60\textdegree{} N/10 hPa, about 450 m2 s-2. The strongest MW, however, was observed in 2008/09 and the weakest in 2006/07. The MW in 2008/09 was triggered by intense wave 2 activity and was a vortex split event. In contrast, strong wave 1 activity led to the MWs of other winters and were vortex displacement events. Large amounts of Eliassen-Palm (EP) and wave 1/2 EP fluxes (about 2-4 \texttimes 105 kg s-2) are estimated shortly before the MWs at 100 hPa averaged over 45-75\textdegree{} N in all winters, suggesting profound tropospheric forcing for the MWs. We observe an increase in the occurrence of MWs (\textasciitilde 1.1 MWs/winter) in recent years (1998/99-2009/10), as there were 13 MWs in the 12 Arctic winters, although the long-term average (1957/58-2009/10) of the frequency stays around its historical value (\textasciitilde 0.7 MWs/winter), consistent with the findings of previous studies. An analysis of the chemical ozone loss in the past 17 Arctic winters (1993/94-2009/10) suggests that the loss is inversely proportional to the intensity and timing of MWs in each winter, where early (December-January) MWs lead to minimal ozone loss. Therefore, this high frequency of MWs in recent Arctic winters has significant implications for stratospheric ozone trends in the northern hemisphere.},
  file = {/Users/oscardimdore-miles/Zotero/storage/QNT9MH4L/Kuttippurath and Nikulin - 2012 - A comparative study of the major sudden stratosphe.pdf},
  journal = {Atmospheric Chemistry \& Physics}
}

@article{Kylash2015,
  title = {Synchronisation of the Equatorial {{QBO}} by the Annual Cycle in Tropical Upwelling in a Warming Climate},
  author = {Rajendran, Kylash and Moroz, Irene and Read, Peter and Osprey, Scott},
  year = {2015},
  doi = {10.1002/qj.2714},
  journal = {Quarterly Journal of the Royal Meteorological Society}
}

@article{labe2019,
  title = {The Effect of {{QBO}} Phase on the Atmospheric Response to Projected Arctic Sea Ice Loss in Early Winter},
  author = {Labe, Zachary and Peings, Yannick and Magnusdottir, Gudrun},
  year = {2019},
  volume = {46},
  pages = {7663--7671},
  journal = {Geophysical Research Letters},
  keywords = {Arctic sea ice,climate variability,Northern Annular Mode,polar vortex,QBO,teleconnection},
  number = {13}
}

@article{latifPerspective2011,
  title = {A Perspective on Decadal Climate Variability and Predictability},
  author = {Latif, Mojib and Keenlyside, Noel S.},
  year = {2011},
  month = sep,
  volume = {58},
  pages = {1880--1894},
  issn = {0967-0645},
  doi = {10.1016/j.dsr2.2010.10.066},
  abstract = {The global surface air temperature record of the last 150 years is characterized by a long-term warming trend, with strong multidecadal variability superimposed. Similar multidecadal variability is also seen in other (societal important) parameters such as Sahel rainfall or Atlantic hurricane activity. The existence of the multidecadal variability makes climate change detection a challenge, since global warming evolves on a similar timescale. The ongoing discussion about a potential anthropogenic signal in the Atlantic hurricane activity is an example. A lot of work was devoted during the last years to understand the dynamics of the multidecadal variability, and external and internal mechanisms were proposed. This review paper focuses on two aspects. First, it describes the mechanisms for internal variability using a stochastic framework. Specific attention is given to variability of the Atlantic Meridional Overturning Circulation (AMOC), which is likely the origin of a considerable part of decadal variability and predictability in the Atlantic Sector. Second, the paper discusses decadal predictability and the factors limiting its realization. These include a poor understanding of the mechanisms involved and large biases in state-of-the-art climate models. Enhanced model resolution, improved subgrid scale parameterisations, and the inclusion of additional climate subsystems, such as a resolved stratosphere, may help overcome these limitations.},
  file = {/Users/oscardimdore-miles/Zotero/storage/9RPF6KT7/Latif and Keenlyside - 2011 - A perspective on decadal climate variability and p.pdf},
  journal = {Deep Sea Research Part II: Topical Studies in Oceanography},
  keywords = {Atlantic Meridional Overturning Circulation,Atlantic Multidecadal Variability,Climate modelling,Climate prediction,Decadal climate variability,Systematic model error},
  language = {English},
  number = {17},
  series = {Climate and the {{Atlantic Meridional Overturning Circulation}}}
}

@article{Lau1995,
  title = {Climate Signal Detection Using Wavelet Transform: {{How}} to Make a Time Series Sing},
  author = {Lau, K.-M. and Weng, Hengyi},
  year = {1995},
  volume = {76},
  pages = {2391--2402},
  doi = {10.1175/1520-0477},
  journal = {Bulletin of the American Meteorological Society},
  number = {12}
}

@article{lauClimate1995,
  title = {Climate {{Signal Detection Using Wavelet Transform}}: {{How}} to {{Make}} a {{Time Series Sing}}},
  shorttitle = {Climate {{Signal Detection Using Wavelet Transform}}},
  author = {Lau, K.-M. and Weng, Hengyi},
  year = {1995},
  month = dec,
  volume = {76},
  pages = {2391--2402},
  publisher = {{American Meteorological Society}},
  issn = {0003-0007, 1520-0477},
  doi = {10.1175/1520-0477(1995)076<2391:CSDUWT>2.0.CO;2},
  abstract = {{$<$}section class="abstract"{$><$}p{$>$}In this paper, the application of the wavelet transform (WT) to climate time series analyses is introduced. A tutorial description of the basic concept of WT, compared with similar concepts used in music, is also provided. Using an analogy between WT representation of a time series and a music score, the authors illustrate the importance of {$<$}em{$>$}local{$<$}/em{$>$} versus {$<$}em{$>$}global{$<$}/em{$>$} information in the time\textendash frequency localization of climate signals. Examples of WT applied to climate data analysis are demonstrated using analytic signals as well as real climate time series. Results of WT applied to two climate time series\textemdash that is, a proxy paleoclimate time series with a 2.5-Myr deep-sea sediment record of {$\delta$}18 {$<$}em{$>$}O{$<$}/em{$>$} and a 140-yr monthly record of Northern Hemisphere surface temperature\textemdash are presented. The former shows the presence of a 40-kyr and a 100-kyr oscillation and an abrupt transition in the oscillation regime at 0.7 Myr before the present, consistent with previous studies. The latter possesses a myriad of oscillatory modes from interannual (2\textendash 5 yr), interdecadal (10\textendash 12 yr, 20\textendash 25 yr, and 40\textendash 60 yr), and century (\textasciitilde 180 yr) scales at different periods of the data record. In spite of the large difference in timescales, common features in time\textendash frequency characteristics of these two time series have been identified. These features suggest that the variations of the earth's climate are consistent with those exhibited by a nonlinear dynamical system under external forcings.{$<$}/p{$><$}/section{$>$}},
  chapter = {Bulletin of the American Meteorological Society},
  file = {/Users/oscardimdore-miles/Zotero/storage/XRN7XXRX/Lau and Weng - 1995 - Climate Signal Detection Using Wavelet Transform .pdf;/Users/oscardimdore-miles/Zotero/storage/DVTSPV6Z/1520-0477_1995_076_2391_csduwt_2_0_co_2.html},
  journal = {Bulletin of the American Meteorological Society},
  language = {English},
  number = {12}
}

@article{lavoieProjections2019,
  title = {Projections of {{Future Trends}} in {{Biogeochemical Conditions}} in the {{Northwest Atlantic Using CMIP5 Earth System Models}}},
  author = {Lavoie, Diane and Lambert, Nicolas and Gilbert, Denis},
  year = {2019},
  month = jan,
  volume = {57},
  pages = {18--40},
  publisher = {{Taylor \& Francis}},
  issn = {0705-5900},
  doi = {10.1080/07055900.2017.1401973},
  abstract = {We use the results from eight of the Earth System Models (ESMs) made available for the Fifth Assessment Report of the Intergovernmental Panel on Climate Change to analyze the projected changes in biogeochemical conditions over the next 50 years in the northwest Atlantic. We looked at the projected changes using the Representative Concentration Pathway 8.5 scenario in the 100\textendash 400 m depth range over a large region and at more specific locations to assess the relevance of using these outputs to force a regional climate downscaling model of the Gulf of St. Lawrence. The projected trends for dissolved oxygen (decrease), pH (decrease), and nitrate (variable although negative in general) represent a continuation of the recently observed trends in the area. For primary production, no firm conclusions can be drawn because of large differences in the trends from one model to another. The consistency of the trends near the regional model lateral boundaries leads us to conclude that the ESM trends can be used to set up future boundary conditions to evaluate regional impacts of climate change although the uncertainty of the results for the Scotian Shelf will be greater than for the Gulf of St. Lawrence.},
  file = {/Users/oscardimdore-miles/Zotero/storage/QIIJBC6X/Lavoie et al. - 2019 - Projections of Future Trends in Biogeochemical Con.pdf;/Users/oscardimdore-miles/Zotero/storage/FI6YZKBR/07055900.2017.html},
  journal = {Atmosphere-Ocean},
  keywords = {2016–2065,climate change,dissolved oxygen,Gulf of St. Lawrence,lateral boundaries,nitrate,pH,regional model,Scotian Shelf},
  number = {1}
}

@article{lawrenceRemarkably2020,
  title = {The {{Remarkably Strong Arctic Stratospheric Polar Vortex}} of {{Winter}} 2020: {{Links}} to {{Record}}-{{Breaking Arctic Oscillation}} and {{Ozone Loss}}},
  shorttitle = {The {{Remarkably Strong Arctic Stratospheric Polar Vortex}} of {{Winter}} 2020},
  author = {Lawrence, Zachary D. and Perlwitz, Judith and Butler, Amy H. and Manney, Gloria L. and Newman, Paul A. and Lee, Simon H. and Nash, Eric R.},
  year = {2020},
  volume = {125},
  pages = {e2020JD033271},
  issn = {2169-8996},
  doi = {10.1029/2020JD033271},
  abstract = {The Northern Hemisphere (NH) polar winter stratosphere of 2019/2020 featured an exceptionally strong and cold stratospheric polar vortex. Wave activity from the troposphere during December\textendash February was unusually low, which allowed the polar vortex to remain relatively undisturbed. Several transient wave pulses nonetheless served to help create a reflective configuration of the stratospheric circulation by disturbing the vortex in the upper stratosphere. Subsequently, multiple downward wave coupling events took place, which aided in dynamically cooling and strengthening the polar vortex. The persistent strength of the stratospheric polar vortex was accompanied by an unprecedentedly positive phase of the Arctic Oscillation in the troposphere during January\textendash March, which was consistent with large portions of observed surface temperature and precipitation anomalies during the season. Similarly, conditions within the strong polar vortex were ripe for allowing substantial ozone loss: The undisturbed vortex was a strong transport barrier, and temperatures were low enough to form polar stratospheric clouds for over 4 months into late March. Total column ozone amounts in the NH polar cap decreased and were the lowest ever observed in the February\textendash April period. The unique confluence of conditions and multiple broken records makes the 2019/2020 winter and early spring a particularly extreme example of two-way coupling between the troposphere and stratosphere.},
  copyright = {\textcopyright 2020. American Geophysical Union. All Rights Reserved.},
  file = {/Users/oscardimdore-miles/Zotero/storage/WHXMRZKU/2020JD033271.html},
  journal = {Journal of Geophysical Research: Atmospheres},
  keywords = {Arctic Oscillation,downward wave coupling,planetary waves,stratospheric ozone,stratospheric polar vortex},
  language = {English},
  number = {22}
}

@article{Lehtonen,
  title = {Observed and Modeled Tropospheric Cold Anomalies Associated with Sudden Stratospheric Warmings},
  author = {Lehtonen, Ilari and Karpechko, Alexey Yu.},
  year = {2016},
  volume = {121},
  pages = {1591--1610},
  journal = {Journal of Geophysical Research: Atmospheres},
  number = {4}
}

@article{lehtonenObserved2016,
  title = {Observed and Modeled Tropospheric Cold Anomalies Associated with Sudden Stratospheric Warmings},
  author = {Lehtonen, Ilari and Karpechko, Alexey Yu},
  year = {2016},
  volume = {121},
  pages = {1591--1610},
  issn = {2169-8996},
  doi = {10.1002/2015JD023860},
  abstract = {Surface weather patterns related to 35 major sudden stratospheric warmings (SSWs) in 1958\textendash 2010 are analyzed based on reanalysis data. Similar analyses are conducted with data from seven stratosphere-resolving Earth system models. The analyses are carried out separately for displacement and splitting SSWs. On the basis of the observational analysis, it is shown that in northern Eurasia, the cold anomalies linked to the SSWs tend to be stronger and more widespread before the central date of the SSWs than during the first 2 months after the event central dates. This is particularly true for the displacement events. The cold anomalies preceding the SSWs are coupled to atmospheric blocking events which trigger the SSWs. While the role of SSWs as important predictors of cold air outbreaks in the Northern Hemisphere is well recognized, our results indicate that the impact of the preceding blocking on near-surface temperatures is, in fact, widely more significant than the downward impact of the SSWs. Thus, stratosphere-troposphere coupling provides only limited predictability for cold air outbreaks in Eurasia. The models reproduce qualitatively well the typical large-scale surface weather patterns following the SSWs, but they largely miss the cooling preceding the SSWs over Europe and western Siberia. Hence, the strongest modeled temperature anomalies related to the SSWs occur after the events. Moreover, the model results indicate that the tropospheric response to SSWs is stronger following split events. At the same time, many models simulate too few splitting SSWs.},
  file = {/Users/oscardimdore-miles/Zotero/storage/C8QPTS4D/Lehtonen and Karpechko - 2016 - Observed and modeled tropospheric cold anomalies a.pdf;/Users/oscardimdore-miles/Zotero/storage/AF5CRKK3/2015JD023860.html},
  journal = {Journal of Geophysical Research: Atmospheres},
  keywords = {CMIP5,stratosphere-troposphere interactions,sudden stratospheric warmings,surface weather anomalies},
  language = {English},
  number = {4}
}

@article{limpasuvanLife2004,
  title = {The {{Life Cycle}} of the {{Northern Hemisphere Sudden Stratospheric Warmings}}},
  author = {Limpasuvan, Varavut and Thompson, David W. J. and Hartmann, Dennis L.},
  year = {2004},
  month = jul,
  volume = {17},
  pages = {2584--2596},
  publisher = {{American Meteorological Society}},
  issn = {0894-8755, 1520-0442},
  doi = {10.1175/1520-0442(2004)017<2584:TLCOTN>2.0.CO;2},
  abstract = {{$<$}section class="abstract"{$><$}h2 class="abstractTitle text-title my-1" id="d474e2"{$>$}Abstract{$<$}/h2{$><$}p{$>$}Motivated by recent evidence of strong stratospheric\textendash tropospheric coupling during the Northern Hemisphere winter, this study examines the evolution of the atmospheric flow and wave fluxes at levels throughout the stratosphere and troposphere during the composite life cycle of a sudden stratospheric warming. The composite comprises 39 major and minor warming events using 44 years of NCEP\textendash NCAR reanalysis data. The incipient stage of the life cycle is characterized by preconditioning of the stratospheric zonal flow and anomalous, quasi-stationary wavenumber-1 forcing in both the stratosphere and troposphere. As the life cycle intensifies, planetary wave driving gives rise to weakening of the stratospheric polar vortex and downward propagation of the attendant easterly wind and positive temperature anomalies. When these anomalies reach the tropopause, the life cycle is marked by momentum flux and mean meridional circulation anomalies at tropospheric levels that are consistent with the negative phase of the Northern Hemisphere annular mode. The anomalous momentum fluxes are largest over the Atlantic half of the hemisphere and are associated primarily with waves of wavenumber 3 and higher.{$<$}/p{$><$}/section{$>$}},
  chapter = {Journal of Climate},
  file = {/Users/oscardimdore-miles/Zotero/storage/ZXG6G3E9/Limpasuvan et al. - 2004 - The Life Cycle of the Northern Hemisphere Sudden S.pdf;/Users/oscardimdore-miles/Zotero/storage/EXPIDYLE/1520-0442_2004_017_2584_tlcotn_2.0.co_2.html},
  journal = {Journal of Climate},
  language = {English},
  number = {13}
}

@article{liuMechanisms2019,
  title = {The {{Mechanisms}} of the {{Atlantic Meridional Overturning Circulation Slowdown Induced}} by {{Arctic Sea Ice Decline}}},
  author = {Liu, Wei and Fedorov, Alexey and S{\'e}vellec, Florian},
  year = {2019},
  month = feb,
  volume = {32},
  pages = {977--996},
  publisher = {{American Meteorological Society}},
  issn = {0894-8755, 1520-0442},
  doi = {10.1175/JCLI-D-18-0231.1},
  abstract = {{$<$}section class="abstract"{$><$}h2 class="abstractTitle text-title my-1" id="d89610081e98"{$>$}Abstract{$<$}/h2{$><$}p{$>$}We explore the mechanisms by which Arctic sea ice decline affects the Atlantic meridional overturning circulation (AMOC) in a suite of numerical experiments perturbing the Arctic sea ice radiative budget within a fully coupled climate model. The imposed perturbations act to increase the amount of heat available to melt ice, leading to a rapid Arctic sea ice retreat within 5 years after the perturbations are activated. In response, the AMOC gradually weakens over the next \textasciitilde 100 years. The AMOC changes can be explained by the accumulation in the Arctic and subsequent downstream propagation to the North Atlantic of buoyancy anomalies controlled by temperature and salinity. Initially, during the first decade or so, the Arctic sea ice loss results in anomalous positive heat and salinity fluxes in the subpolar North Atlantic, inducing positive temperature and salinity anomalies over the regions of oceanic deep convection. At first, these anomalies largely compensate one another, leading to a minimal change in upper ocean density and deep convection in the North Atlantic. Over the following years, however, more anomalous warm water accumulates in the Arctic and spreads to the North Atlantic. At the same time, freshwater that accumulates from seasonal sea ice melting over most of the upper Arctic Ocean also spreads southward, reaching as far as south of Iceland. These warm and fresh anomalies reduce upper ocean density and suppress oceanic deep convection. The thermal and haline contributions to these buoyancy anomalies, and therefore to the AMOC slowdown during this period, are found to have similar magnitudes. We also find that the related changes in horizontal wind-driven circulation could potentially push freshwater away from the deep convection areas and hence strengthen the AMOC, but this effect is overwhelmed by mean advection.{$<$}/p{$><$}/section{$>$}},
  chapter = {Journal of Climate},
  file = {/Users/oscardimdore-miles/Zotero/storage/TZJDNL89/Liu et al. - 2019 - The Mechanisms of the Atlantic Meridional Overturn.pdf;/Users/oscardimdore-miles/Zotero/storage/ZUPHUWYU/jcli-d-18-0231.1.html},
  journal = {Journal of Climate},
  language = {English},
  number = {4}
}

@article{liuOverlooked2017,
  title = {Overlooked Possibility of a Collapsed {{Atlantic Meridional Overturning Circulation}} in Warming Climate},
  author = {Liu, Wei and Xie, Shang-Ping and Liu, Zhengyu and Zhu, Jiang},
  year = {2017},
  month = jan,
  volume = {3},
  pages = {e1601666},
  publisher = {{American Association for the Advancement of Science}},
  issn = {2375-2548},
  doi = {10.1126/sciadv.1601666},
  abstract = {Changes in the Atlantic Meridional Overturning Circulation (AMOC) are moderate in most climate model projections under increasing greenhouse gas forcing. This intermodel consensus may be an artifact of common model biases that favor a stable AMOC. Observationally based freshwater budget analyses suggest that the AMOC is in an unstable regime susceptible for large changes in response to perturbations. By correcting the model biases, we show that the AMOC collapses 300 years after the atmospheric CO2 concentration is abruptly doubled from the 1990 level. Compared to an uncorrected model, the AMOC collapse brings about large, markedly different climate responses: a prominent cooling over the northern North Atlantic and neighboring areas, sea ice increases over the Greenland-Iceland-Norwegian seas and to the south of Greenland, and a significant southward rain-belt migration over the tropical Atlantic. Our results highlight the need to develop dynamical metrics to constrain models and the importance of reducing model biases in long-term climate projection. Climate models underestimate the possibility of a collapsed Atlantic Meridional Overturning Circulation under climate warming. Climate models underestimate the possibility of a collapsed Atlantic Meridional Overturning Circulation under climate warming.},
  chapter = {Research Article},
  copyright = {Copyright \textcopyright{} 2017, The Authors. This is an open-access article distributed under the terms of the Creative Commons Attribution-NonCommercial license, which permits use, distribution, and reproduction in any medium, so long as the resultant use is not for commercial advantage and provided the original work is properly cited.},
  file = {/Users/oscardimdore-miles/Zotero/storage/K7UVQ89I/Liu et al. - 2017 - Overlooked possibility of a collapsed Atlantic Mer.pdf;/Users/oscardimdore-miles/Zotero/storage/YR9YA4MX/e1601666.html},
  journal = {Science Advances},
  language = {English},
  number = {1}
}

@article{liuRectification2007b,
  title = {Rectification of the {{Bias}} in the {{Wavelet Power Spectrum}}},
  author = {Liu, Yonggang and Liang, X. San and Weisberg, Robert H.},
  year = {2007},
  month = dec,
  volume = {24},
  pages = {2093--2102},
  publisher = {{American Meteorological Society}},
  issn = {0739-0572, 1520-0426},
  doi = {10.1175/2007JTECHO511.1},
  abstract = {{$<$}section class="abstract"{$><$}h2 class="abstractTitle text-title my-1" id="d155726553e72"{$>$}Abstract{$<$}/h2{$><$}p{$>$}This paper addresses a bias problem in the estimate of wavelet power spectra for atmospheric and oceanic datasets. For a time series comprised of sine waves with the same amplitude at different frequencies the conventionally adopted wavelet method does not produce a spectrum with identical peaks, in contrast to a Fourier analysis. The wavelet power spectrum in this definition, that is, the transform coefficient squared (to within a constant factor), is equivalent to the integration of energy (in physical space) over the influence period (time scale) the series spans. Thus, a physically consistent definition of energy for the wavelet power spectrum should be the transform coefficient squared divided by the scale it associates. Such adjusted wavelet power spectrum results in a substantial improvement in the spectral estimate, allowing for a comparison of the spectral peaks across scales. The improvement is validated with an artificial time series and a real coastal sea level record. Also examined is the previous example of the wavelet analysis of the Ni\~no-3 SST data.{$<$}/p{$><$}/section{$>$}},
  chapter = {Journal of Atmospheric and Oceanic Technology},
  file = {/Users/oscardimdore-miles/Zotero/storage/XQNIRB83/Liu et al. - 2007 - Rectification of the Bias in the Wavelet Power Spe.pdf;/Users/oscardimdore-miles/Zotero/storage/BHNAB8HA/2007jtecho511_1.html},
  journal = {Journal of Atmospheric and Oceanic Technology},
  language = {English},
  number = {12}
}

@article{lohmannPossible2009,
  title = {A Possible Mechanism for the Strong Weakening of the {{North Atlantic}} Subpolar Gyre in the Mid-1990s},
  author = {Lohmann, Katja and Drange, Helge and Bentsen, Mats},
  year = {2009},
  volume = {36},
  issn = {1944-8007},
  doi = {10.1029/2009GL039166},
  abstract = {The extent and strength of the North Atlantic subpolar gyre (SPG) changed rapidly in the mid-1990s, going from large and strong in 1995 to substantially weakened in the following years. The abrupt change in the intensity of the SPG is commonly linked to the reversal of the North Atlantic Oscillation (NAO) index, changing from strong positive to negative values, in the winter 1995/96. In this study we investigate the impact of the initial SPG state on the subsequent behavior of the SPG by means of an ocean general circulation model driven by NCEP-NCAR reanalysis fields. Our sensitivity integrations suggest that the weakening of the SPG cannot be explained by the change in the atmospheric forcing alone. Rather, for the time period around 1995, the SPG was about to weaken, irrespective of the actual atmospheric forcing, due to the ocean state governed by the persistently strong positive NAO during the preceding seven years (1989\textendash 1995). Our analysis indicates that it was this preconditioning of the ocean, in combination with the sudden drop in the NAO in 1995/96, that lead to the strong and rapid weakening of the SPG in the second half of the 1990s. This hypothesis explains the diverging evolution of the strength of the SPG and the atmospheric forcing (winter NAO) after 1995, as has been suggested recently.},
  copyright = {Copyright 2009 by the American Geophysical Union.},
  file = {/Users/oscardimdore-miles/Zotero/storage/NJYH5EQZ/Lohmann et al. - 2009 - A possible mechanism for the strong weakening of t.pdf},
  journal = {Geophysical Research Letters},
  keywords = {North Atlantic subpolar gyre,ocean's response to NAO},
  language = {English},
  number = {15}
}

@article{Lu14,
  title = {Mechanisms for the {{Holton}}-{{Tan}} Relationship and Its Decadal Variation},
  author = {Lu, Hua and Bracegirdle, Thomas J. and Phillips, Tony and Bushell, Andrew and Gray, Lesley},
  year = {2014},
  volume = {119},
  pages = {2811--2830},
  doi = {10.1002/2013JD021352},
  journal = {Journal of Geophysical Research: Atmospheres},
  number = {6}
}

@article{Lu2008,
  title = {Decadal-Scale Changes in the Effect of the {{QBO}} on the Northern Stratospheric Polar Vortex},
  author = {Lu, Hua and Baldwin, Mark and Gray, L. and Jarvis, Martin},
  year = {2008},
  volume = {113},
  pages = {102--116},
  doi = {10.1029/2007JD009647},
  journal = {Journal of Geophysical Research Atmospheres},
  number = {10}
}

@article{luDecadalscale2008,
  title = {Decadal-Scale Changes in the Effect of the {{QBO}} on the Northern Stratospheric Polar Vortex},
  author = {Lu, Hua and Baldwin, Mark P. and Gray, Lesley J. and Jarvis, Martin J.},
  year = {2008},
  volume = {113},
  issn = {2156-2202},
  doi = {10.1029/2007JD009647},
  abstract = {This study documents decadal-scale changes in the Holton and Tan (HT) relationship, i.e., the influence of the lower stratospheric equatorial quasi-biennial oscillation (QBO) on the northern hemisphere (NH) extratropical circulation. Using a combination of ECMWF ERA-40 Reanalysis and Operational data from 1958\textendash 2006, we find that the Arctic stratosphere is indeed warmer under easterly QBO and colder under westerly QBO. During November to January, composite easterly minus westerly QBO signals in zonal wind extend from the lower stratosphere to the upper stratosphere and are centered at {$\sim$}5 hPa, 55\textendash 65\textdegree N with a magnitude of {$\sim$}10 m s-1. In temperature, the maximum signal is near {$\sim$}20\textendash 30 hPa at the pole with a magnitude of {$\sim$}4 K. During winter, the dominant feature is a poleward and downward transfer of wind and temperature anomalies from the midlatitude upper stratosphere to the high latitude lower stratosphere. For the first time, a statistically significant decadal scale change of the HT relationship during 1977\textendash 1997 is diagnosed. The main feature of the change is that the extratropical QBO signals reverse sign in late winter, resulting in fewer and delayed major stratospheric sudden warmings (SSWs), which occurred more often under westerly QBO. Consistent with earlier studies, it is found that the HT relationship is significantly stronger under solar minima overall, but the solar cycle does not appear to be the primary cause for the detected decadal-scale change. Possible mechanisms related to changes in planetary wave forcing are discussed.},
  copyright = {Copyright 2008 by the American Geophysical Union.},
  file = {/Users/oscardimdore-miles/Zotero/storage/L4A4JH2W/Lu et al. - 2008 - Decadal-scale changes in the effect of the QBO on .pdf;/Users/oscardimdore-miles/Zotero/storage/3W2TL6S8/2007JD009647.html},
  journal = {Journal of Geophysical Research: Atmospheres},
  keywords = {decadal-scale changes,Equatorial quasi-biennial oscillation,stratospheric polar vortex},
  language = {English},
  number = {D10}
}

@article{luDecadalscale2008a,
  title = {Decadal-Scale Changes in the Effect of the {{QBO}} on the Northern Stratospheric Polar Vortex},
  author = {Lu, Hua and Baldwin, Mark P. and Gray, Lesley J. and Jarvis, Martin J.},
  year = {2008},
  volume = {113},
  issn = {2156-2202},
  doi = {10.1029/2007JD009647},
  abstract = {This study documents decadal-scale changes in the Holton and Tan (HT) relationship, i.e., the influence of the lower stratospheric equatorial quasi-biennial oscillation (QBO) on the northern hemisphere (NH) extratropical circulation. Using a combination of ECMWF ERA-40 Reanalysis and Operational data from 1958\textendash 2006, we find that the Arctic stratosphere is indeed warmer under easterly QBO and colder under westerly QBO. During November to January, composite easterly minus westerly QBO signals in zonal wind extend from the lower stratosphere to the upper stratosphere and are centered at {$\sim$}5 hPa, 55\textendash 65\textdegree N with a magnitude of {$\sim$}10 m s-1. In temperature, the maximum signal is near {$\sim$}20\textendash 30 hPa at the pole with a magnitude of {$\sim$}4 K. During winter, the dominant feature is a poleward and downward transfer of wind and temperature anomalies from the midlatitude upper stratosphere to the high latitude lower stratosphere. For the first time, a statistically significant decadal scale change of the HT relationship during 1977\textendash 1997 is diagnosed. The main feature of the change is that the extratropical QBO signals reverse sign in late winter, resulting in fewer and delayed major stratospheric sudden warmings (SSWs), which occurred more often under westerly QBO. Consistent with earlier studies, it is found that the HT relationship is significantly stronger under solar minima overall, but the solar cycle does not appear to be the primary cause for the detected decadal-scale change. Possible mechanisms related to changes in planetary wave forcing are discussed.},
  copyright = {Copyright 2008 by the American Geophysical Union.},
  file = {/Users/oscardimdore-miles/Zotero/storage/USYHSUAH/Lu et al. - 2008 - Decadal-scale changes in the effect of the QBO on .pdf;/Users/oscardimdore-miles/Zotero/storage/MDHXXU2Z/2007JD009647.html},
  journal = {Journal of Geophysical Research: Atmospheres},
  keywords = {decadal-scale changes,Equatorial quasi-biennial oscillation,stratospheric polar vortex},
  language = {English},
  number = {D10}
}

@article{luMechanisms2014,
  title = {Mechanisms for the {{Holton}}-{{Tan}} Relationship and Its Decadal Variation},
  author = {Lu, Hua and Bracegirdle, Thomas J. and Phillips, Tony and Bushell, Andrew and Gray, Lesley},
  year = {2014},
  volume = {119},
  pages = {2811--2830},
  issn = {2169-8996},
  doi = {10.1002/2013JD021352},
  abstract = {This study provides a mechanistic explanation of why the Holton-Tan (HT) effect, a phenomenon in which the strength of northern stratospheric winter polar vortex synchronizes with the equatorial quasi-biennial oscillation (QBO), was disrupted in the middle to late winters of 1978\textendash 1997. In line with recent reassessments of the HT effect, we find that an easterly QBO in the lower stratosphere leads to the formation of a midlatitude wave guide that enhances both the upward propagating planetary waves from the troposphere into the lower stratosphere ( 35\textendash 50\textdegree N, 30\textendash 200 hPa) and the northward wave propagation in the upper to middle stratosphere ( 35\textendash 60\textdegree N, 20\textendash 5 hPa). This enhanced poleward refraction of planetary waves results in a more disturbed polar vortex, causing the HT effect. The weakening of the HT effect in 1978\textendash 1997 was associated with a broader and strengthened polar vortex in November to January. The divergence of wave activity generated by eddies growing within the vortex provided the momentum source and allowed wave activity to propagate meridionally away from the vortex; this interfered with the QBO modulation of planetary wave propagation and led to a weakening of the HT effect during this period. The stronger than average polar vortex in 1978\textendash 1997 was associated with a vertically coherent cooling signature over northeastern Asia in the stratosphere. We suggest that a change of stratospheric circulation and/or a change of the stratosphere-troposphere coupling were the main causes for the disrupted HT effect in 1978\textendash 1997.},
  copyright = {\textcopyright 2014. American Geophysical Union. All Rights Reserved.},
  file = {/Users/oscardimdore-miles/Zotero/storage/5HYEFZD5/Lu et al. - 2014 - Mechanisms for the Holton-Tan relationship and its.pdf;/Users/oscardimdore-miles/Zotero/storage/I6PU22I5/2013JD021352.html},
  journal = {Journal of Geophysical Research: Atmospheres},
  keywords = {decadal variation,Holton-Tan effect,QBO,stratospheric circulation change},
  language = {English},
  number = {6}
}

@article{Manney2005,
  title = {The Remarkable 2003\textendash 2004 Winter and Other Recent Warm Winters in the {{Arctic}} Stratosphere since the Late 1990s},
  author = {Manney, Gloria L. and Kr{\"u}ger, Kirstin and Sabutis, Joseph L. and Sena, Sara Amina and Pawson, Steven},
  year = {2005},
  volume = {110},
  doi = {10.1029/2004JD005367},
  journal = {Journal of Geophysical Research: Atmospheres},
  keywords = {interannual variability,stratospheric temperatures,stratospheric warmings},
  number = {D4}
}

@article{manneyAura2009,
  title = {Aura {{Microwave Limb Sounder}} Observations of Dynamics and Transport during the Record-Breaking 2009 {{Arctic}} Stratospheric Major Warming},
  author = {Manney, Gloria L. and Schwartz, Michael J. and Kr{\"u}ger, Kirstin and Santee, Michelle L. and Pawson, Steven and Lee, Jae N. and Daffer, William H. and Fuller, Ryan A. and Livesey, Nathaniel J.},
  year = {2009},
  volume = {36},
  issn = {1944-8007},
  doi = {10.1029/2009GL038586},
  abstract = {A major stratospheric sudden warming (SSW) in January 2009 was the strongest and most prolonged on record. Aura Microwave Limb Sounder (MLS) observations are used to provide an overview of dynamics and transport during the 2009 SSW, and to compare with the intense, long-lasting SSW in January 2006. The Arctic polar vortex split during the 2009 SSW, whereas the 2006 SSW was a vortex displacement event. Winds reversed to easterly more rapidly and reverted to westerly more slowly in 2009 than in 2006. More mixing of trace gases out of the vortex during the decay of the vortex fragments, and less before the fulfillment of major SSW criteria, was seen in 2009 than in 2006; persistent well-defined fragments of vortex and anticyclone air were more prevalent in 2009. The 2009 SSW had a more profound impact on the lower stratosphere than any previously observed SSW, with no significant recovery of the vortex in that region. The stratopause breakdown and subsequent reformation at very high altitude, accompanied by enhanced descent into a rapidly strengthening upper stratospheric vortex, were similar in 2009 and 2006. Many differences between 2006 and 2009 appear to be related to the different character of the SSWs in the two years.},
  copyright = {Copyright 2009 by the American Geophysical Union.},
  file = {/Users/oscardimdore-miles/Zotero/storage/2YTG6LPS/Manney et al. - 2009 - Aura Microwave Limb Sounder observations of dynami.pdf;/Users/oscardimdore-miles/Zotero/storage/K2AGEC3C/2009GL038586.html},
  journal = {Geophysical Research Letters},
  keywords = {satellite data,stratospheric sudden warming},
  language = {English},
  number = {12}
}

@article{manneyRemarkable2005,
  title = {The Remarkable 2003\textendash 2004 Winter and Other Recent Warm Winters in the {{Arctic}} Stratosphere since the Late 1990s},
  author = {Manney, Gloria L. and Kr{\"u}ger, Kirstin and Sabutis, Joseph L. and Sena, Sara Amina and Pawson, Steven},
  year = {2005},
  volume = {110},
  issn = {2156-2202},
  doi = {10.1029/2004JD005367},
  abstract = {The 2003\textendash 2004 Arctic winter was remarkable in the {$\sim$}50-year record of meteorological analyses. A major warming beginning in early January 2004 led to nearly 2 months of vortex disruption with high-latitude easterlies in the middle to lower stratosphere. The upper stratospheric vortex broke up in late December, but began to recover by early January, and in February and March was the strongest since regular observations began in 1979. The lower stratospheric vortex broke up in late January. Comparison with 2 previous years, 1984\textendash 1985 and 1986\textendash 1987, with prolonged midwinter warming periods shows unique characteristics of the 2003\textendash 2004 warming period: The length of the vortex disruption, the strong and rapid recovery in the upper stratosphere, and the slow progression of the warming from upper to lower stratosphere. January 2004 zonal mean winds in the middle and lower stratosphere were over 2 standard deviations below average. Examination of past variability shows that the recent frequency of major stratospheric warmings (7 in the past 6 years) is unprecedented. Lower stratospheric temperatures were unusually high during 6 of the past 7 years, with 5 having much lower than usual potential for polar stratospheric cloud (PSC) formation and ozone loss (nearly none in 1998\textendash 1999, 2001\textendash 2002, and 2003\textendash 2004, and very little in 1997\textendash 1998 and 2000\textendash 2001). Middle and upper stratospheric temperatures, however, were unusually low during and after February. The pattern of 5 of the last 7 years with very low PSC potential would be expected to occur randomly once every {$\sim$}850 years. This cluster of warm winters, immediately following a period of unusually cold winters, may have important implications for possible changes in interannual variability and for determination and attribution of trends in stratospheric temperatures and ozone.},
  copyright = {Copyright 2005 by the American Geophysical Union.},
  file = {/Users/oscardimdore-miles/Zotero/storage/Z6ZLN9JG/Manney et al. - 2005 - The remarkable 2003–2004 winter and other recent w.pdf;/Users/oscardimdore-miles/Zotero/storage/V8YQZE2V/2004JD005367.html},
  journal = {Journal of Geophysical Research: Atmospheres},
  keywords = {interannual variability,stratospheric temperatures,stratospheric warmings},
  language = {English},
  number = {D4}
}

@article{Mantua_1997,
  title = {A Pacific Interdecadal Climate Oscillation with Impacts on Salmon Production},
  author = {Mantua, Nathan and Hare, Steven and Zhang, Yuan and Wallace, John and Francis, Robert},
  year = {1997},
  volume = {78},
  pages = {1069--1079},
  doi = {10.1175/1520-0477(1997)078\\3C1069:APICOW\\3E2.0.CO;2},
  journal = {Bulletin of the American Meteorological Society}
}

@article{mantuaPacific1997,
  title = {A {{Pacific Interdecadal Climate Oscillation}} with {{Impacts}} on {{Salmon Production}}*},
  author = {Mantua, Nathan J. and Hare, Steven R. and Zhang, Yuan and Wallace, John M. and Francis, Robert C.},
  year = {1997},
  month = jun,
  volume = {78},
  pages = {1069--1080},
  publisher = {{American Meteorological Society}},
  issn = {0003-0007, 1520-0477},
  doi = {10.1175/1520-0477(1997)078<1069:APICOW>2.0.CO;2},
  abstract = {{$<$}section class="abstract"{$><$}p{$>$}Evidence gleaned from the instrumental record of climate data identifies a robust, recurring pattern of ocean\textendash atmosphere climate variability centered over the midlatitude North Pacific basin. Over the past century, the amplitude of this climate pattern has varied irregularly at interannual-to-interdecadal timescales. There is evidence of reversals in the prevailing polarity of the oscillation occurring around 1925, 1947, and 1977; the last two reversals correspond to dramatic shifts in salmon production regimes in the North Pacific Ocean. This climate pattern also affects coastal sea and continental surface air temperatures, as well as streamflow in major west coast river systems, from Alaska to California.{$<$}/p{$><$}/section{$>$}},
  chapter = {Bulletin of the American Meteorological Society},
  file = {/Users/oscardimdore-miles/Zotero/storage/PEU6LANC/Mantua et al. - 1997 - A Pacific Interdecadal Climate Oscillation with Im.pdf;/Users/oscardimdore-miles/Zotero/storage/RYYJJEP4/1520-0477_1997_078_1069_apicow_2_0_co_2.html},
  journal = {Bulletin of the American Meteorological Society},
  language = {English},
  number = {6}
}

@article{Manzini2006,
  title = {The Influence of Sea Surface Temperatures on the Northern Winter Stratosphere: {{Ensemble}} Simulations with the {{MAECHAM5}} Model},
  author = {Manzini, E. and Giorgetta, M. A. and Esch, M. and Kornblueh, L. and Roeckner, E.},
  year = {2006},
  volume = {19},
  pages = {3863--3881},
  doi = {10.1175/JCLI3826.1},
  journal = {Journal of Climate},
  number = {16}
}

@article{Manzini2012,
  title = {Stratosphere-Troposphere Coupling at Inter-Decadal Time Scales: {{Implications}} for the {{North Atlantic Ocean}}},
  author = {Manzini, Elisa and Cagnazzo, Chiara and Fogli, Pier Giuseppe and Bellucci, Alessio and M{\"u}ller, Wolfgang A.},
  year = {2012},
  volume = {39},
  doi = {10.1029/2011GL050771},
  journal = {Geophysical Research Letters},
  number = {5}
}

@article{manziniInfluence2006,
  title = {The {{Influence}} of {{Sea Surface Temperatures}} on the {{Northern Winter Stratosphere}}: {{Ensemble Simulations}} with the {{MAECHAM5 Model}}},
  shorttitle = {The {{Influence}} of {{Sea Surface Temperatures}} on the {{Northern Winter Stratosphere}}},
  author = {Manzini, E. and Giorgetta, M. A. and Esch, M. and Kornblueh, L. and Roeckner, E.},
  year = {2006},
  month = aug,
  volume = {19},
  pages = {3863--3881},
  publisher = {{American Meteorological Society}},
  issn = {0894-8755, 1520-0442},
  doi = {10.1175/JCLI3826.1},
  abstract = {{$<$}section class="abstract"{$><$}h2 class="abstractTitle text-title my-1" id="d1192e2"{$>$}Abstract{$<$}/h2{$><$}p{$>$}The role of interannual variations in sea surface temperatures (SSTs) on the Northern Hemisphere winter polar stratospheric circulation is addressed by means of an ensemble of nine simulations performed with the middle atmosphere configuration of the ECHAM5 model forced with observed SSTs during the 20-yr period from 1980 to 1999. Results are compared to the 40-yr ECMWF Re-Analysis (ERA-40). Three aspects have been considered: the influence of the interannual SST variations on the climatological mean state, the response to El Ni\~no\textendash Southern Oscillation (ENSO) events, and the influence on systematic temperature changes. The strongest influence of SST variations has been found for the warm ENSO events considered. Namely, it has been found that the large-scale pattern associated with the extratropical tropospheric response to the ENSO phenomenon during northern winter enhances the forcing and the vertical propagation into the stratosphere of the quasi-stationary planetary waves emerging from the troposphere. This enhanced planetary wave disturbance thereafter results in a polar warming of a few degrees in the lower stratosphere in late winter and early spring. Consequently, the polar vortex is weakened, and the warm ENSO influence clearly emerges also in the zonal-mean flow. In contrast, the cold ENSO events considered do not appear to have an influence distinguishable from that of internal variability. It is also not straightforward to deduce the influence of the SSTs on the climatological mean state from the simulations performed, because the simulated internal variability of the stratosphere is large, a realistic feature. Moreover, the results of the ensemble of simulations provide weak to negligible evidence for the possibility that SST variations during the two decades considered are substantially contributing to changes in the polar temperature in the winter lower stratosphere.{$<$}/p{$><$}/section{$>$}},
  chapter = {Journal of Climate},
  file = {/Users/oscardimdore-miles/Zotero/storage/I4YFLQXD/Manzini et al. - 2006 - The Influence of Sea Surface Temperatures on the N.pdf;/Users/oscardimdore-miles/Zotero/storage/3R4CG6ER/jcli3826.1.html},
  journal = {Journal of Climate},
  language = {English},
  number = {16}
}

@article{manziniStratospheretroposphere2012,
  title = {Stratosphere-Troposphere Coupling at Inter-Decadal Time Scales: {{Implications}} for the {{North Atlantic Ocean}}},
  shorttitle = {Stratosphere-Troposphere Coupling at Inter-Decadal Time Scales},
  author = {Manzini, Elisa and Cagnazzo, Chiara and Fogli, Pier Giuseppe and Bellucci, Alessio and M{\"u}ller, Wolfgang A.},
  year = {2012},
  volume = {39},
  issn = {1944-8007},
  doi = {10.1029/2011GL050771},
  abstract = {Evidence of stratosphere-troposphere coupling at inter-decadal time scales is searched for in a 260-year simulation performed with a climate model including a state-of-the-art stratosphere. The boundary conditions of the simulation are specified according to preindustrial conditions and are kept constant from year to year. It is shown that long lasting ({$\sim$}20 years) positive and negative anomalies of the northern winter stratospheric polar vortex exist in the simulation. Given that there are no externally imposed low frequency time variations, these persistent variations are due to internal dynamical processes of the modeled coupled atmosphere ocean system. By composite analysis, it is shown that the long lasting stratospheric vortex anomalies are connected through the troposphere to mean sea level pressure, surface temperature and sea ice cover anomalies. These connections are reminiscent of intra-seasonal stratosphere\textendash troposphere coupling. Over the ocean, the surface temperature and sea ice cover anomalies are indicative of the delayed Atlantic meridional overturning circulation response to atmospheric forcing. The latter is indeed found to be anomalously strong/weak during the long lasting positive/negative stratospheric vortex anomalies, providing evidence for a potential role of the stratosphere in decadal prediction.},
  copyright = {Copyright 2012 by the American Geophysical Union},
  file = {/Users/oscardimdore-miles/Zotero/storage/KQ9FIM77/Manzini et al. - 2012 - Stratosphere-troposphere coupling at inter-decadal.pdf;/Users/oscardimdore-miles/Zotero/storage/KTZ24NHW/2011GL050771.html},
  journal = {Geophysical Research Letters},
  keywords = {climate prediction,stratosphere-troposphere coupling},
  language = {English},
  number = {5}
}

@article{Marshall2009,
  title = {Impact of the {{QBO}} on Surface Winter Climate},
  author = {Marshall, Andrew G. and Scaife, Adam A.},
  year = {2009},
  volume = {114},
  doi = {10.1029/2009JD011737},
  journal = {Journal of Geophysical Research: Atmospheres},
  number = {D18}
}

@article{matsunoDynamical1971,
  title = {A {{Dynamical Model}} of the {{Stratospheric Sudden Warming}}},
  author = {Matsuno, Taroh},
  year = {1971},
  month = nov,
  volume = {28},
  pages = {1479--1494},
  publisher = {{American Meteorological Society}},
  issn = {0022-4928, 1520-0469},
  doi = {10.1175/1520-0469(1971)028<1479:ADMOTS>2.0.CO;2},
  abstract = {{$<$}section class="abstract"{$><$}h2 class="abstractTitle text-title my-1" id="d337e2"{$>$}Abstract{$<$}/h2{$><$}p{$>$}The dynamics of the stratosphere sudden warming phenomenon is discussed in terms of the interaction of vertically propagating planetary waves with zonal winds. If global-scale disturbances are generated in the troposphere, they propagate upward into the stratosphere, where the waves act to decelerate the polar night jet through the induction of a meridional circulation. Thus, the distortion and the break-down of the polar vortex occur. If the disturbance is intense and persists, the westerly jet may eventually disappear and an easterly wind may replace it. Then ``critical layer interaction'' takes place. Further intensification of the easterly wind and rapid warming of the polar air are expected to occur as well as weakening of the disturbance. The model is verified by numerical integrations of the adiabatic-geostrophic potential vorticity equation. Computed results possess features similar to those observed in sudden warming phenomena.{$<$}/p{$><$}/section{$>$}},
  chapter = {Journal of the Atmospheric Sciences},
  file = {/Users/oscardimdore-miles/Zotero/storage/CTEHHLAI/Matsuno - 1971 - A Dynamical Model of the Stratospheric Sudden Warm.pdf;/Users/oscardimdore-miles/Zotero/storage/3IRQHS5G/1520-0469_1971_028_1479_admots_2_0_co_2.html},
  journal = {Journal of the Atmospheric Sciences},
  language = {English},
  number = {8}
}

@article{matthewmanStratospheric2011,
  title = {Stratospheric {{Sudden Warmings}} as {{Self}}-{{Tuning Resonances}}. {{Part I}}: {{Vortex Splitting Events}}},
  shorttitle = {Stratospheric {{Sudden Warmings}} as {{Self}}-{{Tuning Resonances}}. {{Part I}}},
  author = {Matthewman, N. Joss and Esler, J. G.},
  year = {2011},
  month = nov,
  volume = {68},
  pages = {2481--2504},
  publisher = {{American Meteorological Society}},
  issn = {0022-4928, 1520-0469},
  doi = {10.1175/JAS-D-11-07.1},
  abstract = {{$<$}section class="abstract"{$><$}h2 class="abstractTitle text-title my-1" id="d290e2"{$>$}Abstract{$<$}/h2{$><$}p{$>$}The fundamental dynamics of ``vortex splitting'' stratospheric sudden warmings (SSWs), which are known to be predominantly barotropic in nature, are reexamined using an idealized single-layer {$<$}em{$>$}f{$<$}/em{$>$}-plane model of the polar vortex. The aim is to elucidate the conditions under which a stationary topographic forcing causes the model vortex to split, and to express the splitting condition as a function of the model parameters determining the topography and circulation.For a specified topographic forcing profile the model behavior is governed by two nondimensional parameters: the topographic forcing height {$<$}em{$>$}M{$<$}/em{$>$} and a surf-zone potential vorticity parameter {$\Omega$}. For relatively low {$<$}em{$>$}M{$<$}/em{$>$}, vortex splits similar to observed SSWs occur only for a narrow range of {$\Omega$} values. Further, a bifurcation in parameter space is observed: a small change in {$\Omega$} (or {$<$}em{$>$}M{$<$}/em{$>$}) beyond a critical value can lead to an abrupt transition between a state with low-amplitude vortex Rossby waves and a sudden vortex split. The model behavior can be fully understood using two nonlinear analytical reductions: the Kida model of elliptical vortex motion in a uniform strain flow and a forced nonlinear oscillator equation. The abrupt transition in behavior is a feature of both reductions and corresponds to the onset of a nonlinear (self-tuning) resonance. The results add an important new aspect to the ``resonant excitation'' theory of SSWs. Under this paradigm, it is not necessary to invoke an anomalous tropospheric planetary wave source, or unusually favorable conditions for upward wave propagation, in order to explain the occurrence of SSWs.{$<$}/p{$><$}/section{$>$}},
  chapter = {Journal of the Atmospheric Sciences},
  file = {/Users/oscardimdore-miles/Zotero/storage/8ZQ96SLI/Matthewman and Esler - 2011 - Stratospheric Sudden Warmings as Self-Tuning Reson.pdf},
  journal = {Journal of the Atmospheric Sciences},
  language = {English},
  number = {11}
}

@article{maycockRegime2020,
  title = {A {{Regime Perspective}} on the {{North Atlantic Eddy}}-{{Driven Jet Response}} to {{Sudden Stratospheric Warmings}}},
  author = {Maycock, Amanda C. and Masukwedza, Gibbon I. T. and Hitchcock, Peter and Simpson, Isla R.},
  year = {2020},
  month = may,
  volume = {33},
  pages = {3901--3917},
  publisher = {{American Meteorological Society}},
  issn = {0894-8755, 1520-0442},
  doi = {10.1175/JCLI-D-19-0702.1},
  abstract = {{$<$}section class="abstract"{$><$}h2 class="abstractTitle text-title my-1" id="d303e2"{$>$}Abstract{$<$}/h2{$><$}p{$>$}Changes to the preferred states, or regime behavior, of the North Atlantic eddy-driven jet (EDJ) following a major sudden stratospheric warming (SSW) is examined using a large ensemble experiment from the Canadian Middle Atmosphere Model in which the stratosphere is nudged toward an SSW. In the 3 months following the SSW (January\textendash March), the North Atlantic EDJ shifts equatorward by \textasciitilde 3\textdegree, on average; this arises from an increased occurrence of the EDJ's south regime and reductions in its north and central regimes. Qualitatively similar behavior is shown in a reanalysis dataset. We show that under SSW conditions the south regime becomes more persistent and that this can explain the overall increase in the EDJ latitude decorrelation time scale. A cluster analysis reveals that, following the SSW, the south EDJ regime is characterized by weaker low-level baroclinicity and eddy heat fluxes in the North Atlantic Ocean. We hypothesize, therefore, that the increased persistence of the south regime is related to the weaker baroclinicity leading to slower growth rates of the unstable modes and hence a slower buildup of eddy heat flux, which has been shown to precede EDJ transitions. In the North Atlantic sector, the surface response to the SSW projects onto a negative North Atlantic Oscillation (NAO) pattern, with almost no change in the east Atlantic (EA) pattern. This behavior appears to be distinct from the modeled intrinsic variability in the EDJ, where the jet latitude index captures variations in both the NAO and EA patterns. The results offer new insight into the mechanisms for stratosphere\textendash troposphere coupling following SSWs.{$<$}/p{$><$}/section{$>$}},
  chapter = {Journal of Climate},
  file = {/Users/oscardimdore-miles/Zotero/storage/GNEIT7JI/Maycock et al. - 2020 - A Regime Perspective on the North Atlantic Eddy-Dr.pdf;/Users/oscardimdore-miles/Zotero/storage/9VI4HH2G/jcli-d-19-0702.1.html},
  journal = {Journal of Climate},
  language = {English},
  number = {9}
}

@article{mccarthyMeasuring2015,
  title = {Measuring the {{Atlantic Meridional Overturning Circulation}} at 26\textdegree{{N}}},
  author = {McCarthy, G. D. and Smeed, D. A. and Johns, W. E. and {Frajka-Williams}, E. and Moat, B. I. and Rayner, D. and Baringer, M. O. and Meinen, C. S. and Collins, J. and Bryden, H. L.},
  year = {2015},
  month = jan,
  volume = {130},
  pages = {91--111},
  issn = {0079-6611},
  doi = {10.1016/j.pocean.2014.10.006},
  abstract = {The Atlantic Meridional Overturning Circulation (AMOC) plays a key role in the global climate system through its redistribution of heat. Changes in the AMOC have been associated with large fluctuations in the earth's climate in the past and projections of AMOC decline in the future due to climate change motivate the continuous monitoring of the circulation. Since 2004, the RAPID monitoring array has been providing continuous estimates of the AMOC and associated heat transport at 26\textdegree N in the North Atlantic. We describe how these measurements are made including the sampling strategy, the accuracies of parameters measured and the calculation of the AMOC. The strength of the AMOC and meridional heat transport are estimated as 17.2Sv and 1.25PW respectively from April 2004 to October 2012. The accuracy of ten day (annual) transports is 1.5Sv (0.9Sv). Improvements to the estimation of the transport above the shallowest instruments and deepest transports (including Antarctic Bottom Water), and the use of the new equation of state for seawater have reduced the estimated strength of the AMOC by 0.6Sv relative to previous publications. As new basinwide AMOC monitoring projects begin in the South Atlantic and sub-polar North Atlantic, we present this thorough review of the methods and measurements of the original AMOC monitoring array.},
  file = {/Users/oscardimdore-miles/Zotero/storage/HX6NUXBQ/McCarthy et al. - 2015 - Measuring the Atlantic Meridional Overturning Circ.pdf;/Users/oscardimdore-miles/Zotero/storage/449FJ2XM/S0079661114001694.html},
  journal = {Progress in Oceanography},
  language = {English}
}

@article{mccarthyObserved2012,
  title = {Observed Interannual Variability of the {{Atlantic}} Meridional Overturning Circulation at 26.5\textdegree{{N}}},
  author = {McCarthy, G. and {Frajka-Williams}, E. and Johns, W. E. and Baringer, M. O. and Meinen, C. S. and Bryden, H. L. and Rayner, D. and Duchez, A. and Roberts, C. and Cunningham, S. A.},
  year = {2012},
  volume = {39},
  issn = {1944-8007},
  doi = {10.1029/2012GL052933},
  abstract = {The Atlantic meridional overturning circulation (MOC) plays a critical role in the climate system and is responsible for much of the heat transported by the ocean. A mooring array, nominally at 26\textdegree N between the Bahamas and the Canary Islands, deployed in Apr 2004 provides continuous measurements of the strength and variability of this circulation. With seven full years of measurements, we now examine the interannual variability of the MOC. While earlier results highlighted substantial seasonal and shorter timescale variability, there had not been significant interannual variability. The mean MOC from 1 Apr 2004 to the 31 March 2009 was 18.5 Sv with the annual means having a standard deviation of only 1.0 Sv. From 1 April 2009 to 31 March 2010, the annually averaged MOC strength was just 12.8 Sv, representing a 30\% decline. This downturn persisted from early 2009 to mid-2010. We show that the cause of the decline was not only an anomalous wind-driven event from Dec 2009\textendash Mar 2010 but also a strengthening of the geostrophic flow. In particular, the southward flow in the top 1100 m intensified, while the deep southward return transport\textemdash particularly in the deepest layer from 3000\textendash 5000 m\textemdash weakened. This rebalancing of the transport from the deep overturning to the upper gyre has implications for the heat transported by the Atlantic.},
  copyright = {\textcopyright 2012. American Geophysical Union. All Rights Reserved.},
  file = {/Users/oscardimdore-miles/Zotero/storage/7UN4RGDP/McCarthy et al. - 2012 - Observed interannual variability of the Atlantic m.pdf;/Users/oscardimdore-miles/Zotero/storage/2CMW78DA/2012GL052933.html},
  journal = {Geophysical Research Letters},
  keywords = {Atlantic meridional overturning circulation,heat transport,thermohaline circulation},
  language = {English},
  number = {19}
}

@article{medhaugMechanisms2012,
  title = {Mechanisms for Decadal Scale Variability in a Simulated {{Atlantic}} Meridional Overturning Circulation},
  author = {Medhaug, I. and Langehaug, H. R. and Eldevik, T. and Furevik, T. and Bentsen, M.},
  year = {2012},
  month = jul,
  volume = {39},
  pages = {77--93},
  issn = {1432-0894},
  doi = {10.1007/s00382-011-1124-z},
  abstract = {Variability in the Atlantic Meridional Overturning Circulation (AMOC) has been analysed using a 600-year pre-industrial control simulation with the Bergen Climate Model. The typical AMOC variability has amplitudes of 1 Sverdrup (1 Sv = 106 m3 s-1) and time scales of 40\textendash 70 years. The model is reproducing the observed dense water formation regions and has very realistic ocean transports and water mass distributions. The dense water produced in the Labrador Sea (1/3) and in the Nordic Seas, including the water entrained into the dense overflows across the Greenland-Scotland Ridge (GSR; 2/3), are the sources of North Atlantic Deep Water (NADW) forming the lower limb of the AMOC's northern overturning. The variability in the Labrador Sea and the Nordic Seas convection is driven by decadal scale air-sea fluxes in the convective region that can be related to opposite phases of the North Atlantic Oscillation. The Labrador Sea convection is directly linked to the variability in AMOC. Linkages between convection and water mass transformation in the Nordic Seas are more indirect. The Scandinavian Pattern, the third mode of atmospheric variability in the North Atlantic, is a driver of the ocean's poleward heat transport (PHT), the overall constraint on northern water mass transformation. Increased PHT is both associated with an increased water mass exchange across the GSR, and a stronger AMOC.},
  file = {/Users/oscardimdore-miles/Zotero/storage/R6ANW89D/Medhaug et al. - 2012 - Mechanisms for decadal scale variability in a simu.pdf},
  journal = {Climate Dynamics},
  language = {English},
  number = {1}
}

@article{Menary2018,
  title = {Preindustrial Control Simulations with {{HadGEM3}}-{{GC3}}.1 for {{CMIP6}}},
  author = {Menary, Matthew B. and Kuhlbrodt, Till and Ridley, Jeff and Andrews, Martin B. and {Dimdore-Miles}, Oscar B. and Deshayes, Julie and Eade, Rosie and Gray, Lesley and Ineson, Sarah and Mignot, Juliette and Roberts, Christopher D. and Robson, Jon and Wood, Richard A. and Xavier, Prince},
  year = {2018},
  volume = {10},
  pages = {3049--3075},
  doi = {10.1029/2018MS001495},
  journal = {Journal of Advances in Modeling Earth Systems},
  number = {12}
}

@article{menaryMultimodel2012,
  title = {A Multimodel Comparison of Centennial {{Atlantic}} Meridional Overturning Circulation Variability},
  author = {Menary, Matthew B. and Park, Wonsun and Lohmann, Katja and Vellinga, Michael and Palmer, Matthew D. and Latif, Mojib and Jungclaus, Johann H.},
  year = {2012},
  month = jun,
  volume = {38},
  pages = {2377--2388},
  issn = {1432-0894},
  doi = {10.1007/s00382-011-1172-4},
  abstract = {A mechanism contributing to centennial variability of the Atlantic Meridional Overturning Circulation (AMOC) is tested with multi-millennial control simulations of several coupled general circulation models (CGCMs). These are a substantially extended integration of the 3rd Hadley Centre Coupled Climate Model (HadCM3), the Kiel Climate Model (KCM), and the Max Plank Institute Earth System Model (MPI-ESM). Significant AMOC variability on time scales of around 100 years is simulated in these models. The centennial mechanism links changes in the strength of the AMOC with oceanic salinities and surface temperatures, and atmospheric phenomena such as the Intertropical Convergence Zone (ITCZ). 2 of the 3 models reproduce all aspects of the mechanism, with the third (MPI-ESM) reproducing most of them. A comparison with a high resolution paleo-proxy for Sea Surface Temperatures (SSTs) north of Iceland over the last 4,000 years, also linked to the ITCZ, suggests that elements of this mechanism may also be detectable in the real world.},
  file = {/Users/oscardimdore-miles/Zotero/storage/8WTLU38W/Menary et al. - 2012 - A multimodel comparison of centennial Atlantic mer.pdf},
  journal = {Climate Dynamics},
  language = {English},
  number = {11}
}

@article{menaryPreindustrial2018,
  title = {Preindustrial {{Control Simulations With HadGEM3}}-{{GC3}}.1 for {{CMIP6}}},
  author = {Menary, Matthew B. and Kuhlbrodt, Till and Ridley, Jeff and Andrews, Martin B. and {Dimdore-Miles}, Oscar B. and Deshayes, Julie and Eade, Rosie and Gray, Lesley and Ineson, Sarah and Mignot, Juliette and Roberts, Christopher D. and Robson, Jon and Wood, Richard A. and Xavier, Prince},
  year = {2018},
  volume = {10},
  pages = {3049--3075},
  issn = {1942-2466},
  doi = {10.1029/2018MS001495},
  abstract = {Preindustrial control simulations with the third Hadley Centre Global Environmental Model, run in the Global Coupled configuration 3.1 of the Met Office Unified Model (HadGEM3-GC3.1) are presented at two resolutions. These are N216ORCA025, which has a horizontal resolution of 60 km in the atmosphere and 0.25\textdegree{} in the ocean, and N96ORCA1, which has a horizontal resolution of 130 km in the atmosphere and 1\textdegree{} in the ocean. The aim of this study is to document the climate variability in these simulations, make comparisons against present-day observations (albeit under different forcing), and discuss differences arising due to resolution. In terms of interannual variability in the leading modes of climate variability the two resolutions behave generally very similarly. Notable differences are in the westward extent of El Ni\~no and the pattern of Atlantic multidecadal variability, in which N216ORCA025 compares more favorably to observations, and in the Antarctic Circumpolar Current, which is far too weak in N216ORCA025. In the North Atlantic region, N216ORCA025 has a stronger and deeper Atlantic Meridional Overturning Circulation, which compares well against observations, and reduced biases in temperature and salinity in the North Atlantic subpolar gyre. These simulations are being provided to the sixth Coupled Model Intercomparison Project (CMIP6) and provide a baseline against which further forced experiments may be assessed.},
  copyright = {\textcopyright 2018 Crown copyright. This article is published with the permission of the Controller of HMSO and the Queen's Printer for Scotland.},
  file = {/Users/oscardimdore-miles/Zotero/storage/QL82FMZA/Menary et al. - 2018 - Preindustrial Control Simulations With HadGEM3-GC3.pdf;/Users/oscardimdore-miles/Zotero/storage/2J4VPT8Q/2018MS001495.html},
  journal = {Journal of Advances in Modeling Earth Systems},
  keywords = {climate,climate modeling,CMIP6,IPCC,preindustrial,resolution},
  language = {English},
  number = {12}
}

@article{mielkeObserved2013,
  title = {Observed and {{Simulated Variability}} of the {{AMOC}} at 26\textdegree{{N}} and 41\textdegree{{N}}},
  author = {Mielke, C. and {Frajka-Williams}, E. and Baehr, J.},
  year = {2013},
  volume = {40},
  pages = {1159--1164},
  issn = {1944-8007},
  doi = {10.1002/grl.50233},
  abstract = {Time series of the observational estimate of the Atlantic meridional overturning circulation (AMOC) have recently become available, but so far, no contemporaneous relation has been documented between them. Here, we analyze the variability of the 26\textdegree N Rapid Climate Change programme (RAPID) and the 41\textdegree N Argo-based AMOC estimates on seasonal timescales, and we compare them to a simulation from a high-resolution National Centers for Environmental Prediction (NCEP)-forced ocean model. In our analysis of the observed time series, we find that the seasonal cycles of the non-Ekman component of the AMOC between 26\textdegree N and 41\textdegree N are 180-degrees out-of-phase. Removing the mean seasonal cycle from each time series, the residuals have a non-stationary covariability. Our results demonstrate that the AMOC is meridionally covariable between 26\textdegree N and 41\textdegree N at seasonal timescales. We find the same covariability in the model, although the phasing differs from the observed phasing. This may offer the possibility of inferring AMOC variations and associated climate anomalies throughout the North Atlantic from discontinuous observations.},
  copyright = {\textcopyright 2013. American Geophysical Union. All Rights Reserved.},
  file = {/Users/oscardimdore-miles/Zotero/storage/UG42HTVW/Mielke et al. - 2013 - Observed and simulated variability of the AMOC at .pdf;/Users/oscardimdore-miles/Zotero/storage/MX56INYN/grl.html},
  journal = {Geophysical Research Letters},
  keywords = {AMOC,North Atlantic,Seasonal variability},
  language = {English},
  number = {6}
}

@article{Minobe,
  title = {Resonance in Bidecadal and Pentadecadal Climate Oscillations over the {{North Pacific}}: {{Role}} in Climatic Regime Shifts},
  author = {Minobe, Shoshiro},
  year = {1999},
  volume = {26},
  pages = {855--858},
  doi = {10.1029/1999GL900119},
  journal = {Geophysical Research Letters},
  number = {7}
}

@misc{moatAtlantic2020,
  title = {Atlantic Meridional Overturning Circulation Observed by the {{RAPID}}-{{MOCHA}}-{{WBTS}} ({{RAPID}}-{{Meridional Overturning Circulation}} and {{Heatflux Array}}-{{Western Boundary Time Series}}) Array at {{26N}} from 2004 to 2018 ({{V2018}}.2).},
  author = {Moat, Ben I and {Frajka-Williams}, Eleanor and Smeed, David and Rayner, Darren and {Sanchez-Franks}, Alejandra and Johns, William E and Baringer, Molly O and Volkov, Denis L and Collins, Julie},
  year = {2020},
  publisher = {{British Oceanographic Data Centre, National Oceanography Centre, NERC, UK}},
  doi = {10.5285/AA57E879-4CCA-28B6-E053-6C86ABC02DE5},
  abstract = {The RAPID-MOCHA-WBTS (RAPID-Meridional Overturning Circulation and Heatflux Array-Western Boundary Time Series) programme has produced a continuous time series of the Atlantic Meridional Overturning Circulation (AMOC) at 26N that started in April 2004. This release of the time series covers the period from April 2004 to Sept 2018, but differs from the previous release by using a different latitude for the Ekman transport. The 26N AMOC time series is derived from measurements of temperature, salinity, pressure and water velocity from an array of moored instruments that extend from the east coast of the Bahamas to the continental shelf off Africa east of the Canary Islands. The AMOC calculation also uses estimates of the transport in the Florida Strait derived from sub-sea cable measurements calibrated by regular hydrographic cruises. The component of the AMOC associated with the wind driven Ekman layer is derived from satellite scatterometer measurements. This release of the data includes a document with a brief description of the calculation of the AMOC time series and references to more detailed description in published papers. The 26N AMOC time series and the data from the moored array are curated by the British Oceanographic Data Centre (BODC). The RAPID-MOCHA-WBTS programme is a joint effort between NERC in the UK (Principle Investigator Eleanor Frajka-Williams since 2020, David Smeed 2012 to 2020, and Stuart Cunningham from 2004 to 2012), NOAA (PIs Molly Baringer and Denis Volkov) and NSF (PI Prof. Bill Johns, Uni. Miami) in the USA.},
  keywords = {elevation,oceans},
  language = {English}
}

@article{Mulcahy2018,
  title = {Improved Aerosol Processes and Effective Radiative Forcing in {{HadGEM3}} and {{UKESM1}}},
  author = {Mulcahy, J. P. and Jones, C. and Sellar, A. and Johnson, B. and Boutle, I. A. and Jones, A. and Andrews, T. and Rumbold, S. T. and Mollard, J. and Bellouin, N. and Johnson, C. E. and Williams, K. D. and Grosvenor, D. P. and McCoy, D. T.},
  year = {2018},
  doi = {10.1029/2018MS001464},
  journal = {Journal of Advances in Modeling Earth Systems}
}

@article{mulcahyImproved2018,
  title = {Improved {{Aerosol Processes}} and {{Effective Radiative Forcing}} in {{HadGEM3}} and {{UKESM1}}},
  author = {Mulcahy, J. P. and Jones, C. and Sellar, A. and Johnson, B. and Boutle, I. A. and Jones, A. and Andrews, T. and Rumbold, S. T. and Mollard, J. and Bellouin, N. and Johnson, C. E. and Williams, K. D. and Grosvenor, D. P. and McCoy, D. T.},
  year = {2018},
  volume = {10},
  pages = {2786--2805},
  issn = {1942-2466},
  doi = {10.1029/2018MS001464},
  abstract = {Aerosol processes and, in particular, aerosol-cloud interactions cut across the traditional physical-Earth system boundary of coupled Earth system models and remain one of the key uncertainties in estimating anthropogenic radiative forcing of climate. Here we calculate the historical aerosol effective radiative forcing (ERF) in the HadGEM3-GA7 climate model in order to assess the suitability of this model for inclusion in the UK Earth system model, UKESM1. The aerosol ERF, calculated for the year 2000 relative to 1850, is large and negative in the standard GA7 model leading to an unrealistic negative total anthropogenic forcing over the twentieth century. We show how underlying assumptions and missing processes in both the physical model and aerosol parameterizations lead to this large aerosol ERF. A number of model improvements are investigated to assess their impact on the aerosol ERF. These include an improved representation of cloud droplet spectral dispersion, updates to the aerosol activation scheme, and black carbon optical properties. One of the largest contributors to the aerosol forcing uncertainty is insufficient knowledge of the preindustrial aerosol climate. We evaluate the contribution of uncertainties in the natural marine emissions of dimethyl sulfide and organic aerosol to the ERF. The combination of model improvements derived from these studies weakens the aerosol ERF by up to 50\% of the original value and leads to a total anthropogenic historical forcing more in line with assessed values.},
  copyright = {\textcopyright 2018. Crown copyright. This article is published with the permission of the Controller of HMSO and the Queen's Printer for Scotland.},
  file = {/Users/oscardimdore-miles/Zotero/storage/ILEB5FFU/Mulcahy et al. - 2018 - Improved Aerosol Processes and Effective Radiative.pdf},
  journal = {Journal of Advances in Modeling Earth Systems},
  keywords = {aerosol forcing,climate models,effective radiative forcing,model development},
  language = {English},
  number = {11}
}

@article{Nakamura2016,
  title = {The Stratospheric Pathway for {{Arctic}} Impacts on Midlatitude Climate},
  author = {Nakamura, Tetsu and Yamazaki, Koji and Iwamoto, Katsushi and Honda, Meiji and Miyoshi, Yasunobu and Ogawa, Yasunobu and Tomikawa, Yoshihiro and Ukita, Jinro},
  year = {2016},
  volume = {43},
  pages = {3494--3501},
  journal = {Geophysical Research Letters},
  keywords = {Arctic Oscillation,Arctic sea ice reduction,polar amplification,polar vortex weakening,severe weather,wave mean flow interaction},
  number = {7}
}

@article{Newman2016,
  title = {The Pacific Decadal Oscillation, Revisited},
  author = {Newman, Matthew and Alexander, Michael A. and Ault, Toby R. and Cobb, Kim M. and Deser, Clara and Lorenzo, Emanuele Di and Mantua, Nathan J. and Miller, Arthur J. and Minobe, Shoshiro and Nakamura, Hisashi and Schneider, Niklas and Vimont, Daniel J. and Phillips, Adam S. and Scott, James D. and Smith, Catherine A.},
  year = {2016},
  volume = {29},
  pages = {4399--4427},
  journal = {Journal of Climate},
  number = {12}
}

@article{nishiiModulations2009,
  title = {Modulations in the Planetary Wave Field Induced by Upward-Propagating {{Rossby}} Wave Packets Prior to Stratospheric Sudden Warming Events: {{A}} Case-Study},
  shorttitle = {Modulations in the Planetary Wave Field Induced by Upward-Propagating {{Rossby}} Wave Packets Prior to Stratospheric Sudden Warming Events},
  author = {Nishii, Kazuaki and Nakamura, Hisashi and Miyasaka, Takafumi},
  year = {2009},
  volume = {135},
  pages = {39--52},
  issn = {1477-870X},
  doi = {10.1002/qj.359},
  abstract = {A diagnostic framework is introduced in which anomalous zonally averaged Rossby wave-activity injection into the stratosphere is decomposed into a contribution solely from zonally confined upward-propagating Rossby wave packets and another from interaction of the wave packets with the climatological planetary waves. To pinpoint the tropospheric sources of the wave packets, a particular form of wave-activity flux is evaluated for the associated circulation anomalies. The framework is applied to reanalysis data for the period prior to a stratospheric sudden warming (SSW) event in January 2006, which was associated with two successive events of above-normal wave-activity injection from the troposphere. In the earlier event, a pair of wave packets that emanated from tropospheric anomalies over the North Pacific and over Europe enhanced the upward wave-activity injection, which was augmented further by their interaction with the climatological planetary wave. In contrast, in the later period a wave packet that emanated from an anticyclonic anomaly over the North Atlantic is found to be the primary contributor to the enhanced planetary wave-activity injection, while its interaction with the climatological planetary wave contributed negatively. The predominant importance of the sole contribution from a single wave packet is also found in a major SSW event observed over Antarctica in September 2002. These results indicate that the diagnostic framework presented in this study is a useful tool for understanding the interaction between anomalies associated with zonally confined wave packets and climatological-mean planetary waves in the study of stratosphere\textendash troposphere dynamical coupling. Copyright \textcopyright{} 2009 Royal Meteorological Society},
  copyright = {Copyright \textcopyright{} 2009 Royal Meteorological Society},
  file = {/Users/oscardimdore-miles/Zotero/storage/UQM6NI3U/Nishii et al. - 2009 - Modulations in the planetary wave field induced by.pdf;/Users/oscardimdore-miles/Zotero/storage/39QUUBCT/qj.html},
  journal = {Quarterly Journal of the Royal Meteorological Society},
  keywords = {blocking,low-frequency variability,wave-activity flux},
  language = {English},
  number = {638}
}

@article{Nitta1989,
  title = {Recent Warming of Tropical Sea Surface Temperature and Its Relationship to the Northern Hemisphere Circulation},
  author = {Tsuyoshi, N. and Shingo, Y.},
  year = {1989},
  volume = {67:3},
  doi = {10.2151/jmsj1965.67.3_375},
  journal = {Journal of the Meteorological Society of Japan}
}

@article{ocallaghanEffects2014,
  title = {The {{Effects}} of {{Different Sudden Stratospheric Warming Type}} on the {{Ocean}}},
  author = {O'Callaghan, Amee and Joshi, Manoj and Stevens, David and Mitchell, Daniel},
  year = {2014},
  month = nov,
  volume = {41},
  doi = {10.1002/2014GL062179},
  abstract = {There is a confirmed link between sudden stratospheric warmings (SSWs) and surface weather. Here we find significant differences in the strength of surface and ocean responses for splitting and displacement SSWs, classified using a new straightforward moment analysis technique. In an intermediate general circulation model splitting SSWs possess an enhanced ability to affect the surface climate demonstrating the need to treat the two types individually. Following SSWs the North Atlantic surface wind stress curl weakens, compared to its climatological winter state, for over 30 days: this is also evident in NCEP/NCAR reanalysis. The effect of anomalies associated with SSWs on the ocean is analysed in the IGCM4. The splitting SSW composite displays strong anomalies in the implied Ekman heat flux and net atmosphere-surface flux, modifying the mixed layer heat budget. Our results highlight that different SSW types need to be simulated in coupled stratospheric/tropospheric/ocean models.},
  file = {/Users/oscardimdore-miles/Zotero/storage/YX7QYKPA/O'Callaghan et al. - 2014 - The Effects of Different Sudden Stratospheric Warm.pdf},
  journal = {Geophysical Research Letters}
}

@article{oconnorAssessment2021,
  title = {Assessment of Pre-Industrial to Present-Day Anthropogenic Climate Forcing in {{UKESM1}}},
  author = {O'Connor, Fiona M. and Abraham, N. Luke and Dalvi, Mohit and Folberth, Gerd A. and Griffiths, Paul T. and Hardacre, Catherine and Johnson, Ben T. and Kahana, Ron and Keeble, James and Kim, Byeonghyeon and Morgenstern, Olaf and Mulcahy, Jane P. and Richardson, Mark and Robertson, Eddy and Seo, Jeongbyn and Shim, Sungbo and Teixeira, Jo{\~a}o C. and Turnock, Steven T. and Williams, Jonny and Wiltshire, Andrew J. and Woodward, Stephanie and Zeng, Guang},
  year = {2021},
  month = jan,
  volume = {21},
  pages = {1211--1243},
  publisher = {{Copernicus GmbH}},
  issn = {1680-7316},
  doi = {10.5194/acp-21-1211-2021},
  abstract = {{$<$}p{$><$}strong class="journal-contentHeaderColor"{$>$}Abstract.{$<$}/strong{$>$} Quantifying forcings from anthropogenic perturbations to the Earth system (ES) is important for understanding changes in climate since the pre-industrial (PI) period. Here, we quantify and analyse a wide range of present-day (PD) anthropogenic effective radiative forcings (ERFs) with the UK's Earth System Model (ESM), UKESM1, following the protocols defined by the Radiative Forcing Model Intercomparison Project (RFMIP) and the Aerosol and Chemistry Model Intercomparison Project (AerChemMIP). In particular, quantifying ERFs that include rapid adjustments within a full ESM enables the role of various chemistry\textendash aerosol\textendash cloud interactions to be investigated.{$<$}/p{$>$} {$<$}p{$>$}Global mean ERFs for the PD (year 2014) relative to the PI (year 1850) period for carbon dioxide (CO{$_2$}), nitrous oxide (N{$_2$}O), ozone-depleting substances (ODSs), and methane (CH{$_4$}) are 1.89 {$\pm$} 0.04, 0.25 {$\pm$} 0.04, -0.18 {$\pm$} 0.04, and 0.97 {$\pm$} 0.04 W m{$^{-2}$}, respectively. The total greenhouse gas (GHG) ERF is 2.92 {$\pm$} 0.04 W m{$^{-2}$}.{$<$}/p{$>$} {$<$}p{$>$}UKESM1 has an aerosol ERF of -1.09 {$\pm$} 0.04 W m{$^{-2}$}. A relatively strong negative forcing from aerosol\textendash cloud interactions (ACI) and a small negative instantaneous forcing from aerosol\textendash radiation interactions (ARI) from sulfate and organic carbon (OC) are partially offset by a substantial forcing from black carbon (BC) absorption. Internal mixing and chemical interactions imply that neither the forcing from ARI nor ACI is linear, making the aerosol ERF less than the sum of the individual speciated aerosol ERFs.{$<$}/p{$>$} {$<$}p{$>$}Ozone (O{$_3$}) precursor gases consisting of volatile organic compounds (VOCs), carbon monoxide (CO), and nitrogen oxides (NO\textsubscript{\emph{x}}), but excluding CH{$_4$}, exert a positive radiative forcing due to increases in O{$_3$}. However, they also lead to oxidant changes, which in turn cause an indirect aerosol ERF. The net effect is that the ERF from PD\textendash PI changes in NO\textsubscript{\emph{x}} emissions is negligible at 0.03 {$\pm$} 0.04 W m{$^{-2}$}, while the ERF from changes in VOC and CO emissions is 0.33 {$\pm$} 0.04 W m{$^{-2}$}. Together, aerosol and O{$_3$} precursors (called near-term climate forcers (NTCFs) in the context of AerChemMIP) exert an ERF of -1.03 {$\pm$} 0.04 W m{$^{-2}$}, mainly due to changes in the cloud radiative effect (CRE). There is also a negative ERF from land use change (-0.17 {$\pm$} 0.04 W m{$^{-2}$}). When adjusted from year 1850 to 1700, it is more negative than the range of previous estimates, and is most likely due to too strong an albedo response. In combination, the net anthropogenic ERF (1.76 {$\pm$} 0.04 W m{$^{-2}$}) is consistent with other estimates.{$<$}/p{$>$} {$<$}p{$>$}By including interactions between GHGs, stratospheric and tropospheric O{$_3$}, aerosols, and clouds, this work demonstrates the importance of ES interactions when quantifying ERFs. It also suggests that rapid adjustments need to include chemical as well as physical adjustments to fully account for complex ES interactions.{$<$}/p{$>$}},
  file = {/Users/oscardimdore-miles/Zotero/storage/92MJRQ2P/O'Connor et al. - 2021 - Assessment of pre-industrial to present-day anthro.pdf;/Users/oscardimdore-miles/Zotero/storage/BHE3XP2A/acp-21-1211-2021.html},
  journal = {Atmospheric Chemistry and Physics},
  language = {English},
  number = {2}
}

@article{orrImproved2010,
  title = {Improved {{Middle Atmosphere Climate}} and {{Forecasts}} in the {{ECMWF Model}} through a {{Nonorographic Gravity Wave Drag Parameterization}}},
  author = {Orr, Andrew and Bechtold, Peter and Scinocca, John and Ern, Manfred and Janiskova, Marta},
  year = {2010},
  month = nov,
  volume = {23},
  pages = {5905--5926},
  publisher = {{American Meteorological Society}},
  issn = {0894-8755, 1520-0442},
  doi = {10.1175/2010JCLI3490.1},
  abstract = {{$<$}section class="abstract"{$><$}h2 class="abstractTitle text-title my-1" id="d63937505e96"{$>$}Abstract{$<$}/h2{$><$}p{$>$}In model cycle 35r3 (Cy35r3) of the ECMWF Integrated Forecast System (IFS), the momentum deposition from small-scale nonorographic gravity waves is parameterized by the Scinocca scheme, which uses hydrostatic nonrotational wave dynamics to describe the vertical evolution of a broad, constant, and isotropic spectrum of gravity waves emanating from the troposphere. The Cy35r3 middle atmosphere climate shows the following: (i) an improved representation of the zonal-mean circulation and temperature structure; (ii) a realistic parameterized gravity wave drag; (iii) a reasonable stationary planetary wave structure and stationary wave driving in July and an underestimate of the generation of stationary wave activity in the troposphere and stationary wave driving in January; (iv) an improved representation of the tropical variability of the stratospheric circulation, although the westerly phase of the semiannual oscillation is missing; and (v) a realistic horizontal distribution of momentum flux in the stratosphere. By contrast, the middle atmosphere climate is much too close to radiative equilibrium when the Scinocca scheme is replaced by Rayleigh friction, which was the standard method of parameterizing the effects of nonorographic gravity waves in the IFS prior to Cy35r3. Finally, there is a reduction in Cy35r3 short-range high-resolution forecast error in the upper stratosphere.{$<$}/p{$><$}/section{$>$}},
  chapter = {Journal of Climate},
  file = {/Users/oscardimdore-miles/Zotero/storage/2Z8DHNAP/Orr et al. - 2010 - Improved Middle Atmosphere Climate and Forecasts i.pdf;/Users/oscardimdore-miles/Zotero/storage/PHFKUFKA/2010jcli3490.1.html},
  journal = {Journal of Climate},
  language = {EN},
  number = {22}
}

@article{OspEA10,
  title = {The Climatology of the Middle Atmosphere in a Vertically Extended Version of the {{Met Office}}'s Climate Model. {{Part II}}: {{Variability}}},
  author = {Osprey, Scott M and Gray, Lesley J and Hardiman, Steven C and Butchart, Neal and Bushell, Andrew C and Hinton, Tim J},
  year = {2010},
  volume = {67},
  pages = {3637--3651},
  journal = {Journal of the atmospheric sciences},
  number = {11}
}

@article{ospreyClimatology2010,
  title = {The {{Climatology}} of the {{Middle Atmosphere}} in a {{Vertically Extended Version}} of the {{Met Office}}'s {{Climate Model}}. {{Part II}}: {{Variability}}},
  shorttitle = {The {{Climatology}} of the {{Middle Atmosphere}} in a {{Vertically Extended Version}} of the {{Met Office}}'s {{Climate Model}}. {{Part II}}},
  author = {Osprey, Scott M. and Gray, Lesley J. and Hardiman, Steven C. and Butchart, Neal and Bushell, Andrew C. and Hinton, Tim J.},
  year = {2010},
  month = nov,
  volume = {67},
  pages = {3637--3651},
  publisher = {{American Meteorological Society}},
  issn = {0022-4928, 1520-0469},
  doi = {10.1175/2010JAS3338.1},
  abstract = {{$<$}section class="abstract"{$><$}h2 class="abstractTitle text-title my-1" id="d132662404e99"{$>$}Abstract{$<$}/h2{$><$}p{$>$}Stratospheric variability is examined in a vertically extended version of the Met Office global climate model. Equatorial variability includes the simulation of an internally generated quasi-biennial oscillation (QBO) and semiannual oscillation (SAO). Polar variability includes an examination of the frequency of sudden stratospheric warmings (SSW) and annular mode variability. Results from two different horizontal resolutions are also compared. Changes in gravity wave filtering at the higher resolution result in a slightly longer QBO that extends deeper into the lower stratosphere. At the higher resolution there is also a reduction in the occurrence rate of sudden stratospheric warmings, in better agreement with observations. This is linked with reduced levels of resolved waves entering the high-latitude stratosphere. Covariability of the tropical and extratropical stratosphere is seen, linking the phase of the QBO with disturbed NH winters, although this linkage is sporadic, in agreement with observations. Finally, tropospheric persistence time scales and seasonal variability for the northern and southern annular modes are significantly improved at the higher resolution, consistent with findings from other studies.{$<$}/p{$><$}/section{$>$}},
  chapter = {Journal of the Atmospheric Sciences},
  file = {/Users/oscardimdore-miles/Zotero/storage/MQFQ7FHW/Osprey et al. - 2010 - The Climatology of the Middle Atmosphere in a Vert.pdf;/Users/oscardimdore-miles/Zotero/storage/7KB8Q6SS/2010jas3338.1.html},
  journal = {Journal of the Atmospheric Sciences},
  language = {English},
  number = {11}
}

@article{Overland1999,
  title = {Decadal Variability of the Aleutian Low and Its Relation to High-Latitude Circulation},
  author = {Overland, James E. and Adams, Jennifer Miletta and Bond, Nicholas A.},
  year = {1999},
  volume = {12},
  pages = {1542--1548},
  issn = {0894-8755},
  doi = {10.1175/1520-0442},
  journal = {Journal of Climate},
  number = {5}
}

@article{overlandDecadal1999,
  title = {Decadal {{Variability}} of the {{Aleutian Low}} and {{Its Relation}} to {{High}}-{{Latitude Circulation}}},
  author = {Overland, James and Adams, Jennifer and Bond, Nicholas},
  year = {1999},
  month = may,
  volume = {12},
  pages = {1542--1548},
  doi = {10.1175/1520-0442(1999)012<1542:DVOTAL>2.0.CO;2},
  abstract = {The January-February mean central pressure of the Aleutian low is investigated as an index of North Pacific variability on interannual to decadal timescales. Since the turn of the century, 37\% of the winter interannual variance of the Aleutian low is on timescales greater than 5 yr. An objective algorithm detects zero crossings of Aleutian low central pressure anomalies in 1925, 1931, 1939, 1947, 1959, 1968, 1976, and 1989. No single midtropospheric teleconnection pattern is sufficient to capture the variance of the Aleutian low. The Aleutian low covaries primarily with the Pacific-North American (PNA) pattern but also with the Arctic Oscillation (AO). The change to a prominent deep Aleutian low after 1977 is seen in indices of both the PNA and AO; the return to average conditions after 1989 was also associated with a change in the AO. The authors' analysis suggests an increasing covariability of the high- and midlatitude atmosphere after 1970.},
  file = {/Users/oscardimdore-miles/Zotero/storage/LJEQ4Q5I/Overland et al. - 1999 - Decadal Variability of the Aleutian Low and Its Re.pdf},
  journal = {Journal of Climate - J CLIMATE}
}

@article{Pascoe2005,
  title = {The Quasi-Biennial Oscillation: {{Analysis}} Using {{ERA}}-40 Data},
  author = {Pascoe, Charlotte L. and Gray, Lesley J. and Crooks, Simon A. and Juckes, Martin N. and Baldwin, Mark P.},
  year = {2005},
  volume = {110},
  pages = {1--13},
  doi = {10.1029/2004JD004941},
  journal = {Journal of Geophysical Research D: Atmospheres},
  number = {8}
}

@article{pascoeQuasibiennial2005,
  title = {The Quasi-Biennial Oscillation: {{Analysis}} Using {{ERA}}-40 Data},
  shorttitle = {The Quasi-Biennial Oscillation},
  author = {Pascoe, Charlotte L. and Gray, Lesley J. and Crooks, Simon A. and Juckes, Martin N. and Baldwin, Mark P.},
  year = {2005},
  volume = {110},
  issn = {2156-2202},
  doi = {10.1029/2004JD004941},
  abstract = {The ERA-40 data set is used to examine the equatorial quasi-biennial oscillation (QBO). The data set extends from the ground to 0.1 hPa ({$\sim$}65 km) and covers a 44-year period (January 1958 to December 2001), including 18.5 QBO cycles. Analysis of this data set of unprecedented spatial and temporal coverage has revealed a threefold structure in height in the QBO zonal wind anomalies at the equator. In addition to the well-known twofold structure in the lower and middle stratosphere, that is, easterlies overlying westerlies or vice versa, there is a third anomaly in the upper stratosphere. The sign of this upper stratospheric anomaly is the same as the lower stratospheric anomaly, thus forming anomalies of alternating sign throughout the depth of the equatorial stratosphere. The amplitude of this upper stratospheric anomaly is {$\sim$}10 m s-1, approximately one third of the amplitude of the lower stratospheric signal. The frequency and descent rates of the east and west QBO phases are analyzed in detail, with particular attention to any 11-year solar cycle influence. In addition to the observed solar modulation of the duration of the QBO west phase the analysis shows a solar modulation of the mean descent rate of the easterly shear zone. The mean time required for the easterly shear zone to descend from 20 to 44 hPa is 2 months less under solar maximum conditions than under solar minimum conditions (7.4 months versus 9.7 months). This rapid descent of the easterly shear zone cuts short the west phase of the QBO in the lower stratosphere during solar maximum periods.},
  copyright = {Copyright 2005 by the American Geophysical Union.},
  file = {/Users/oscardimdore-miles/Zotero/storage/4V2IZRCV/Pascoe et al. - 2005 - The quasi-biennial oscillation Analysis using ERA.pdf;/Users/oscardimdore-miles/Zotero/storage/MRMXYFNB/2004JD004941.html},
  journal = {Journal of Geophysical Research: Atmospheres},
  keywords = {QBO,solar,stratosphere},
  language = {English},
  number = {D8}
}

@article{Pawson1999,
  title = {The Cold Winters of the Middle 1990s in the Northern Lower Stratosphere},
  author = {Pawson, Steven and Naujokat, Barbara},
  year = {1999},
  volume = {104},
  pages = {14209--14222},
  doi = {10.1029/1999JD900211},
  journal = {Journal of Geophysical Research: Atmospheres},
  number = {D12}
}

@article{pawsonCold1999,
  title = {The Cold Winters of the Middle 1990s in the Northern Lower Stratosphere},
  author = {Pawson, Steven and Naujokat, Barbara},
  year = {1999},
  volume = {104},
  pages = {14209--14222},
  issn = {2156-2202},
  doi = {10.1029/1999JD900211},
  abstract = {Lower stratospheric temperatures in the northern winters of 1994/1995, 1995/1996, and 1996/1997 were low enough to support polar stratospheric cloud (PSC) formation for prolonged periods. While the seasonal evolution of each winter was quite different, there are some common characteristics: notably, the occurrence of extremely cold periods of long duration and the coldness of the late winter in each year. Comparison with observations over more than three decades indicate the stratosphere was atypically cold in these three years, with the largest anomalies occurring in the late winter and spring. In January and February the coldness seems to be determined by the interannual variability of the circulation, while in March the persistence of the polar vortex dominated the circulation in these three years. This may be related to the lack of major midwinter warmings in those years. Comparison with other winters shows that although the persistence of the polar vortex well into the spring is not unprecedented, this did not occur frequently in the previous two decades. Further, there is a general temperature decrease in the northern lower stratosphere which contributed to the coldness of the three winters. Comparison of the late winter and spring of 1997 with 1967, both of which were forced only weakly by dynamics, supports the idea that this is due to a change in the radiative balance (with equilibrium at a lower temperature), although there are many caveats to this conclusion.},
  copyright = {Copyright 1999 by the American Geophysical Union.},
  file = {/Users/oscardimdore-miles/Zotero/storage/MVA8WBTH/Pawson and Naujokat - 1999 - The cold winters of the middle 1990s in the northe.pdf;/Users/oscardimdore-miles/Zotero/storage/GBLA8UY6/1999JD900211.html},
  journal = {Journal of Geophysical Research: Atmospheres},
  language = {English},
  number = {D12}
}

@article{polvaniDistinguishing2017,
  title = {Distinguishing {{Stratospheric Sudden Warmings}} from {{ENSO}} as {{Key Drivers}} of {{Wintertime Climate Variability}} over the {{North Atlantic}} and {{Eurasia}}},
  author = {Polvani, Lorenzo M. and Sun, Lantao and Butler, Amy H. and Richter, Jadwiga H. and Deser, Clara},
  year = {2017},
  month = mar,
  volume = {30},
  pages = {1959--1969},
  publisher = {{American Meteorological Society}},
  issn = {0894-8755, 1520-0442},
  doi = {10.1175/JCLI-D-16-0277.1},
  abstract = {{$<$}section class="abstract"{$><$}h2 class="abstractTitle text-title my-1" id="d1072e2"{$>$}Abstract{$<$}/h2{$><$}p{$>$}Stratospheric conditions are increasingly being recognized as an important driver of North Atlantic and Eurasian climate variability. Mindful that the observational record is relatively short, and that internal climate variability can be large, the authors here analyze a new 10-member ensemble of integrations of a stratosphere-resolving, atmospheric general circulation model, forced with the observed evolution of sea surface temperature (SST) during 1952\textendash 2003. Previous studies are confirmed, showing that El Ni\~no conditions enhance the frequency of occurrence of stratospheric sudden warmings (SSWs), whereas La Ni\~na conditions do not appear to affect it. However, large differences are noted among ensemble members, suggesting caution when interpreting the relatively short observational record. More importantly, it is emphasized that the majority of SSWs are not caused by anomalous tropical Pacific SSTs. Comparing composites of winters with and without SSWs in each ENSO phase separately, it is demonstrated that stratospheric variability gives rise to large and statistically significant anomalies in tropospheric circulation and surface conditions over the North Atlantic and Eurasia. This indicates that, for those regions, climate variability of stratospheric origin is comparable in magnitude to variability originating from tropical Pacific SSTs, so that the occurrence of a single SSW in a given winter is able to completely alter seasonal climate predictions based solely on ENSO conditions. These findings, corroborating other recent studies, highlight the importance of accurately forecasting SSWs for improved seasonal prediction of North Atlantic and Eurasian climate.{$<$}/p{$><$}/section{$>$}},
  chapter = {Journal of Climate},
  file = {/Users/oscardimdore-miles/Zotero/storage/RRJYEUT8/Polvani et al. - 2017 - Distinguishing Stratospheric Sudden Warmings from .pdf},
  journal = {Journal of Climate},
  language = {English},
  number = {6}
}

@article{Raible2005,
  title = {Northern Hemispheric Trends of Pressure Indices and Atmospheric Circulation Patterns in Observations, Reconstructions, and Coupled {{GCM}} Simulations},
  author = {Raible, Christoph and Stocker, Thomas and Yoshimori, M and Renold, Manuel and Beyerle, Urs and Casty, Carlo and Luterbacher, J{\"u}rg},
  year = {2005},
  volume = {18},
  doi = {10.1175/JCLI3511.1},
  journal = {Journal of Climate}
}

@article{rajendranSynchronisation2016,
  title = {Synchronisation of the Equatorial {{QBO}} by the Annual Cycle in Tropical Upwelling in a Warming Climate},
  author = {Rajendran, Kylash and Moroz, Irene M. and Read, Peter L. and Osprey, Scott M.},
  year = {2016},
  volume = {142},
  pages = {1111--1120},
  issn = {1477-870X},
  doi = {10.1002/qj.2714},
  abstract = {The response of the period of the quasi-biennial oscillation (QBO) to increases in tropical upwelling are considered using a one-dimensional model. We find that the imposition of the annual cycle in tropical upwelling creates substantial variability in the period of the QBO. The annual cycle creates synchronisation regions in the wave forcing space, within which the QBO period locks onto an integer multiple of the annual forcing period. Outside of these regions, the QBO period undergoes discrete jumps as it attempts to find a stable relationship with the oscillator forcing. The resulting set of QBO periods can be either discrete or broad-banded, depending on the intrinsic period of the QBO. We use the same model to study the evolution of the QBO period as the strength of tropical upwelling increases, as would be expected in a warmer climate. The QBO period lengthens and migrates closer towards 36- and 48-month locking regions as upwelling increases. The QBO period does not vary continuously with increased upwelling, however, but instead transitions through a series of two- and three-cycles before becoming locked to the annual cycle. Finally, some observational evidence for the cyclical behaviour of the QBO periods in the real atmosphere is presented.},
  copyright = {\textcopyright{} 2015 The Authors. Quarterly Journal of the Royal Meteorological Society published by John Wiley \& Sons Ltd on behalf of the Royal Meteorological Society.},
  file = {/Users/oscardimdore-miles/Zotero/storage/2ML5K7NP/Rajendran et al. - 2016 - Synchronisation of the equatorial QBO by the annua.pdf;/Users/oscardimdore-miles/Zotero/storage/F7VKSZBK/qj.html},
  journal = {Quarterly Journal of the Royal Meteorological Society},
  keywords = {Brewer–Dobson circulation,climate change,QBO,synchronisation},
  language = {English},
  number = {695}
}

@article{Rao2015,
  title = {A Decomposition of {{ENSO}}'s Impacts on the Northern Winter Stratosphere: {{Competing}} Effect of {{SST}} Forcing in the Tropical {{Indian Ocean}}},
  author = {Rao, Jian and Ren, Rongcai},
  year = {2015},
  volume = {46},
  doi = {10.1007/s00382-015-2797-5},
  journal = {Climate Dynamics}
}

@article{Rao2017,
  title = {Varying Stratospheric Responses to Tropical {{Atlantic SST}} Forcing from Early to Late Winter},
  author = {Rao, Jian and Ren, Rongcai},
  year = {2017},
  doi = {10.1007/s00382-017-3998-x},
  journal = {Climate Dynamics}
}

@article{rao2019,
  title = {Combined Impact of El {{Ni\~no}}\textendash{{Southern}} Oscillation and Pacific Decadal Oscillation on the Northern Winter Stratosphere},
  author = {Rao, Jian and Ren, Rongcai and Xia, Xin and Shi, Chunhua and Guo, Dong},
  year = {2019},
  volume = {10},
  pages = {211},
  doi = {10.3390/atmos10040211},
  journal = {Atmosphere}
}

@article{Rao2020,
  title = {Impact of the Quasi-Biennial Oscillation on the Northern Winter Stratospheric Polar Vortex in {{CMIP5}}/6 Models},
  author = {Rao, Jian and Garfinkel, Chaim I. and White, Ian P.},
  year = {2020},
  volume = {33},
  pages = {4787--4813},
  publisher = {{American Meteorological Society}},
  address = {{Boston MA, USA}},
  doi = {10.1175/JCLI-D-19-0663.1},
  journal = {Journal of Climate},
  number = {11}
}

@article{raoDecomposition2016,
  title = {A Decomposition of {{ENSO}}'s Impacts on the Northern Winter Stratosphere: {{Competing}} Effect of {{SST}} Forcing in the Tropical {{Indian Ocean}}},
  shorttitle = {A Decomposition of {{ENSO}}'s Impacts on the Northern Winter Stratosphere},
  author = {Rao, Jian and Ren, Rongcai},
  year = {2016},
  month = jun,
  volume = {46},
  pages = {3689--3707},
  issn = {1432-0894},
  doi = {10.1007/s00382-015-2797-5},
  abstract = {This study applies WACCM, a stratosphere-resolving model to dissect the stratospheric responses in the northern winter extratropics to the imposed ENSO-related SST anomalies in the tropics. It is found that the anomalously warmer and weaker stratospheric polar vortex during warm ENSO is basically a balance of the opposite effects between the SST anomalies in the tropical Pacific (TPO) and that over the tropical Indian Ocean basin (TIO). Specifically, the ENSO-related SST anomalies over the TIO are to induce an anomalously colder and stronger stratospheric polar vortex during warm ENSO, which acts to partially cancel out the much stronger warmer and weaker polar vortex response to the SST anomalies over the TPO. Further analysis indicates that, while the SST forcing from the TPO contributes to the anomalously positive Pacific North America (PNA) pattern in the troposphere and the enhancement of the stationary wavenumber (WN)-1 in the stratosphere during warm ENSO, the TIO SST forcing is to induce an anomalously negative PNA and a reduction of both WN-1 and WN-2 in the stratosphere. Diagnosis of E\textendash P flux confirms that, the anomalously upward propagation of stationary waves in the extratropics mainly lies over the western coast of North America during warm ENSO, which is mainly associated with the TPO-induced positive PNA response and is partially suppressed by the effect of the accompanying TIO SST forcing.},
  file = {/Users/oscardimdore-miles/Zotero/storage/SWNRXZN4/Rao and Ren - 2016 - A decomposition of ENSO’s impacts on the northern .pdf},
  journal = {Climate Dynamics},
  language = {English},
  number = {11}
}

@article{raoModulation2019,
  title = {Modulation of the {{Northern Winter Stratospheric El Ni\~no}}\textendash{{Southern Oscillation Teleconnection}} by the {{PDO}}},
  author = {Rao, Jian and Garfinkel, Chaim I. and Ren, Rongcai},
  year = {2019},
  month = sep,
  volume = {32},
  pages = {5761--5783},
  publisher = {{American Meteorological Society}},
  issn = {0894-8755, 1520-0442},
  doi = {10.1175/JCLI-D-19-0087.1},
  abstract = {{$<$}section class="abstract"{$><$}h2 class="abstractTitle text-title my-1" id="d719e2"{$>$}Abstract{$<$}/h2{$><$}p{$>$}Using the CMIP5 multimodel ensemble (MME) historical experiments, the modulation of the stratospheric El Ni\~no\textendash Southern Oscillation (ENSO) teleconnection by the Pacific decadal oscillation (PDO) is investigated in this study. El Ni\~no (La Ni\~na) significantly impacts the extratropical stratosphere mainly during the positive (negative) PDO in the MME. Although the composite tropical ENSO SST intensities are similar during the positive and negative PDO in models, the Pacific\textendash North American (PNA) responses are only significant when the PDO and ENSO are in phase. The local SST anomalies in the North Pacific can constructively (destructively) interfere with the tropical ENSO forcing to influence the extratropical eddy height anomalies when the PDO and ENSO are in (out of) phase. The difference between the positive and negative PDO in El Ni\~no or La Ni\~na winters filters out the tropical SST forcing, permitting the deduction of the extratropical SST contribution to the atmospheric response. The composite shows that the cold (warm) SST anomalies in the central North Pacific associated with the positive (negative) PDO have a similar impact to that of the warm (cold) SST anomalies in the tropical Pacific, exhibiting a positive (negative) PNA-like response, enhancing (weakening) the upward propagation of waves over the western coast of North America. The composite difference between the positive and negative PDO in El Ni\~no or La Ni\~na winters, as well as in eastern Pacific ENSO or central Pacific ENSO winters, presents a highly consistent atmospheric response pattern, which may imply a linear interference of the PDO's impact with ENSO's.{$<$}/p{$><$}/section{$>$}},
  chapter = {Journal of Climate},
  file = {/Users/oscardimdore-miles/Zotero/storage/BDHVSZLT/Rao et al. - 2019 - Modulation of the Northern Winter Stratospheric El.pdf;/Users/oscardimdore-miles/Zotero/storage/UQABXT5E/jcli-d-19-0087.1.html},
  journal = {Journal of Climate},
  language = {English},
  number = {18}
}

@article{raoModulation2019a,
  title = {Modulation of the {{Northern Winter Stratospheric El Ni\~no}}\textendash{{Southern Oscillation Teleconnection}} by the {{PDO}}},
  author = {Rao, Jian and Garfinkel, Chaim I. and Ren, Rongcai},
  year = {2019},
  month = sep,
  volume = {32},
  pages = {5761--5783},
  publisher = {{American Meteorological Society}},
  issn = {0894-8755, 1520-0442},
  doi = {10.1175/JCLI-D-19-0087.1},
  abstract = {{$<$}section class="abstract"{$><$}h2 class="abstractTitle text-title my-1" id="d46e2"{$>$}Abstract{$<$}/h2{$><$}p{$>$}Using the CMIP5 multimodel ensemble (MME) historical experiments, the modulation of the stratospheric El Ni\~no\textendash Southern Oscillation (ENSO) teleconnection by the Pacific decadal oscillation (PDO) is investigated in this study. El Ni\~no (La Ni\~na) significantly impacts the extratropical stratosphere mainly during the positive (negative) PDO in the MME. Although the composite tropical ENSO SST intensities are similar during the positive and negative PDO in models, the Pacific\textendash North American (PNA) responses are only significant when the PDO and ENSO are in phase. The local SST anomalies in the North Pacific can constructively (destructively) interfere with the tropical ENSO forcing to influence the extratropical eddy height anomalies when the PDO and ENSO are in (out of) phase. The difference between the positive and negative PDO in El Ni\~no or La Ni\~na winters filters out the tropical SST forcing, permitting the deduction of the extratropical SST contribution to the atmospheric response. The composite shows that the cold (warm) SST anomalies in the central North Pacific associated with the positive (negative) PDO have a similar impact to that of the warm (cold) SST anomalies in the tropical Pacific, exhibiting a positive (negative) PNA-like response, enhancing (weakening) the upward propagation of waves over the western coast of North America. The composite difference between the positive and negative PDO in El Ni\~no or La Ni\~na winters, as well as in eastern Pacific ENSO or central Pacific ENSO winters, presents a highly consistent atmospheric response pattern, which may imply a linear interference of the PDO's impact with ENSO's.{$<$}/p{$><$}/section{$>$}},
  chapter = {Journal of Climate},
  file = {/Users/oscardimdore-miles/Zotero/storage/LYUI8Y64/Rao et al. - 2019 - Modulation of the Northern Winter Stratospheric El.pdf;/Users/oscardimdore-miles/Zotero/storage/C4IIVIZ9/jcli-d-19-0087.1.html},
  journal = {Journal of Climate},
  language = {English},
  number = {18}
}

@article{raoVarying2018,
  title = {Varying Stratospheric Responses to Tropical {{Atlantic SST}} Forcing from Early to Late Winter},
  author = {Rao, Jian and Ren, Rongcai},
  year = {2018},
  month = sep,
  volume = {51},
  pages = {2079--2096},
  issn = {1432-0894},
  doi = {10.1007/s00382-017-3998-x},
  abstract = {Using multiple reanalysis datasets and model simulations, we begin in this study by isolating the tropical Atlantic Ocean (TAO) sea surface temperature (SST) signals that are independent from ENSO, and then investigate their influences on the northern winter stratosphere. It is revealed that TAO SST forcing does indeed have significant effects on the northern winter stratosphere, but these effects vary from early to late winter in a way that explains the overall insignificant effect when the seasonal average is considered. The stratospheric polar vortex is anomalously weaker/warmer in November\textendash December, stronger/colder in January\textendash March, and weaker/warmer again in April\textendash May during warm TAO years. The varying impacts of the TAO forcing on the extratropical stratosphere are related to a three-stage response of the extratropical troposphere to the TAO forcing during cold season. The tropospheric circulation exhibits a negative North Atlantic Oscillation\textendash like response during early winter, an eastward propagating Rossby wave pattern in mid-to-late winter, and a meridional dipole over North America in spring. Associated with this is varying planetary wave activity in the stratosphere, manifested as an increase in early winter, a decrease in mid-to-late winter, and an increase again in spring. The varying modulation of stratospheric circulation by TAO forcing is consistently confirmed in three reanalysis datasets, and model simulations (fully coupled model and its component AGCM). The exception to the robustness of this verification is that the circumpolar wind response in the fully coupled model is relatively weaker, and that in its component AGCM appears a month later than observed.},
  file = {/Users/oscardimdore-miles/Zotero/storage/6CWIZAUG/Rao and Ren - 2018 - Varying stratospheric responses to tropical Atlant.pdf},
  journal = {Climate Dynamics},
  language = {English},
  number = {5}
}

@article{reichlerStratospheric2012,
  title = {A Stratospheric Connection to {{Atlantic}} Climate Variability},
  author = {Reichler, Thomas and Kim, Junsu and Manzini, Elisa and Kr{\"o}ger, J{\"u}rgen},
  year = {2012},
  month = nov,
  volume = {5},
  pages = {783--787},
  publisher = {{Nature Publishing Group}},
  issn = {1752-0908},
  doi = {10.1038/ngeo1586},
  abstract = {Stratospheric circulation is known to affect weather in the troposphere. Climate modelling reveals a connection between variations in the stratospheric and North Atlantic ocean circulation over the past 30 years, and demonstrates that the stratosphere is an important component of climate over multidecadal timescales.},
  copyright = {2012 Nature Publishing Group},
  file = {/Users/oscardimdore-miles/Zotero/storage/CPURJD8A/Reichler et al. - 2012 - A stratospheric connection to Atlantic climate var.pdf;/Users/oscardimdore-miles/Zotero/storage/YVGHAB2E/ngeo1586.html},
  journal = {Nature Geoscience},
  language = {English},
  number = {11}
}

@article{Richter2015,
  title = {Effects of Stratospheric Variability on El Ni\~no Teleconnections},
  author = {Richter, J. and Deser, C. and Sun, L.},
  year = {2015},
  volume = {10},
  pages = {124021},
  doi = {10.1088/1748-9326/10/12/124021},
  journal = {Environmental Research Letters},
  number = {12}
}

@article{richterEffects2015,
  title = {Effects of Stratospheric Variability on {{El Ni\~no}} Teleconnections},
  author = {Richter, J. H. and Deser, C. and Sun, L.},
  year = {2015},
  month = dec,
  volume = {10},
  pages = {124021},
  publisher = {{IOP Publishing}},
  issn = {1748-9326},
  doi = {10.1088/1748-9326/10/12/124021},
  file = {/Users/oscardimdore-miles/Zotero/storage/TUYM8N3L/Richter et al. - 2015 - Effects of stratospheric variability on El Niño te.pdf;/Users/oscardimdore-miles/Zotero/storage/3G6HMM55/meta.html},
  journal = {Environmental Research Letters},
  language = {English},
  number = {12}
}

@article{Ridley2018,
  title = {The Sea Ice Model Component of {{HadGEM3}}-{{GC3}}.1},
  author = {Ridley, J. K. and Blockley, E. W. and Keen, A. B. and Rae, J. G. L. and West, A. E. and Schroeder, D.},
  year = {2018},
  volume = {11},
  pages = {713--723},
  doi = {10.5194/gmd-11-713-2018},
  journal = {Geoscientific Model Development},
  number = {2}
}

@article{ridleySea2018,
  title = {The Sea Ice Model Component of {{HadGEM3}}-{{GC3}}.1},
  author = {Ridley, Jeff K. and Blockley, Edward W. and Keen, Ann B. and Rae, Jamie G. L. and West, Alex E. and Schroeder, David},
  year = {2018},
  month = feb,
  volume = {11},
  pages = {713--723},
  publisher = {{Copernicus GmbH}},
  issn = {1991-959X},
  doi = {10.5194/gmd-11-713-2018},
  abstract = {{$<$}p{$><$}strong class="journal-contentHeaderColor"{$>$}Abstract.{$<$}/strong{$>$} A new sea ice configuration, GSI8.1, is implemented in the Met Office global coupled configuration HadGEM3-GC3.1 which will be used for all CMIP6 (Coupled Model Intercomparison Project Phase 6) simulations. The inclusion of multi-layer thermodynamics has required a semi-implicit coupling scheme between atmosphere and sea ice to ensure the stability of the solver. Here we describe the sea ice model component and show that the Arctic thickness and extent compare well with observationally based data.{$<$}/p{$>$}},
  file = {/Users/oscardimdore-miles/Zotero/storage/73I8WP6N/Ridley et al. - 2018 - The sea ice model component of HadGEM3-GC3.1.pdf;/Users/oscardimdore-miles/Zotero/storage/E5RZGPHI/2018.html},
  journal = {Geoscientific Model Development},
  language = {English},
  number = {2}
}

@article{robsonCauses2012,
  title = {Causes of the {{Rapid Warming}} of the {{North Atlantic Ocean}} in the {{Mid}}-1990s},
  author = {Robson, Jon and Sutton, Rowan and Lohmann, Katja and Smith, Doug and Palmer, Matthew D.},
  year = {2012},
  month = jun,
  volume = {25},
  pages = {4116--4134},
  publisher = {{American Meteorological Society}},
  issn = {0894-8755, 1520-0442},
  doi = {10.1175/JCLI-D-11-00443.1},
  abstract = {{$<$}section class="abstract"{$><$}h2 class="abstractTitle text-title my-1" id="d7e2"{$>$}Abstract{$<$}/h2{$><$}p{$>$}In the mid-1990s, the subpolar gyre of the North Atlantic underwent a remarkable rapid warming, with sea surface temperatures increasing by around 1\textdegree C in just 2 yr. This rapid warming followed a prolonged positive phase of the North Atlantic Oscillation (NAO) but also coincided with an unusually negative NAO index in the winter of 1995/96. By comparing ocean analyses and carefully designed model experiments, it is shown that this rapid warming can be understood as a delayed response to the prolonged positive phase of the NAO and not simply an instantaneous response to the negative NAO index of 1995/96. Furthermore, it is inferred that the warming was partly caused by a surge and subsequent decline in the meridional overturning circulation and northward heat transport of the Atlantic Ocean. These results provide persuasive evidence of significant oceanic memory on multiannual time scales and are therefore encouraging for the prospects of developing skillful predictions.{$<$}/p{$><$}/section{$>$}},
  chapter = {Journal of Climate},
  file = {/Users/oscardimdore-miles/Zotero/storage/YV54Q3CQ/Robson et al. - 2012 - Causes of the Rapid Warming of the North Atlantic .pdf;/Users/oscardimdore-miles/Zotero/storage/TMWG86EM/jcli-d-11-00443.1.html},
  journal = {Journal of Climate},
  language = {English},
  number = {12}
}

@article{robsonEvaluation2020,
  title = {The {{Evaluation}} of the {{North Atlantic Climate System}} in {{UKESM1 Historical Simulations}} for {{CMIP6}}},
  author = {Robson, Jon and Aksenov, Yevgeny and Bracegirdle, Thomas J. and {Dimdore-Miles}, Oscar and Griffiths, Paul T. and Grosvenor, Daniel P. and Hodson, Daniel L. R. and Keeble, James and MacIntosh, Claire and Megann, Alex and Osprey, Scott and Povey, Adam C. and Schr{\"o}der, David and Yang, Mingxi and Archibald, Alexander T. and Carslaw, Ken S. and Gray, Lesley and Jones, Colin and Kerridge, Brian and Knappett, Diane and Kuhlbrodt, Till and Russo, Maria and Sellar, Alistair and Siddans, Richard and Sinha, Bablu and Sutton, Rowan and Walton, Jeremy and Wilcox, Laura J.},
  year = {2020},
  volume = {12},
  pages = {e2020MS002126},
  issn = {1942-2466},
  doi = {10.1029/2020MS002126},
  abstract = {Earth system models enable a broad range of climate interactions that physical climate models are unable to simulate. However, the extent to which adding Earth system components changes or improves the simulation of the physical climate is not well understood. Here we present a broad multivariate evaluation of the North Atlantic climate system in historical simulations of the UK Earth System Model (UKESM1) performed for CMIP6. In particular, we focus on the mean state and the decadal time scale evolution of important variables that span the North Atlantic climate system. In general, UKESM1 performs well and realistically simulates many aspects of the North Atlantic climate system. Like the physical version of the model, we find that changes in external forcing, and particularly aerosol forcing, are an important driver of multidecadal change in UKESM1, especially for Atlantic Multidecadal Variability and the Atlantic Meridional Overturning Circulation. However, many of the shortcomings identified are similar to common biases found in physical climate models, including the physical climate model that underpins UKESM1. For example, the summer jet is too weak and too far poleward; decadal variability in the winter jet is underestimated; intraseasonal stratospheric polar vortex variability is poorly represented; and Arctic sea ice is too thick. Forced shortwave changes may be also too strong in UKESM1, which, given the important role of historical aerosol forcing in shaping the evolution of the North Atlantic in UKESM1, motivates further investigation. Therefore, physical model development, alongside Earth system development, remains crucial in order to improve climate simulations.},
  file = {/Users/oscardimdore-miles/Zotero/storage/RIJ5YRM2/Robson et al. - 2020 - The Evaluation of the North Atlantic Climate Syste.pdf;/Users/oscardimdore-miles/Zotero/storage/GATU3T8S/2020MS002126.html},
  journal = {Journal of Advances in Modeling Earth Systems},
  keywords = {CMIP6,Earth system model,model evaluation,North Atlantic},
  language = {English},
  number = {9}
}

@article{Rodionov2005,
  title = {The Aleutian Low and Winter Climatic Conditions in the Bering Sea. {{Part I}}: {{Classification}}},
  author = {Rodionov, S. N. and Overland, J. E. and Bond, N. A.},
  year = {2005},
  volume = {18},
  pages = {160--177},
  doi = {10.1175/JCLI3253.1},
  journal = {Journal of Climate},
  number = {1}
}

@article{rodionovSpatial2005,
  title = {Spatial and Temporal Variability of the {{Aleutian}} Climate},
  author = {Rodionov, Sergei N. and Overland, James E. and Bond, Nicholas A.},
  year = {2005},
  volume = {14},
  pages = {3--21},
  issn = {1365-2419},
  doi = {10.1111/j.1365-2419.2005.00363.x},
  abstract = {The objective of this paper is to highlight those characteristics of climate variability that may pertain to the climate hypothesis regarding the long-term population decline of Steller sea lions (Eumetopias jubatus). The seasonal changes in surface air temperature (SAT) across the Aleutian Islands are relatively uniform, from 5 to 10\textdegree C in summer to near freezing temperatures in winter. The interannual and interdecadal variations in SAT, however, are substantially different for the eastern and western Aleutians, with the transition found at about 170\textdegree W. The eastern Aleutians experienced a regime shift toward a warmer climate in 1977, simultaneously with the basin-wide shift in the Pacific Decadal Oscillation (PDO). In contrast, the western Aleutians show a steady decline in winter SATs that started in the 1950s. This cooling trend was accompanied by a trend toward more variable SAT, both on the inter- and intra-annual time scale. During 1986\textendash 2002, the variance of winter SATs more than doubled compared to 1965\textendash 1985. At the same time in Southeast Alaska, the SAT variance diminished by half. Much of the increase in the intra-seasonal variability for the western Aleutians is associated with a warming trend in November and a cooling trend in January. As a result, the rate of seasonal cooling from November to January has doubled since the late 1950s. We hypothesize that this trend in SAT variability may have increased the environmental stress on the western stock of Steller sea lions and hence contributed to its decline.},
  file = {/Users/oscardimdore-miles/Zotero/storage/JF4PY76W/Rodionov et al. - 2005 - Spatial and temporal variability of the Aleutian c.pdf;/Users/oscardimdore-miles/Zotero/storage/79AAAM2K/j.1365-2419.2005.00363.html},
  journal = {Fisheries Oceanography},
  keywords = {500 hPa height,Aleutian Islands,Aleutian Low,regime shift,sea level pressure,storm tracks,surface air temperature,trend},
  language = {English},
  number = {s1}
}

@article{Santoso2017,
  title = {The Defining Characteristics of {{ENSO}} Extremes and the Strong 2015/2016 El Ni\~no},
  author = {Santoso, Agus and Mcphaden, Michael J. and Cai, Wenju},
  year = {2017},
  volume = {55},
  pages = {1079--1129},
  doi = {10.1002/2017RG000560},
  journal = {Reviews of Geophysics},
  keywords = {El Nino,ENSO,extremes,greenhouse warming,La Nina,Pacific},
  number = {4}
}

@article{santosoDefining2017,
  title = {The {{Defining Characteristics}} of {{ENSO Extremes}} and the {{Strong}} 2015/2016 {{El Ni\~no}}},
  author = {Santoso, Agus and Mcphaden, Michael J. and Cai, Wenju},
  year = {2017},
  volume = {55},
  pages = {1079--1129},
  issn = {1944-9208},
  doi = {10.1002/2017RG000560},
  abstract = {The year 2015 was special for climate scientists, particularly for the El Ni\~no Southern Oscillation (ENSO) research community, as a major El Ni\~no finally materialized after a long pause since the 1997/1998 extreme El Ni\~no. It was scientifically exciting since, due to the short observational record, our knowledge of an extreme El Ni\~no has been based only on the 1982/1983 and 1997/1998 events. The 2015/2016 El Ni\~no was marked by many environmental disasters that are consistent with what is expected for an extreme El Ni\~no. Considering the dramatic impacts of extreme El Ni\~no, and the risk of a potential increase in frequency of ENSO extremes under greenhouse warming, it is timely to evaluate how the recent event fits into our understanding of ENSO extremes. Here we provide a review of ENSO, its nature and dynamics, and through analysis of various observed key variables, we outline the processes that characterize its extremes. The 2015/2016 El Ni\~no brings a useful perspective into the state of understanding of these events and highlights areas for future research. While the 2015/2016 El Ni\~no is characteristically distinct from the 1982/1983 and 1997/1998 events, it still can be considered as the first extreme El Ni\~no of the 21st century. Its extremity can be attributed in part to unusually warm condition in 2014 and to long-term background warming. In effect, this study provides a list of physically meaningful indices that are straightforward to compute for identifying and tracking extreme ENSO events in observations and climate models.},
  copyright = {\textcopyright 2017. American Geophysical Union. All Rights Reserved.},
  file = {/Users/oscardimdore-miles/Zotero/storage/CR88H79A/Santoso et al. - 2017 - The Defining Characteristics of ENSO Extremes and .pdf},
  journal = {Reviews of Geophysics},
  keywords = {El Nino,ENSO,extremes,greenhouse warming,La Nina,Pacific},
  language = {English},
  number = {4}
}

@article{Scaife2014,
  title = {Predictability of the Quasi-Biennial Oscillation and Its Northern Winter Teleconnection on Seasonal to Decadal Timescales},
  author = {Scaife, Adam A. and Athanassiadou, Maria and Andrews, Martin and Arribas, Alberto and Baldwin, Mark and Dunstone, Nick and Knight, Jeff and MacLachlan, Craig and Manzini, Elisa and M{\"u}ller, Wolfgang A. and Pohlmann, Holger and Smith, Doug and Stockdale, Tim and Williams, Andrew},
  year = {2014},
  volume = {41},
  pages = {1752--1758},
  journal = {Geophysical Research Letters},
  number = {5}
}

@article{Scaife2016,
  title = {Tropical Rainfall, {{Rossby}} Waves and Regional Winter Climate Predictions},
  author = {Scaife, Adam A. and Comer, Ruth E. and Dunstone, Nick J. and Knight, Jeff R. and Smith, Doug M. and MacLachlan, Craig and Martin, Nicola and Peterson, K. Andrew and Rowlands, Dan and Carroll, Edward B. and Belcher, Stephen and Slingo, Julia},
  year = {2017},
  volume = {143},
  pages = {1--11},
  doi = {10.1002/qj.2910},
  journal = {Quarterly Journal of the Royal Meteorological Society},
  number = {702}
}

@article{scaifeSignaltonoise2018,
  title = {A Signal-to-Noise Paradox in Climate Science},
  author = {Scaife, Adam A. and Smith, Doug},
  year = {2018},
  month = jul,
  volume = {1},
  pages = {1--8},
  publisher = {{Nature Publishing Group}},
  issn = {2397-3722},
  doi = {10.1038/s41612-018-0038-4},
  abstract = {We review the growing evidence for a widespread inconsistency between the low strength of predictable signals in climate models and the relatively high level of agreement they exhibit with observed variability of the atmospheric circulation. This discrepancy is particularly evident in the climate variability of the Atlantic sector, where ensemble predictions using climate models generally show higher correlation with observed variability than with their own simulations, and higher correlations with observations than would be expected from their small signal-to-noise ratios, hence a `signal-to-noise paradox'. This unusual behaviour has been documented in multiple climate prediction systems and in the response to a number of different sources of climate variability. However, we also note that the total variance in the models is often close in magnitude to the observed variance, and so it is not a simple matter of models containing too much variability. Instead, the proportion of Atlantic climate variance that is predictable in climate models appears to be too weak in amplitude by a factor of two, or perhaps more. In this review, we provide a range of examples from existing studies to build the case for a problem that is common across different climate models, common to several different sources of climate variability and common across a range of timescales. We also discuss the wider implications of this intriguing paradox.},
  copyright = {2018 The Author(s)},
  file = {/Users/oscardimdore-miles/Zotero/storage/G7EKZXDN/Scaife and Smith - 2018 - A signal-to-noise paradox in climate science.pdf;/Users/oscardimdore-miles/Zotero/storage/UYGWIJ6E/s41612-018-0038-4.html},
  journal = {npj Climate and Atmospheric Science},
  language = {English},
  number = {1}
}

@article{scaifeTropical2017,
  title = {Tropical Rainfall, {{Rossby}} Waves and Regional Winter Climate Predictions},
  author = {Scaife, Adam A. and Comer, Ruth E. and Dunstone, Nick J. and Knight, Jeff R. and Smith, Doug M. and MacLachlan, Craig and Martin, Nicola and Peterson, K. Andrew and Rowlands, Dan and Carroll, Edward B. and Belcher, Stephen and Slingo, Julia},
  year = {2017},
  volume = {143},
  pages = {1--11},
  issn = {1477-870X},
  doi = {10.1002/qj.2910},
  abstract = {Skilful climate predictions of the winter North Atlantic Oscillation and Arctic Oscillation out to a few months ahead have recently been demonstrated, but the source of this predictability remains largely unknown. Here we investigate the role of the Tropics in this predictability. We show high levels of skill in tropical rainfall predictions, particularly over the Pacific but also the Indian and Atlantic Ocean basins. Rainfall fluctuations in these regions are associated with clear signatures in tropical and extratropical atmospheric circulation that are approximately symmetric about the Equator in boreal winter. We show how these patterns can be explained as steady poleward propagating linear Rossby waves emanating from just a few key source regions. These wave source `hotspots' become more or less active as tropical rainfall varies from winter to winter but they do not change position. Finally, we show that predicted tropical rainfall explains a highly significant fraction of the predicted year-to-year variation of the winter North Atlantic Oscillation.},
  copyright = {\textcopyright{} 2016 Crown Copyright, Met Office. Quarterly Journal of the Royal Meteorological Society published by John Wiley \& Sons Ltd on behalf of the Royal Meteorological Society.},
  file = {/Users/oscardimdore-miles/Zotero/storage/ZS4EXG86/Scaife et al. - 2017 - Tropical rainfall, Rossby waves and regional winte.pdf;/Users/oscardimdore-miles/Zotero/storage/XUQND8YS/qj.html},
  journal = {Quarterly Journal of the Royal Meteorological Society},
  keywords = {North Atlantic Oscillation,Rossby wave,teleconnection,tropical rainfall},
  language = {English},
  number = {702}
}

@article{Schimanke2011,
  title = {Multi-Decadal Variability of Sudden Stratospheric Warmings in an {{AOGCM}}},
  author = {Schimanke, S. and K{\"o}rper, J. and Spangehl, T. and Cubasch, U.},
  year = {2011},
  volume = {38},
  pages = {1--6},
  doi = {10.1029/2010GL045756},
  journal = {Geophysical Research Letters},
  number = {1}
}

@article{schimankeMultidecadal2011,
  title = {Multi-Decadal Variability of Sudden Stratospheric Warmings in an {{AOGCM}}},
  author = {Schimanke, Semjon and Zittel, Janina and Spangehl, Thomas and Cubasch, Ulrich},
  year = {2011},
  month = jan,
  volume = {38},
  doi = {10.1029/2010GL045756},
  abstract = {The variability in the number of major sudden stratospheric warmings (SSWs) is analyzed in a multi-century simulation under constant forcing using a stratosphere resolving atmosphere-ocean general circulation model. A wavelet-analysis of the SSW time series identifies significantly enhanced power at a period of 52 years. The coherency of this signal with tropospheric and oceanic parameters is investigated. The strongest coherence is found with the North Atlantic ocean-atmosphere heat-flux from November to January. Here, an enhanced heat-flux from the ocean into the atmosphere is related to an increase in the number of SSWs. Furthermore, a correlation is found with Eurasian snow cover in October and the number of blockings in October/November. These results suggest that the multi-decadal variability is generated within the ocean-troposphere-stratosphere system. A two-way interaction of the North Atlantic and the atmosphere buffers and amplifies stratospheric anomalies, leading to a coupled multi-decadal mode.},
  file = {/Users/oscardimdore-miles/Zotero/storage/YUBBVA9B/Schimanke et al. - 2011 - Multi-decadal variability of sudden stratospheric .pdf},
  journal = {Geophysical Research Letters - GEOPHYS RES LETT}
}

@article{scottInternal2006,
  title = {Internal {{Variability}} of the {{Winter Stratosphere}}. {{Part I}}: {{Time}}-{{Independent Forcing}}},
  shorttitle = {Internal {{Variability}} of the {{Winter Stratosphere}}. {{Part I}}},
  author = {Scott, R. K. and Polvani, L. M.},
  year = {2006},
  month = nov,
  volume = {63},
  pages = {2758--2776},
  publisher = {{American Meteorological Society}},
  issn = {0022-4928, 1520-0469},
  doi = {10.1175/JAS3797.1},
  abstract = {{$<$}section class="abstract"{$><$}h2 class="abstractTitle text-title my-1" id="d1184e2"{$>$}Abstract{$<$}/h2{$><$}p{$>$}This paper examines the nature and robustness of internal stratospheric variability, namely the variability resulting from the internal dynamics of the stratosphere itself, as opposed to that forced by external sources such as the natural variability of the free troposphere. Internal stratospheric variability arises from the competing actions of radiative forcing, which under perpetual winter conditions strengthens the polar vortex, and planetary wave breaking, which weakens it. The results from a stratosphere-only model demonstrate that strong internal stratospheric variability, consisting of repeated sudden warming-type events, exists over a wide range of realistic radiative and wave forcing conditions, and is largely independent of other physical and numerical parameters. In particular, the coherent form of the variability persists as the number of degrees of freedom is increased, and is therefore not an artifact of severe model truncation. Various diagnostics, including three-dimensional representations of the potential vorticity, illustrate that the variability is determined by the vertical structure of the vortex and the extent to which upward wave propagation is favored or inhibited. In this paper, the variability arising from purely internal stratosphere dynamics is isolated by specifying thermal and wave forcings that are completely time independent. In a second paper, the authors investigate the relative importance of internal and external variability by considering time-dependent wave forcing as a simple representation of tropospheric variability.{$<$}/p{$><$}/section{$>$}},
  chapter = {Journal of the Atmospheric Sciences},
  file = {/Users/oscardimdore-miles/Zotero/storage/MWRW96MW/Scott and Polvani - 2006 - Internal Variability of the Winter Stratosphere. P.pdf;/Users/oscardimdore-miles/Zotero/storage/7TT7P8IM/jas3797.1.html},
  journal = {Journal of the Atmospheric Sciences},
  language = {English},
  number = {11}
}

@article{Seviour2017,
  title = {Weakening and Shift of the {{Arctic}} Stratospheric Polar Vortex: {{Internal}} Variability or Forced Response?},
  author = {Seviour, William J. M.},
  year = {2017},
  volume = {44},
  pages = {3365--3373},
  doi = {10.1002/2017GL073071},
  journal = {Geophysical Research Letters},
  number = {7}
}

@article{shawLife2013,
  title = {The {{Life Cycle}} of {{Northern Hemisphere Downward Wave Coupling}} between the {{Stratosphere}} and {{Troposphere}}},
  author = {Shaw, Tiffany A. and Perlwitz, Judith},
  year = {2013},
  month = mar,
  volume = {26},
  pages = {1745--1763},
  publisher = {{American Meteorological Society}},
  issn = {0894-8755, 1520-0442},
  doi = {10.1175/JCLI-D-12-00251.1},
  abstract = {{$<$}section class="abstract"{$><$}h2 class="abstractTitle text-title my-1" id="d72372863e70"{$>$}Abstract{$<$}/h2{$><$}p{$>$}The life cycle of Northern Hemisphere downward wave coupling between the stratosphere and troposphere via wave reflection is analyzed. Downward wave coupling events are defined by extreme negative values of a wave coupling index based on the leading principal component of the daily wave-1 heat flux at 30 hPa. The life cycle occurs over a 28-day period. In the stratosphere there is a transition from positive to negative total wave-1 heat flux and westward to eastward phase tilt with height of the wave-1 geopotential height field. In addition, the zonal-mean zonal wind in the upper stratosphere weakens leading to negative vertical shear.Following the evolution in the stratosphere there is a shift toward the positive phase of the North Atlantic Oscillation (NAO) in the troposphere. The pattern develops from a large westward-propagating wave-1 anomaly in the high-latitude North Atlantic sector. The subsequent equatorward propagation leads to a positive anomaly in midlatitudes. The near-surface temperature and circulation anomalies are consistent with a positive NAO phase. The results suggest that wave reflection events can directly influence tropospheric weather.Finally, winter seasons dominated by extreme wave coupling and stratospheric vortex events are compared. The largest impacts in the troposphere occur during the extreme negative seasons for both indices, namely seasons with multiple wave reflection events leading to a positive NAO phase or seasons with major sudden stratospheric warmings (weak vortex) leading to a negative NAO phase. The results reveal that the dynamical coupling between the stratosphere and NAO involves distinct dynamical mechanisms that can only be characterized by separate wave coupling and vortex indices.{$<$}/p{$><$}/section{$>$}},
  chapter = {Journal of Climate},
  file = {/Users/oscardimdore-miles/Zotero/storage/TF55FY4H/Shaw and Perlwitz - 2013 - The Life Cycle of Northern Hemisphere Downward Wav.pdf;/Users/oscardimdore-miles/Zotero/storage/TWUYA77N/jcli-d-12-00251.1.html},
  journal = {Journal of Climate},
  language = {English},
  number = {5}
}

@article{Shindell1999,
  title = {Simulation of Recent Northern Winter Climate Trends by Greenhouse-Gas Forcing},
  author = {Shindell, D. T. and Miller, R. L. and Schmidt, G. A. and Pandolfo, L.},
  year = {1999},
  volume = {399},
  pages = {452--455},
  doi = {10.1038/20905},
  journal = {Nature}
}

@misc{smeedAtlantic2019,
  title = {Atlantic Meridional Overturning Circulation Observed by the {{RAPID}}-{{MOCHA}}-{{WBTS}} ({{RAPID}}-{{Meridional Overturning Circulation}} and {{Heatflux Array}}-{{Western Boundary Time Series}}) Array at {{26N}} from 2004 to 2018.},
  author = {Smeed, David and Moat, Ben I and Rayner, Darren and Johns, William E and Baringer, Molly O and Volkov, Denis L and {Frajka-Williams}, Eleanor},
  year = {2019},
  publisher = {{British Oceanographic Data Centre, National Oceanography Centre, NERC, UK}},
  doi = {10.5285/8CD7E7BB-9A20-05D8-E053-6C86ABC012C2},
  abstract = {The RAPID-MOCHA-WBTS (RAPID-Meridional Overturning Circulation and Heatflux Array-Western Boundary Time Series) programme has produced a continuous time series of the Atlantic Meridional Overturning Circulation (AMOC) at 26N that started in April 2004. This release of the time series extends the data to September 2018. The 26N AMOC time series is derived from measurements of temperature, salinity, pressure and water velocity from an array of moored instruments that extend from the east coast of the Bahamas to the continental shelf off Africa east of the Canary Islands. The AMOC calculation also uses estimates of the transport in the Florida Strait derived from sub-sea cable measurements calibrated by regular hydrographic cruises. The component of the AMOC associated with the wind driven Ekman layer is derived from satellite scatterometer measurements. This release of the data includes a document with a brief description of the calculation of the AMOC time series and references to more detailed description in published papers. The 26N AMOC time series and the data from the moored array are curated by the British Oceanographic Data Centre (BODC). The RAPID-MOCHA-WBTS programme is a joint effort between NERC in the UK (Principle Investigator David Smeed since 2012 and Stuart Cunningham from 2004 to 2012), NOAA (PIs Molly Baringer and Denis Volkov) and NSF (PI Prof. Bill Johns, Uni. Miami) in the USA.},
  keywords = {elevation,oceans},
  language = {English}
}

@article{smeedNorth2018,
  title = {The {{North Atlantic Ocean Is}} in a {{State}} of {{Reduced Overturning}}},
  author = {Smeed, D. A. and Josey, S. A. and Beaulieu, C. and Johns, W. E. and Moat, B. I. and {Frajka-Williams}, E. and Rayner, D. and Meinen, C. S. and Baringer, M. O. and Bryden, H. L. and McCarthy, G. D.},
  year = {2018},
  volume = {45},
  pages = {1527--1533},
  issn = {1944-8007},
  doi = {10.1002/2017GL076350},
  abstract = {The Atlantic Meridional Overturning Circulation (AMOC) is responsible for a variable and climatically important northward transport of heat. Using data from an array of instruments that span the Atlantic at 26\textdegree N, we show that the AMOC has been in a state of reduced overturning since 2008 as compared to 2004\textendash 2008. This change of AMOC state is concurrent with other changes in the North Atlantic such as a northward shift and broadening of the Gulf Stream and altered patterns of heat content and sea surface temperature. These changes resemble the response to a declining AMOC predicted by coupled climate models. Concurrent changes in air-sea fluxes close to the western boundary reveal that the changes in ocean heat transport and sea surface temperature have altered the pattern of ocean-atmosphere heat exchange over the North Atlantic. These results provide strong observational evidence that the AMOC is a major factor in decadal-scale variability of North Atlantic climate.},
  copyright = {\textcopyright 2018. The Authors.},
  file = {/Users/oscardimdore-miles/Zotero/storage/SEA7J9F9/Smeed et al. - 2018 - The North Atlantic Ocean Is in a State of Reduced .pdf;/Users/oscardimdore-miles/Zotero/storage/PG54VGQ5/2017GL076350.html},
  journal = {Geophysical Research Letters},
  keywords = {AMOC,Atlantic,circulation,overturning},
  language = {English},
  number = {3}
}

@article{smeedObserved2014,
  title = {Observed Decline of the {{Atlantic}} Meridional Overturning Circulation 2004\&ndash;2012},
  author = {Smeed, D. A. and McCarthy, G. D. and Cunningham, S. A. and {Frajka-Williams}, E. and Rayner, D. and Johns, W. E. and Meinen, C. S. and Baringer, M. O. and Moat, B. I. and Duchez, A. and Bryden, H. L.},
  year = {2014},
  month = feb,
  volume = {10},
  pages = {29--38},
  publisher = {{Copernicus GmbH}},
  issn = {1812-0784},
  doi = {10.5194/os-10-29-2014},
  abstract = {{$<$}p{$><$}strong class="journal-contentHeaderColor"{$>$}Abstract.{$<$}/strong{$>$} The Atlantic meridional overturning circulation (AMOC) has been observed continuously at 26\textdegree{} N since April 2004. The AMOC and its component parts are monitored by combining a transatlantic array of moored instruments with submarine-cable-based measurements of the Gulf Stream and satellite derived Ekman transport. The time series has recently been extended to October 2012 and the results show a downward trend since 2004. From April 2008 to March 2012, the AMOC was an average of 2.7 Sv (1 Sv = 10{$^6$} m{$^3$} s\textsuperscript{\&minus;1}) weaker than in the first four years of observation (95\% confidence that the reduction is 0.3 Sv or more). Ekman transport reduced by about 0.2 Sv and the Gulf Stream by 0.5 Sv but most of the change (2.0 Sv) is due to the mid-ocean geostrophic flow. The change of the mid-ocean geostrophic flow represents a strengthening of the southward flow above the thermocline. The increased southward flow of warm waters is balanced by a decrease in the southward flow of lower North Atlantic deep water below 3000 m. The transport of lower North Atlantic deep water slowed by 7\% per year (95\% confidence that the rate of slowing is greater than 2.5\% per year).{$<$}/p{$>$}},
  file = {/Users/oscardimdore-miles/Zotero/storage/LP8VTEQB/Smeed et al. - 2014 - Observed decline of the Atlantic meridional overtu.pdf;/Users/oscardimdore-miles/Zotero/storage/H6BQPFMB/2014.html},
  journal = {Ocean Science},
  language = {English},
  number = {1}
}

@article{Smith2008,
  title = {Decadal-Scale Periodicities in the Stratosphere Associated with the Solar Cycle and the {{QBO}}},
  author = {Smith, Anne K. and Matthes, Katja},
  year = {2008},
  volume = {113},
  doi = {10.1029/2007JD009051},
  journal = {Journal of Geophysical Research: Atmospheres},
  keywords = {ozone,solar cycle,stratosphere},
  number = {D5}
}

@article{Smith2012,
  title = {Linear Interference and the Initiation of Extratropical Stratosphere-Troposphere Interactions},
  author = {Smith, Karen L. and Kushner, Paul J.},
  year = {2012},
  volume = {117},
  doi = {10.1029/2012JD017587},
  journal = {Journal of Geophysical Research: Atmospheres},
  keywords = {annular modes,stationary waves,stratosphere-tropshere interactions,stratospheric sudden warmings,stratospheric variability},
  number = {D13}
}

@article{smithLinear2012,
  title = {Linear Interference and the Initiation of Extratropical Stratosphere-Troposphere Interactions},
  author = {Smith, Karen L. and Kushner, Paul J.},
  year = {2012},
  volume = {117},
  issn = {2156-2202},
  doi = {10.1029/2012JD017587},
  abstract = {Vertical fluxes of wave activity from the troposphere to the stratosphere correlate negatively with the Northern Annular Mode (NAM) index in the stratosphere and subsequently in the troposphere. Recent studies have shown that stratospheric NAM variability is also negatively correlated with the amplitude of the wave pattern coherent with the large-scale climatological stationary wavefield; when the climatological stationary wavefield is amplified or attenuated, the stratospheric jet correspondingly weakens or strengthens. Here we quantify the importance of this linear interference effect in initiating stratosphere-troposphere interactions by performing a decomposition of the vertical wave activity flux using reanalysis data. The interannual variability in vertical wave activity flux in both the Northern and Southern Hemisphere extratropics is dominated by linear interference of quasi-stationary waves during the season of strongest stratosphere-troposphere coupling. Composite analysis of anomalous vertical wave activity flux events reveals the significant role of linear interference and shows that ``linear'' and ``nonlinear'' events are essentially independent. Linear interference is the dominant contribution to the vertical wave activity flux anomalies preceding displacement stratospheric sudden warmings (SSWs) while split SSWs are preceded by nonlinear wave activity flux anomalies. Wave activity variability controls the timing of stratospheric final warmings, and this variability is shown to be dominated by linear interference, particularly in the Southern Hemisphere. The persistence of the linear interference component of the vertical wave activity flux, corresponding to persistent constructive or destructive interference between the wave-1 component of climatological stationary wave and the wave anomaly, may help improve wintertime extratropical predictability.},
  copyright = {\textcopyright 2012. American Geophysical Union. All Rights Reserved.},
  file = {/Users/oscardimdore-miles/Zotero/storage/ZHWH8NFC/Smith and Kushner - 2012 - Linear interference and the initiation of extratro.pdf;/Users/oscardimdore-miles/Zotero/storage/GTUG928R/2012JD017587.html},
  journal = {Journal of Geophysical Research: Atmospheres},
  keywords = {annular modes,stationary waves,stratosphere-tropshere interactions,stratospheric sudden warmings,stratospheric variability},
  language = {English},
  number = {D13}
}

@article{songRevisiting2018,
  title = {Revisiting the {{ENSO}}\textendash{{SSW Relationship}}},
  author = {Song, Kanghyun and Son, Seok-Woo},
  year = {2018},
  month = mar,
  volume = {31},
  pages = {2133--2143},
  publisher = {{American Meteorological Society}},
  issn = {0894-8755, 1520-0442},
  doi = {10.1175/JCLI-D-17-0078.1},
  abstract = {{$<$}section class="abstract"{$><$}h2 class="abstractTitle text-title my-1" id="d1590e2"{$>$}Abstract{$<$}/h2{$><$}p{$>$}Stratospheric sudden warming (SSW) events exhibit pronounced interannual variability. Based on zonal wind reversals at 60\textdegree N and 10 hPa, it has been suggested that SSW events occur more preferentially during El Ni\~no\textendash Southern Oscillation (ENSO) winters (both El Ni\~no and La Ni\~na winters) than during ENSO-neutral winters. This relationship is reevaluated here by considering seven different SSW definitions. For all definitions, SSW events are detected more frequently during El Ni\~no winters than during ENSO-neutral winters, in agreement with a strengthened planetary-scale wave activity. However, such a systematic relationship is not found during La Ni\~na winters. While three SSW definitions, including the wind-reversal definition, show a higher SSW frequency during La Ni\~na winters than during ENSO-neutral winters, other definitions show no difference or even lower SSW frequency during La Ni\~na winters. This result, which is qualitatively insensitive to the choice of reanalysis datasets, ENSO indices, and SST datasets, indicates that the reported ENSO\textendash SSW relationship is dependent on the details of the SSW definition. This result is interpreted in terms of different background wind, latitudinal extent of wind reversal, and planetary-scale wave activity during El Ni\~no and La Ni\~na winter SSW events.{$<$}/p{$><$}/section{$>$}},
  chapter = {Journal of Climate},
  file = {/Users/oscardimdore-miles/Zotero/storage/F8YAU2UC/Song and Son - 2018 - Revisiting the ENSO–SSW Relationship.pdf},
  journal = {Journal of Climate},
  language = {English},
  number = {6}
}

@article{Storkey2018,
  title = {{{UK}} Global Ocean {{GO6}} and {{GO7}}: A Traceable Hierarchy of Model Resolutions},
  author = {Storkey, David and Blaker, Adam and Mathiot, Pierre and Megann, Alex and Aksenov, Yevgeny and W. Blockley, Edward and Calvert, Daley and Graham, Tim and Hewitt, Helene and Hyder, Patrick and Kuhlbrodt, Till and G. L. Rae, Jamie and Sinha, Bablu},
  year = {2018},
  pages = {1--43},
  doi = {10.5194/gmd-2017-263},
  journal = {Geoscientific Model Development Discussions}
}

@article{storkeyUK2018,
  title = {{{UK Global Ocean GO6}} and {{GO7}}: {{A}} Traceable Hierarchy of Model Resolutions},
  shorttitle = {{{UK Global Ocean GO6}} and {{GO7}}},
  author = {Storkey, David and Blaker, Adam T. and Mathiot, Pierre and Megann, Alex and Aksenov, Yevgeny and Blockley, Edward W. and Calvert, Daley and Graham, Tim and Hewitt, Helene T. and Hyder, Patrick and Kuhlbrodt, Till and Rae, Jamie G. L. and Sinha, Bablu},
  year = {2018},
  month = aug,
  volume = {11},
  pages = {3187--3213},
  publisher = {{Copernicus GmbH}},
  issn = {1991-959X},
  doi = {10.5194/gmd-11-3187-2018},
  abstract = {{$<$}p{$><$}strong class="journal-contentHeaderColor"{$>$}Abstract.{$<$}/strong{$>$} Versions 6 and 7 of the UK Global Ocean configuration (known as GO6 and GO7) will form the ocean components of the Met Office GC3.1 coupled model and UKESM1 earth system model to be used in CMIP6{$<$}a href="\#fn\_Ch1.Footn1"{$>^{1}<$}/a{$>$} simulations. The label ``GO6'' refers to a traceable hierarchy of three model configurations at nominal 1, 1/4 and {$<$}math xmlns="http://www.w3.org/1998/Math/MathML" id="M3" display="inline" overflow="scroll" dspmath="mathml"{$><$}mrow{$><$}mn mathvariant="normal"{$>$}1{$<$}/mn{$><$}mo{$>$}/{$<$}/mo{$><$}msup{$><$}mn mathvariant="normal"{$>$}12{$<$}/mn{$><$}mo{$>\circ<$}/mo{$><$}/msup{$><$}/mrow{$><$}/math{$><$}svg:svg xmlns:svg="http://www.w3.org/2000/svg" width="31pt" height="14pt" class="svg-formula" dspmath="mathimg" md5hash="52d83408a281a6be147fb692aa06146b"{$><$}svg:image xmlns:xlink="http://www.w3.org/1999/xlink" xlink:href="gmd-11-3187-2018-ie00001.svg" width="31pt" height="14pt" src="gmd-11-3187-2018-ie00001.png"/{$><$}/svg:svg{$>$} resolutions. The GO6 configurations are described in detail with particular focus on aspects which have been updated since the previous version (GO5). Results of 30-year forced ocean-ice integrations with the {$<$}math xmlns="http://www.w3.org/1998/Math/MathML" id="M4" display="inline" overflow="scroll" dspmath="mathml"{$><$}mrow{$><$}mn mathvariant="normal"{$>$}1{$<$}/mn{$><$}mo{$>$}/{$<$}/mo{$><$}msup{$><$}mn mathvariant="normal"{$>$}4{$<$}/mn{$><$}mo{$>\circ<$}/mo{$><$}/msup{$><$}/mrow{$><$}/math{$><$}svg:svg xmlns:svg="http://www.w3.org/2000/svg" width="25pt" height="14pt" class="svg-formula" dspmath="mathimg" md5hash="1336b0c128a3cdfeedc0b3d16a0bf247"{$><$}svg:image xmlns:xlink="http://www.w3.org/1999/xlink" xlink:href="gmd-11-3187-2018-ie00002.svg" width="25pt" height="14pt" src="gmd-11-3187-2018-ie00002.png"/{$><$}/svg:svg{$>$} model are presented, in which GO6 is coupled to the GSI8.1 sea ice configuration and forced with CORE2{$<$}a href="\#fn\_Ch1.Footn2"{$>^{2}<$}/a{$>$} fluxes. GO6-GSI8.1 shows an overall improved simulation compared to GO5-GSI5.0, especially in the Southern Ocean where there are more realistic summertime mixed layer depths, a reduced near-surface warm and saline biases, and an improved simulation of sea ice. The main drivers of the improvements in the Southern Ocean simulation are tuning of the vertical and isopycnal mixing parameters. Selected results from the full hierarchy of three resolutions are shown. Although the same forcing is applied, the three models show large-scale differences in the near-surface circulation and in the short-term adjustment of the overturning circulation. The GO7 configuration is identical to the GO6 {$<$}math xmlns="http://www.w3.org/1998/Math/MathML" id="M5" display="inline" overflow="scroll" dspmath="mathml"{$><$}mrow{$><$}mn mathvariant="normal"{$>$}1{$<$}/mn{$><$}mo{$>$}/{$<$}/mo{$><$}msup{$><$}mn mathvariant="normal"{$>$}4{$<$}/mn{$><$}mo{$>\circ<$}/mo{$><$}/msup{$><$}/mrow{$><$}/math{$><$}svg:svg xmlns:svg="http://www.w3.org/2000/svg" width="25pt" height="14pt" class="svg-formula" dspmath="mathimg" md5hash="b3e7aebd1bc2e8d9f3c98042acad90ea"{$><$}svg:image xmlns:xlink="http://www.w3.org/1999/xlink" xlink:href="gmd-11-3187-2018-ie00003.svg" width="25pt" height="14pt" src="gmd-11-3187-2018-ie00003.png"/{$><$}/svg:svg{$>$} configuration except that the cavities under the ice shelves are opened. Opening the ice shelf cavities has a local impact on temperature and salinity biases on the Antarctic shelf with some improvement in the biases in the Weddell Sea.{$<$}/p{$>$}},
  file = {/Users/oscardimdore-miles/Zotero/storage/2HQNL3KI/Storkey et al. - 2018 - UK Global Ocean GO6 and GO7 a traceable hierarchy.pdf;/Users/oscardimdore-miles/Zotero/storage/YC3ASCYA/2018.html},
  journal = {Geoscientific Model Development},
  language = {English},
  number = {8}
}

@article{SUGIMOTO2009,
  title = {Decadal and Interdecadal Variations of the Aleutian Low Activity and Their Relation to Upper Oceanic Variations over the North Pacific},
  author = {Sugimoto, Shusaku and Hanawa, Kimio},
  year = {2009},
  volume = {87},
  pages = {601--614},
  doi = {10.2151/jmsj.87.601},
  journal = {Journal of the Meteorological Society of Japan. Ser. II},
  number = {4}
}

@article{sungAltered2014,
  title = {Altered Atmospheric Responses to Eastern {{Pacific}} and Central {{Pacific El Ni\~nos}} over the {{North Atlantic}} Region Due to Stratospheric Interference},
  author = {Sung, Mi-Kyung and Kim, Baek-Min and An, Soon-Il},
  year = {2014},
  month = jan,
  volume = {42},
  pages = {159--170},
  issn = {1432-0894},
  doi = {10.1007/s00382-012-1661-0},
  abstract = {The two types of El Ni\~no that have been identified, namely the eastern Pacific (EP) and central Pacific (CP) El Ni\~nos, are known to exert different climatic impacts on the North Atlantic region during winter. Here, we investigate the characteristics of the teleconnection of the two El Ni\~no types with a focus on the stratosphere-troposphere coupling. During the EP El Ni\~no, polar stratospheric warming and polar vortex weakening frequently occur with a strong tendency for downward propagation near the tropopause. Consequently, the atmospheric pattern within the troposphere over the North Atlantic sector during midwinter closely resembles the negative North Atlantic Oscillation pattern. In contrast, during CP El Ni\~no events stratospheric warming events exhibit a much weaker downward propagation tendency. This difference in the stratospheric circulation response arises from the different seasonal evolution of the tropospheric wave response to the two El Ni\~no types. For the EP El Ni\~no, the Aleutian Low begins growing during December and is sustained throughout the entire winter (December to February), which provides favorable conditions for the continuous downward propagation of the stratospheric warming. We also discuss the origin of the difference in the teleconnections from the two types of El Ni\~no associated with the distinct longitudinal position of the warm SST anomaly that determines troposphere-stratosphere coupling.},
  file = {/Users/oscardimdore-miles/Zotero/storage/2CMXV2PT/Sung et al. - 2014 - Altered atmospheric responses to eastern Pacific a.pdf},
  journal = {Climate Dynamics},
  language = {English},
  number = {1}
}

@article{suttonAtlantic2005,
  title = {Atlantic {{Ocean Forcing}} of {{North American}} and {{European Summer Climate}}},
  author = {Sutton, Rowan T. and Hodson, Daniel L. R.},
  year = {2005},
  month = jul,
  volume = {309},
  pages = {115--118},
  publisher = {{American Association for the Advancement of Science}},
  issn = {0036-8075, 1095-9203},
  doi = {10.1126/science.1109496},
  abstract = {Recent extreme events such as the devastating 2003 European summer heat wave raise important questions about the possible causes of any underlying trends, or low-frequency variations, in regional climates. Here, we present new evidence that basin-scale changes in the Atlantic Ocean, probably related to the thermohaline circulation, have been an important driver of multidecadal variations in the summertime climate of both North America and western Europe. Our findings advance understanding of past climate changes and also have implications for decadal climate predictions. Climate model results indicate that decadal variations in the circulation of the Atlantic Ocean have a dominant influence on summer climates of North America and western Europe. Climate model results indicate that decadal variations in the circulation of the Atlantic Ocean have a dominant influence on summer climates of North America and western Europe.},
  chapter = {Report},
  copyright = {American Association for the Advancement of Science},
  file = {/Users/oscardimdore-miles/Zotero/storage/Y9WJGBLE/Sutton and Hodson - 2005 - Atlantic Ocean Forcing of North American and Europ.pdf;/Users/oscardimdore-miles/Zotero/storage/UZ3AICET/115.html},
  journal = {Science},
  language = {English},
  number = {5731},
  pmid = {15994552}
}

@article{Taguchi2006,
  title = {Increased Occurrence of Stratospheric Sudden Warmings during El Ni\~no as Simulated by {{WACCM}}},
  author = {Taguchi, Masakazu and Hartmann, Dennis L.},
  year = {2006},
  volume = {19},
  pages = {324},
  journal = {Journal of Climate},
  number = {3}
}

@article{taguchiIncreased2006,
  title = {Increased {{Occurrence}} of {{Stratospheric Sudden Warmings}} during {{El Ni\~no}} as {{Simulated}} by {{WACCM}}},
  author = {Taguchi, Masakazu and Hartmann, Dennis L.},
  year = {2006},
  month = feb,
  volume = {19},
  pages = {324--332},
  publisher = {{American Meteorological Society}},
  issn = {0894-8755, 1520-0442},
  doi = {10.1175/JCLI3655.1},
  abstract = {{$<$}section class="abstract"{$><$}h2 class="abstractTitle text-title my-1" id="d1706e2"{$>$}Abstract{$<$}/h2{$><$}p{$>$}Experiments with Whole Atmosphere Community Climate Model (WACCM) under perpetual January conditions indicate that stratospheric sudden warmings (SSWs) are twice as likely to occur in El Ni\~no winters than in La Ni\~na winters, in basic agreement with the limited observational dataset. Tropical SST anomalies that mimic El Ni\~no and La Ni\~na lead to changes in the shape of probability distribution functions (PDFs) of stratospheric day-to-day variability, resulting in a warmer pole and weaker vortex on average for El Ni\~no conditions. The tropical SST forcing induces a response similar to the observed response in the enhancement of the planetary wave of zonal wavenumber 1 (wave 1) and the weakening of wave 2 in the upper troposphere and stratosphere of high latitudes. The enhanced wave 1 contributes to a shift of the PDFs of poleward eddy heat flux in the lower stratosphere, or wave forcing entering the stratosphere. The shift of the PDFs includes an increase of strong wave events that induce more frequent SSWs.{$<$}/p{$><$}/section{$>$}},
  chapter = {Journal of Climate},
  file = {/Users/oscardimdore-miles/Zotero/storage/V9WDXJWB/Taguchi and Hartmann - 2006 - Increased Occurrence of Stratospheric Sudden Warmi.pdf;/Users/oscardimdore-miles/Zotero/storage/MDW8FH2Q/jcli3655.1.html},
  journal = {Journal of Climate},
  language = {English},
  number = {3}
}

@article{taguchiThere2008,
  title = {Is {{There}} a {{Statistical Connection}} between {{Stratospheric Sudden Warming}} and {{Tropospheric Blocking Events}}?},
  author = {Taguchi, Masakazu},
  year = {2008},
  month = apr,
  volume = {65},
  pages = {1442--1454},
  publisher = {{American Meteorological Society}},
  issn = {0022-4928, 1520-0469},
  doi = {10.1175/2007JAS2363.1},
  abstract = {{$<$}section class="abstract"{$><$}h2 class="abstractTitle text-title my-1" id="d206e2"{$>$}Abstract{$<$}/h2{$><$}p{$>$}This paper presents statistical analyses of possible associations between major stratospheric sudden warming (SSW) and tropospheric blocking events in the Northern Hemisphere (NH) with 49 yr of NCEP\textendash NCAR reanalysis data from 1957/58 to 2005/06. Using a random shuffling or ``bootstrap'' method, these analyses explore two hypotheses claiming that blocking events occur preferentially and last longer in association with SSWs (pre- and post-SSW periods are considered separately). In the shuffling method, the defined SSWs are randomly redistributed to evaluate the statistical significance of linked cases in the original data. The author's analyses generally do not support either hypothesis for the pre- or post-SSW period when treating the SSW events all together, suggesting that such associations are not dominant modes of coupling.{$<$}/p{$><$}/section{$>$}},
  chapter = {Journal of the Atmospheric Sciences},
  file = {/Users/oscardimdore-miles/Zotero/storage/WPB6N7E4/Taguchi - 2008 - Is There a Statistical Connection between Stratosp.pdf},
  journal = {Journal of the Atmospheric Sciences},
  language = {English},
  number = {4}
}

@article{Thompson2002,
  title = {Stratospheric Connection to Northern Hemisphere Wintertime Weather: {{Implications}} for Prediction},
  author = {Thompson, D. W. J.},
  year = {2003},
  month = jul,
  volume = {16},
  pages = {2433--2433},
  journal = {Journal of Climate}
}

@article{thompsonStratospheric2002,
  title = {Stratospheric {{Connection}} to {{Northern Hemisphere Wintertime Weather}}: {{Implications}} for {{Prediction}}},
  shorttitle = {Stratospheric {{Connection}} to {{Northern Hemisphere Wintertime Weather}}},
  author = {Thompson, David W. J. and Baldwin, Mark P. and Wallace, John M.},
  year = {2002},
  month = jun,
  volume = {15},
  pages = {1421--1428},
  publisher = {{American Meteorological Society}},
  issn = {0894-8755, 1520-0442},
  doi = {10.1175/1520-0442(2002)015<1421:SCTNHW>2.0.CO;2},
  abstract = {{$<$}section class="abstract"{$><$}h2 class="abstractTitle text-title my-1" id="d896e2"{$>$}Abstract{$<$}/h2{$><$}p{$>$}The dynamical coupling between the stratospheric and tropospheric circulations yields a statistically significant level of potential predictability for extreme cold events throughout much of the Northern Hemisphere (NH) mid\textendash high latitudes on both month-to-month and winter-to-winter timescales. Pronounced weakenings of the NH wintertime stratospheric polar vortex tend to be followed by episodes of anomalously low surface air temperatures and increased frequency of occurrence of extreme cold events throughout densely populated regions such as eastern North America, northern Europe, and eastern Asia that persist for {$\sim$}2 months. Strengthenings of the vortex tend to be followed by surface temperature anomalies in the opposite sense. During midwinter, the quasi-biennial oscillation (QBO) in the equatorial stratosphere has a similar but somewhat weaker impact on NH weather, presumably through its impact on the strength and stability of the stratospheric polar vortex; that is, the easterly phase of the QBO favors an increased incidence of extreme cold events, and vice versa. The signature of the QBO in NH wintertime temperatures is roughly comparable in amplitude to that observed in relation to the El Ni\~no\textendash Southern Oscillation phenomenon.{$<$}/p{$><$}/section{$>$}},
  chapter = {Journal of Climate},
  file = {/Users/oscardimdore-miles/Zotero/storage/T32C9G3Z/Thompson et al. - 2002 - Stratospheric Connection to Northern Hemisphere Wi.pdf;/Users/oscardimdore-miles/Zotero/storage/RMB2VH9E/1520-0442_2002_015_1421_sctnhw_2.0.co_2.html},
  journal = {Journal of Climate},
  language = {English},
  number = {12}
}

@article{timmermannENSO2005,
  title = {{{ENSO Suppression}} Due to {{Weakening}} of the {{North Atlantic Thermohaline Circulation}}},
  author = {Timmermann, A. and An, S.-I. and Krebs, U. and Goosse, H.},
  year = {2005},
  month = aug,
  volume = {18},
  pages = {3122--3139},
  publisher = {{American Meteorological Society}},
  issn = {0894-8755, 1520-0442},
  doi = {10.1175/JCLI3495.1},
  abstract = {{$<$}section class="abstract"{$><$}h2 class="abstractTitle text-title my-1" id="d881e2"{$>$}Abstract{$<$}/h2{$><$}p{$>$}Changes of the North Atlantic thermohaline circulation (THC) excite wave patterns that readjust the thermocline globally. This paper examines the impact of a freshwater-induced THC shutdown on the depth of the Pacific thermocline and its subsequent modification of the El Ni\~no\textendash Southern Oscillation (ENSO) variability using an intermediate-complexity global coupled atmosphere\textendash ocean\textendash sea ice model and an intermediate ENSO model, respectively. It is shown by performing a numerical eigenanalysis and transient simulations that a THC shutdown in the North Atlantic goes along with reduced ENSO variability because of a deepening of the zonal mean tropical Pacific thermocline. A transient simulation also exhibits abrupt changes of ENSO behavior, depending on the rate of THC change. The global oceanic wave adjustment mechanism is shown to play a key role also on multidecadal time scales. Simulated multidecadal global sea surface temperature (SST) patterns show a large degree of similarity with previous climate reconstructions, suggesting that the observed pan-oceanic variability on these time scales is brought about by oceanic waves and by atmospheric teleconnections.{$<$}/p{$><$}/section{$>$}},
  chapter = {Journal of Climate},
  file = {/Users/oscardimdore-miles/Zotero/storage/86AGBXQW/Timmermann et al. - 2005 - ENSO Suppression due to Weakening of the North Atl.pdf;/Users/oscardimdore-miles/Zotero/storage/MG6CFTDD/jcli3495.1.html},
  journal = {Journal of Climate},
  language = {English},
  number = {16}
}

@article{Tomassini2012,
  title = {The Role of Stratosphere-Troposphere Coupling in the Occurrence of Extreme Winter Cold Spells over Northern {{Europe}}},
  author = {Tomassini, Lorenzo and Gerber, Edwin P. and Baldwin, Mark P. and Bunzel, Felix and Giorgetta, Marco},
  year = {2012},
  volume = {4},
  doi = {10.1029/2012MS000177},
  journal = {Journal of Advances in Modeling Earth Systems},
  number = {4}
}

@article{tomassiniRole2012,
  title = {The Role of Stratosphere-Troposphere Coupling in the Occurrence of Extreme Winter Cold Spells over Northern {{Europe}}},
  author = {Tomassini, Lorenzo and Gerber, Edwin P. and Baldwin, Mark P. and Bunzel, Felix and Giorgetta, Marco},
  year = {2012},
  volume = {4},
  issn = {1942-2466},
  doi = {10.1029/2012MS000177},
  abstract = {Extreme cold spells over Northern Europe during winter are examined in order to address the question to what degree and in which ways stratospheric dynamics may influence the state of the troposphere. The study is based on 500 years of a pre-industrial control simulation with a comprehensive global climate model which well resolves the stratosphere, the MPI Earth System Model. Geopotential height anomalies leading to cold air outbreaks leave imprints throughout the atmosphere including the middle and lower stratosphere. A significant connection between tropospheric winter cold spells over Northern Europe and erosion of the stratospheric polar vortex is detected up to 30 hPa. In about 40 percent of the cases, the extreme cold spells are preceded by dynamical disturbances in the stratosphere. The strong warmings associated with the deceleration of the stratospheric jet cause the tropopause height to decrease over high latitudes. The compression of the tropospheric column below favors the development of high pressure anomalies and blocking signatures over polar regions. This in turn leads to the advection of cold air towards Northern Europe and the establishment of a negative annular mode pattern in the troposphere. Anomalies in the residual mean meridional circulation during the stratospheric weak vortex events contribute to the warming of the lower stratosphere, but are not key in the mechanism through which the stratosphere impacts the troposphere.},
  copyright = {\textcopyright 2012. American Geophysical Union. All Rights Reserved.},
  file = {/Users/oscardimdore-miles/Zotero/storage/9BS3VUJ7/Tomassini et al. - 2012 - The role of stratosphere-troposphere coupling in t.pdf},
  journal = {Journal of Advances in Modeling Earth Systems},
  keywords = {coupling,stratosphere-troposphere},
  language = {English},
  number = {4}
}

@article{Torrence1998,
  title = {A Practical Guide to Wavelet Analysis},
  author = {Torrence, Christopher and Compo, Gilbert P.},
  year = {1998},
  doi = {10.1175/1520-0477},
  journal = {Bulletin of the American Meteorological Society}
}

@article{torrencePractical1998,
  title = {A {{Practical Guide}} to {{Wavelet Analysis}}.},
  author = {Torrence, Christopher and Compo, Gilbert P.},
  year = {1998},
  month = jan,
  volume = {79},
  pages = {61--78},
  doi = {10.1175/1520-0477(1998)079<0061:APGTWA>2.0.CO;2},
  abstract = {A practical step-by-step guide to wavelet analysis is given, with examples taken from time series of the El Ni\~no-Southern Oscillation (ENSO). The guide includes a comparison to the windowed Fourier transform, the choice of an appropriate wavelet basis function, edge effects due to finite-length time series, and the relationship between wavelet scale and Fourier frequency. New statistical significance tests for wavelet power spectra are developed by deriving theoretical wavelet spectra for white and red noise processes and using these to establish significance levels and confidence intervals. It is shown that smoothing in time or scale can be used to increase the confidence of the wavelet spectrum. Empirical formulas are given for the effect of smoothing on significance levels and confidence intervals. Extensions to wavelet analysis such as filtering, the power Hovm\"oller, cross-wavelet spectra, and coherence are described.The statistical significance tests are used to give a quantitative measure of changes in ENSO variance on interdecadal timescales. Using new datasets that extend back to 1871, the Ni\~no3 sea surface temperature and the Southern Oscillation index show significantly higher power during 1880-1920 and 1960-90, and lower power during 1920-60, as well as a possible 15-yr modulation of variance. The power Hovm\"oller of sea level pressure shows significant variations in 2-8-yr wavelet power in both longitude and time.},
  journal = {Bulletin of the American Meteorological Society}
}

@article{Trenberth1994,
  title = {Decadal Atmosphere-Ocean Variations in the Pacific},
  author = {Trenberth, Kevin and Hurrell, J.W},
  year = {1994},
  volume = {9},
  pages = {303--319},
  doi = {10.1007/BF00204745},
  journal = {Climate Dynamics}
}

@article{Trenberth2001,
  title = {Indices of El Ni\~no Evolution},
  author = {Trenberth, Kevin E. and Stepaniak, David P.},
  year = {2001},
  volume = {14},
  pages = {1697--1701},
  doi = {10.1175/1520-0442},
  journal = {Journal of Climate},
  number = {8}
}

@article{trenberthIndices2001,
  title = {Indices of {{El Ni\~no Evolution}}},
  author = {Trenberth, Kevin E. and Stepaniak, David P.},
  year = {2001},
  month = apr,
  volume = {14},
  pages = {1697--1701},
  publisher = {{American Meteorological Society}},
  issn = {0894-8755, 1520-0442},
  doi = {10.1175/1520-0442(2001)014<1697:LIOENO>2.0.CO;2},
  abstract = {{$<$}section class="abstract"{$><$}h2 class="abstractTitle text-title my-1" id="d447e2"{$>$}Abstract{$<$}/h2{$><$}p{$>$}To characterize the nature of El Ni\~no\textendash Southern Oscillation (ENSO), sea surface temperature (SST) anomalies in different regions of the Pacific have been used. An optimal characterization of both the distinct character and the evolution of each El Ni\~no or La Ni\~na event is suggested that requires at least two indices: (i) SST anomalies in the Ni\~no-3.4 region (referred to as N3.4), and (ii) a new index termed here the Trans-Ni\~no Index (TNI), which is given by the difference in normalized anomalies of SST between Ni\~no-1+2 and Ni\~no-4 regions. The first index can be thought of as the mean SST throughout the equatorial Pacific east of the date line and the second index is the gradient in SST across the same region. Consequently, they are approximately orthogonal. TNI leads N3.4 by 3 to 12 months prior to the climate shift in 1976/77 and also follows N3.4 but with opposite sign 3 to 12 months later. However, after 1976/77, the sign of the TNI leads and lags are reversed.{$<$}/p{$><$}/section{$>$}},
  chapter = {Journal of Climate},
  file = {/Users/oscardimdore-miles/Zotero/storage/ZJCBFKTN/Trenberth and Stepaniak - 2001 - Indices of El Niño Evolution.pdf;/Users/oscardimdore-miles/Zotero/storage/W23TLYUE/1520-0442_2001_014_1697_lioeno_2.0.co_2.html},
  journal = {Journal of Climate},
  language = {English},
  number = {8}
}

@article{tullochExploring2012,
  title = {Exploring {{Mechanisms}} of {{Variability}} and {{Predictability}} of {{Atlantic Meridional Overturning Circulation}} in {{Two Coupled Climate Models}}},
  author = {Tulloch, Ross and Marshall, John},
  year = {2012},
  month = jun,
  volume = {25},
  pages = {4067--4080},
  publisher = {{American Meteorological Society}},
  issn = {0894-8755, 1520-0442},
  doi = {10.1175/JCLI-D-11-00460.1},
  abstract = {{$<$}section class="abstract"{$><$}h2 class="abstractTitle text-title my-1" id="d979e2"{$>$}Abstract{$<$}/h2{$><$}p{$>$}Multidecadal variability in the Atlantic meridional overturning circulation (AMOC) of the ocean is diagnosed in the NCAR Community Climate System Model, version 3 (CCSM3), and the GFDL Coupled Model (CM2.1). Common diagnostic approaches are applied to draw out similarities and differences between the two models. An index of AMOC variability is defined, and the manner in which key variables covary with it is determined. In both models the following is found. (i) AMOC variability is associated with upper-ocean (top 1 km) density anomalies (dominated by temperature) on the western margin of the basin in the region of the Mann eddy with a period of about 20 years. These anomalies modulate the trajectory and strength of the North Atlantic Current. The importance of the western margin is a direct consequence of the thermal wind relation and is independent of the mechanisms that create those density anomalies. (ii) Density anomalies in this key region are part of a larger-scale pattern that propagates around the subpolar gyre and acts as a ``pacemaker'' of AMOC variability. (iii) The observed variability is consistent with the primary driving mechanism being stochastic wind curl forcing, with Labrador Sea convection playing a secondary role. Also, ``toy models'' of delayed oscillator form are fitted to power spectra of key variables and are used to infer ``quality factors'' (Q-factors), which characterize the bandwidth relative to the center frequency and hence AMOC predictability horizons. The two models studied here have Q-factors of around 2, suggesting that prediction is possible out to about two cycles, which is likely larger than the real AMOC.{$<$}/p{$><$}/section{$>$}},
  chapter = {Journal of Climate},
  file = {/Users/oscardimdore-miles/Zotero/storage/2LSVJE87/Tulloch and Marshall - 2012 - Exploring Mechanisms of Variability and Predictabi.pdf;/Users/oscardimdore-miles/Zotero/storage/B4YQ4LLT/jcli-d-11-00460.1.html},
  journal = {Journal of Climate},
  language = {English},
  number = {12}
}

@phdthesis{verena2016a,
  title = {Tropical Stratosphere Variability and Extratropical Teleconnections},
  author = {Schenzinger, Verena},
  year = {2016},
  school = {University of Oxford / University of Oxford}
}

@article{vialSudden2013,
  title = {Sudden {{Stratospheric Warmings}} and Tropospheric Blockings in a Multi-Century Simulation of the {{IPSL}}-{{CM5A}} Coupled Climate Model},
  author = {Vial, Jessica and Osborn, Timothy and Lott, Francois},
  year = {2013},
  month = may,
  volume = {40},
  doi = {10.1007/s00382-013-1675-2},
  abstract = {The relation between sudden stratospheric warmings (SSWs) and blocking events is analyzed in a multi-centennial pre-industrial simulation of the Institut Pierre Simon Laplace coupled model (IPSL-CM5A), prepared for the fifth phase of the coupled model intercomparison project. The IPSL model captures a fairly realistic distribution of both SSWs and tropospheric blocking events, albeit with a tendency to overestimate the frequency of blocking in the western Pacific and underestimate it in the Euro-Atlantic sector. The 1000-year long simulation reveals statistically significant differences in blocking frequency and duration over the 40-day periods preceding and following the onset of SSWs. More specifically, there is an enhanced blocking frequency over Eurasia before SSWs, followed by an westward displacement of blocking anomalies over the Atlantic region as SSWs evolve and then decline. The frequency of blocking is reduced over the western Pacific sector during the life-cycle of SSWs, while the model simulates no significant relationship with eastern Pacific blocks. Finally, these changes in blocking frequency tend to be associated with a shift in the distribution of blocking lifetime toward longer-lasting blocking events before the onset of SSWs and shorter-lived blocks after the warmings. This study systematically verifies that the results are consistent with the two pictures that (1) blockings produce planetary scale anomalies that can force vertically propagating Rossby waves and then SSWs when the waves break and (2) SSWs affect blockings in return, for instance via the effect they have on the North Atlantic Oscillation.},
  journal = {Climate Dynamics}
}

@article{visbeckOcean1998,
  title = {An Ocean Model's Response to {{North Atlantic Oscillation}}-like Wind Forcing},
  author = {Visbeck, Martin and Cullen, Heidi and Krahmann, Gerd and Naik, Naomi},
  year = {1998},
  volume = {25},
  pages = {4521--4524},
  issn = {1944-8007},
  doi = {10.1029/1998GL900162},
  abstract = {The response of the Atlantic Ocean to North Atlantic Oscillation (NAO)-like wind forcing was investigated using an ocean-only general circulation model coupled to an atmospheric boundary layer model. A series of idealized experiments was performed to investigate the interannual to multi-decadal frequency response of the ocean to a winter wind anomaly pattern. Overall, the strength of the SST response increased slightly with longer forcing periods. In the subpolar gyre, however, the model showed a broad response maximum in the decadal band (12\textendash 16 years).},
  copyright = {Copyright 1998 by the American Geophysical Union.},
  file = {/Users/oscardimdore-miles/Zotero/storage/CTU6DMVH/Visbeck et al. - 1998 - An ocean model's response to North Atlantic Oscill.pdf;/Users/oscardimdore-miles/Zotero/storage/MUX5R4VB/1998GL900162.html},
  journal = {Geophysical Research Letters},
  language = {English},
  number = {24}
}

@article{Wallace1993,
  title = {Representation of the Equatorial Stratospheric Quasi-Biennial Oscillation in {{EOF}} Phase Space.},
  author = {Wallace, John M. and Panetta, R. Lee and Estberg, Jerry},
  year = {1993},
  volume = {50},
  pages = {1751--1762},
  doi = {10.1175/1520-0469},
  journal = {Journal of Atmospheric Sciences},
  number = {12}
}

@article{Walters2011,
  title = {The Met Office Unified Model Global Atmosphere 3.0/3.1 and {{JULES}} Global Land 3.0/3.1 Configurations},
  author = {Walters, D. N. and Best, M. J. and Bushell, A. C. and Copsey, D. and Edwards, J. M. and Falloon, P. D. and Harris, C. M. and Lock, A. P. and Manners, J. C. and Morcrette, C. J. and Roberts, M. J. and Stratton, R. A. and Webster, S. and Wilkinson, J. M. and Willett, M. R. and Boutle, I. A. and Earnshaw, P. D. and Hill, P. G. and MacLachlan, C. and Martin, G. M. and {Moufouma-Okia}, W. and Palmer, M. D. and Petch, J. C. and Rooney, G. G. and Scaife, A. A. and Williams, K. D.},
  year = {2011},
  volume = {4},
  pages = {919--941},
  doi = {10.5194/gmd-4-919-2011},
  journal = {Geoscientific Model Development},
  number = {4}
}

@article{Walters2019,
  title = {The Met Office Unified Model Global Atmosphere 7.0/7.1 and {{JULES}} Global Land 7.0 Configurations},
  author = {Walters, D. and Baran, A. J. and Boutle, I. and Brooks, M. and Earnshaw, P. and Edwards, J. and Furtado, K. and Hill, P. and Lock, A. and Manners, J. and Morcrette, C. and Mulcahy, J. and Sanchez, C. and Smith, C. and Stratton, R. and Tennant, W. and Tomassini, L. and Van Weverberg, K. and Vosper, S. and Willett, M. and Browse, J. and Bushell, A. and Carslaw, K. and Dalvi, M. and Essery, R. and Gedney, N. and Hardiman, S. and Johnson, B. and Johnson, C. and Jones, A. and Jones, C. and Mann, G. and Milton, S. and Rumbold, H. and Sellar, A. and Ujiie, M. and Whitall, M. and Williams, K. and Zerroukat, M.},
  year = {2019},
  volume = {12},
  pages = {1909--1963},
  doi = {10.5194/gmd-12-1909-2019},
  journal = {Geoscientific Model Development},
  number = {5}
}

@article{waltersMet2019,
  title = {The {{Met Office Unified Model Global Atmosphere}} 7.0/7.1 and {{JULES Global Land}} 7.0 Configurations},
  author = {Walters, David and Baran, Anthony J. and Boutle, Ian and Brooks, Malcolm and Earnshaw, Paul and Edwards, John and Furtado, Kalli and Hill, Peter and Lock, Adrian and Manners, James and Morcrette, Cyril and Mulcahy, Jane and Sanchez, Claudio and Smith, Chris and Stratton, Rachel and Tennant, Warren and Tomassini, Lorenzo and Van Weverberg, Kwinten and Vosper, Simon and Willett, Martin and Browse, Jo and Bushell, Andrew and Carslaw, Kenneth and Dalvi, Mohit and Essery, Richard and Gedney, Nicola and Hardiman, Steven and Johnson, Ben and Johnson, Colin and Jones, Andy and Jones, Colin and Mann, Graham and Milton, Sean and Rumbold, Heather and Sellar, Alistair and Ujiie, Masashi and Whitall, Michael and Williams, Keith and Zerroukat, Mohamed},
  year = {2019},
  month = may,
  volume = {12},
  pages = {1909--1963},
  publisher = {{Copernicus GmbH}},
  issn = {1991-959X},
  doi = {10.5194/gmd-12-1909-2019},
  abstract = {{$<$}p{$><$}strong class="journal-contentHeaderColor"{$>$}Abstract.{$<$}/strong{$>$} We describe Global Atmosphere 7.0 and Global Land 7.0 (GA7.0/GL7.0), the latest science configurations of the Met Office Unified Model (UM) and the Joint UK Land Environment Simulator (JULES) land surface model developed for use across weather and climate timescales. GA7.0 and GL7.0 include incremental developments and targeted improvements that, between them, address four critical errors identified in previous configurations: excessive precipitation biases over India, warm and moist biases in the tropical tropopause layer (TTL), a source of energy non-conservation in the advection scheme and excessive surface radiation biases over the Southern Ocean. They also include two new parametrisations, namely the UK Chemistry and Aerosol (UKCA) GLOMAP-mode (Global Model of Aerosol Processes) aerosol scheme and the JULES multi-layer snow scheme, which improve the fidelity of the simulation and were required for inclusion in the Global Atmosphere/Global Land configurations ahead of the 6th Coupled Model Intercomparison Project (CMIP6).{$<$}/p{$>$} {$<$}p{$>$}In addition, we describe the GA7.1 branch configuration, which reduces an overly negative anthropogenic aerosol effective radiative forcing (ERF) in GA7.0 whilst maintaining the quality of simulations of the present-day climate. GA7.1/GL7.0 will form the physical atmosphere/land component in the HadGEM3\textendash GC3.1 and UKESM1 climate model submissions to the CMIP6.{$<$}/p{$>$}},
  file = {/Users/oscardimdore-miles/Zotero/storage/P5PH8YET/Walters et al. - 2019 - The Met Office Unified Model Global Atmosphere 7.0.pdf},
  journal = {Geoscientific Model Development},
  language = {English},
  number = {5}
}

@article{wangCharacteristic2019,
  title = {Characteristic Evolution of the {{Atlantic Meridional Overturning Circulation}} from 1990 to 2015: {{An}} Eddy-Resolving Ocean Model Study},
  shorttitle = {Characteristic Evolution of the {{Atlantic Meridional Overturning Circulation}} from 1990 to 2015},
  author = {Wang, Zeliang and Brickman, David and Greenan, Blair J. W.},
  year = {2019},
  month = jul,
  volume = {149},
  pages = {103056},
  issn = {0967-0637},
  doi = {10.1016/j.dsr.2019.06.002},
  abstract = {A 1/12\textdegree{} North Atlantic model is used to investigate the Atlantic Meridional Overturning Circulation (AMOC) variability from 1990 to 2015. The seasonality of the AMOC in depth and density spaces is dominated by the Ekman transport from low to high latitudes. At interannual timescales, the AMOC in depth and density spaces has different characteristics, mostly in high latitudes, which is attributed to strong doming of isopycnals. An Empirical Orthogonal Function analysis of the AMOC in depth space demonstrates that the AMOC can be decomposed into two portions \textendash{} one associated with Labrador Sea winter convection and one related to Ekman transport. The Ekman transport portion of the depth-space AMOC has a general out-of-phase relationship between high and low latitudes and the AMOC becomes more meridionally coherent after the Ekman transport portion is removed. Our study suggests that the AMOC in depth space appears to have two regimes, a strong one before 2001, and a weak one after 2001, which is seen to be associated with the westward movement of the winter North Atlantic Oscillation centers of action starting from 2001. The two-regime behavior is best reflected in the variability of the western part of the Labrador Current, while the general decline over this period is also seen in the Labrador Sea convection depth, the strength of the eastern part of the Labrador Current, and the downward movement of isopycals in the deep layers of the Labrador Sea The latter suggests that hydrographic changes in Labrador Sea deep layers could be potentially used as a proxy for AMOC variability.},
  file = {/Users/oscardimdore-miles/Zotero/storage/QS6CXK3A/Wang et al. - 2019 - Characteristic evolution of the Atlantic Meridiona.pdf;/Users/oscardimdore-miles/Zotero/storage/FB4EIYKH/S096706371830311X.html},
  journal = {Deep Sea Research Part I: Oceanographic Research Papers},
  keywords = {AMOC,Deep convection,Ekman transport,NAO,Out-of-phase relationship},
  language = {English}
}

@article{wangVariability2015,
  title = {Variability of Sea Surface Height and Circulation in the {{North Atlantic}}: {{Forcing}} Mechanisms and Linkages},
  shorttitle = {Variability of Sea Surface Height and Circulation in the {{North Atlantic}}},
  author = {Wang, Zeliang and Lu, Youyu and Dupont, Frederic and W. Loder, John and Hannah, Charles and G. Wright, Daniel},
  year = {2015},
  month = mar,
  volume = {132},
  pages = {273--286},
  issn = {0079-6611},
  doi = {10.1016/j.pocean.2013.11.004},
  abstract = {Simulations with a coarse-resolution global ocean model during 1958\textendash 2004 are analyzed to understand the inter-annual and decadal variability of the North Atlantic. Analyses of Empirical Orthogonal Functions (EOFs) suggest relationships among basin-scale variations of sea surface height (SSH) and depth-integrated circulation, and the winter North Atlantic Oscillation (NAO) or the East Atlantic Pattern (EAP) indices. The linkages between the atmospheric indices and ocean variables are shown to be related to the different roles played by surface momentum and heat fluxes in driving ocean variability. In the subpolar region, variations of the gyre strength, SSH in the central Labrador Sea and the NAO index are highly correlated. Surface heat flux is important in driving variations of SSH and circulation in the upper ocean and decadal variations of the Atlantic Meridional Overturning Circulation (AMOC). Surface momentum flux drives a significant barotropic component of flow and makes a noticeable contribution to the AMOC. In the subtropical region, momentum flux plays a dominant role in driving variations of the gyre circulation and AMOC; there is a strong correlation between gyre strength and SSH at Bermuda.},
  file = {/Users/oscardimdore-miles/Zotero/storage/JSV2ELA3/S0079661113002231.html},
  journal = {Progress in Oceanography},
  language = {English},
  series = {Oceanography of the {{Arctic}} and {{North Atlantic Basins}}}
}

@article{Watson2014,
  title = {How Does the Quasi-Biennial Oscillation Affect the Stratospheric Polar Vortex?},
  author = {Watson, Peter A.G. and Gray, Lesley J.},
  year = {2014},
  volume = {71},
  pages = {391--409},
  doi = {10.1175/JAS-D-13-096.1},
  journal = {Journal of the Atmospheric Sciences},
  number = {1}
}

@article{whiteGeneric2020,
  title = {The {{Generic Nature}} of the {{Tropospheric Response}} to {{Sudden Stratospheric Warmings}}},
  author = {White, Ian P. and Garfinkel, Chaim I. and Gerber, Edwin P. and Jucker, Martin and Hitchcock, Peter and Rao, Jian},
  year = {2020},
  month = jul,
  volume = {33},
  pages = {5589--5610},
  publisher = {{American Meteorological Society}},
  issn = {0894-8755, 1520-0442},
  doi = {10.1175/JCLI-D-19-0697.1},
  abstract = {{$<$}section class="abstract"{$><$}h2 class="abstractTitle text-title my-1" id="d6811580e132"{$>$}Abstract{$<$}/h2{$><$}p{$>$}The tropospheric response to midwinter sudden stratospheric warmings (SSWs) is examined using an idealized model. SSW events are triggered by imposing high-latitude stratospheric heating perturbations of varying magnitude for only a few days, spun off from a free-running control integration (CTRL). The evolution of the thermally triggered SSWs is then compared with naturally occurring SSWs identified in CTRL. By applying a heating perturbation, with no modification to the momentum budget, it is possible to isolate the tropospheric response directly attributable to a change in the stratospheric polar vortex, independent of any planetary wave momentum torques involved in the initiation of an SSW. Zonal-wind anomalies associated with the thermally triggered SSWs first propagate downward to the high-latitude troposphere after \textasciitilde 2 weeks, before migrating equatorward and stalling at midlatitudes, where they straddle the near-surface jet. After \textasciitilde 3 weeks, the circulation and eddy fluxes associated with thermally triggered SSWs evolve very similarly to SSWs in CTRL, despite the lack of initial planetary wave driving. This suggests that at longer lags, the tropospheric response to SSWs is generic and it is found to be linearly governed by the strength of the lower-stratospheric warming, whereas at shorter lags, the initial formation of the SSW potentially plays a large role in the downward coupling. In agreement with previous studies, synoptic waves are found to play a key role in the persistent tropospheric jet shift at long lags. Synoptic waves appear to respond to the enhanced midlatitude baroclinicity associated with the tropospheric jet shift, and preferentially propagate poleward in an apparent positive feedback with changes in the high-latitude refractive index.{$<$}/p{$><$}/section{$>$}},
  chapter = {Journal of Climate},
  file = {/Users/oscardimdore-miles/Zotero/storage/BE9XI9FE/White et al. - 2020 - The Generic Nature of the Tropospheric Response to.pdf;/Users/oscardimdore-miles/Zotero/storage/Q3E3ITY7/jcli-d-19-0697.1.html},
  journal = {Journal of Climate},
  language = {English},
  number = {13}
}

@article{Williams2018,
  title = {The Met Office Global Coupled Model 3.0 and 3.1 ({{GC3}}.0 and {{GC3}}.1) Configurations},
  author = {Williams, K. D. and Copsey, D. and Blockley, E. W. and {Bodas-Salcedo}, A. and Calvert, D. and Comer, R. and Davis, P. and Graham, T. and Hewitt, H. T. and Hill, R. and Hyder, P. and Ineson, S. and Johns, T. C. and Keen, A. B. and Lee, R. W. and Megann, A. and Milton, S. F. and Rae, J. G. L. and Roberts, M. J. and Scaife, A. A. and Schiemann, R. and Storkey, D. and Thorpe, L. and Watterson, I. G. and Walters, D. N. and West, A. and Wood, R. A. and Woollings, T. and Xavier, P. K.},
  year = {2018},
  volume = {10},
  pages = {357--380},
  doi = {10.1002/2017MS001115},
  journal = {Journal of Advances in Modeling Earth Systems},
  number = {2}
}

@article{williamsMet2018,
  title = {The {{Met Office Global Coupled Model}} 3.0 and 3.1 ({{GC3}}.0 and {{GC3}}.1) {{Configurations}}},
  author = {Williams, K. D. and Copsey, D. and Blockley, E. W. and {Bodas-Salcedo}, A. and Calvert, D. and Comer, R. and Davis, P. and Graham, T. and Hewitt, H. T. and Hill, R. and Hyder, P. and Ineson, S. and Johns, T. C. and Keen, A. B. and Lee, R. W. and Megann, A. and Milton, S. F. and Rae, J. G. L. and Roberts, M. J. and Scaife, A. A. and Schiemann, R. and Storkey, D. and Thorpe, L. and Watterson, I. G. and Walters, D. N. and West, A. and Wood, R. A. and Woollings, T. and Xavier, P. K.},
  year = {2018},
  volume = {10},
  pages = {357--380},
  issn = {1942-2466},
  doi = {10.1002/2017MS001115},
  abstract = {The Global Coupled 3 (GC3) configuration of the Met Office Unified Model is presented. Among other applications, GC3 is the basis of the United Kingdom's submission to the Coupled Model Intercomparison Project 6 (CMIP6). This paper documents the model components that make up the configuration (although the scientific descriptions of these components are in companion papers) and details the coupling between them. The performance of GC3 is assessed in terms of mean biases and variability in long climate simulations using present-day forcing. The suitability of the configuration for predictability on shorter time scales (weather and seasonal forecasting) is also briefly discussed. The performance of GC3 is compared against GC2, the previous Met Office coupled model configuration, and against an older configuration (HadGEM2-AO) which was the submission to CMIP5. In many respects, the performance of GC3 is comparable with GC2, however, there is a notable improvement in the Southern Ocean warm sea surface temperature bias which has been reduced by 75\%, and there are improvements in cloud amount and some aspects of tropical variability. Relative to HadGEM2-AO, many aspects of the present-day climate are improved in GC3 including tropospheric and stratospheric temperature structure, most aspects of tropical and extratropical variability and top-of-atmosphere and surface fluxes. A number of outstanding errors are identified including a residual asymmetric sea surface temperature bias (cool northern hemisphere, warm Southern Ocean), an overly strong global hydrological cycle and insufficient European blocking.},
  copyright = {\textcopyright{} 2017. The Authors and Crown copyright. This article is published with the permission of the Controller of HMSO and the Queen's Printer for Scotland.},
  file = {/Users/oscardimdore-miles/Zotero/storage/AXSFERAH/Williams et al. - 2018 - The Met Office Global Coupled Model 3.0 and 3.1 (G.pdf;/Users/oscardimdore-miles/Zotero/storage/H27ESB47/2017MS001115.html},
  journal = {Journal of Advances in Modeling Earth Systems},
  keywords = {model description,model evaluation},
  language = {English},
  number = {2}
}

@article{Woo2015,
  title = {Connection between Weak Stratospheric Vortex Events and the {{Pacific Decadal Oscillation}}},
  author = {Woo, Sung-Ho and Sung, Mi-Kyung and Son, Seok-Woo and Kug, Jong-Seong},
  year = {2015},
  volume = {45},
  pages = {3481--3492},
  doi = {10.1007/s00382-015-2551-z},
  journal = {Climate Dynamics},
  number = {11-12}
}

@article{wooConnection2015,
  title = {Connection between Weak Stratospheric Vortex Events and the {{Pacific Decadal Oscillation}}},
  author = {Woo, Sung-Ho and Sung, Mi-Kyung and Son, Seok-Woo and Kug, Jong-Seong},
  year = {2015},
  month = dec,
  volume = {45},
  pages = {3481--3492},
  issn = {1432-0894},
  doi = {10.1007/s00382-015-2551-z},
  abstract = {We investigate the possible impacts of the Pacific Decadal Oscillation (PDO) on the occurrence of weak stratospheric polar vortex (WSV) events in the Northern Hemisphere winter. WSV events, which are defined when polar-cap geopotential height anomalies at 50 hPa fall below the 10th percentile in winter, are observed more frequently during positive PDO phases than during negative PDO phases. Additionally, tropospheric wave forcings that drive WSV events are remarkably different between the two phases of the PDO. During the positive PDO phase, the vertical propagation of wavenumber-one waves plays a predominant role with a rather minor contribution of wavenumber-two waves. This contrasts with the negative PDO phase when the WSV events are primarily caused by wavenumber-two waves. This difference is partly related to the PDO-induced tropospheric circulation anomalies over the North Pacific whose zonal wavenumber-one component constructively (destructively) interferes with climatological planetary-scale waves during positive (negative) PDO winters. The predominant wavenumber-two wave forcings during the negative PDO phase are likely related to the enhanced tropospheric eddy activity over Alaska that results from the poleward shift of the Pacific jet in response to the negative PDO.},
  file = {/Users/oscardimdore-miles/Zotero/storage/96W68PXA/Woo et al. - 2015 - Connection between weak stratospheric vortex event.pdf},
  journal = {Climate Dynamics},
  language = {English},
  number = {11}
}

@article{xuIntraseasonal2014,
  title = {Intraseasonal to Interannual Variability of the {{Atlantic}} Meridional Overturning Circulation from Eddy-Resolving Simulations and Observations},
  author = {Xu, Xiaobiao and Chassignet, Eric P. and Johns, William E. and Schmitz, William J. and Metzger, E. Joseph},
  year = {2014},
  volume = {119},
  pages = {5140--5159},
  issn = {2169-9291},
  doi = {10.1002/2014JC009994},
  abstract = {Results from two 1/12\textdegree{} eddy-resolving simulations, together with data-based transport estimates at 26.5\textdegree N and 41\textdegree N, are used to investigate the temporal variability of the Atlantic meridional overturning circulation (AMOC) during 2004\textendash 2012. There is a good agreement between the model and the observation for all components of the AMOC at 26.5\textdegree N, whereas the agreement at 41\textdegree N is primarily due to the Ekman transport. We found that (1) both observations and model results exhibit higher AMOC variability on seasonal and shorter time scales than on interannual and longer time scales; (2) on intraseasonal and interannual time scales, the AMOC variability is often coherent over a wide latitudinal range, but lacks an overall consistent coherent pattern over the entire North Atlantic; and (3) on seasonal time scales, the AMOC variability exhibits two distinct coherent regimes north and south of 20\textdegree N, due to different wind stress variability in the tropics and subtropics. The high AMOC variability south of 20\textdegree N in the tropical Atlantic comes primarily from the Ekman transport of the near-surface water, and is modulated to some extent by the transport of the Antarctic Intermediate water below the thermocline. These results highlight the importance of the surface wind in driving the AMOC variability.},
  copyright = {\textcopyright{} 2014. American Geophysical Union. All Rights Reserved.},
  file = {/Users/oscardimdore-miles/Zotero/storage/C953ZZKD/Xu et al. - 2014 - Intraseasonal to interannual variability of the At.pdf},
  journal = {Journal of Geophysical Research: Oceans},
  keywords = {AMOC,eddy-resolving simulation,variability},
  language = {English},
  number = {8}
}

@article{Yang2016,
  title = {Attribution of Variations in the Quasi-Biennial Oscillation Period from the Duration of Easterly and Westerly Phases},
  author = {Yang, M. and Yu, Y.},
  year = {2016},
  volume = {47},
  pages = {1943--1959},
  doi = {10.1007/s00382-015-2943-0},
  journal = {Climate Dynamics}
}

@article{yangLocal2015,
  title = {Local and Remote Wind Stress Forcing of the Seasonal Variability of the {{Atlantic Meridional Overturning Circulation}} ({{AMOC}}) Transport at 26.5\textdegree{{N}}},
  author = {Yang, Jiayan},
  year = {2015},
  volume = {120},
  pages = {2488--2503},
  issn = {2169-9291},
  doi = {10.1002/2014JC010317},
  abstract = {The transport of the Atlantic Meridional Overturning Circulation (AMOC) varies considerably on the seasonal time scale at 26.5\textdegree N, according to observations made at the RAPID-MOCHA array. Previous studies indicate that the local wind stress at 26.5\textdegree N, especially a large wind stress curl at the African coast, is the leading contributor to this seasonal variability. The purpose of the present study is to examine whether nonlocal wind stress forcing, i.e., remote forcing from latitudes away from 26.5\textdegree N, affects the seasonal AMOC variability at the RAPID-MOCHA array. Our tool is a two-layer and wind-driven model with a realistic topography and an observation-derived wind stress. The seasonal cycle of the modeled AMOC transport agrees well with RAPID-MOCHA observations while the amplitude is in the lower end of the observational range. In contrast to previous studies, the seasonal AMOC variability at 26.5\textdegree N is not primarily forced by the wind stress curl at the eastern boundary, but is a result of a basin-wide adjustment of ocean circulation to seasonal changes in wind stress. Both the amplitude and phase of the seasonal cycle at 26.5\textdegree N are strongly influenced by wind stress forcing from other latitudes, especially from the subpolar North Atlantic. The seasonal variability of the AMOC transport at 26.5\textdegree N is due to the seasonal redistribution of the water mass volume and is driven by both local and remote wind stress. Barotropic processes make significant contributions to the seasonal AMOC variability through topography-gyre interactions.},
  copyright = {\textcopyright{} 2015. American Geophysical Union. All Rights Reserved.},
  file = {/Users/oscardimdore-miles/Zotero/storage/DNE63VCC/Yang - 2015 - Local and remote wind stress forcing of the season.pdf;/Users/oscardimdore-miles/Zotero/storage/YZTCRGF5/2014JC010317.html},
  journal = {Journal of Geophysical Research: Oceans},
  keywords = {Atlantio Overturning Circulation,RAPID-MOCHA array,remote forcing,seasonal variability,transport,wind stress},
  language = {English},
  number = {4}
}

@article{yeagerSensitivity2012,
  title = {Sensitivity of {{Atlantic Meridional Overturning Circulation Variability}} to {{Parameterized Nordic Sea Overflows}} in {{CCSM4}}},
  author = {Yeager, Stephen and Danabasoglu, Gokhan},
  year = {2012},
  month = mar,
  volume = {25},
  pages = {2077--2103},
  publisher = {{American Meteorological Society}},
  issn = {0894-8755, 1520-0442},
  doi = {10.1175/JCLI-D-11-00149.1},
  abstract = {{$<$}section class="abstract"{$><$}h2 class="abstractTitle text-title my-1" id="d334e2"{$>$}Abstract{$<$}/h2{$><$}p{$>$}The inclusion of parameterized Nordic Sea overflows in the ocean component of the Community Climate System Model version 4 (CCSM4) results in a much improved representation of the North Atlantic tracer and velocity distributions compared to a control CCSM4 simulation without this parameterization. As a consequence, the variability of the Atlantic meridional overturning circulation (AMOC) on decadal and longer time scales is generally lower, but the reduction is not uniform in latitude, depth, or frequency\textendash space. While there is dramatically less variance in the overall AMOC maximum (at about 35\textdegree N), the reduction in AMOC variance at higher latitudes is more modest. Also, it is somewhat enhanced in the deep ocean and at low latitudes (south of about 30\textdegree N). The complexity of overturning response to overflows is related to the fact that, in both simulations, the AMOC spectrum varies substantially with latitude and depth, reflecting a variety of driving mechanisms that are impacted in different ways by the overflows. The usefulness of reducing AMOC to a single index is thus called into question. This study identifies two main improvements in the ocean mean state associated with the overflow parameterization that tend to damp AMOC variability: enhanced stratification in the Labrador Sea due to the injection of dense overflow waters and a deepening of the deep western boundary current. Direct driving of deep AMOC variance by overflow transport variations is found to be a second-order effect.{$<$}/p{$><$}/section{$>$}},
  chapter = {Journal of Climate},
  file = {/Users/oscardimdore-miles/Zotero/storage/6JERMTV5/Yeager and Danabasoglu - 2012 - Sensitivity of Atlantic Meridional Overturning Cir.pdf},
  journal = {Journal of Climate},
  language = {English},
  number = {6}
}

@article{Yool20,
  title = {Spin-up of {{UK}} Earth System Model 1 ({{UKESM1}}) for {{CMIP6}}},
  author = {Yool, A. and Palmi{\'e}ri, J. and Jones, C. G. and Sellar, A. A. and {de Mora}, L. and Kuhlbrodt, T. and Popova, E. E. and Mulcahy, J. P. and Wiltshire, A. and Rumbold, S. T. and Stringer, M. and Hill, R. S. R. and Tang, Y. and Walton, J. and Blaker, A. and Nurser, A. J. G. and Coward, A. C. and Hirschi, J. and Woodward, S. and Kelley, D. I. and Ellis, R. and {Rumbold-Jones}, S.},
  year = {2020},
  volume = {12},
  doi = {10.1029/2019MS001933},
  journal = {Journal of Advances in Modeling Earth Systems},
  keywords = {carbon cycle,CMIP6,Earth system model,equilibrium,spin-up},
  number = {8}
}

@article{Yool2013,
  title = {{{MEDUSA}}-2.0: {{An}} Intermediate Complexity Biogeochemical Model of the Marine Carbon Cycle for Climate Change and Ocean Acidification Studies},
  author = {Yool, A. and Popova, E. E. and Anderson, T. R.},
  year = {2013},
  issn = {1991959X},
  doi = {10.5194/gmd-6-1767-2013},
  journal = {Geoscientific Model Development}
}

@article{yoolSpinup2020,
  title = {Spin-up of {{UK Earth System Model}} 1 ({{UKESM1}}) for {{CMIP6}}},
  author = {Yool, A. and Palmi{\'e}ri, J. and Jones, C. G. and Sellar, A. A. and {de Mora}, L. and Kuhlbrodt, T. and Popova, E. E. and Mulcahy, J. P. and Wiltshire, A. and Rumbold, S. T. and Stringer, M. and Hill, R. S. R. and Tang, Y. and Walton, J. and Blaker, A. and Nurser, A. J. G. and Coward, A. C. and Hirschi, J. and Woodward, S. and Kelley, D. I. and Ellis, R. and {Rumbold-Jones}, S.},
  year = {2020},
  volume = {12},
  pages = {e2019MS001933},
  issn = {1942-2466},
  doi = {10.1029/2019MS001933},
  abstract = {For simulations intended to study the influence of anthropogenic forcing on climate, temporal stability of the Earth's natural heat, freshwater, and biogeochemical budgets is critical. Achieving such coupled model equilibration is scientifically and computationally challenging. We describe the protocol used to spin-up the UK Earth system model (UKESM1) with respect to preindustrial forcing for use in the sixth Coupled Model Intercomparison Project (CMIP6). Due to the high computational cost of UKESM1's atmospheric model, especially when running with interactive full chemistry and aerosols, spin-up primarily used parallel configurations using only ocean/land components. For the ocean, the resulting spin-up permitted the carbon and heat contents of the ocean's full volume to approach equilibrium over 5,000 years. On land, a spin-up of 1,000 years brought UKESM1's dynamic vegetation and soil carbon reservoirs toward near-equilibrium. The end-states of these parallel ocean- and land-only phases then initialized a multicentennial period of spin-up with the full Earth system model, prior to this simulation continuing as the UKESM1 CMIP6 preindustrial control (piControl). The realism of the fully coupled spin-up was assessed for a range of ocean and land properties, as was the degree of equilibration for key variables. Lessons drawn include the importance of consistent interface physics across ocean- and land-only models and the coupled (parent) model, the extreme simulation duration required to approach equilibration targets, and the occurrence of significant regional land carbon drifts despite global-scale equilibration. Overall, the UKESM1 spin-up underscores the expense involved and argues in favor of future development of more efficient spin-up techniques.},
  copyright = {\textcopyright 2020. The Authors.},
  file = {/Users/oscardimdore-miles/Zotero/storage/UBBFWI7J/Yool et al. - 2020 - Spin-up of UK Earth System Model 1 (UKESM1) for CM.pdf},
  journal = {Journal of Advances in Modeling Earth Systems},
  keywords = {carbon cycle,CMIP6,Earth system model,equilibrium,spin-up},
  language = {English},
  number = {8}
}

@article{Zhang1997,
  title = {{{ENSO}}-like Interdecadal Variability: 1900\textendash 93},
  author = {Zhang, Yuan and Wallace, John M. and Battisti, David S.},
  year = {1997},
  volume = {10},
  pages = {1004--1020},
  issn = {0894-8755},
  doi = {10.1175/1520-0442},
  journal = {Journal of Climate},
  number = {5}
}

@article{Zhang2019,
  title = {Estimates of Decadal Climate Predictability from an Interactive Ensemble Model},
  author = {Zhang, Wei and Kirtman, Ben},
  year = {2019},
  volume = {46},
  pages = {3387--3397},
  doi = {10.1029/2018GL081307},
  journal = {Geophysical Research Letters},
  number = {6}
}

@article{zhangLatitudinal2010,
  title = {Latitudinal Dependence of {{Atlantic}} Meridional Overturning Circulation ({{AMOC}}) Variations},
  author = {Zhang, Rong},
  year = {2010},
  volume = {37},
  issn = {1944-8007},
  doi = {10.1029/2010GL044474},
  abstract = {AMOC variations are often thought to propagate with the Kelvin wave speed, resulting in a short time lead between high and low latitudes AMOC variations. However as shown in this paper using a coupled climate model (GFDL CM2.1), with the existence of interior pathways of North Atlantic Deep Water (NADW) from Flemish Cap to Cape Hatteras as that observed recently, AMOC variations estimated in density space propagate with the advection speed in this region, resulting in a much longer time lead (several years) between subpolar and subtropical AMOC variations and providing a more useful predictability. The results suggest that AMOC variations have significant meridional coherence in density space, and monitoring AMOC variations in density space at higher latitudes might reveal a stronger signal with a several-year time lead.},
  copyright = {Copyright 2010 by the American Geophysical Union.},
  file = {/Users/oscardimdore-miles/Zotero/storage/8E65BESE/Zhang - 2010 - Latitudinal dependence of Atlantic meridional over.pdf;/Users/oscardimdore-miles/Zotero/storage/VLDI3RL9/2010GL044474.html},
  journal = {Geophysical Research Letters},
  keywords = {AMOC},
  language = {English},
  number = {16}
}



