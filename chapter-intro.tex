\chapter{Introduction}
\label{cha:intro}

\section{Overview and aims}
\label{sec:overview}
Major Sudden Stratospheric Sudden Warming (SSW) events involve significant disruption of the Northern Hemisphere (NH) stratospheric polar vortex and represent the largest mode of interannual variability in the boreal winter stratosphere \citep{Butler2017,baldwin2020}.

- They are associated with modes of surface variability as well as coupling with other parts of the stratosphere e.g. equatorial region

- As a result, these events warrant a large degree of scientific investigation.

- Despite the large body of work examining these events, the polar vortex and its coupling.... key features are still not fully understood.

- In observation based datasets known as reanalyses, SSWs occur at an average rate of 0.6 events/winter but this varies markedly over the record \citep{Butler2015} suggesting the possibility of variability on much longer timescales. For example, observational studies have noted a hiatus in the 1990s when very few major SSW events occurred \citep{Butler2015,Pawson1999,Shindell1999} while, in contrast, the early 21st century displayed a remarkable number of consecutive winters containing SSW events \citep{Manney2005}. 

- The work presented in this thesis investigates variability in polar vortex strength on multiple timescales as well as the role such variability plays in coupling between the vortex and other parts of the climate system. 

The primary original contributions to this field of study contained in this thesis are as follows:

\begin{itemize}
    \item Chapter xxx considers multidecadal variability in the occurrence of SSW events in a 1000 year pre industrial control simulation of a fully coupled GCM. I utilise a wavelet analysis and identify non stationary periodic variability in these events corresponding to periods around 90-100 years as well as less persistent signals at 60 year scales. I also examine possible origins for this variability and find that similar variations in the amplitude and vertical coherence of the equatorial winds (the Quasi Biennial Oscillation) are likely driving signals in SSWs. This represents a potential novel source of predictability of SSW events.
    
    \item
    Chapter xxxx considers the influence of multi-decadal vortex variations found in chapter xxx on surface variability with a particular focus on low frequency modes in the Ocean. I find an assoication between the vortex strength and the AMOC 
\end{itemize}

\section{Relation to published work}

Chapter 3 broadly corresponds to a paper written by myself, Lesley Gray and Scott Osprey and published in Weather and Climate Dynamics (ref here). However, the analysis presented here includes an extension to the published work and has been rewritten for inclusion in the thesis. Chapter xxx also corresponds to a paper written by myself, Lesley and Scott (ref) with the significant help of Jon Robson, Bablu Sindhu and Rowan Sutton  A paper based on the work in chapter 4 is currently being constructed and will hopefully be submitted in the near future. 

In all papers, I conducted the analysis and was in charge all writing and generated all figures. I am extremely grateful to all my coauthors for their input and comments during the writing of these papers as well as all those involved in reviewing both works (all of whom are anonymous).  

\section{Thesis structure}
Chapter 2 presents a background overview of middle atmospheric dynamics and the variability in the polar and equatorial stratosphere including SSWs and the QBO. This includes the current understanding of these modes and their coupling as well as their variability on decadal to multi-decadal timescales. It also outlines some of the key features of current understanding of stratosphere-troposphere coupling. This includes interactions between the stratosphere and the North Atlantic Oscillation, Aleutian Low, El Nino Southern Oscillation as well as oceanic modes such as the Atlantic Meridional Overturning Cirulation. Results from the original analysis described in section \ref{sec:overview} is presented in chapters xx xxx and xxx. Conclusions as well as possible further work are presented in chapter xxx.



%%% Local Variables:
%%% mode: latex
%%% TeX-master: "thesis"
%%% End:







