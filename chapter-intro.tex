\chapter{Introduction}
\label{cha:intro}

\section{Overview and aims}
\label{sec:overview}

There is a variety of reasons that the Northern Hemisphere (NH) stratospheric polar vortex warrants significant scientific study. Variations in its strength, particularly disruption events known as Sudden Stratospheric Warmings (SSWs), are among the most dramatic dynamical events observed in the Earth's atmosphere. They also play an important role in stratosphere-troposphere coupling as they have been shown to both respond to and influence variations in surface weather and climate. Despite significant developments in understanding of vortex variability, some key features of this phenomenon are still not fully understood. Namely: 

\begin{itemize}
    \item The nature of variability in the strength of the vortex, including the frequency of SSWs, is yet to be fully diagnosed. This is particularly an issue when considering variations on decadal to multi-decadal timescales.  
    
    \item the full extent of interactions between the vortex and other parts of the climate system such as the equatorial stratosphere, troposphere, surface and ocean variations is also not understood. 

\end{itemize}

These are significant and ongoing issues in the field. Research efforts are often hampered by the limited observational record and works rely heavily on state of the art climate models for diagnosis of climate behaviour. While it is beyond the remit of this work to provide comprehensive resolutions to these issues, this thesis aims to contribute to the understanding of variability in polar vortex strength on multiple timescales as well as the role such variability plays in coupling between the vortex and other parts of the climate system. This understanding is not only motivated by scientific curiosity. Vortex variations and SSWs play a key role in seasonal to sub-seasonal forecasts of NH weather \citep{Domeison2019, Domeison2019-2}. As a result, an improved understanding of their variability and coupling could aid in improving predictability of the climate system. The primary original contributions to this field of study contained in this thesis are as follows:

\begin{itemize}
    \item Chapter 3 considers multidecadal variability in the occurrence of SSW events in a 1000 year pre-industrial control simulation of a coupled GCM. A wavelet spectral decomposition analysis is utilised to identify non-stationary periodic variations in events corresponding to periods around 90-100 years as well as less persistent signals at 60 year scales. Possible origins for this variability is examined and it is shown that similar variations in the amplitude and vertical coherence of the equatorial winds (the Quasi Biennial Oscillation) are likely driving signals in SSWs. This represents a potential novel source of predictability of SSW events.
    
    \item Chapter 4 considers the influence of multi-decadal vortex variations found in chapter 3 on surface variability with a particular focus on low frequency modes in the North Atlantic region. Co-variations are found between the appearance of intervals of winters which exhibit the same type of anomalous behaviour (strong or weak) and the Atlantic Meridional Overturning Circulation on 50 and 30 year timescales. This interaction appears to involve a pathway through the vortex's influence over tropospheric circulation and the North Atlantic Oscillation. This result builds on previous work suggesting a similar connection and demonstrates the key role non-stationarities may play in the teleconnection. It also suggests a potentially key role of the stratosphere in recent observed trends in the AMOC. 
    
    \item Chapter 5 presents an analysis of the role of vertical coherence in the Quasi Biennial Oscillation in teleconnections with NH atmospheric and surface variability. This analysis utilises a set of GCM experiments which apply a relaxation of the QBO winds to a set of idealised fields which impose different degrees of vertical coherence. We find that some elements of teleconnections, for example early winter QBO-vortex coupling, are enhanced in a simulation with a high degree of vertical coherence compared to a simulation which imposes a low degree of coherence as well as a free-running simulation. However, the surface and vortex responses in the vertically coherent QBO experiment are dominated by influence from upper stratospheric equatorial winds (the semiannual oscillation, SAO) which exhibit unrealistically large anomalies. In this case, the magnitude of the SAO's influence over the vortex and surface obscures other pathways for QBO teleconnections.

\end{itemize}

\section{Relation to published work}

Chapter 3 broadly corresponds to a paper written by myself, Lesley Gray and Scott Osprey and published in Weather and Climate Dynamics \citep{dimdore-milesOrigins2020b}. However, the analysis presented here includes an extension to the published work and has been rewritten for inclusion in the thesis. Chapter 4 also corresponds to \cite{dimdore-milesInteractions2021}, a paper written by myself, Lesley Gray, Scott Osprey, Jon Robson, Bablu Sinha and Rowan Sutton which is currently under review by Atmospheric Chemistry and Physics. A paper based on the work in chapter 5 is currently being constructed and will hopefully be submitted in the near future. 
In all papers, I conducted the analysis, was in charge of all writing and generated all figures. I am extremely grateful to all my co-authors for their input and comments during the writing of these papers as well as all those involved in reviewing both works (all of whom are anonymous).  

\section{Thesis structure}
Chapter 2 presents a background overview of middle atmospheric dynamics and the variability in the polar and equatorial stratosphere including SSWs and the QBO. This includes the current understanding of these modes and their interactions as well as their variability on decadal to multi-decadal timescales. It also outlines some of the key features of current understanding of stratosphere-troposphere coupling. This includes interactions between the stratosphere and the North Atlantic Oscillation, Aleutian Low, El Ni\~{n}o Southern Oscillation as well as oceanic modes such as the Atlantic Meridional Overturning Circulation. Results from the original analysis summarised above are presented in chapters 3, 4 and 5. These chapters also include an outline of the analysis tools and data utilised in each analysis. Conclusions as well as possible further work are presented in chapter 6.

%%% Local Variables:
%%% mode: latex
%%% TeX-master: "thesis"
%%% End:







