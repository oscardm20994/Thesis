\chapter{A geometrical description of vortex variability}
\begin{quotation}
  Much of the work contained in this chapter is based upon \citet{Seviour2013},
  published in \emph{Geophysical Research Letters}, although the analysis
  presented here has been significantly extended.
\end{quotation}

\label{cha:moments}

% To Do: 
% - Discussion of sensitivity of the clustering algorithm
% - Description of dripping paint plot
% - Significance for dripping paint plot
% - CP07 and M13 events in table (or maybe disagreeing events highlighed
%   - plot disagreeing events seperately to check
% - Possibly add seasnal cycle of moment diagnostics with percentiles 



\section{Introduction}
\label{sec:moments-introduction}
A quantitative description of stratospheric polar vortex variability is
desirable for a number of reasons; it allows for the comparison of different
studies, observational data sets, and model simulations, as well as permitting
robust definitions of extreme events.  Traditional methods to quantify vortex
variability have been based on zonal-mean diagnostics, such as the zonal-mean
zonal wind \citep[e.g.,][]{Andrews1987}. This was motivated both by the
simplicity of these diagnostics and the physical reasoning that the strength of
the zonal flow controls the propagation of planetary waves \citep[][Section
\ref{sec:plan-waves-strat}]{Charney1961}. \citet{McInturff1978} provided the
first quantitative definition of SSWs\footnote{In the literature, this is often
  called ``the WMO definition'', although at least two different definitions can
  be attributed to the WMO \citep{Butler2014a}.} (referred to in that text as
``major stratospheric warmings'') using zonal mean quantities as below.
\begin{quotation}
\emph{A stratospheric warming can be said to be major if at 10~mb or below the
latitudinal mean temperature increases poleward from 60 degrees latitude and an
associated circulation reversal is observed (i.e., mean westerly winds poleward
of 60$^{\circ}$ latitude are succeeded by mean easterlies in the same area).}
\end{quotation}
A number of variations of this definition have since appeared in the
literature. Most commonly, the temperature gradient criterion has been neglected
and/or zonal wind reversals at a particular latitude (usually $60^{\circ}$N)
used instead of the stricter criterion of a reversal everywhere poleward of
$60^{\circ}$N \citep[e.g.,][]{Labitzke2000, Christiansen2001,
  Reichler2012}. 

Although the reversal of zonal-mean zonal wind is physically relevant for the
propagation of planetary waves, the choice of $60^{\circ}$N and 10~hPa in the
definition of SSWs is less physically significant. Indeed, different numbers of
SSWs are identified if these locations are varied. \citet{Butler2014a} found
that a greater number of events are identified if the threshold is located
either equatorward or poleward of $60^{\circ}$N. Some studies have aimed to
avoid this sensitivity to spatial location by quantifying vortex variability
through empirical orthogonal function (EOF) analysis, using fields over a larger
area. This includes the Northern Annular Mode (NAM) (calculated either from the
three-dimensional geopotential height field \citep{Baldwin2001a} or zonal-mean
geopotential height \citep{Baldwin2009}), EOFs of zonal wind
\citep{Limpasuvan2004}, and vertical profiles of polar cap-averaged temperature
\citep{Kuroda2004}. SSW events are then defined by a threshold in the principal
component of the relevant EOF.

As it has become increasingly recognised that SSWs generally occur as either
split or displaced vortex events, studies have aimed to objectively distinguish
these two types of event. Commonly this has been achieved through Fourier
decomposition of the zonal wave structure. For instance, \citet{Nakagawa2006}
defined SSWs through a polar temperature criterion and then split these events
into two groups depending on whether the 150~hPa Eliassen-Palm (EP) flux prior
to the events was dominated by zonal wavenumber one or two. \citet{Charlton2007}
(hereafter CP07) introduced a new classification method, which does not rely on
Fourier decomposition; first they identified events using the traditional wind
reversal at $60^{\circ}$N, 10~hPa criterion, then they calculated the circulation
around the two largest contours of relative vorticity on the vortex edge. If
these two contours have a circulation ratio of 2:1 or lower the event is
classified as a split, and all other events are automatically classed as
displacements. 

Both Fourier decomposition of the zonal wave structure and the method of CP07
rely on an Eulerian framework, with fields analysed at a fixed spatial
location. \citet{Waugh1997} first applied two-dimensional moment diagnostics
(otherwise known as elliptical diagnostics) to the stratospheric polar vortex to
provide an alternative semi-Lagrangian (or vortex-oriented) framework. These
diagnostics are calculated by fitting an ellipse to a contour and then
determining its properties such as the centre, orientation, aspect ratio, and
area (a further diagnostic, excess kurtosis--a measure of the `peakedness' of
the distribution--was introduced by \citet{Matthewman2009}). This allows the
movement and elongation of the vortex to be quantified. \citet{Waugh1997} also
compared these diagnostics to the traditional Fourier decomposition. He showed
that wave-1 and 2 amplitudes relate most strongly to the displacement and
elongation of the vortex respectively, however, these relationships were not
found to be strong, with correlations of daily values less than 0.5. These weak
relationships were attributed to the fact that planetary wave propagation can be
affected by changes in the meridional PV gradient, even if the vortex shape and
location are fixed. Furthermore, the wave-1 amplitude depends to some extent on
the elongation of the vortex as well as the location of the centre (and
similarly for the wave-2 amplitude). He concluded that it is difficult to
extract quantitative information about the shape and location of the vortex
based on wave amplitudes alone, highlighting the advantages of the moment
diagnostics.

\citet{Hannachi2010} then applied a hierarchical clustering algorithm to daily
values of the area, centroid latitude, and aspect ratio diagnostics and found
that the vortex falls preferably into three clusters corresponding to
undisturbed, split, and displaced states. These groupings were used by
\citet{Mitchell2013} (hereafter M13) to identify split and displaced vortex
events; if the vortex remained in the split or displaced cluster for at least
five consecutive days it was classified as the corresponding
event. Significantly, as discussed in Section \ref{sec:observ-evid}, M13
demonstrated that split vortex events penetrated deep into the troposphere and
resulted in significant surface anomalies, while anomalies associated with
displaced vortex events do not descend far below the tropopause. This is in
agreement with \citet{Nakagawa2006} who found tropospheric anomalies to be
larger following SSWs with dominant wave 2 amplitude, however, it contrasts with
CP07, who found little difference in the tropospheric impact of split and
displaced vortex events. This highlights the potential importance of the method
of classification of split and displaced vortex events in any study.

In this chapter we wish to develop a method for the classification of split and
displaced vortex events with the following properties:
\begin{itemize}
\item It is based on vortex moment diagnostics.
\item It can be easily applied to a range of data sets, including climate model
  simulations.
\item It is computationally inexpensive. 
\end{itemize}
The motivation for the use of moment diagnostics includes their advantages in
quantifying the shape and location of the vortex, as noted above. This, in turn,
is desirable because the location of the vortex near the tropopause may be
important for understanding the regional tropospheric effect of stratospheric
anomalies \citep[e.g.,][Section \ref{sec:mechanisms}]{Ambaum2002}. Previous
calculations of vortex moment diagnostics have been based on the distributions
of quasi-conservative tracers such as PV on isentropic surfaces
\citep{Mitchell2011} or long-lived tracer (e.g., N$_{2}$O) concentrations
\citep{Waugh1997}. These quantities have strong meridional gradients allowing
for clear determination of the vortex edge \citep{Nash1996}. Unfortunately, many
climate models do not output PV or tracer concentrations, and these are often
computationally expensive or impractical to calculate. As such, we wish to
develop a method which uses geopotential height, a variable which is output by
all contemporary climate models. This effort will also allow us to test the
robustness of the result of M13 regarding the different surface impacts of split
and displaced vortex events using a semi-independent classification method and
extended data set.

The remainder of this chapter is structured as follows. The next section
introduces the necessary theoretical background for the calculation of moment
diagnostics. Section \ref{sec:methodology} describes the methods used for the
classification of split and displaced vortex events, and compares these events
with those determined by M13 and CP07. Section \ref{sec:moments_analysis}
contrasts the surface impacts of split and displaced vortex events calculated
using the new method and discusses potential mechanisms behind any differences. 


% Stratospheric sudden warmings (SSWs) are extreme events in which the strong
% westerly winds that usually dominate the winter polar stratosphere become highly
% disturbed (here, for reasons outlined below, we use the term SSW to encompass a
% wider range of variability than its traditional definition). These events lead
% to the mixing of mid-latitude air into the polar vortex region, causing an
% increase in temperatures by several tens of kelvin over the course of a few
% days. Traditional methods to identify stratospheric sudden warmings (SSWs) have
% relied on either zonal-mean \citep{Andrews1987} or annular mode
% \citep{Baldwin2001a} diagnostics. Neither method explicitly deals with the
% inherent zonal asymmetry in vortex variability. In particular, SSWs are observed
% to occur in one of two manners: displaced vortex events, where the vortex moves
% far from the pole, and split vortex events, where the vortex separates into two
% `child' vortices. These two types have a very different spatial structure and
% evolution timescale \citep{Matthewman2009}. Displaced and split vortex events
% are predominantly associated with vertically propagating Rossby waves of
% wavenumber 1 and 2 respectively, and many previous studies have classified SSWs
% based on wavenumber \citep[e.g.][]{Nakagawa2006}. However, this method does not
% provide a description of the location of the polar vortex itself, which
% theoretical arguments suggest may be important for understanding
% stratosphere-troposphere coupling \citep{Ambaum2002}. In an improvement to these
% traditional SSW definitions, \citet{Charlton2007} (hereafter CP07) introduced a
% classification in which a split vortex event is identified when two vortices
% with a circulation ratio of 2:1 or higher are present, and all other SSWs are
% automatically classed as displaced vortex events. However, they maintained the
% traditional SSW identification which requires there to be a reversal of the
% zonal-mean zonal wind at 10~hPa and $60^{\circ}$N.

% An increased understanding of stratospheric variability can be gained by using
% vortex-centric diagnostics, such as two-dimensional (2D) vortex moments
% \citep{Waugh1997, Waugh1999, Mitchell2011, Mitchell2011a}, which provide a
% geometrical description of the vortex and have no reliance on zonal-mean
% properties. Using a classification based on these diagnostics,
% \citet{Mitchell2013} (hereafter M13) identified a greater number of SSWs than
% CP07. This is primarily because they did not use a zonal mean threshold
% criterion. Importantly, M13 also demonstrated that split vortex events
% penetrated deep into the troposphere and resulted in significant surface
% anomalies, while anomalies associated with displaced vortex events do not
% descend far below the tropopause. Their result supported a similar conclusion by
% \citet{Nakagawa2006}, who found that the impact of events associated with an
% enhanced upward flux of wavenumber-2 planetary waves was more likely to reach
% the surface. These results underline the need to correctly identify the precise
% type of SSW, in order to understand stratosphere-troposphere coupling within
% climate models.

% Distinguishing between displaced and split vortex events using the method of M13
% requires the use of potential vorticity (PV), which is not commonly output by
% climate models. For this reason, previous attempts to apply PV-based techniques
% in a multi-model study have led to the majority of models being excluded
% \citep{Mitchell2012a}. Furthermore, their method used a hierarchical clustering
% technique \citep{Hannachi2010}, which is very sensitive to the exact shape of
% the distribution of vortex variability, so is unsuitable for application to a
% range of models with different climatologies. In this chapter, we develop an
% improved method which; (a), is based on the geometry of the vortex, but requires
% only the 10~hPa geopotential height; and (b), identifies events using a simple
% threshold instead of a clustering technique. We apply this new method to the
% ERA-40 and ERA-Interim reanalysis datasets and demonstrate that the method
% captures a similar number of events which are in good agreement with, and at
% least as extreme as, those of M13.

\section{Vortex moment diagnostics}
\label{sec:vort-moment-diagn}

The moments, $M_{n}$, of a one-dimensional distribution can be classified by
their order, $n$, and provide familiar parameters. These are the area under the
distribution (0th order), mean (1st order), variance (2nd order), skewness (3rd
order), and kurtosis (4th order), given by
\begin{equation}
M_{n} = \int_{S} x^{n}f(x)~\mathrm{d}x \, ,
\end{equation}
where $S$ represents the extent of the distribution, $f(x)$, to be integrated
over. The extension of this for a two-dimensional distribution is
straightforwardly
\begin{equation}
M_{nm} = \iint_{S} x^{n}y^{m}f(x,y)~\mathrm{d}x\mathrm{d}y \, ,
\label{eq:2D_moment}
\end{equation}
where the order of the moment is now defined as $m+n$, meaning it is possible to
have different diagnostics with the same order (e.g., $M_{01}$,
$M_{10}$). Although these diagnostics can be further extended to three
dimensions, this has been demonstrated to be highly computationally expensive
\citep{Li1994}, and would require assumptions about the lower and upper bounds
of the vortex region. We therefore calculate two-dimensional moment diagnostics
for the stratospheric polar vortex on quasi-horizontal surfaces. We use two
variables; geopotential height ($f(x,y) = Z(x,y)$) on the 10~hPa pressure level,
and potential vorticity ($f(x,y) = q(x,y)$) on the 850~K potential temperature
(isentropic) surface, which lies close to 10~hPa. Following \citet{Waugh1997},
the calculation of moment diagnostics is simplified by transforming the
spherical data $q(\phi,\lambda)$ and $Z(\phi,\lambda)$, where $\phi$ is latitude
and $\lambda$ longitude, to Cartesian coordinates using the polar stereographic
projection
\begin{equation}
x = \frac{\cos\lambda\cos\phi}{1 \pm \sin\phi}\, , \quad
y = \frac{\pm\sin\lambda\cos\phi}{1 \pm \sin\phi}\, , 
\end{equation}  
where the positive sign is used in the NH and negative in the SH. The use of
planar geometry introduces some biases (e.g., an ellipse centred off the pole
would appear slightly banana-shaped), however these biases were shown by
\citet{Waugh1997} to be small in the case of stratospheric polar
vortices. Hence, although it is possible to calculate moment diagnostics in
spherical geometry \citep{Dritschel1993}, the simpler planar formulae are used
here.

In order to calculate moment diagnostics for the stratospheric polar vortex we
must first isolate the vortex region by defining the vortex edge. Different
methods have previously been used for this calculation; \citet{Waugh1999} used
the mean PV at the maximum of the mean meridional PV gradient, while
\citet{Matthewman2009} defined the vortex edge on a daily basis, using the
average value of PV poleward of $45^{\circ}$N nine days before the onset of a
SSW (their SSWs were defined by zonal-mean zonal wind reversal, as in CP07). A
more complex method due to \citet{Nash1996}, starts by transforming PV to
`equivalent latitude' \citep{Butchart1986} coordinates, before defining the
vortex edge as the position of the largest gradient in a plot of PV against
equivalent latitude. This method was applied in \citet{Mitchell2011} to
calculate the vortex edge. 

None of the three methods outlined above are found to be appropriate for the
present study. We wish to directly compare the PV and geopotential
height-derived moments, but the methods of \citet{Waugh1999} and
\citet{Nash1996} rely on meridional gradients in PV and so may not be
transferable to geopotential height. Furthermore, the method of
\citet{Matthewman2009} is impractical because we wish to define the events from
the moment diagnostics, so will not know their dates before
calculation. Instead, we pick a simple definition; PV ($q_{b}$) or geopotential
height ($Z_{b}$) on the vortex edge is defined as the value of the
December-March (DJFM) mean at $60^{\circ}$N for the NH and the
June-September (JJAS) mean at $60^{\circ}$S for the SH. This is
seen to lie close to contours defined by the above methods, and results are
insensitive to small changes in the latitude chosen.

Having defined the vortex edge, we extend the method \citet{Matthewman2009} to
isolate the vortex region by introducing a transformed PV field, $\hat{q}$,
given by
\begin{equation}
 \hat{q}(x,y) = 
 \begin{cases}
   q(x,y) - q_{b} & \text{if $q(x,y) > q_{b}$} \, , \\
   0 & \text{if $q(x,y) \leq q_{b}$} \, , 
 \end{cases}
\end{equation}
and conversely for geopotential height 
\begin{equation}
 \hat{Z}(x,y) = 
 \begin{cases}
   Z(x,y) - Z_{b} & \text{if $Z(x,y) < Z_{b}$} \, , \\
   0 & \text{if $Z(x,y) \geq Z_{b}$} \, . 
 \end{cases}
\end{equation}
By substituting $f(x,y) = \hat{q}(x,y)$ or $f(x,y) = \hat{Z}(x,y)$ in equation
\ref{eq:2D_moment} it is then possible to calculate the moment diagnostics. The
zeroth order moment diagnostic, $M_{00}$ can be used to define the `equivalent
area', $A_{\mathrm{eq}}$ \citep{Matthewman2009}, as
\begin{equation} 
A_{\mathrm{eq}} = \frac{M_{00}}{q_{b}}\quad \text{or} \quad A_{\mathrm{eq}} = \frac{M_{00}}{Z_b}\, ,
\end{equation}
depending on whether PV or geopotential height based diagnostics are
calculated. Because $M_{00} \approx Aq$, where A is the vortex area, the
equivalent area can be considered a measure of both vortex strength and
area. The first order moment diagnostic can be used to calculate the vortex
centroid,
\begin{equation}
(\bar{x}, \bar{y}) = \left( \frac{M_{10}}{M_{00}}, \frac{M_{01}}{M_{00}} \right)
\, . 
\end{equation}

In order for higher order moment diagnostics to be useful, the moment equation
(\ref{eq:2D_moment}), must be transformed to the \emph{centralised moment}
form \citep{Hall2005}. This calculates moments relative to the vortex centroid,
and is given by
\begin{equation}
J_{mn} = \iint_{S} f(x,y)(x-\bar{x})^n(y-\bar{y})^m~\mathrm{d}x\mathrm{d}y \, .
\end{equation}
Two useful parameters can be derived from the second-order centralised moment
diagnostics, the vortex orientation, $\psi$ (defined as the angle between the
major axis of the ellipse and the $x$-axis) and the aspect ratio, $r$ (defined
as the ratio of the lengths of the major to minor axes), given by
\begin{equation}
\psi = \frac{1}{2} \tan^{-1} \left( \frac{2J_{11}}{J_{20}-J_{02}} \right) \, ,
\end{equation}
\begin{equation}
r = \left| \frac{(J_{20}+J_{02})+\sqrt{4J_{11}^2+(J_{20}-J_{02})^2}}
  {(J_{20}+J_{02})-\sqrt{4J_{11}^2+(J_{20}-J_{02})^2}} \right|^{1/2} \, .
\end{equation}
Using the area, centroid, orientation, and aspect ratio, the \emph{equivalent
  ellipse} can be uniquely defined. Figure \ref{fig:displaced_ellipse} shows the
equivalent ellipse calculated from both PV and geopotential height fields over a
16-day period centred on a displaced vortex event (classified using the method
in Section \ref{sec:methodology}). It can be seen that the equivalent ellipse
provides a qualitatively good fit to the vortex, although this is less good in
Figures \ref{fig:displaced_ellipse}(c,f) when the vortex becomes less elliptical
and filamentation occurs. Greater fine-scale structure and filamentation is
visible in the PV field due to its quasi-conservative properties, however
reasonable agreement can be seen between the PV and geopotential height
ellipses. 

\begin{figure}
 \centering
 \noindent\includegraphics[width=\textwidth]{figures/chapter-moments/PV_GPH_2006.pdf}
 \caption[Equivalent ellipse for a displaced vortex event]{PV on the 850~K
   $\theta$ surface (a,b,c) and geopotential height at 10~hPa (d,e,f) 8 days
   before (a,d), at onset (b,e), and 8 days following the onset (c,f) of a
   displaced vortex event. Contours of $q_{b}$ and $Z_{b}$ are shown in thin
   black lines, the equivalent ellipse in a thick dark line, and its centroid
   with a white cross. Data are transformed to Cartesian coordinates with a
   polar stereographic projection.}
 \label{fig:displaced_ellipse}
\end{figure}

Equivalent ellipses for an example of a split vortex event are shown in Figure
\ref{fig:split_ellipse}. It can be seen that after the vortex has separated the
equivalent ellipse becomes less physically significant, as it spans the two
vortices. \citet{Matthewman2009} introduced the 4th order moment diagnostic,
``excess kurtosis'', in order to identify splits of the polar vortex; it is
given by
\begin{equation}
\kappa_4 = M_{00}\frac{J_{40}+2J_{22}+J_{04}}{(J_{20}+J_{02})^2}-\frac{2}{3}\left[\frac{3r^4+2r^2+3}{(r^2+1)^2}\right]\,.
\end{equation}
This has the property of being negative for a vortex with a ``pinched'' shape,
zero for a perfectly elliptical vortex, and positive for a vortex with a strong
central core. When negative kurtosis was detected \citet{Matthewman2009} split
the PV field into two regions along the minor axis of the equivalent ellipse and
re-calculated moment diagnostics for the vortices in these regions separately.

In this study we do not make use of the excess kurtosis or calculate separate
diagnostics for split vortices for three reasons. First, as a 4th order
diagnostic it is a highly skewed variable, making its use in event
classification problematic (this was also found by
\citet{Hannachi2010}). Second, this procedure is more computationally expensive,
requiring about three times the number of calculations during split vortex
events. Third, kurtosis is highly sensitive to horizontal resolution
\citep{Mitchell2011}, and so may not be a suitable diagnostic in the comparison
of climate models with different resolutions. Hence, we calculate single moment
diagnostics even when the vortex has split, but bear in mind that these may not
represent the properties of any real vortex.

Code for the calculation of moment diagnostics using the method described in
this section is available from \url{https://github.com/wseviour/vortex-moments}.

\begin{figure}
 \centering
 \noindent\includegraphics[width=\textwidth]{figures/chapter-moments/PV_GPH_1979.pdf}
 \caption[Equivalent ellipse for a spit vortex event]{As Figure
   \ref{fig:displaced_ellipse} but for a split vortex event.}
 \label{fig:split_ellipse}
\end{figure}


\section{Data and methods}
\label{sec:methodology}

\subsection{Reanalysis data}
\label{sec:reanalysis-data}
For the analysis in this chapter NH winter daily-mean data for
December-March (DJFM) are employed from the European Centre for Medium-Range
Weather Forecasts (ECMWF) reanalyses. The ERA-40 data set \citep{Uppala2005} is
used from 1958-1978 and ERA-Interim \citep{Dee2011} from 1979-2009. The
combination of these two data sets is chosen in order to maximise the total
number of years entering the analysis (ERA-40 runs only to 2002), as well as to
compare results from the more recent ERA-Interim with previous studies using
only ERA-40, such as \citet{Charlton2007} and \citet{Mitchell2013}.

ERA-Interim is similar to ERA-40 but uses a four-dimensional variational data
assimilation system (4D-Var) as opposed to the 3D-Var system used in ERA-40. It
also has higher horizontal and vertical resolution, improved humidity analysis,
model physics, data quality control, bias handling and other improvements as
noted in \citet{Simmons2007}. The majority of observational data for the
stratosphere entering both reanalyses are from radiosonde and satellite
measurements. It is important to note that in the pre-satellite era (1958-1971)
observations in the stratosphere were much more sparse, leading to greater
errors in reanalyses during this time \citep{Uppala2005}.

A number of studies have evaluated the stratospheric circulation in ERA-40 and
ERA-Interim against other observations or reanalyses. \citet{Randel2004} found
ERA-40 to closely match measurements of the zonal stratospheric circulation
derived from radiosonde, rocketsonde and lidar
measurements. \citet{Karpetchko2005} found that the representation of the polar
vortices in ERA-40 agrees well with the NCEP/NCAR reanalysis, and CP07
demonstrated that this also holds for the occurrence of
SSWs. \citet{Seviour2012} showed that the strength of the stratospheric
meridional mean stratospheric circulation in ERA-Interim agrees well with
previous reanalysis, but that the residual vertical velocity is more smoothly
represented.

In order to perform a consistent analysis across the two data sets, ERA-Interim
data is linearly interpolated to the lower resolution ERA-40
($1.125^{\circ} \times 1.125^{\circ}$) Gaussian grid. PV is also interpolated
from pressure levels to the 850~K isentropic surface (which lies close to
10~hPa), as this quantity has the property of being conserved under adiabatic
flow. Both in the calculation of the vortex edge (climatological mean $q$ or $Z$
at $60^{\circ}$N/S) and the moment diagnostics themselves, no clear jumps were
seen between ERA-40 and ERA-Interim data sets. As such, the two are considered
together with no bias corrections. For the remainder of this thesis, this
combined ERA-40 and ERA-Interim data set is referred to as \emph{ERA}.


\subsection{Moment diagnostic calculation}
\label{sec:vort-geom-calc}

In order to calculate the moment diagnostics, the values of PV and geopotential
height on the vortex edge ($q_b$ and $Z_b$) must first be determined. These are
the $60^{\circ}$N DJFM/$60^{\circ}$S JJAS mean values of PV at 850~K ($q_{850}$)
and 10~hPa geopotential height ($Z_{10}$) respectively. They are found to be
$q_b = 460$~PVU ($\mathrm{1~PVU = 10^{-6}~Km^2kg^{-1}s^{-1}}$) and
$Z_b = 30.2$~km for the NH, and $q_b = 618$~PVU and $Z_b = 29.0$~km for the
SH. Using these values the moment diagnostics are calculated from ERA data for
1958--2009 using the method described in Section \ref{sec:vort-moment-diagn}.

As discussed in Section \ref{sec:vort-moment-diagn} the excess kurtosis
diagnostic is not used in the present analysis. In the interests of simplicity,
only the aspect ratio and centroid latitude diagnostics are used to identify
events, and the centroid longitude, orientation and equivalent area are not
used. The aspect ratio and centroid latitude are the most intuitive diagnostics
for this purpose, with a high aspect ratio and poleward centroid latitude
expected during split vortex events, and a low aspect ratio and equatorward
centroid latitude expected during displaced vortex events.

Figure \ref{fig:pv_z_moments_distribution} shows the distributions of these two
quantities calculated from $q_{850}$ and $Z_{10}$ for the NH vortex. The
centroid latitude distributions are almost identical, with a peak near
$80^{\circ}$N which is in agreement with previous studies
\citep{Waugh1999,Mitchell2011}. The aspect ratio distributions have a similar
shape, with a peak at about 1.3, but the PV based diagnostic has a larger
tail. This is because the PV field contains more small-scale filamentary
structures than geopotential height (e.g. Figures \ref{fig:displaced_ellipse}
and \ref{fig:split_ellipse}), making high aspect ratios more likely. As well as
having similar distributions, the time series of the PV and geopotential height
derived diagnostics (not shown) are significantly correlated, with correlation
coefficients of 0.9 for daily centroid latitude and 0.6 for aspect
ratio. Overall, these results suggest that geopotential height-derived moment
diagnostics are appropriate for the identification of split and displaced vortex
events.


\begin{figure}
  \centering
  \includegraphics[width=\textwidth]{figures/chapter-moments/moments_distribution_crop.pdf}
  \caption[NH distributions of $Z_{10}$ and PV-based moment
  diagnostics]{Distributions of the December--March centroid latitude (a) and
    aspect ratio (b), of the NH stratospheric polar vortex over
    1958--2009. Diagnostics are calculated from geopotential height at 10~hPa
    ($Z_{10}$) and potential vorticity at 850~K (PV). Thresholds of
    66$^{\circ}$N in centroid latitude and 2.4 in aspect ratio are used to
    define events, and are indicated by the black vertical lines.}
  \label{fig:pv_z_moments_distribution}
\end{figure}

\begin{figure}
  \centering
  \includegraphics[width=\textwidth]{figures/chapter-moments/moments_distribution_crop_sh.pdf}
  \caption[SH distributions of $Z_{10}$ and PV-based moment diagnostics]{As
    Figure \ref{fig:pv_z_moments_distribution} but for moment diagnostics
    calculated for the SH stratospheric polar vortex over
    June--September.}
  \label{fig:pv_z_moments_distribution_sh}
\end{figure}

Figure \ref{fig:pv_z_moments_distribution_sh} shows the same distribution as
Figure \ref{fig:pv_z_moments_distribution}, but for the SH vortex aspect ratio
and centroid latitude. As in the case of the Northern Hemisphere, the
geopotential height and PV-based distributions have very similar shapes, with
the PV-based aspect ratio having a slightly larger tail. Comparing the Northern
and SH distributions it can be seen that there is much less variability in both
aspect ratio and centroid latitude in the Southern Hemisphere. This is because
of the reduced planetary wave propagation into the SH stratosphere, in turn a
result of lesser forcing from orography and land-sea temperature contrasts. The
peak in the Southern Hemisphere centroid latitude is at about $86^{\circ}$S; the
same as that found by \citet{Waugh1999}.

A result of this reduced SH vortex variability is that only one SSW has been
observed in the SH (discussed further in Chapter \ref{cha:seas}). The rest of
this chapter relates to the classification and impacts of split and displaced
vortex events and so focuses only on the NH. However, it should be noted that
all the methods below can also be applied to the SH.


\subsection{Event identification}
\label{sec:event-definition}

Previous attempts to identify SSW events have used a clustering method
\citep{K.Coughlin2009,Hannachi2010}. These methods attempt to classify the
vortex state for each day into a number of groups, which may be specified
beforehand or determined by the clustering algorithm. Individual days within the
same cluster should be physically similar, while those in different clusters
distinct. More precisely, clustering aims to maximise the between-cluster
variance while minimising the within-cluster variance. In the case of the
stratospheric polar vortex, clusters may represent, for instance, stable, split,
and displaced states. Events are then typically defined by the vortex persisting
in a particular cluster for a number of days.

\begin{figure}
 \centering
 \noindent\includegraphics[width=0.7\textwidth]{figures/chapter-moments/clustering.pdf}
 \caption[Clustering algorithms applied to the moment diagnostics]{Three
   clustering algorithms and a threshold division applied to the moment
   diagnostics in centroid latitude-aspect ratio space. For the $K$-means and
   hierarchical algorithms three clusters were specified. The mean-shift
   algorithm determined the number of clusters to be 4. Colours are used only to
   distinguish members of different clusters.}
 \label{fig:clusters}
\end{figure}

A large number of clustering algorithms exist, and some may be more appropriate
than others for certain uses. Here, three different algorithms are applied to
the moment diagnostics in centroid latitude-aspect ratio space, and their
outcomes shown in Figure \ref{fig:clusters}(a,b,c). Details of the three
algorithms are given below:

\bigskip\noindent\textbf{(a) \textit{K}-means} clustering requires the number of
clusters, $K$, to be specified beforehand (in Figure \ref{fig:clusters},
$K=3$). The algorithm begins by randomly selecting $K$ data points to be the
centroids of the initial clusters, all other data points are assigned to the
cluster with the nearest centroid. Having assigned the initial cluster
membership, the algorithm proceeds as follows:
\begin{enumerate}[1.]
\item Compute the centroids (the vector means), $\mathbf{\overline{x}}_k$ of
  each cluster.
\item Calculate the distance between the current data point, $\mathbf{x}_i$, and
  each of the $K$ $\mathbf{\overline{x}}_k$s. (Various distance measures can be
  used; in Figure \ref{fig:clusters}(a), the Euclidean distance is used).
\item If $\mathbf{x}_i$ is not in the group with the closest mean then reassign
  it to that group, otherwise repeat step 2 for $\mathbf{x}_{i+1}$.
\end{enumerate}
This is repeated until a full cycle through each $\mathbf{x}_i$ produces no
reassignments. An advantage of this method is that it is computationally
efficient, but the major disadvantage is that the number of clusters must be
pre-determined. Several methods exist to estimate the ideal number of clusters,
which generally have the aim of finding the best compromise between minimising
within-cluster variance and maximising between-cluster
variance. \citet{K.Coughlin2009} applied $K$-means clustering to several
variables representing the stratospheric polar vortex. They used the method of
\emph{silhouette values} \citep{Rousseeuw1987} to determine the ideal number of
clusters to be two (representing stable and disturbed vortex states). However,
three clusters has been imposed in Figure \ref{fig:clusters} in order to attempt
to identify stable, split, and displaced states.


\bigskip\noindent\textbf{(b) Mean-shift} clustering aims to discover `blobs' in
a data set. It works by updating candidates for centroids to be the mean of the
points within a given region. That is, given a candidate centroid $\mathbf{x}_i$
for iteration $t$, the candidate is updated according to
\begin{equation}
  \mathbf{x}^{t+1}_{i} = \mathbf{x}^t_i + \mathbf{m}(\mathbf{x}^t_i) \, ,
\end{equation}
where $\mathbf{m}$ is the mean shift vector. This is calculated as
\begin{equation}
  \mathbf{m}(\mathbf{x}_i) = \frac{\sum_{\mathbf{x}_j \in N(\mathbf{x}_i)}K(\mathbf{x}_j-\mathbf{x}_i)\mathbf{x}_j}{\sum_{\mathbf{x}_j
      \in N(\mathbf{x}_i)}K(\mathbf{x}_j-\mathbf{x}_i)} \, ,
\end{equation}
where $K(\mathbf{x}_j-\mathbf{x}_i)$ is a kernel function which determines the
weight of nearby points. Typically, and in Figure \ref{fig:clusters}(b), a
Gaussian kernel is used,
$K(\mathbf{x}_j-\mathbf{x}_i) = e^{-c\|\mathbf{x}_j-\mathbf{x}_i\|^2}$.
$N(\mathbf{x}_i)$ represents the set of points for which
$K(\mathbf{x}_i) \ne 0$. This shifting is repeated until $\mathbf{m}$
converges. Following this calculation, the candidates are then filtered to
remove near duplicates. The greatest advantage of this method is that it
automatically sets the number of clusters, so no prior assumptions about the
data set are required. A disadvantage is that it requires multiple nearest
neighbour searches during each iteration, and so may not be scalable to large
data sets. In Figure \ref{fig:clusters}(b) the number of clusters was determined
to be four (because the fourth cluster is very small it is not easily visible in
Figure \ref{fig:clusters}(b), but represents points with high aspect ratio).

\bigskip\noindent\textbf{(c) Hierarchical} clustering proceeds by calculating a
series of nested clusters. To begin with, all data points are considered each as
a separate cluster and then at each iteration the nearest two clusters are
merged. There are a number of methods to identify the distance between clusters
when those clusters consist of more than one member. Following
\citet{Hannachi2010} the \emph{complete-linkage} method is used here, defining
the distance as the largest distance between members in the two groups. As with
the $K$-means clustering, the number of clusters desired must be pre-determined,
otherwise the algorithm will run to completion with a single cluster consisting
of all data points. Again, many methods exist to determine the optimum number of
clusters. \citet{Hannachi2010} used the gap statistic method
\citep{Tibshirani2001} with vortex area, centroid latitude, and aspect ratio
moment diagnostics, and found a slight preference for three clusters. As such,
three clusters are used in Figure \ref{fig:clusters}(c).


\bigskip Figure \ref{fig:clusters} demonstrates that the three clustering
methods produce very different results. While the $K$-means clustering produces
clusters approximately representing stable (blue), split (green), and displaced
(red) vortex states, this distinction is less clear for other methods such as
mean-shift. As well as the size and extent of the clusters, there is also
disagreement between this and past studies on the optimum number of clusters;
\citet{K.Coughlin2009} found two clusters using a silhouette values method,
\citet{Hannachi2010} found three clusters using the gap statistic, while the
mean-shift algorithm applied here produces four clusters. Further sensitivity
tests were performed by randomly removing 1\% of the data and re-calculating the
clustering. It was found that very different clusters were calculated with this
small alteration to the data, suggesting that these clusterings may not be
robust if applied to different data sets, such as climate model simulations. The
likely reason for this sensitivity is that the data itself is not highly
clustered; as can be seen in Figure \ref{fig:pv_z_moments_distribution} no clear
bimodality is present. Rather, it is more appropriate to view the split and
displaced vortex states as the tails of a distribution rather than distinct
clusters or regimes. Consequently, no physical significance is ascribed to the
clusters displayed in Figure \ref{fig:clusters}.

For the reasons above, clustering methods are deemed inappropriate for the
present study, and a simpler, more robust, thresholds-based method is
introduced. Days with an aspect ratio $>2.4$ (11\% of all days) or a centroid
latitude $<66^{\circ}$N (5\% of all days) are classified as split and displaced
states respectively. A small number of days lie beyond both thresholds, and
these are classified as a mixed state (1\% of all days). The vast majority of
days (83\%) lie in the stable state, where neither threshold is exceeded. The
choice of thresholds is somewhat subjective but the results presented below are
not sensitive to the exact choice of threshold. They were chosen to give a
similar frequency of split and displaced vortex events (identified using the
method below) as CP07 and M13. 

\citet{Mitchell2011} found that above certain thresholds the aspect ratio and
centroid latitude follow a generalised Pareto distribution, which is used to
model extreme values \citep{Cole}. Both thresholds chosen here lie beyond these
extreme value thresholds of their respective distributions (these were found to
be 2.3 for aspect ratio and $72^{\circ}$N for centroid latitude). Some
theoretical motivation for the aspect ratio threshold can also be provided by
the theoretical stability of an idealised elliptical vortex. \citet{Love1893}
found that the Kirchoff ellipse (an elliptical patch of uniform vorticity in a
quiescent fluid) is linearly unstable if the aspect ratio exceeds 3. The aspect
ratio threshold of 2.4 used here lies below this limit, and so under this
idealised model it might expect that some split vortex events do not display a
full separation into two vortices.

Having classified each day into these four groups (split, displaced, mixed, and
stable), a persistence criterion is introduced in order to identify split and
displaced vortex \emph{events}. A displaced vortex event requires the centroid
latitude to remain equatorward of 66$^{\circ}$N for 7 days or more, while a
split vortex event requires the aspect ratio to remain higher than 2.4 for 7
days or more. A mixed event is identified if both thresholds are exceeded for 7
days or more. The onset date is defined as the day that the appropriate
threshold is first exceeded, and to ensure that no events are counted twice,
these onset dates are required to be spaced at least 30 days apart, chosen to
reflect radiative timescales in the lower stratosphere \citep{Newman1997}. Using
this method with geopotential height data, 17 displaced and 18 split vortex
events (listed in Table 3.1) are identified over the 52 winters, an average of 7
per decade (no mixed events were identified). This frequency lies between the
values of CP07 (6 per decade) and M13 (8 per decade). Although data is
restricted to DJFM in this analysis, no measures are taken to exclude early
final warmings which may occur in late March. This is motivated by the fact that
these are highly dynamically driven events which may have significant impacts on
the troposphere \citep{Hardiman2011}. The events defined here may therefore
include some which would traditionally be classed as final warmings (i.e. the
zonal-mean zonal wind does not return to westerly after the event). For this
reason, these events are not referred to as SSWs, but simply as split and
displaced vortex events.

A disadvantage of this method for classifying events (as well as the CP07 and
M13 methods) is that if events happen in quick succession (although not faster
than the 7 days required to identify an event), it will only identify the first
event. For example, if the vortex becomes displaced and then splits, the event
will be classed as a displacement, or (less plausibly) if the vortex splits then
reforms in a displaced location it will be classed a split. An extension of the
present method could therefore allow it to classify such sequences of events,
although in the interests of simplicity this is not carried out here. 

\begin{figure}[htbp]
 \centering
 \noindent\includegraphics[width=0.8\textwidth]{figures/chapter-moments/GPH_all_events_splits.pdf}
 \caption[Geopotential height at the peak of split vortex events]{10~hPa
   geopotential height at the peak of each of the 18 split vortex events
   identified in ERA. The peak is defined as the day with the largest aspect
   ratio during the two weeks following the onset date. The vortex edge is shown
   as a thin black contour, the equivalent ellipse the thick black contour and
   its centroid as a white cross. Data are transformed to Cartesian coordinates
   with a polar stereographic projection.}
 \label{fig:gph_all_events_splits}
\end{figure}

\begin{figure}[htbp]
 \centering
 \noindent\includegraphics[width=0.8\textwidth]{figures/chapter-moments/GPH_all_events_displs.pdf}
 \caption[Geopotential height at the peak of displaced vortex events]{As Figure
   \ref{fig:gph_all_events_splits} but for the 17 displaced vortex events
   identified in ERA.}
 \label{fig:gph_all_events_displs}
\end{figure}

Figures \ref{fig:gph_all_events_splits} and \ref{fig:gph_all_events_displs} show
geopotential height at the peak of each of the split and displaced vortex
events. The peak is defined as the day with the maximum aspect ratio or minimum
centroid latitude in the two weeks following the onset date of split and
displaced vortex events respectively. Almost all of the split vortex events show
two clearly separated vortices or a pinched vortex shape, which is approximately
symmetrical about the North Pole. Two exceptions are Figures
\ref{fig:gph_all_events_splits}(a) and (k), in which the vortex is highly
elliptical but not clearly split. Figure \ref{fig:gph_all_events_splits}(n)
shows an event with a highly elliptical vortex that is also somewhat displaced
from the pole, indicating that it has some displaced nature. The majority of
split vortex events are seen to occur along the $90^{\circ}$E-$90^{\circ}$W
axis, in line with the climatological wave-2 pattern \citep{Andrews1987}. Figure
\ref{fig:gph_all_events_splits}(h) shows an exception to this, with an
orientation orthogonal to the majority of events.

The displacement events mostly show a smaller and weaker vortex, owing to the
fact that they are more common later in winter (see Figure
\ref{fig:z_m13_cp07_histogram}). Some events, particularly those occurring in
late March, are also likely to be events which would traditionally be defined as
final warmings. It can be seem that the majority of displacement events occur in
the direction of the 0-90$^{\circ}$E quadrant, again in line with the
climatological wave-1 pattern. However, there are some exceptions to this, for
instance Figures \ref{fig:gph_all_events_displs} (c) and (i), which  show a
westward-displaced vortex. 



\subsection{Comparison with CP07 and M13}

\begin{table}
  \begin{centering}
    \begin{tabular}{llcrr}  \hline
    No. & Event onset & Event type & $\Delta \mathrm{T}_{10}$~(K) &
                                                                    $\overline{u}_{10}~(\mathrm{m~s^{-1}})$ \\ \hline
    1*  & 1961-3-9    & D          & 10.2       & 2.7 \\
    2*  & 1962-1-30   & S          & 1.9        & 38.9 \\
    3*  & 1962-3-7    & S          & -1.0       & 16.9 \\
    4*  & 1964-3-15   & D          & 11.9       & 1.3 \\
    5$\dagger$  & 1966-2-26   & D          & 2.5        & -5.9 \\
    6   & 1967-12-29  & S          & 13.0       & 19.4 \\
    7$\dagger$   & 1970-1-5    & S          & 8.5        & -4.0 \\
    8   & 1971-1-15   & S          & 10.8       & -1.7 \\
    9*  & 1972-2-4    & S          & -1.6       & 33.6 \\
    10$\dagger$  & 1973-2-4    & S          & 7.3        & -6.6 \\
    11* & 1974-3-12   & D          & 5.3        & -4.8 \\
    12* & 1975-3-16   & D          & 7.6        & -8.0 \\
    13* & 1976-3-31   & D          & 8.2        & -13.3 \\
    14$\dagger$  & 1977-1-7    & S          & 7.6        & -5.5 \\
    15*$\dagger$ & 1978-3-25   & D          & 2.5        & -9.3 \\
    16  & 1979-2-18   & S          & 5.6        & -0.4 \\
    17  & 1984-2-25   & D          & 11.6       & -4.4 \\
    18  & 1984-12-25  & S          & 15.0       & -1.7 \\
    19* & 1986-1-7    & S          & 3.4        & 29.9 \\
    20* & 1986-3-21   & D          & 9.1        & -12.2 \\
    21  & 1987-1-20   & D          & 8.3        & -7.7 \\
    22  & 1987-12-10  & S          & 9.8        & -3.0 \\
    23* & 1992-3-22   & D          & 7.6        & -4.4 \\
    24*$\dagger$ & 1995-2-2    & D          & 5.6        & 7.7 \\
    25  & 1998-12-15  & D          & 8.2        & 8.1 \\
    26  & 1999-2-24   & S          & 6.6        & -12.7 \\
    27  & 2001-2-7    & S          & 5.2        & -7.2 \\
    28* & 2001-3-15   & S          & -6.8       & 12.1 \\
    29* & 2002-3-21   & S          & -1.5       & 5.1 \\
    30  & 2003-1-17   & S          & 6.1        & 16.8 \\
    31  & 2004-1-2    & D          & 5.8        & -4.8 \\
    32* & 2005-3-11   & D          & 3.1        & -5.0 \\
    33  & 2006-1-17   & D          & 4.2        & -14.3 \\
    34  & 2008-2-18   & D          & 4.6        & 2.3 \\
    35  & 2009-1-18   & S          & 13.2       & 16.9 \\ \hline
    \end{tabular}
    \caption{A summary table of displaced (D) and split (S) vortex events,
      identified from 10~hPa geopotential height data from 1958--2009.
      $\Delta \mathrm{T}_{10}$ represents the mean area-weighted
      $50^{\circ}$-$90^{\circ}$N polar cap temperature anomaly at 10 hPa
      calculated 5 days either side of the event onset date. $\overline{u}_{10}$
      represents $\overline{u}$ at $60^{\circ}$N and 10~hPa averaged over the
      same period. Asterisks (*) represent those numbers that do not coincide
      (i.e. within 10 days and of the same type) with events defined by CP07 and
      daggers ($\dagger$) events which do not coincide with events of M13.}
  \end{centering}
  \label{tab:events}
\end{table}

The split and displaced vortex events identified using the above method are now
compared with those of the CP07 and M13 methods. Table 3.1 identifies those
events which do not coincide with the events of CP07 and M13, where `coincide'
indicates events within 10 days and of the same type. Of the 35 events
identified, 16 were found not to coincide with events of CP07 (10 displacement
and 6 split). Six events were found not to coincide with those of M13 (3
displacement and 3 split), although this comparison only covers the 28 events
from 1958-2002, as it was not possible to reproduce the M13 method over the
longer period studied here because of the difficulties with hierarchical
clustering discussed in Section \ref{sec:event-definition}. Just two completely
new events were identified (i.e. not coinciding with either CP07 or M13); these
are the displaced vortex events with onset dates 1978-3-25 and 1995-2-2.

Table 3.1 also shows polar cap averaged 10~hPa temperature anomalies
($\Delta \mathrm{T}_{10}$), averaged 5 days either side of the event to give a
measure of the event magnitude. The events of CP07 show a larger average anomaly
than events identified with the current method, although the two are not
statistically significantly different: CP07 average 8.6~K [6.1, 10.9] split and
7.8~K [5.5, 9.9] for displaced vortex events, while the current method averages
5.7~K [3.0, 8.3] for split and 6.8~K [5.5, 8.2] for displaced vortex events
(numbers in square brackets represent the 95\% uncertainty range, calculated
using a bootstrap test). It can be seen that while the vast majority of events
show positive values of $\Delta \mathrm{T}_{10}$ (i.e. warming), four events
show negative values. All of these events are also identified by M13, and they
attributed the negative values to the presence of a strong, cold vortex prior to
the event. Zonal-mean zonal wind at $60^{\circ}$N and 10~hPa
($\overline{u}_{10}$), averaged over the same period is also shown in Table
3.1. The majority of events show negative values, in line with the traditional
wind reversal criterion, although some show positive values. Again this results
from a strong vortex prior to these events, as well as the fact that the new
method detects some events with a distorted but strong vortex (seen in Figures
\ref{fig:gph_all_events_splits} and \ref{fig:gph_all_events_displs}).

% Mean values 
% -----------
% DT: Split: 5.7 pm 5.7 (3.0, 8.3)
%     Displ: 6.8 pm 2.9 (5.5, 8.2)
%     Split (CP07): 8.6 pm 4.6 (6.1, 10.9)
%     Displ (CP07): 7.8 pm 3.9 (5.5, 9.9)


The seasonal distribution of split and displaced vortex events identified by the
current method ($Z_{10}$), M13, and CP07, is shown in Figure
\ref{fig:z_m13_cp07_histogram}. In all three methods split vortex events are
more frequent in early-mid winter, with a peak in January. For displaced vortex
events, both the current method and M13 show a skew towards events occurring
later in winter. However, there is less similarity with the CP07 distribution of
displaced vortex events. CP07 indicates an approximately flat distribution
throughout winter, and many fewer displaced vortex events overall. It should be
noted that the seasonal distribution of split vortex events from the moment
based methods does not arise from the underlying climatology of aspect ratio,
which remains approximately constant throughout winter (e.g., Figure
\ref{fig:cmip5_moments_stats_seas}). The centroid latitude does however, show a
small equatorwards trend throughout winter, which may to some extent account for
the seasonal distribution of displaced vortex events \citep{Mitchell2011}.

\begin{figure}
  \centering
  \noindent\includegraphics[width=\textwidth]{figures/chapter-moments/splits_displacements_histogram.pdf}
  \caption[Seasonal distribution of displaced and split vortex
  events]{Histogram of the seasonal distribution of displaced and split vortex
    events, from the new geopotential height-based method ($Z_{10}$), M13 and
    CP07.}
  \label{fig:z_m13_cp07_histogram}
\end{figure}

\begin{figure}
 \centering
 \noindent\includegraphics[width=\textwidth]{figures/chapter-moments/pv_composites_colbar_crop.pdf}
 \caption[PV composites for split and displaced vortex events]{Composites of
   potential vorticity at the 850~K $\theta$ surface from the ERA reanalysis
   over 1958--2009. Composites are taken over the 5 days following the onset
   date of split vortex events (a,b,c) and displaced vortex events (d,e,f). The
   new ($Z_{10}$) method (a,b) is compared with that of M13 (b,e) and CP07
   (c,f).}
 \label{fig:pv_composites_m13_cp07}
\end{figure}

Figure \ref{fig:pv_composites_m13_cp07} compares the average shape of the
stratospheric polar vortex following the split and displaced vortex events
identified by the three methods. Composites of PV in the mid-stratosphere
(850~K) are shown averaged 5 days following each event. For the split vortex
events, the new method ($Z_{10}$) method clearly shows two separated
vortices, one centred over Canada and the other over Siberia. For the M13 events
the split vortex composite shows the vortex stretched across the same
$90^{\circ}$W-$90^{\circ}$E line, although not as clearly split, while the
composite for the CP07 events looks very different. This has a weak vortex
centred over Canada, with the other over Northern Europe in a similar location
to the composite for displaced events. All three composites for displaced events
show a vortex centred over Northern Europe, but this extends most westward in
the CP07 composite, suggesting that there may be some contamination from
misdiagnosed split vortex events.

Overall, Figure \ref{fig:pv_composites_m13_cp07} demonstrates that the new
method succeeds (in a composite sense) in identifying displaced and split vortex
events at least as well as the methods of M13 or CP07. When comparing the three
methods, CP07 is the clear outlier. This is most likely because the CP07
approach employs a zonal-mean threshold which cannot accurately capture some
extreme events. For instance, the CP07 method will not capture those events with
a distorted but strong vortex, that still have a westerly zonal mean
wind. Furthermore, the CP07 method defines displacements by default, if the
criterion for a split is not reached, rather than defining both explicitly as in
the present method. Further comparison between moment-based diagnostics and the
CP07 method is given in M13.


\subsection{Comparison with zonal wave amplitudes}
\label{sec:comp-zonal-wave}

Many studies have characterised stratospheric polar variability by its zonal
wave structure \citep[e.g,][]{Randel1988,Yoden1999,Nakagawa2006,Bancala2012}. It
is therefore instructive to compare this wave analysis with the moment
diagnostics developed above. Here this is carried out for the displaced and
split vortex events shown in Figures \ref{fig:displaced_ellipse} and
\ref{fig:split_ellipse} respectively. A similar comparison was shown by
\citet{Waugh1997} and \citet{Waugh1999}, and the results here are consistent
with their findings.

Zonal wavenumber decomposition is carried out by taking the Fourier transform of
the 60$^{\circ}$N, 10~hPa geopotential height field over all longitudes. The
amplitude of wave-$n$ on a given day is then given by the modulus of the $n$th
Fourier component on that day. In ERA data, the amplitude of DJFM wave-2 is, on
average, about 30\% that of wave-1, and wave-3 13\% of wave-1, indicating a
Charney-Drazin filtering of zonal wavenumbers, as discussed in Section
\ref{sec:planetary-waves}.

The correlation of geopotential height-derived daily aspect ratio and the wave-2
amplitude over DJFM is 0.30 and that of centroid latitude and wave-1 amplitude
is $-0.22$, both of which are statistically significant at the 99\%
level. Figure \ref{fig:wave_moments_ts} illustrates these relationships for the
example split and displaced vortex events previously shown in Figures
\ref{fig:displaced_ellipse} and \ref{fig:split_ellipse} . In the case of the
1979 split vortex event the wave-2 amplitude peaks approximately 5 days before
the peak of the aspect ratio. Wave-2 amplitude is also more variable than aspect
ratio in early winter, although the two are correlated at this time. In the case
of the 2006 displaced vortex event, the difference is even greater. The wave-1
amplitude peaks about three weeks before the centroid latitude, and is actually
anomalously small at the peak of centroid latitude. Before and after the event,
the wave-1 amplitude and centroid latitude are highly correlated.

A result of these differences is that not all split vortex events are defined as
wave-2 warmings and not all displaced vortex events as wave-1 warmings. For
example, the split vortex events with onset dates 1973-2-7, 1987-12-10, and
1999-2-24 are classified as wave-1 warmings by \citet{Bancala2012}. However, the
structure of the vortex appears clearly split in these three examples (see
Figures \ref{fig:gph_all_events_splits} (g), (l) and (m)), highlighting the
differences in these classifications.

The physical reasons for these differences are investigated in Figure
\ref{fig:wave_moments_vortex}. This compares the vortex structure at the peak
wave amplitude and peak aspect ratio/centroid latitude for the two events. For
the 1979 split vortex event, the vortex appears split at both times but there is
a greater separation between the vortices at the time of maximum aspect
ratio. The wave-2 amplitude is greatest at the earlier time, however, because
the vortices (particularly the eastern vortex) are stronger. Similarly for the
2006 event, at the time of maximum wave-1 amplitude the vortex is not displaced
far from its average position but its strength means that the wave-1 amplitude
is greater than the much more displaced vortex found later. Generally, the
sensitivity of wave amplitudes to vortex strength means that wave amplitudes may
actually decline during periods of intense wave breaking due to the weakening of
the vortex, even if that vortex is significantly distorted.

Overall, these results show that as the vortex departs from zonal symmetry
linear wave theory breaks down and changes in the wave-1 and wave-2 amplitudes
cannot be simply interpreted as changes in the position and elongation of the
vortex respectively.


\begin{figure}
 \centering
 \noindent\includegraphics[width=\textwidth]{figures/chapter-moments/waves_moments_ts.pdf}
 \caption[Wave amplitude and moment diagnostic time series]{(a) Aspect ratio
   (solid line) and wave-2 amplitude (dashed line) over the winter
   1978-1979. (b) Centroid latitude (solid line) and wave-1 amplitude (dashed
   line) over the winter 2005-2006. Centroid latitude is expressed as its
   deviation from the North Pole.}
 \label{fig:wave_moments_ts}
\end{figure}

\begin{figure}
 \centering
 \noindent\includegraphics[width=\textwidth]{figures/chapter-moments/wave_moment_vortices.pdf}
 \caption[Vortices at maximum of moment diagnostics and wave amplitudes]{10~hPa
   geopotential height on the day of maximum wave-2 amplitude (a) and maximum
   aspect ratio (b) for the 1979 split vortex event; and maximum wave-1
   amplitude (c) and minimum centroid latitude (d) for the 2006 displaced vortex
   event. The vortex edge contour, equivalent ellipse, and centroid latitude are
   shown as Figure \ref{fig:displaced_ellipse}.}
 \label{fig:wave_moments_vortex}
\end{figure}


\section{Stratosphere-troposphere coupling}
\label{sec:moments_analysis}

\subsection{Tropospheric response}

Having verified that this new method identifies split and displaced vortex
events as skillfully as previous methods, it is now possible to study their
influence on the troposphere. This is motivated by the result of M13 who, as
discussed in Section \ref{sec:observ-evid}, found tropospheric anomalies to be
larger following split vortex events that displaced vortex events. Figure
\ref{fig:dripping_paint}(a,b) shows time-height composites of the NAM over the
90 days following split and displaced vortex events. Here the method of
\citet{Baldwin2009} is used to define the NAM as the leading empirical
orthogonal function (EOF) of daily wintertime (November-April) zonal mean
geopotential height anomalies poleward of $20^{\circ}$N. The anomalies are
calculated by subtracting the seasonal cycle which has been smoothed with a
90-day low-pass filter. The daily NAM anomalies are then determined by
projecting daily geopotential anomalies onto the leading EOF patterns. Finally,
the NAM is normalised so that the time series at each level has unit variance.

\begin{figure}
 \centering
 \noindent\includegraphics[width=\textwidth]{figures/chapter-moments/dripping_paint.png}
 \caption[NAM composites for split and displaced vortex events]{Composites of
   the time-height evolution of the NAM during the (a) 18 split vortex events
   and (b) 17 displaced vortex events calculated from ERA data. (c) shows the
   difference in these composites, and hashed regions represent those that are
   95\% significant according to a two-tailed bootstrap test. Lag 0 is at the
   onset of an event as measured at 10 hPa. Contour intervals are 0.25 and the
   region between -0.25 and 0.25 is unshaded.}
 \label{fig:dripping_paint}
\end{figure}


In agreement with M13, it can be seen that the tropospheric NAM is more negative
during the 60 days following split vortex events than displaced vortex
events. Also similar to M13 is the fact the vertical evolution for the two
events greatly differs, with split vortex events occurring almost
instantaneously throughout the depth of the atmosphere and displaced vortex
taking almost two weeks to propagate through the stratosphere. The
near-barotropic nature of split vortex events suggests that resonant excitation
of the barotropic normal mode \citep{Esler2005} may be an important influence in
this case.
% Furthermore, the rapid onset of split vortex events is ... (Love 1893).

The difference in the NAM composites (split minus displaced) is shown in Figure
\ref{fig:dripping_paint}(c). Statistical significance of this difference is
calculated with the null hypothesis that there is no difference between the NAM
response to split and displaced vortex events, and assessed using a two tailed
bootstrap test with the following procedure:
\begin{enumerate}[i.]
\item The labels `split' and `displacement' are randomly re-assigned to the 35
  events.
\item NAM composites and the composite difference of these randomly assigned events
  are calculated. 
\item The above steps are repeated $10~000$ times, to form a distribution
  of random composite differences. If the true composite difference lies
  $<2.5\%$ or $>97.5\%$ within this distribution, then it can be said to be 95\%
  significant.
\end{enumerate}
Some significant differences are seen between the split and displaced vortex
composites. For instance, a more positive stratospheric NAM is seen to precede
displaced vortex events, while the dipole in the upper stratospheric and
tropospheric NAM near lag 0 represents the difference in baroclinicity of the
two types of event. Some regions of significant differences are seen in the
tropospheric NAM 0-60 days after the event, but there are also some regions that
are not significant. Care must be taken when interpreting the importance of
small significant regions these may arise by chance, even if no physical
relationship exists.

\begin{figure}
 \centering
 \noindent\includegraphics[width=0.6\textwidth]{figures/chapter-moments/nam_difference_sig.pdf}
 \caption[Significance test of surface NAM difference]{Distribution of 0-30 day
   mean surface NAM composite differences between split and displaced vortex
   events, formed by randomly shuffling the labels `split' and `displacement'
   between events. The 95\% significant region (according to a two-tailed test;
   i.e. $<2.5$\% and $>97.5$\%) is shaded and the true composite difference is
   at the 94th percentile.}
 \label{fig:nam_comp_diff}
\end{figure}

The difference in the tropospheric anomalies following split and displaced
vortex events can be tested more robustly by examining surface anomalies
averaged over the 30 days following onset. This difference is again tested using
the bootstrap procedure outlined above. The distribution of randomly calculated
surface NAM composite differences, and the actual surface NAM composite
difference are shown in Figure \ref{fig:nam_comp_diff}. It can be seen that the
true NAM difference does not lie in the 95\% significant region, so the null
hypothesis that there is no difference between surface NAM anomalies following
split and displaced vortex events cannot be rejected. It should be noted that
the statistical test here is different to that carried out by M13. They tested
whether the surface NAM following split and displaced vortex events were
different from randomly selected winter dates, finding that anomalies following
splits are, but those following displacements are not. They did not, however,
test the \emph{difference} between split and displaced vortex events.

\begin{figure}
 \centering
 \noindent\includegraphics[width=0.7\textwidth]{figures/chapter-moments/mslp_composites_colbar_crop.pdf}
 \caption[Mean sea-level pressure composites for split and displaced vortex
 events]{Composites of mean sea-level pressure anomalies in the 30 days before
   (a,b) and 30 days after (c,d) the onset dates of displaced (a,c) and split
   (b,d) vortex events, calculated from ERA data using the new ($Z_{10}$)
   method. Anomalies are calculated for each day and gridpoint from the
   climatology for that day of the year and gridpoint. Grey contours indicate
   regions of greater than 95\% statistical significance according to a
   bootstrap significance test.}
 \label{fig:mslp_composites}
\end{figure}

The surface NAM does not provide the full description of surface variability,
and so in Figure \ref{fig:mslp_composites} composites of MSLP 30 days before and
30 days following the onset dates of displaced and split vortex events are
presented. Statistical significance is calculated against the null hypothesis
that anomalies before and after split and displaced vortex events are
indistinguishable from other winter dates. This is again estimated from a
two-tailed bootstrap test, in which $10~000$ composites of equal size are formed
from randomly selected winter dates, and the percentile of the true composite
calculated from this distribution.

The strongest precursor is found for displaced vortex events, with a wave-1
pattern that is similar to the climatological stationary wave pattern
\citep[e.g.,][]{Garfinkel2008}, suggesting increased wave-1 propagation into the
stratosphere. However, the strongest anomalies following events occur after
split vortex events, with a pattern resembling the negative phase of the NAM,
though with a southern centre of action shifted towards Europe. A further
difference between the split and displaced vortex composites is that there is a
more negative MSLP anomaly over Scandinavia and Siberia following displaced
vortex event. Overall, the main features of Figure \ref{fig:mslp_composites}
compare very well with the corresponding diagnostics from M13.



\subsection{Tropopause response}

\begin{figure}
 \centering
 \noindent\includegraphics[width=\textwidth]{figures/chapter-moments/PV_430K.pdf}
 \caption[Composites of 430~K PV]{Composite of PV anomalies on the 430~K
   isentropic surface averaged over the 10 days following displaced (a) and
   split (b) vortex events. DJFM average of 430~K PV (c). In (c), units are PVU
   and the contour interval is 2~PVU. Data are restricted to the ERA-Interim
   period (1979--2009), meaning a total of 10 displaced and 10 split vortex
   events enter the composites. }
 \label{fig:430K_PV}
\end{figure}
 

\begin{figure}
 \centering
 \noindent\includegraphics[width=0.8\textwidth]{figures/chapter-moments/tropopause_height_composites_nam_crop.png}
 \caption[Tropopause height composites for split and displaced vortex
 events]{Composites of tropopause height anomalies averaged 10 days before
   (a,b), 10 days after (c,d) and 10-20 days after displaced and split vortex
   events (filled contours). Anomalies are calculated for each day and gridpoint
   from the climatology for that day of the year and gridpoint. Stippling
   indicates regions of greater than 95\% statistical significance according to
   a Monte-Carlo significance test. Grey contours indicate the first EOF of NH
   mean sea-level pressure, which explains 33\% of the variance (dashes
   represent negative values).}
 \label{fig:tropopause_height}
\end{figure}

The mechanism of the stratosphere's influence on the troposphere proposed by
\citet{Ambaum2002} states that changes in the PV near the tropopause affect the
tropopause height and induce tropospheric anomalies below (more details are
given in Section \ref{sec:mechanisms}). In order to investigate this mechanism,
composites of PV anomalies at the 430~K isentropic surface (which lies close to
100~hPa, just above the tropopause), over the 10 days following displaced and
split vortex events are shown in Figures \ref{fig:430K_PV}(a,b). Here,
composites are limited to the ERA-Interim (1979-2009) period, meaning 10 events
of each type enter the composites. The shorter 10-day period was chosen to
reflect the typical time scale for the split or displacement of the vortex,
rather than the longer time scale taken for the re-formation of the
vortex. However, composites taken over 30 days, as in Figure
\ref{fig:mslp_composites}, show similar structure but with reduced magnitude
(similarly, composites taken over 5 days, as in Figure
\ref{fig:pv_composites_m13_cp07}, show slightly increased magnitude). In the
displaced composite case a region of high PV is seen over Siberia and
Scandinavia, consistent with the movement of the vortex over this region. Note
that this is shifted further east than the position of the vortex at 850~K
($\sim 10$~hPa) (Figure \ref{fig:pv_composites_m13_cp07}(d)), again indicating
the more baroclinic nature of displaced vortex events (this westward tilt with
height was also found by \citet{Matthewman2009}). The split vortex composite
shows two regions of raised PV which are approximately co-located with the two
vortices at 850~K.

% In a modelling study, \citet{Hitchcock2013} argued that a stronger surface
% response seen following split vortex events could be attributed to larger
% lower-stratospheric anomalies following these events, rather than their
% spatial patterns. However, Figure \ref{fig:430K_PV} clearly shows that this is
% not the case; indeed the overall magnitude of anomalies appears greater
% folliwng displaced vortex events.

Again following the reasoning of \citet{Ambaum2002}, composites of tropopause
height averaged over the 10 days before, 0-10 days after, and 10-20 days after
split and displaced vortex events are now shown in Figure
\ref{fig:tropopause_height} (these composites now use the full ERA (1958-2009)
data set). The measure of tropopause height used is that of \citet{Wilcox2012},
who construct a blended thermal and dynamical tropopause. Significance is again
calculated using a two-tailed bootstrap test.

In line with the MSLP anomalies shown in Figure \ref{fig:mslp_composites},
tropopause height anomalies are seen to be larger prior to displaced vortex
events, with a wave-1-like structure. Following the events, tropopause height
anomalies are seen to approximately mirror the stratospheric PV anomalies
(Figure \ref{fig:430K_PV}). That is, following displaced vortex events an
elevated tropopause is seen over Europe and Scandinavia, with a lowered
tropopause over Canada, and following split vortex events two regions of
elevated tropopause are present over Canada and Siberia with a depression in
between.

It is possible to quantitatively examine (although only approximately) whether
these tropopause anomalies are consistent with the changes in stratospheric PV
above. Changes in tropopause pressure, $\Delta p_{\mathrm{trop}}$, are related
to changes in stratospheric PV, $\Delta q$, through
\begin{equation}
\Delta q \approx -q(1+\mathrm{Bu})\frac{\Delta
  p_{\mathrm{trop}}}{p_{\mathrm{trop}}} \, , 
\label{eqn:pv_trop}
\end{equation}
where $\mathrm{Bu}$ is the Burger number, which is approximately equal to one
under the QG approximation \citep{Ambaum2002}. The change in tropopause height,
$\Delta h_{\mathrm{trop}}$ can be calculated using the hydrostatic relation
\begin{equation}
\Delta h_{\mathrm{trop}} = -\frac{\Delta p_{\mathrm{trop}}}{p_{\mathrm{trop}}}
\frac{R T_{\mathrm{trop}}}{g} \, , 
\end{equation} 
where $T_{\mathrm{trop}}$ is the tropopause temperature. Hence
\begin{equation}
\Delta h_{\mathrm{trop}} = \frac{\Delta q}{q}
\frac{RT_{\mathrm{trop}}}{g(1+\mathrm{Bu})} \, .
\end{equation}
From Figures \ref{fig:430K_PV}(a,b) a typical 430~K PV anomaly is 2~PVU, and the
background climatology is approximately 20~PVU (Figure \ref{fig:430K_PV}(c)), so
$\Delta q/q \approx 0.1$. With a typical value of
$T_{\mathrm{trop}} = 210~\mathrm{K}$, this then gives a change of tropopause
height of $\Delta h_{\mathrm{trop}} \approx 300~\mathrm{m}$, which is indeed
approximately in line with the tropopause height anomalies seen in Figure
\ref{fig:tropopause_height}. This, along with the fact that the pattern in
tropopause height anomalies approximately mirrors that of stratospheric PV
anomalies, suggests that these tropopause height anomalies are induced by
changes in stratospheric PV above.

Also shown in Figure \ref{fig:tropopause_height} is the surface NAM pattern (the
leading EOF of DJFM daily MSLP). It can be seen that following split vortex
events more than displaced (especially days 0-10), the negative tropopause
height north of Iceland aligns more closely with the minimum in the NAM (this
region is also a node of the NAO). This may be significant if it is expected
that the fluctuation-dissipation theorem (FDT) \citep{Nyquist1928} holds in the
tropospheric response to stratospheric forcing. For systems in which the FDT
holds (which relies on a small applied forcing), the response of a system
projected on a mode of variability should linearly scale with the projection of
the forcing on that mode \citep{Ring2008}. Under the assumption that the
tropopause height perturbation represents the ``forcing'', appears to project
more strongly on the NAM/NAO following split vortex events\footnote{This
  stronger projection can also be seen by considering the fact that the overall
  magnitude of tropopause height anomalies in Figures
  \ref{fig:tropopause_height}(c) and (d) are similar, but the
  lower-stratospheric NAM in Figure \ref{fig:dripping_paint} is much larger over
  the 0--10 days following split vortex events than displaced vortex events.},
consistent with a greater surface response to these events. However, the pattern
correlations between the split and displaced vortex tropopause height anomalies
and the NAM are not statistically significantly different because the tropopause
height field is very noisy. In order to give a more detailed analysis a greater
number of events would be needed.


\section{Conclusions}

Recent research has demonstrated the need to distinguish between split and
displaced stratospheric polar vortex events because of their different dynamics
and impacts on the troposphere. However, previous methods to identify these
events are impractical for application to climate model or seasonal prediction
simulations because they are highly sensitive to model climatology or rely on
non-standard variables. Motivated by this, we have developed a new method to
identify displaced and split vortex events which requires only geopotential
height at 10~hPa. The method is summarised as follows:
\begin{enumerate}[i.]
\item To identify the vortex region, a single contour of 10~hPa geopotential
  height is selected. This is the value of the DJFM mean zonal-mean at
  $60^{\circ}$N.
\item Using this contour the centroid latitude and aspect ratio moment
  diagnostics can be calculated.
\item Events are identified using a threshold criterion: Displaced events are
  said to occur if the centroid latitude remains equatorward $66^{\circ}$N for 7
  days or more. Split events are said to occur if the aspect ratio remains above
  2.4 for 7 days or more. In order to ensure that events are not counted twice,
  no two events may occur within 30 days.
\end{enumerate}
Results show that vortex moment diagnostics derived from geopotential height in
this way are highly correlated with those derived from PV, although fewer high
aspect ratio values are seen. The use of geopotential height here is motivated
by the fact that it is commonly output by climate models, whereas PV is
not. However, in cases where PV is available (such as in reanalyses) its use is
preferable because of its quasi-conservative properties and smaller-scale
features. The above method can be easily adapted for use with PV-based vortex
moments.

Analysis of the stratosphere following events identified by this method
demonstrates that it is able to accurately identify split and displaced vortex
events. Most of the events identified coincide with those of M13, and about
half with events identified by CP07. Composite analysis indicates that the
position of the stratospheric polar vortex following these events is at least as
extreme as that from the previous methods. 

Having identified these events, their impact on the troposphere has been
investigated. Composites of the NAM indicate a more negative surface NAM over
the month following split vortex events than following displaced vortex
events. This supports the finding of M13, using a different event identification
method and extended data set. However, using a bootstrap test the composite
\emph{difference} of the surface NAM is not found to be statistically
significant.

Anomalies of tropopause height following split and displaced vortex events are
found to be co-located with lower-stratospheric PV anomalies. They are also of a
magnitude consistent with being induced by changes in the stratospheric polar
vortex. Surface anomalies induced by changes in tropopause height may therefore
explain the different surface anomalies following split and displaced vortex
events. However, it is not possible to draw firm conclusions on this because of
the relatively small number of events and the noise of the MSLP and tropopause
height fields.

Overall, statistically significant results about the difference in the
tropospheric response to split and displaced vortex events will require a larger
number of events. This is achieved through the analysis of climate model
simulations in the next chapter.  





%%% Local Variables:
%%% mode: latex
%%% TeX-master: "thesis"
%%% End:







