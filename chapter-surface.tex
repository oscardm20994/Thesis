\chapter{Interactions between the Stratospheric Polar Vortex and Atlantic circulation} 
\begin{quotation}
  Much of the work contained in this chapter is based upon xxx,
  published in \emph{Journal of Climate}.
\end{quotation}

\label{cha:surface}

\section{Introduction}
Chapter 3 demonstrated that the appearance of SSWs, specifically intervals of hiatus in events, vary on multi-decadal timescales in a pi-control simulation of UKESM. Given the well studied downward influence of vortex variations over the Atlantic sector (see section \ref{sec:Downward_influence}), a natural question that follows from this finding is to what extent do these multidecadal signals influence modes in tropospheric and surface climate? The majority of studies that have examined the associations between the stratospheric polar vortex and surface anomalies have considered the in-season impact of individual events (notably \cite{baldwinStratospheric2001a}), however the Atlantic region exhibits key modes of multi-decadal variation, namely the the AMOC (section \ref{sec:multi-decadal_background}) which is in turn key a key driver of climate variability (e.g. \cite{frankignoulInfluence2013b}). 

%The effect of decadal to multi-decadal scale variation in the vortex on the surface has been considered in previous works, but its true nature is not fully understood. \cite{manziniStratospheretroposphere2012} examines decadal fluctuations in SSW events in a 260 year prescribed SST simulation of a GCM and analyse impacts of these signals at the surface. They show that decadal vortex variability excites similar timescale variations in surface temperature and sea ice coverage between Greenland and Norway over the Atlantic sector. They propose this connection to be indicative of a delayed response of the AMOC to stratospheric forcing via the NAO which  subsequently influences northward Atlantic heat transfer and sea ice melt rates as well as surface temperatures anomalies. \cite{haaseImportance2018} analyse the in-season impact of SSW events on the strength of deep convection in the North Atlantic that occurs through the impact of SSWs on the NAO using a model (CESM1 WACCM). The study notes the presence of an anomalously shallow  mixed layer depth in the Labrador Sea following an SSW event. 

An association between stratospheric polar vortex variability and the AMOC on decadal timescales has been previously investigated \citep{reichlerStratospheric2012b, schimankeMultidecadal2011b} but the mechanism of its influence remains unclear. For example, \cite{reichlerStratospheric2012b} examine the response of the AMOC to strong and weak polar vortex events and show a lagged, oscillatory response in the AMOC. They propose a pathway involving alterations of wind stress and ocean-atmosphere heat flux anomalies in the West Atlantic due to the changed NAO patterns following the vortex events. The effect is prominent in a pi-cntrl simulation of a single model (GFDL-CM2.1) and to some extent in a suite of CMIP5 models. An impact of long-term changes in the NAO on the strength of the AMOC is supported by a number of studies \citep{visbeckOcean1998b, delworthImplications2000b, delworthMultidecadal2000b, edenMechanism2001b, lohmannResponse2009, robsonCauses2012c}. Most recently \cite{delworthImpact2016c} used a set of idealised GCM experiments in which they impose a perpetual ocean-atmosphere heat flux pattern associated with different NAO phases. They find significantly different AMOC mean states depending on the imposed pattern (a stronger AMOC under positive NAO flux conditions than a control simulation). \cite{haaseImportance2018b} also analyse the in-season influence of SSW events on the NAO and ocean-atmosphere heat fluxes that then impacts the strength of deep convection in the North Atlantic, using the CESM1 WACCM model. The study notes the presence of an anomalously shallow  mixed layer depth in the Labrador Sea following an SSW event. 

A key result from \cite{reichlerStratospheric2012b} is the decadal modulation of the vortex-AMOC covariability. However, decadal to multi-decadal variability in vortex strength is not well understood. Some studies have focused on potential polar vortex impacts at the surface but suffer from low statistical significance due to the short observational record. \cite{garfinkelStratospheric2017b}, \cite{garfinkelEffect2015b} and \cite{cohenDecadal2009b} link decadal fluctuations in vortex strength with modulation of the global warming signal in Eurasian surface temperature in both reanalyses and CMIP3 models. \cite{schimankeMultidecadal2011b} demonstrate multi-decadal signals in SSW occurrence in a multi-century GCM simulation and propose an influence of these signals on similar period variability in Eurasian snow cover and Atlantic SSTs. Results from this study are difficult to interpret however, given that the GCM used (EGMAM: ECHO‐G with Middle Atmosphere Model) exhibits significant bias in its vortex representation with a mean SSW rate of only 2 events per decade compared to 6 events per decade in most reanalyses \citep{ayarzaguenaRepresentation2019a}. They note that repeating their study with a more advanced model is required to corroborate their findings. \cite{manziniStratospheretroposphere2012b} examined decadal fluctuations in SSW events in a 260 year prescribed SST simulation of a GCM and analyse impacts of these signals at the surface. They show that decadal vortex variability excites similar timescale variations in surface temperature and sea ice coverage between Greenland and Norway over the Atlantic sector. They propose this connection to be indicative of a delayed response of the AMOC to stratospheric forcing via the NAO which  subsequently influences northward Atlantic heat transfer and sea ice melt rates as well as surface temperatures anomalies. 

While a degree of connection between the vortex and AMOC has been proposed in works outlined above, aspects of the interaction evade definitive explanation. Namely, the true mechanisms as well as the timescales involved. In this chapter we utilise the same 1000-yr pi-control simulation as chapter 3, to extend the analysis there and examine the links between long-term vortex variability and the North Atlantic including oceanic modes. We first examine the near-surface in-season response to extreme vortex events to demonstrate that the model is able to reproduce corresponding anomalies in mean sea level pressure (MSLP), ocean-atmosphere heat flux and SSTs consistent with previous studies. We then study The interactions and feedback mechanisms between the stratosphere, troposphere and surface on different timescales using a combination of lag/lead composite analysis as well the wavelet spectral decomposition method employed in chapter 3. The amplitude and lag of the AMOC response to a multi-year interval of strong polar vortex in the model is determined. This is then used to estimate the potential contribution of the observed 8 consecutive years with a very stable, undisturbed stratospheric vortex in the 1990s \citep{pawsonCold1999} to the recent observed negative trend in the strength of the AMOC.

\section{Data And Methods}
\subsection{Model Configuration and Observation Data}

We analyse output from the same GCM simulation as presented in chapter 3 - the pi-control simulation of UKESM (see section \ref{sec:model_config}). As in chapter 3, we choose the pi-control due to the length of integration relative to the timescales we wish to consider and to analyse the internal variability of the system in this model. 

To estimate the contribution of stratospheric variations to recent observed AMOC trends we also make use of observation based datasets of the atmosphere and oceans. First, we utilise reanalysis data from the ECMWF: ERA5 \citep{hersbachERA52020c} for observation based geopotential height (GPH) fields. Similar to its predecessor, ERA-Interim (section \ref{sec:ERA_data}), ERA5 consists of a set of observations assimilated through 4D-var using ECMWF Integrated Forecast System (IFS). However, ERA5 utilises the latest version of the IFS which operates at a greater horizontal grid resolution to ERA-Interim (31km vs 80km) as well as a higher vertical resolution and maximum height (60 levels up to 10hPa vs 137 levels to 1 hPa) \citep{hersbachERA52020c}. Most relevant for stratospheric representation is the inclusion of an improved gravity wave parameterisation scheme in the latest IFS, a model which exhibits marginal improvement in the predictability of SSW events compared to its predecessor \citep{orrImproved2010a}. Despite these improvements in the underlying model, \cite{hersbachERA52020c} reports no significant improvement in stratospheric representation between ERA5 and ERA-Interim due to a combination of IFS model bias in the middle to upper stratosphere and sparse observations in the same region. We choose to utilise ERA5 as opposed to ERA-Interim for analysis in this chapter as we consider observations of the AMOC and the vortex up to near present day. ERA5 provides data up till present day while ERA-Interim has been discontinued as of 2019 so ERA5 is a preferable dataset in this case. 
Second, we use the Rapid Array Data set which provides time-depth profiles for the meridional overturning mass streamfunction in the Atlantic region at 26$^{\circ}$N \citep{moatAtlantic2020c}. These data are measured through a combination of ocean mooring, ship based, satellite and submarine telephone cable observations to estimate the strength of primary contributions to the meridional overturning circulation: Ekman transport (through wind stress), transport through the Florida Straits and transport driven by EW density gradients between the American and African continents \citep{mccarthyMeasuring2015b}.

\subsection{Model Diagnostics}
\label{sec:model_diagnostics_surface}
We utilise the NAM (section \ref{sec:NAM}) as a metric for the strength of the vortex as used by \cite{baldwinStratospheric2001a} as well as numerous subsequent studies. The NAM is defined as the 1st principle component of the zonal mean, deseasonalised geopotential height field evaluated at latitudes north of $20^{\circ}N$ over the NH winter season (Dec-Mar) on a given pressure level. To measure the vortex strength we evaluate the NAM on a level in close proximity to the vortex edge, 10hPa, and the resulting index is henceforth known as NAM$_{10}$. We choose to utilise a continuous vortex metric for this work as opposed to the event based measured used in chapter 3 as the NAM captures both types of anomalous vortex behaviour (strong and weak). Cross spectral analysis of the SSW time-series and the QBO (figure \ref{fig:SSW_low_rate_QBO}) suggested an important role of SSW hiatuses in multi-decadal modes of variation. As a result, a continuous metric that distinguishes between a weak, strong and neutral vortex is highly useful for this analysis. An individual vortex event (either strong or weak) is recorded when the daily NAM$_{10}$ crosses +1.5 (strong) or -2 (weak). The day on which this reversal occurs is referred to as the central date. After this date, the NAM$_10$ must recover to neutral (between -2 and 1.5) for a period of at least 10 consecutive days (which is the approximate radiative timescale of the mid-stratosphere) before another event can be recorded. The strong threshold value for events is chosen in accordance with the methodology of \cite{baldwinStratospheric2001a} and the weak threshold selected such that it results in approximately the same rate of weak events (SSWs) as is reported in chapter 3 (figure \ref{fig:SSW_histogram}) using the same simulation of UKESM but with a zonal wind definition of SSWs (0.54 events/winter).

We also use the NAM$_{10}$ to derive an index for the appearance of intervals of consecutive winters which show persistent vortex behaviour. The persistent NAM$_{10}$ interval index is defined as follows: First, the NAM$_{10}$ is averaged over each NH winter season (December-March), this gives a measure of mean vortex strength for each winter. Second, this index is smoothed using a Gaussian filter which is carried out through a convolution of the time-series with a 1D Gaussian kernel in the time domain given by

\begin{equation} \label{Gaussian_filter}
f(t, \sigma) = \frac{1}{\sqrt{2 \pi \sigma^2}} e^{-\frac{1}{2}\big(\frac{t}{\sigma}\big)^2}
\end{equation}

where $\sigma$ is the standard deviation of the distribution defined by the kernel. We choose $\sigma$ = 2 years following the method of \cite{reichlerStratospheric2012} and as a method analogous to the 5 year smoothing applied to an SSW timeseries in chapter 3. The selection of $\sigma = 2$ years allows contributions to the smoothed value from values approximately 7 years either side of the central year as the value of the Gaussian window decays to near 0 approximately $3.5\sigma$ from its mean. However the largest contributions come from 3-4 years either side of the central year. This allows the smoothing to capture instances of $\sim 6-8$ consecutive years with persistent vortex behaviour, a similar length to intervals observed in reanalysis (e.g. the 1990s, \cite{pawsonCold1999b}). 

We subsequently define persistent NAM$_{10}$ intervals, when the vortex exhibits the same type of behaviour for a number of consecutive years, using extreme values of the smoothed NAM$_{10}$ index. A persistent NAM$_{10}$ interval is recorded when the smoothed NAM$_{10}$ index value falls within the top 5 percentile values. Once such an interval occurs, another cannot be recorded for 15 years after to avoid choosing multiple central years within the same interval. Using 5 percentile values gives approximately the same rate of persistent vortex intervals as is reported in \cite{reichlerStratospheric2012b} so we proceed with this threshold throughout for a direct comparison with this study. Tests were also carried out to assess the sensitivity of our results to this threshold and are reported in section \ref{sec:sensitivity}.

We define an AMOC index following the procedure in \cite{reichlerStratospheric2012b}. The AMOC is defined using the overturning streamfunction field in the Atlantic sector. At each time point the AMOC index is the maximum streamfunction value at any depth at a chosen latitude. We evaluate the index at 30N, 45N and 50N and measure the response and co-variability with the NAM$_{10}$ timeseries and other climate indices defined below. We derive the observed AMOC index from the Rapid Array data as the maximum MOC at each time point at 26N. We also utilise a definition of the North Atlantic Oscillation from \cite{hurrellNorth2003c}. The NAO index is defined as the 1st PC of the Dec-Mar MSLP in the region $20^{\circ}-80^{\circ}N, 90^{\circ}W-40^{\circ}E$. The PC is calculated by taking the first EOF of deseasonalised MSLP anomalies and projecting this EOF onto the anomaly field to produce the 1st PC. We additionally derive an Ocean-Atmosphere heat flux field defined as the sum of latent and sensible heat fluxes between the ocean surface and the atmosphere (i.e. positive values indicate exchange of heat from the ocean to the atmosphere). We derive an index for the occurrence of deep convection anomalies in the equatorial eastern Pacific region. This index is defined by the top of atmosphere outgoing longwave radiation (OLR) averaged over the box 10$^{\circ}$\,S–10$^{\circ}$\,N, 240$^{\circ}$\,–290$^{\circ}$\,E. The OLR field is utilised as it acts as a proxy for the occurrence of convection anomalies - When deep convection is enhanced, cloud top height is increased and therefore OLR is reduced. The East Pacific box is selected following a sensitivity analysis to establish the region which exhibits 90 year timescales variations in OLR. It is also a similar region to studies which consider east pacific ENSO patterns which is identified as a separate mode of variability to the traditionally used central pacific ENSO region \citep{johnsonHow2013f}.

Finally, we utilise the same QBO metric as in the wavelet analysis of chapter 3 (see section \ref{sec:model_diagnostics}). This is defined as the Zonal Mean Zonal Wind (ZMZW) averaged over the latitude range 5$^{\circ}$\,S–5$^{\circ}$\,N,  and the pressure level range 15-30hPa. As with chapter 3 we also consider the Hilbert amplitude of the deep QBO as an instantaneous amplitude metric (see equations \ref{eq:hilbert1}-\ref{eq:hilbert3}). 

Finally, We analyse the relationship between the magnitude of smoothed stratospheric NAM$_{10}$ extremes and 17-year lagged AMOC anomalies using linear regression with a single predictor (the lagged AMOC). We analyse the strength of linear relationship using a correlation coefficient, $r$, and estimate a significance level for this value using a bootstrapping which assesses the probability such a value results if the phases in signals in the NAM$_{10}$ and AMOC are randomly assigned but the overall autocorrelation structure is retained. We do this by comparing the $r$ value calculated with real data with those produced from a set of synthetic NAM$_{10}$ series. These synthetic data are generated by Fourier transforming the smoothed NAM$_{10}$ index, randomly shuffling the Fourier phases and subsequently inverse Fourier transforming to generate a surrogate timeseries with the same Fourier power spectrum as the real data. Repeating this data generation and calculating the correlation between the magnitude of positive extremes in the surrogate NAM$_{10}$s and the 17 year lagged AMOC builds up a PDF for the $r$ value which can be used to estimate the significance level for a real $r$ value.


\section{In-season surface responses to anomalous vortex events}
We begin by diagnosing the in-season response to anomalous vortex events exhibited by surface variability in the model to assess its suitability for studying interactions on longer timescales. Figure \ref{fig:surface_comp_all} shows the mean sea level pressure (MSLP) composite differences between strong and weak polar vortex years (figure \ref{fig:surface_comp_all}, top row). The composites have been determined by selecting MSLP values associated with events in which the daily NAM$_{10}$ values cross the +1.5 (strong) or -2 (weak) threshold (see section \ref{sec:model_diagnostics_surface}). The composite differences demonstrate a significant lagged MSLP response, with strong (weak) vortex years  corresponding to a positive (negative) NAO pattern, in agreement with previous model and observational studies (see section 1). The NAO anomalies peak in magnitude at a lag of 1-2 months following the vortex anomalies with significant anomalies still visible for up to 3 months. There is also a weak positive NAO pattern that leads the vortex anomalies by up to one month ($-1 - 0$ month lags). This may be an indication that an anomalous NAO pattern is a precursor for an anomalous vortex, or it could be a response to the initiation of the vortex anomaly since this usually commences in the upper stratosphere and pre-dates the event's central date defined at 10hPa. Further exploration of this weak NAO signal is outside the scope of this paper. Additionally, a much stronger significant positive anomaly over the AL region is evident in the month leading up to the vortex anomaly consistent with previous work on this connection outlined in section \ref{sec:external_influence_AL}. We also analysed this like in chapter 3 and found a similar statistically significant relationship between the AL and the frequency of SSWs but the regression coefficients were small in comparison with the QBO influence (table \ref{table:regression_SSW}). Here the association between the AL and the vortex strength appears marginally stronger (r = 0.39 with the NAM) which may be due to the NAM's ability to capture both types of vortex anomalies (strong and weak). In chapter 3 we found that the AL exhibited minimal decadal to multi-decadal variability that was coherent with the same timescale variations in the vortex and for this reason the role of the AL is not considered in detail in the remainder of this chapter.

\begin{center}
\begin{figure}[h!]
\noindent\includegraphics[width = \linewidth]{Figures/Figures-surface/in_season_response_NAM_combined.png}
\caption[Surface patterns associated with anomalous winter stratospheric NAM$_{10}$ events.]{Surface patterns associated with anomalous winter stratospheric NAM$_{10}$ events. \textbf{Top row}: Monthly mean sea level pressure anomaly (hPa), \textbf{middle row}: Ocean-Atmosphere heat flux defined as the sum of latent and sensible heat fluxes ($wm^{-2}$)and \textbf{bottom row}: SSTs (K). Coloured shading shows where the composite differences between strong and weak NAM$_{10}$ events are statistically significant at the 95\% level under a 2 tailed students t-test. The title of each sub-figure indicates the month range relative to the central date of each NAM$_{10}$ anomaly. Signals at negative times indicate that the surface anomaly leads the stratospheric NAM$_{10}$ anomaly. Signals at positive  times indicate that the stratospheric NAM$_{10}$ anomaly leads the surface response.}
\label{fig:surface_comp_all}
\end{figure}
\end{center}

The model also exhibits significant responses in ocean-atmosphere heat flux (figure \ref{fig:surface_comp_all}, middle row). The largest flux anomalies are seen within 30 days (lag 0-1 month) and their spatial pattern resembles that of a North Atlantic tripole with positive anomalies over the subpolar North Atlantic between approximately 50$^\circ$-65$^\circ$N, negative anomalies off the east coast of the USA and a second positive anomaly off the coast of north east Africa. This pattern is consistent with the model response found by \cite{reichlerStratospheric2012b} to anomalous stratospheric NAM$_{10}$ events as well as the pattern associated with positive NAO phases in \cite{delworthImpact2016c}. As with the MSLP composites, there are visible anomalies 30 days leading up to the identified events (lag -1 - 0 months) both over the North Atlantic and Pacific regions. The Atlantic pattern may correspond to early responses to a disrupted or strengthened vortex as well as possible precursors to events. The Pacific anomalies preceding events are considerably smaller than the Atlantic anomalies and are concentrated over the AL region.

The SST response to anomalous stratospheric NAM$_{10}$ events (figure \ref{fig:surface_comp_all}, bottom row) over the North Atlantic lags behind the heat flux anomalies by around 2 months, with the largest amplitude anomalies at around 2-4 month lags. The anomaly pattern resembles that of the heat flux anomalies (with a change of sign), consistent with a mechanism in which the SSTs respond to the anomalous heat fluxes. A prominent negative tropical East Pacific anomaly is obvious in the months leading up to anomalous vortex events together with  anomalies that resemble the Pacific Decadal Oscillation (PDO) in the region of the Aleutian Low \citep{mantuaPacific1997a} and these features persists for several months. Variability in this region is dominated by El Ni\~{n}o Southern Oscillation (ENSO) variations and a significant body of work (e.g. \cite{domeisenTeleconnection2019d}) has proposed teleconnections between ENSO and vortex strength, consistent with the type of association exhibited here, i.e. negative (positive) SSTs or la Ni\~{n}a (el Ni\~{n}o) conditions associated with an anomalously strong (weak) vortex. 

%----- updated from paper manuscript to here.

\section{Surface impacts of persistent vortex anomalies}
\label{persistent}
The in-season anomaly patterns associated with anomalous stratospheric NAM$_{10}$ events shown in figure \ref{surface_comp_all} confirm that the model is able to reproduce the observed influence of vortex anomalies at the surface, particularly over the Atlantic region. We now extend the analysis to examine decadal-scale variability. Following the approach of \cite{reichlerStratospheric2012b} we smooth the NAM$_{10}$ index  (figure \ref{NAM_and_filtered}) and then select the upper and lower 5 percentiles of this index to identify intervals with a persistent consecutive strong or weak polar vortex (see section \ref{sec:model_diagnostics_surface} for more details). The red and blue dots on figure \ref{NAM_and_filtered} indicate the central year of intervals identified with persistent consecutive vortex anomalies (each dot represents the centre of intervals of approximately 8 years, see section \ref{sec:model_diagnostics_surface}). Characteristic surface responses associated with these intervals are then analysed by compiling composites surrounding the central year of each positive and negative interval at lags of -40 (before the intervals) and 40 (after intervals) years. Calculating the (positive minus negative) composite difference can then be used to assess the potential surface impacts to observed intervals of persistent consecutive vortex anomalies, such as the consecutive strong anomalies throughout most of the 1990s and the consecutive weak anomalies in the early 2000s. 
 

\begin{figure}[h!]
\begin{center}
\noindent\includegraphics[width = \linewidth]{Figures/Figures-surface/NAM_and_filtered.png} 
\caption[Time series of the December-March mean NAM$_{10}$ index and smoothed NAM$_{10}$]{Time series of the December-March mean NAM$_{10}$ index (green) level (\textbf{green}) and smoothed NAM$_{10}$ (purple) evaluated at the 10hPa. The smoothed series is calculated by applying a Gaussian filter ($\sigma = 2$ years) to the green series. Red and blue dots indicate the occurrence of persistent strong and weak vortex intervals respectively defined as extreme values (top and bottom 5 percentiles) in the filtered NAM$_{10}$ index. interval central years are selected such that at least 10 years lies between consecutive intervals.}
\label{NAM_and_filtered}
\end{center}
\end{figure}

A lead-lag analysis of composite differences in the AMOC strength at three different latitudes is shown in  (figure \ref{AMOC_comp_NAM}). The figure can be directly compared with figure 4c of \cite{reichlerStratospheric2012b} who suggest that decadal scale variability in vortex strength acts to amplify a similar timescale of variability in the AMOC through resonance of the two signals. Similar to that work, an oscillatory AMOC response to the stratospheric anomalies is evident here, with significant positive anomalies in the AMOC at 45N and 50N at lags of approximately 3-5 years after persistent NAM$_{10}$ intervals followed by negative anomalies at lags between 15 and 20 years. This response pattern is more clearly evident by taking the low pass filtered versions of the AMOC responses (figure \ref{AMOC_comp_NAM}b, d and f). Even after the high frequency signals have been removed, there are significant composite differences of up to 1.5Sv at 15-20 year lags. (We note, however, that this low pass filtering of the AMOC time-series reduces the overall variance so that the threshold for a composite difference to pass the significance test is lower, and this may increase the responses that are deemed significant shown in figure \ref{AMOC_comp_NAM}). 

Oscillatory response behaviour is not exhibited clearly by the AMOC at 30N, although negative anomalies at lags of 15-20 years after intervals are still visible. Response patterns at this latitude are also significantly smaller than those at 45N and 50N in the filtered composites (figure \ref{AMOC_comp_NAM}a and b). One possible explanation of this is that the coupling mechanism between the NAM$_{10}$ and AMOC may involve an AMOC response that originates at higher latitudes and then propagates equator-ward therefore exerting less forcing on the AMOC evaluated further south. \cite{zhangLatitudinal2010b} also note latitudinal differences in the AMOC response so these differences between 30N, 45N and 50N are not unexpected. The AMOC signals at 45N and 50N also exhibit significant positive anomalies preceding the persistent vortex intervals, at a lead of approximately 20 years and this is also present, albeit smaller in magnitude, in the low pass filtered indices. This precursor to persistent NAM$_{10}$ intervals is not found in corresponding results from \cite{reichlerStratospheric2012b} and the role of this feature is considered in more detail in section \ref{surface-strat_forcing}.

\begin{center}
\begin{figure}[h!]
\noindent\includegraphics[width = \linewidth]{Figures/Figures-surface/AMOC_responses_low_and_highf_combined_FINAL.png} 
\caption[Lagged response of the AMOC index to persistent NAM$_{10}$ intervals.]{Lagged response of the AMOC index to persistent NAM$_{10}$ intervals. Blue time-series shows AMOC composite difference values between positive and negative NAM$_{10}$ intervals defined in section \ref{sec:model_diagnostics_surface}. The $x$ axis denotes the lead (negative values) or lag (positive values) relative to the interval's central year. Black dots denote composite differences significant at the 95\% level under a 2-tailed student's t-test. Panels a, c and e shows monthly AMOC composites while b,d and f show smoothed AMOC composites using a Gaussian filter ($\sigma$ = 2 years).}
\label{AMOC_comp_NAM}
\end{figure}
\end{center}

We now examine this vortex-AMOC teleconnection in closer detail to explore possible physical pathways responsible for an AMOC response to persistent NAM$_{10}$ intervals. Figure \ref{NAO_AMOC_T_response}b shows that there is also an oscillatory response in the NAO. This consists of a positive zero-lag response difference (consistent with figure \ref{surface_comp_all}), a significant negative NAO anomaly between lags of 10-18 years followed by a positive NAO response at lags of around 28 years but with smaller amplitude than the zero-lag response. 

Also evident from figure \ref{NAO_AMOC_T_response}b is that the oscillatory responses of the NAO and AMOC are similar, with the NAO leading the AMOC response by 2-3 years. Both responses vary with periods of 28-30 years but, interestingly, the negative responses in both signals are larger and longer-lasting at 10-20 year lags than those at zero and 28-30 year lags. This difference in the magnitude and persistence suggests that the NAO and AMOC response patterns cannot be explained as a straightforward oscillatory response to the NAM$_{10}$ forcing at zero-lag. If this were the case then the response amplitude would be expected to decay with time and the negative response at 10-20 year lag would be smaller than the initial positive response.  Instead, a form of feedback mechanism is required to explain  the amplified 10-20 year lagged responses or a resonant mechanism as proposed in \cite{reichlerStratospheric2012b}. If such a feedback mechanism were present then one might also expect to see some oscillatory behaviour in the smoothed NAM$_{10}$ time-series, in response to the feedback from the surface. To investigate this, figure \ref{NAO_AMOC_T_response}a shows the corresponding lead-lag difference analysis for the NAM$_{10}$ index (i.e. the smoothed NAM$_{10}$ composited around its own extreme values). This supports the presence of a feedback mechanism since it also exhibits oscillatory behaviour with the same period of around 30 years. However, this oscillation is largely evident from the two positive peaks at zero-lag and 30yr-lag, with the latter being substantially damped. There is no significant response at lags of 10-20 years, where the NAO and AMOC responses were largest. This suggests that the negative NAO and AMOC responses at these lags are unlikely to be due to resonance with the NAM$_{10}$ signal. 


Alternatively, a possible physical pathway could involve an amplifying feedback mechanism between the NAO and AMOC responses. In this scenario, the positive zero-lag NAO response would drive a positive ocean-atmosphere heat flux anomaly over the sub-polar North Atlantic as seen in the response patterns to individual vortex events in figure \ref{surface_comp_all}. This heat flux anomaly would then lead to persistent negative anomalies in the near-surface ocean temperatures as heat is removed from the ocean via variations in wind stress and evaporation. The black line in figure \ref{NAO_AMOC_T_response} shows the lead-lag difference for ocean-atmosphere heat flux in the region encompassing the Subpolar North Atlantic (45$^{\circ}$\,–65$^{\circ}$\,N, 15$^{\circ}$\,–60$^{\circ}$\,W). This region was selected to encompass the region with the largest heat flux response to individual vortex events in figure \ref{surface_comp_all} (see middle row). Figure \ref{NAO_AMOC_T_response}c the corresponding depth profile of ocean temperature response from the same region. A positive heat flux response as well as an upper ocean (0-200m depth) cooling is evident at 0-1yr lags. This heat flux perturbation would in turn drive a positive AMOC anomaly at 2-3 year lags via changes in the mixed layer depth and deep convection in the sub-polar North Atlantic, an effect discussed in \cite{delworthInterdecadal1993b} and \cite{medhaugMechanisms2012b}. This increase in AMOC strength would subsequently increase the Labrador Sea temperature via poleward transport of heat. This is confirmed by the positive, deep (down to 2000m) ocean temperature anomaly at a lag of 10-20 years in figure \ref{NAO_AMOC_T_response}c. In turn, the reversal of the Labrador Sea temperatures can feedback onto the NAO (see e.g. \cite{frankignoulInfluence2013b}, inducing a negative NAO phase at 10 years lags as the increased Labrador Sea heat content alters the ocean-atmosphere heat fluxes in the same region. Finally, this switch in the NAO phase would lead to a subsequent negative AMOC anomaly via the same heat flux mechanism outlined above for an opposite NAO phase. This sequence of feedbacks would thus act to enhance the persistence and magnitude of the secondary extreme in the NAO and AMOC. \cite{reichlerStratospheric2012b} also briefly suggest a similar mechanism to account for the AMOC response in their simulations, but involving a negative feedback of the AMOC onto itself as well as a role for the NAO. 


\begin{figure}[h!]
\begin{center}
\noindent\includegraphics[width = 0.7\linewidth]{Figures/Figures-surface/Ocean_T_AMOC_NAO_responses_FINAL.png} 
\caption[Composite differences of NAO, heat flux and ocean temperature around extreme NAM$_{10}$ intervals.]{\textbf{a}: Composite differences of the smoothed NAM$_{10}$ index around extreme NAM$_{10}$ intervals (positive - negative intervals).
\textbf{b}: Like figure \ref{AMOC_comp_NAM}f for the AMOC at 50N (blue), the Dec-Mar mean NAO index and the ocean-atmosphere heat flux (sum of latent and sensible fluxes) averaged over an Atlantic box defined by 45$^{\circ}$–65$^{\circ}$N, 15$^{\circ}$–60$^{\circ}$W. All indices are smoothed with a Gaussian filter ($\sigma$ = 2 years). \textbf{c}: lagged responses of ocean temperature anomaly depth profiles to persistent NAM$_{10}$ intervals. Composite differences between strong and weak intervals are shown for the same Atlantic box as the heat flux index. Hatching indicates composite differences that are not significant at the 95\% level under a 2-tailed student's t-test.}
\label{NAO_AMOC_T_response}
\end{center}
\end{figure}

\section{Response to strong and weak vortex intervals}
So far, we have considered composite difference responses to persistent strong and weak  vortex intervals. However, we know that the vortex evolution during strong and weak vortex years is very different and this is likely to lead to differing interactions with surface and ocean variability. The surface responses to the two extremes are therefore unlikely to be equal and opposite. For example, weak vortex winters are mostly associated with SSWs whose impact at the surface is observed on average 0-60 days after their central date \citep{baldwinStratospheric2001a}. Furthermore,  the vortex often exhibits a pre-conditioned state (see e.g. \cite{charltonNew2007c} and \cite{bancalaPreconditioning2012b}) in which it becomes anomalously strong in the weeks running up to an SSW. So the timing of SSW events within a given season will dictate both the overall strength of the NAM$_{10}$ measured over the winter season (which we use to construct the persistent NAM$_{10}$ index) and the overall strength of the subsequent surface response. In contrast, winters that exhibit an anomalously strong vortex will, on average, exhibit such behaviour throughout the whole season so the impact on the surface will be present for a larger fraction of the winter season.

To assess the influence from each type of vortex extreme separately, figure \ref{NAO_AMOC_response_individual_types} shows the lead-lag composite analysis of the NAO, AMOC, NAM$_{10}$ and heat flux signals for the persistent positive (strong vortex) and persistent negative (weak vortex) NAM$_{10}$ intervals separately. The AMOC patterns associated with each NAM$_{10}$ type are slightly different. The persistently strong vortex composite shows clear oscillatory behaviour with a period of approximately 28 years. These patterns resemble many of the features observed in the AMOC composite differences (figure \ref{NAO_AMOC_T_response}d). Specifically, a positive AMOC anomaly is present at lags of 2-3 years, a negative AMOC anomaly at 15-20 years and a second positive anomaly at approximately 30 years. On the other hand, the persistently weak vortex composites exhibit a more complicated response, with double peaked minima at lags of -20 and -11 years and double-peaked maxima at 14 and 25 years. The weak vortex composites also exhibit no significant AMOC response at 2-3 year lags unlike the strong vortex composites. Both event types are associated with NAO anomalies at zero-lag but the response to strong vortex intervals is larger in magnitude, which is consistent with the larger 2-3 year lagged AMOC response to this event type. The 0-lag NAO responses is followed by an extreme of the opposite sign at approximately 16yr and 14yr lags for strong and weak intervals respectively. As with the AMOC composites, the NAO response to persistently strong vortex intervals exhibits a pronounced oscillatory behaviour of periods around 28 years. The corresponding NAM$_{10}$ analysis also shows oscillatory behaviour with periods of around 28 years. (We note that the NAM$_{10}$ results  show statistical significance  at both lead and lag times. However, the lead/lag interpretation is less meaningful in this case since the NAM$_{10}$ is used both as the signal and in the selection of the composites. The significance at both lead and lag times simply confirms that there is oscillatory behaviour). The double peaked behaviour of the AMOC associated with weak intervals is also reflected somewhat in the sub-polar North Atlantic heat flux response (figure \ref{NAO_AMOC_response_individual_types}c) with positive response peaks at approximately 11 and 20 year lags. There is also a 0-lag heat flux anomaly associated with weak vortex intervals corresponding to the negative NAO response but the corresponding response to strong intervals is not significant. 

The asymmetry between the AMOC and NAO responses to persistently strong and persistently weak vortex intervals and the complexity of the separate responses show that the interactions between the NAM$_{10}$ and these surface modes are complex, with some suggestion of oscillatory behaviour on different timescales. In the following sections we address these complexities in more detail by analysing the frequency spectra of the time-series and show that some of these complexities can be explained in terms of the non-stationarity of the signals.

\begin{figure}[h!]
\begin{center}
\noindent\includegraphics[width =0.6\linewidth]{Figures/Figures-surface/AMOC_NAO_NAM_responses_each_event_type_final.png} 
\caption[AMOC, NAO and heat flux responses to strong and weak NAM$_{10}$ intervals]{\textbf{a}: Composites of (a) Gaussian smoothed AMOC anomalies at 50N, (b) NAO, (c) NAM$_{10}$ and (d) subpolar NA heat flux associated with persistent vortex intervals of different types. On each sub-figure the red (blue) plots show the lead/lag responses to composites of strong (weak) persistent NAM$_{10}$ intervals. Solid dots denote composite anomalies are significant to the 95\% level under a 2 tailed student's t-test.}
\label{NAO_AMOC_response_individual_types}
\end{center}
\end{figure}

\section{Non-Stationary Variability}
Analysis in chapter 3 showed that variability of the stratospheric polar vortex  occurs on a range of timescales and is highly non-stationary  (figure \ref{fig:SSW_series_5yr_wavelet}). Although the composite analysis presented above shows oscillatory behaviour with periods of approximately 30-yr the results are likely complicated by the presence of  non-stationary variability at other periodicities.  We therefore analyse the frequency characteristics of the filtered NAM$_{10}$ index. Figure \ref{NAM_wavelet} shows the wavelet power spectrum of this index. It reveals variability on a range of timescales. As expected from the composite analyses, the spectrum exhibits intermittent power throughout the whole simulation at periods between 15 and 40 years. There is also significant power   corresponding to a period of approximately 90-100 years that persists for $\sim$300 years of the simulation (year numbers 520-820; approximately 3 cycles) as well as power at the 50 year period timescale that persists for 120 years (year numbers 500-620; approximately 2 cycles). 

We note that much of the $\sim$30-yr periodicity seen in the strong composite analysis is likely associated with significant wavelet power in the interval between years 300 and 410 because this interval displays the largest number of positive extremes in the smoothed NAM$_{10}$ time-series (compare the 30-yr wavelet power in figure 6 with  the red dots in figure 1). The large number of elevated NAM$_{10}$ extremes selected in this interval is also likely affected by the presence of extremely long timescales variability. Qualitative inspection of figure \ref{NAM_and_filtered} shows that the underlying NAM$_{10}$ amplitude increases from around year 200, reaches a peak at $\sim$year 380 and thereafter declines. This means that years within this interval are more likely to reach the top 5 percentile and qualify as an anomalously strong vortex. The origin of this multi-centennial variability is unclear and robust analysis of such a low frequency signal is difficult as the wavelet power of this multi-century variability is located mostly outside the so called "cone of influence" that marks the boundary at which edge effects become significant. 

\begin{figure}[h!]
\begin{center}
\noindent\includegraphics[width = 0.8\linewidth]{Figures/Figures-surface/NAM_wavelet_UKESM.png}
\caption[Wavelet power spectrum of smoothed NAM$_{10}$ index]{\textbf{Top left}: Dec-Mar NAM$_{10}$ values smoothed with a with a Gaussian filter ($\sigma$ = 2 years). \textbf{Bottom left}: Wavelet power spectrum of time series in top left. Hatching represents area outside the cone of influence in which edge effects are significant and power should not be considered. blue contours represent the 95\% confidence level assuming mean background AR1 red noise. \textbf{Top Right}: Morlet wavelet used for the wavelet transform in the time domain. \textbf{Bottom right:} Global power spectrum, the wavelet power averaged over the whole simulation (blue line), and global 95\% confidence spectrum (red dashed line).}
\label{NAM_wavelet}
\end{center}
\end{figure}

We also analyse the frequency characteristics of AMOC variability. Figure \ref{NAM_AMOC_Cross} shows the corresponding wavelet spectra of the AMOC at 50N and also the cross wavelet spectra between the filtered NAM$_{10}$ and the AMOC which gives a time varying measure of co-variability between the two indices at different periods. The wavelet spectrum for the AMOC at 50N exhibits a peak in spectral power corresponding to approximately 130 years that persists for nearly 400 years of the simulation (and also at longer periods up to 250 years but boundary effects are an issue at these multi-centennial timescales, as discussed above). There are also portions of significant power at approximately 30 year and 50 year periods, both of which persist for approximately 2 cycles ($\sim$60 years and $\sim$100 years respectively) which are also apparent in the global power spectrum. We also note that the main 30-yr power comes from the 300-400 year interval, coinciding with the interval of most activity in both the NAM$_{10}$ analysis (figure \ref{NAM_wavelet}) and the selected high NAM$_{10}$ percentiles (figure 1). 
 
The cross power spectrum between the filtered NAM$_{10}$ and the AMOC shows 3 distinct features corresponding broadly to the three timescales prominent in the individual spectra of both indices. Significant cross power is evident at 90-100 year periods for approximately 350 years (between 450 and 800 years; 3-4 cycles). The phase relationship between the signals (indicated by arrows on figure \ref{NAM_AMOC_Cross} within this portion of the cross spectrum  show a mixture of left-pointing arrows in the earlier portion, which indicate an anti-correlated relationship  ($\pi$ out of phase) and downward pointing arrows in the later portion that indicate a $\frac{\pi}{2}$ phase relationship. This latter can be interpreted in a number ways, with maxima in the AMOC leading to maxima in the NAM$_{10}$, minima in the NAM$_{10}$ leading to maxima in the AMOC or maxima in the NAM$_{10}$ index coinciding with maxima in the rate of change of the AMOC (see section \ref{sec:Wavelet_Analysis} for more details of the cross spectra arrows and how they are derived). There is also significant cross spectral power centred around 30 years (between 300-400 years; 3 cycles) and 50 years (between 500-600 years; 2 cycles). In contrast to the 90-yr periodicity, the phase arrows point to the right and slightly upwards, indicating that maxima in the NAM$_{10}$ index lead maxima in the AMOC by a small fraction of the cycle. This phase relationship is consistent with the composite analysis presented in figures \ref{AMOC_comp_NAM} and \ref{NAO_AMOC_response_individual_types} that indicated that a positive (negative) NAM$_{10}$ leads to a positive (negative) AMOC response approximately 2-3 years later. The wavelet spectra for the AMOC at 30N and 45N are provided in figures \ref{NAM_AMOC_Cross_30}a and \ref{NAM_AMOC_Cross_45}a. They show broadly the same features as that of the AMOC at 50N however it is notable that the AMOC at 30N does not exhibit significant variability on the 50 and 30 year timescales. This is reflected in the cross spectrum with the smoothed NAM$_{10}$ index which shows minimal cross power on these timescales (figure \ref{NAM_AMOC_Cross_30}b). 

\begin{center}
\begin{figure}[h!]
\noindent\includegraphics[width = \linewidth]{Figures/Figures-surface/AMOC_NAM_filtered_subplot.png}
\caption[Wavelet power spectrum of the AMOC at 50N and cross spectrum with the NAM$_{10}$ index]{\textbf{(a, top)}: AMOC time series at 50N, \textbf{a, bottom left}: Wavelet power spectrum (shaded contours represent wavelet power and yellow contours the 95\% significance level compared to an AR1 process), \textbf{a, bottom right}: global wavelet power spectrum (blue) and 95\% confidence level (dashed red). \textbf{b}: Cross spectra between Filtered Dec-Mar NAM$_{10}$ series and the AMOC index. \textbf{b, top}: NAM$_{10}$ and AMOC time series. \textbf{b, bottom}: Cross power spectrum. Shading indicates cross power, yellow contours the 95\% confidence interval and arrows the relative phase angle between signals in the time series (to the right: in phase, vertically upwards: $\frac{\pi}{2}$ out of phase with positive peaks in the NAM$_{10}$ leading those in the AMOC, to the left: $\pi$ out of phase, vertically downwards: $\frac{\pi}{2}$ out of phase with positive peaks in the AMOC leading those in the NAM).}
\label{NAM_AMOC_Cross}
\end{figure}
\end{center}

\begin{center}
\begin{figure}[h!]
\noindent\includegraphics[width = \linewidth]{Figures/Figures-surface/AMOC_NAM_filtered_subplot_30N.png}
\caption[Wavelet power spectrum of the AMOC at 30N and cross spectrum with the NAM$_{10}$ index]{like figure \ref{NAM_AMOC_Cross} for the AMOC at 30N. \textbf{a} shows the wavelet power spectrum of the AMOC and \textbf{b} the cross power spectrum between the AMOC and the NAM$_{10}$ index.}
\label{NAM_AMOC_Cross_30}
\end{figure}
\end{center}

\begin{center}
\begin{figure}[h!]
\noindent\includegraphics[width = \linewidth]{Figures/Figures-surface/AMOC_NAM_filtered_subplot_45N.png}
\caption[Wavelet power spectrum of the AMOC at 45N and cross spectrum with the NAM$_{10}$ index]{like figure \ref{NAM_AMOC_Cross} for the AMOC at 45N. \textbf{a} shows the wavelet power spectrum of the AMOC and \textbf{b} the cross power spectrum between the AMOC and the NAM$_{10}$ index.}
\label{NAM_AMOC_Cross_45}
\end{figure}
\end{center}
  
To understand these non-stationary signals in the context of the proposed mechanism for vortex-AMOC interactions involving the NAO, we also analyse the power spectrum for the Dec-Mar NAO (figure \ref{NAO_NAM_Cross}). The wavelet power spectrum for the NAO (figure \ref{NAO_NAM_Cross}a) exhibits a portion of significant power at periods of 30 years between $\sim$300-400 years as well as a feature at 50 years between $\sim$500-600 years similar to the NAM$_{10}$ and the AMOC. Furthermore, the cross power spectrum between the NAO and the NAM$_{10}$ indicate that signals in the two indices on the $\sim$30 and $\sim$50 year periods are also coincident in time for the 70-100 years they persist for. The phase relationship between these signals is small (arrows pointing to the right) indicating an in-phase relationship between the indices, which is consistent with the zero-lag relationship between the NAO and filtered NAM$_{10}$ extremes presented in the composite analysis (figures \ref{NAO_AMOC_T_response} and \ref{NAO_AMOC_response_individual_types}). The NAO wavelet analysis shows no significant power on the 90-100 year timescale. This suggests that the co-variability between the NAM$_{10}$ and AMOC on these longer timescales does not involve the NAO and is likely to arise through different mechanisms. We return to examine this feature in more detail in section \ref{surface-strat_forcing}.

\begin{figure}[h!]
\begin{center}
\noindent\includegraphics[width = \linewidth]{Figures/Figures-surface/NAM_NAO_filtered_subplot.png}
\caption[Wavelet power spectrum of the NAO and cross spectrum with the NAM$_{10}$ index]{like figure \ref{NAM_AMOC_Cross} for the Dec-Mar NAO index. a shows the wavelet power spectrum of the NAO and b the cross power spectrum between the NAO and the NAM$_{10}$ index.}
\label{NAO_NAM_Cross}
\end{center}
\end{figure}

The wavelet analysis suggests that while 30-yr periodicity is most apparent in the composite patterns of figure \ref{NAO_AMOC_T_response} there may also be contributions involving the 50-yr and 90-yr periodicities. This complication is more clearly evident in figure \ref{NAO_AMOC_response_individual_types} where the composites associated with persistent positive and negative NAM$_{10}$ intervals were examined separately. Firstly, we note that the contribution to the composites from the interval exhibiting $\sim$30-year oscillatory behaviour ($\sim$300-400 years) comes solely from a collection of persistent strong NAM$_{10}$ intervals (see the red dots on figure \ref{NAM_and_filtered}). In contrast, a high proportion of the weak NAM$_{10}$ contributions to the composite analysis (8 out of 13) occur within the interval between $\sim$500-600 years which exhibits variability at both 50yr and 90yr periodicity. The complicated double peaked behaviour of the AMOC response following persistent weak vortex intervals can now be better understood, taking into account the influence of the $\sim$50yr and $\sim$90yr periodicities. The double minima in AMOC response at 10yr and 20yr leads and at 15yr and 25yr lags can now be explained as manifestations of the 50-yr and 90-yr AMOC responses e.g. a half-cycle between the minimum at 10yr lead and maximum at 15yr lag (thus a half-cycle of 25 years) gives a periodicity of 50yrs, while a half-cycle between the minimum at 20yr lead and 25yr lag (thus a half-cycle of 45yrs) give a periodicity of 90yrs.  Additional support for this interpretation comes from the fact that the NAO composite analysis (figure \ref{NAO_AMOC_T_response}b) shows a response that corresponds broadly with the AMOC minimum at 10yr lead and maximum at 15yr lag, suggesting a mechanism that involves the NAO on the 50-yr timescale but there is no corresponding response in the NAO at the 90-yr periodicity, in agreement with  the wavelet spectra and cross spectra in figure \ref{NAO_NAM_Cross} which indicates that the NAO does not exhibit variability on timescales of $\sim$90 years. 

\section{Surface Forcing of the Stratosphere}\label{surface-strat_forcing}
The absence of an NAO signal that corresponds to the 90yr periodicities seen in both the NAM$_{10}$ and AMOC wavelet spectra and in the  AMOC-NAM cross spectrum (figure \ref{NAM_AMOC_Cross}) indicates that the mechanism for the stratosphere - AMOC teleconnection on this longer timescale may be distinct in nature to those observed at the 30 and 50 year periodicities. As noted earlier, the phase relationship associated with this feature is also different to the shorter timescales ($\frac{\pi}{2}$ out of phase) suggesting that the direction of causality is also switched i.e. that the AMOC leads the stratosphere response on these timescales rather than vice versa. To study this more closely we investigate possible pathways involving variability on this timescale.

Our earlier analysis of this UKESM pi-control simulation (chapter 3) highlighted the $\sim$90yr variability in frequency of SSWs and demonstrated that this was closely associated with similar timescale variations in the amplitude of the QBO (and in particular, the westerly QBO phase). Figure \ref{OLR_wavelet}a shows the wavelet spectrum of the same QBO amplitude index employed chapter 3 after smoothing with the same Gaussian filter utilised throughout this work (see section \ref{sec:model_diagnostics} for a description of how the QBO index was derived using the Hilbert amplitude). A portion of significant power at $\sim$90 year periods persists for approximately 200 years of the simulation between years 600-800, which coincides with the significant response in the same interval of the NAM$_{10}$ and AMOC spectra at 90yr periodicity. Cross spectra of the QBO index with the smoothed NAM$_{10}$ index (figure \ref{OLR_wavelet}b) also corroborates the findings of chapter 3, with coincident signals at the 90 year timescales and right-pointing arrows that indicate an in-phase relationship, so that an interval of persistently strong positive (westerly) QBO anomalies coincides with an interval of persistent positive (strong) polar vortex anomaly. The sign of this teleconnection is consistent a Holton-Tan teleconnection \citep{luDecadalscale2008c, luMechanisms2014c} but it is present on much longer timescales. 

While the long-timescale QBO-vortex teleconnection was demonstrated by work in chapter 3 (and confirmed here), the cause of the long-term westerly QBO variability was not established (see section \ref{sec:outlook_origins}). chapter 3 presented a wavelet analysis of equatorial SSTs in various regions (figure \ref{fig:tropical_SST_wavelet}), including the equatorial East Pacific, to explore whether the $\sim$90-yr QBO variability could be explained by SST triggering of convective activity that generates the gravity and other equatorial waves that contribute to the QBO. However, no 90-year periodicity in equatorial SSTs was found. Nevertheless, the extremely long period of the QBO and vortex variations suggests a driving mechanism that is likely linked to the ocean, because of the characteristic long oceanic timescales.

To extend this investigation into 90 year timescale QBO modulation by examining the variability of the East Pacific top-of-atmosphere outgoing longwave radiation (OLR) as a proxy for deep convection (instead of using the East Pacific SSTs, as in chapter 3). When deep convection is enhanced, cloud top height is increased and therefore OLR is reduced. Figure \ref{OLR_wavelet}c shows the wavelet analysis of the Sep-Nov OLR in the East Pacific. It exhibits 90 year periodicity, as well as significant cross power with the smoothed QBO amplitude and the NAM$_{10}$ index (figures \ref{OLR_wavelet}d,e). The signals in the OLR and both the QBO and NAM$_{10}$ are anti-correlated (left pointing arrows indicating a $\pi$ phase difference). This is consistent with reduced OLR (increased deep convection) leading to greater QBO amplitude through increased wave forcing.  The corresponding cross spectra between the AMOC and the OLR metric (figure \ref{OLR_wavelet}f) also indicates a significant portion of cross power in the interval $\sim$600-800 years co-located with the feature seen in the NAM$_{10}$ spectrum (see the dashed contours in figure \ref{OLR_wavelet} which indicate the region with  significant power in the NAM$_{10}$ spectrum). The phase relationship in this case is mostly $\frac{\pi}{2}$ (the majority of arrows pointing upwards) indicating that one of the quantities depends on the time rate of change of the other. This result is similar to the study of \cite{timmermannENSO2005d} who found sensitivity of the equatorial Pacific region to periodic forcing of the AMOC. Their study showed a dependence of the Pacific thermocline on the rate of change of the AMOC. (We note, however, that the $\sim$7.5 Sv AMOC perturbation imposed in their study was considerably larger than the AMOC variations in our simulation).  Similarly, a lagged, cross-basin connection between the NA overturning circulation and Pacific Sea Surface Height (SSH) was proposed by \cite{cessiGlobal2004} who interpreted it in terms of the propagation of oceanic Kelvin and Rossby waves with anomalies communicated between Atlantic and Pacific via the Indian Ocean as well as through the Drake passage. We therefore suggest this as a possible pathway for the influence of the AMOC on the polar vortex at 90-yr timescales in out simulation, via modulation of deep convection in the East Pacific that influences the amplitude of the QBO. A more detailed examination of the intermediate steps in this proposed physical pathway is required to confirm this, but this is outside the scope of the study.  

\begin{figure}[h!]
\begin{center}
\noindent\includegraphics[width = 0.7\linewidth]{Figures/Figures-surface/OLR_wavelet.png}
\caption[Wavelet power spectra for the NAM$_{10}$, deep QBO amplitude and associated surface metrics]{\textbf(a): Wavelet power spectrum for the Sep-Nov deep QBO amplitude (see section \ref{sec:model_diagnostics}). \textbf(b) cross power spectra between deep QBO Amplitude (Sep-Nov) and smoothed NAM$_{10}$ (Dec-Mar). \textbf(c): Wavelet power spectrum for the Sep-Nov  Area weighted average equatorial East Pacific OLR (see section \ref{sec:model_diagnostics_surface}). \textbf{d-f}: cross power spectra between combinations of East Pacific OLR, deep QBO Amplitude (both Sep-Nov means), AMOC at 50N (annual mean, all months) and NAM$_{10}$ (Dec-Mar) indices all smoothed with the same Gaussian filter as is used throughout ($\sigma = 2$ years). Indices involved are indicated by the sub-figure titles. Shading indicates the cross power, using the same colour scale as in figures 6-8,. Solid contours indicate the 95\% confidence interval for the power spectrum and dashed contours show the 95\% confidence interval for the NAM$_{10}$ spectrum, for ease of comparison. Arrows  indicate the relative phase angle between the signals in the indices.}
\label{OLR_wavelet}
\end{center}
\end{figure}

\section{Contribution of the stratosphere to recent AMOC changes}
\label{sec:AMOC_obs_contribution}
Our analysis of the UKESM simulation has identified co-variability between modes of variability in stratospheric circulation and the AMOC. Intervals in which the winter stratospheric polar vortex is consistently strong are, on average, followed by an extended negative anomaly in AMOC strength with a lag of approximately 15-20 years (figure 3f). Recent observations of the AMOC have shown a negative trend in circulation strength of approximately $-2.7 Sv$ between 2004 and 2012 \citep{smeedNorth2018} before a marginal recovery after 2012 \citep{smeedAtlantic2019c}. Modelling studies have proposed a key role for anthropogenic forcing in AMOC slowdown over the 20th century and into the future \citep{liuOverlooked2017a, bakkerFate2016b, liuMechanisms2019b}. However, the drivers of observed AMOC trends in the 21st century are not well understood. The results shown in the previous sections suggest the possibility that stratospheric variability has also contributed to the observed AMOC changes, in response to nearly a decade of strong vortex years in the 1990s followed by a sequence of years with a weak, disturbed vortex in the early 2000s. 

Using the relationship seen in the model between the modelled NAM, NAO and AMOC we can now compare the observed NAM, AMOC and NAO indices for the interval 1979-2020 from the ERA5 and RAPID array datasets and assess the potential contribution of stratospheric variability to the observed AMOC trend (figure \ref{ERA5_series}). The NAM$_{10}$ index (red bars) is characterised by an interval of strong vortex winters between 1988 and 1997 in which all but 2 winters exhibited a positive NAM$_{10}$. This interval also contains no SSWs \citep{pawsonCold1999}. This is followed by a run of winters between 1998 and 2005 which exhibit anomalously weak NAM$_{10}$ values with SSWs almost every year  \cite{manneyRemarkable2005a}. The filtered NAM$_{10}$ index (red dashed line) reflects the presence of these time intervals with a peak in positive values centred around 1995 followed by a negative extreme centred around 2003. The smoothed NAO (green dashed line) from the same dataset reflects some of these variations in the NAM$_{10}$ with positive NAO extremes in the 1990s. The long-term envelope of the NAO does not remain positive for as long as the NAM, primarily due to the anomalous negative NAO in 1996 \citep{halpertClimate1997b}. The presence of this anomalous negative NAO in 1996 and the absence of a clear NAO anomaly in the following year despite the presence of strong positive anomalies in the stratospheric NAM$_{10}$ is indicative of the fact that there are many other factors that influence the NAO in addition to the stratospheric influence. The AMOC strength estimated from the Rapid Array observations between 2005 and 2019 is also shown (blue curve) and shows a negative trend between 2005 and 2012 followed by a recovery from 2012 onwards \citep{smeedNorth2018b, smeedAtlantic2019c}. 

\begin{figure}[h!]
\begin{center}
\noindent\includegraphics[width = 0.9\linewidth]{Figures/Figures-surface/ERA5_series_allf.png} \caption[Time series of NAO, NAM$_{10}$ and AMOC from the ERA5 and  RAPID array datasets]{Time series of the Dec-Mar NAO index (green bars), NAM$_{10}$ index (red bars) from the ERA5 dataset. Dashed lines correspond to indices shown by bars smoothed with a Gaussian filter ($\sigma$ = 2 years). Also included is the annual AMOC timeseries at 26N from the rapid array dataset (blue).}
\label{ERA5_series}
\end{center}
\end{figure}

The interval of observed consecutive strong NAM winters in the 1990s is anomalous in the reanalysis period  although the available data record is rather short to allow a robust assessment. The amplitude and longevity of the observed anomaly is also large when compared to the UKESM simulation. Only 2 intervals in the UKESM simulation exhibit at least as many consecutive winters with strong (high NAM$_{10}$) conditions. These 2 intervals occur in the 300-400 year interval (centred around years 349 and 376) as shown in figure \ref{special_events}a. They each exhibit a sequence of 10 consecutive Dec-Mar anomalously positive NAM$_{10}$ values. The 2nd interval exhibits 14 strong or marginally weak consecutive winters with the allowance for the small negative NAM$_{10}$ value at year number 379. The presence of these two intervals is reflected in the smoothed NAM$_{10}$ values (figure \ref{special_events}b) and they represent the two of the three largest values of the filtered NAM$_{10}$ index (0.66 and 0.67). The corresponding smoothed AMOC index during these 2 intervals (blue curve in figure \ref{special_events}a) shows a positive AMOC response at lags of 2-3 years followed by a negative response at 17-20 years, in good agreement with figure \ref{AMOC_comp_NAM}f. The negative responses at 17-20 years following these 2 intervals of strong NAM$_{10}$ years are the 1st and 3rd greatest in magnitude compared to all other responses to persistent strong intervals. This is confirmed by figure 11c which shows the lagged AMOC response following all of the identified intervals with persistent positive NAM$_{10}$ anomalies (the two identified around 349 and 376 years are shown in black). 

To estimate the response amplitude of the AMOC to an interval of persistently strong vortex winters, figure \ref{special_events}d shows a scatter plot of the central NAM$_{10}$ index of these intervals against the AMOC anomaly at 50N lagged  by 17 years. This reveals a strong linear relationship ($r = -0.908$) between the size of the persistent vortex anomaly and the subsequent negative anomaly in the AMOC at 50N 17 years later. The PDF of surrogate correlations used to assess the statistical significance (following the method outlined in section 2) is displayed in figure \ref{cors_stat_sigs}a along with the correlation generated by the observed NAM$_{10}$ data. It shows that the $r$ value lies well outside the distribution of surrogate correlations, indicating the high level of significance of the linear relationship. A similar analysis of the relationship between the magnitude of smoothed negative (weak) NAM$_{10}$ extremes and the lagged AMOC response at 50N (not shown) yields a much weaker relationship ($r = -0.21$) that is not statistically significant compared to the corresponding PDF of surrogate correlations (figure \ref{cors_stat_sigs}b). This asymmetry in the vortex-AMOC relationship between extreme positive and negative NAMs  is perhaps not surprising, given that the surface impact of SSWs (that give rise to the negative NAM events) depends on the timing of the SSW within the winter season whereas strong positive NAM intervals exhibit strong vortex conditions throughout the winter. 

\begin{figure}[h!]
\begin{center}
\noindent\includegraphics[width =\linewidth]{Figures/Figures-surface/AMOC_response_special_events.png} 
\caption[Regression of NAM$_{10}$ extreme magnitude vs AMOC anomaly at 50N]{\textbf{a}: Dec-Mar NAM$_{10}$ index (red bars) from the UKESM simulation between year numbers 300 and 400. \textbf{b}: AMOC (blue) and NAM$_{10}$ (red) indices smoothed with the Gaussian filter ($\sigma$ = 2 years) between year numbers 300 and 400 of the UKESM simulation. Black vertical lines show the location of the largest smoothed NAM$_{10}$  \textbf{c}: AMOC response to to persistent strong NAM$_{10}$ intervals. Light blue lines denote lagged AMOC responses of the AMOC from the whole UKESM simulation, black lines show AMOC responses to NAM$_{10}$ intervals marked in \textbf{b} by vertical black lines. \textbf{d}: Scatter plot of filtered NAM$_{10}$ index values occurring at persistent strong NAM$_{10}$ intervals throughout the whole UKESM simulation ($y$ axis) against AMOC anomalies at 50N lagged 17 years after persistent intervals' central year ($x$ axis). Blue points indicate persistent intervals and black dots represent the 2 intervals displayed in b. The dotted line represents the linear line of best fit for the points in black and blue. Also included is the 17 year lag AMOC anomaly predicted by projecting the regression coefficients used to construct the linear fit onto the maximum smoothed NAM$_{10}$ index in the ERA5 dataset (purple point).}
\label{special_events}
\end{center}
\end{figure}

\begin{center}
\begin{figure}[h!]
\noindent\includegraphics[width = \linewidth]{Figures/Figures-surface/correlation_stat_sigs.png}
\caption[PDFs for correlations between the magnitude of NAM$_{10}$ extreme and lagged AMOC]{Probability distribution functions (PDFs) for correlations between the magnitude of NAM$_{10}$ extreme positive (\textbf{a}) and negative (\textbf{b}) values from surrogate NAM$_{10}$ data and anomalies in the AMOC at 50N evaluated 17 years later. Each NAM$_{10}$ surrogate is generated by Fourier transforming the smoothed NAM$_{10}$ index, randomly shuffling the Fourier phases and inverse transforming. Each PDF is built with 10000 surrogates and the correlation between the NAM$_{10}$ extreme magnitude and 17 year lagged AMOC anomaly at 50N is shown by vertical black lines in both sub-figures.}
\label{cors_stat_sigs}
\end{figure}
\end{center}

A linear regression analysis on the data in figure \ref{special_events}d yields an estimated relationship between the variables which satisfies

\begin{equation}
AMOC'_{+17} = -6.54\space NAM max + 3.11,
\end{equation}

where $AMOC'_{+17}$ is the 17 year lagged AMOC anomaly at 50N and $NAM max$ is the magnitude of the positive extreme in smoothed NAM$_{10}$ at the centre of each interval. We can then use this relationship to predict the AMOC response to the observed sequence of strong vortex years in the 1990s exhibited by ERA5. The maximum smoothed NAM$_{10}$ in ERA5 occurs in 1996 so, using this relationship, the maximum AMOC response associated with the stratosphere would be expected 17 years later (2013) with amplitude of -0.89 Sv (figure \ref{special_events}d, purple point).

This prediction suggests that approximately 30\% percent of the observed reduction in AMOC strength  between 2005 and 2013 (0.89 Sv compared with 2.9 Sv in total) could be due to the response of the ocean to persistent forcing from consecutive strong vortex winters that occurred during the 1990s. We note, however, that our derivation of the modelled vortex-AMOC relationship is based on the AMOC response at 50N, where the response amplitude is largest, whereas the Rapid Array dataset that provides the most direct measurement of AMOC strength over the last 15 years so is based at 26N.  Figure \ref{special_events_30} shows the scatter plot of filtered NAM$_{10}$ extreme magnitudes and lagged AMOC responses from the model at 30N. At this latitude, the linear relationship is significantly weaker than at 50N ($r = 0.652$ for 30N vs $r = 0.908$ for 50N) but the correlation coefficient remains significant at the 95\% level. The predicted contribution from the strong vortex interval in the 1990s to the AMOC strength at 30N is reduced to $-0.49Sv$,  (figure \ref{special_events_30}, purple dot) suggesting that  approximately 17\% of the negative trend in the RAPID AMOC data may be due stratospheric forcing from the 1990s. This is consistent with composite analysis in figures in \ref{AMOC_comp_NAM}a and b which indicate that the modulation of the AMOC by the smoothed NAM$_{10}$ is less pronounced at 30N than at higher latitudes.

\begin{figure}[h!]
\begin{center}
\noindent\includegraphics[width = 0.6\linewidth]{Figures/Figures-surface/AMOC_response_special_events_30N_AMOC.png} 
\caption[Regression of NAM$_{10}$ extreme magnitude vs AMOC anomaly at 30N]{Like figure \ref{special_events}\textbf{d} for AMOC responses at 30N: Scatter plot of filtered NAM$_{10}$ index values occurring at persistent strong NAM$_{10}$ intervals throughout the whole UKESM simulation ($y$ axis) against AMOC anomalies at 30N lagged 17 years after persistent intervals' central year ($x$ axis). Blue points indicate persistent intervals and the dotted line represents the linear line of best fit for the points in black and blue. Also included is the 17 year lag AMOC anomaly predicted by projecting the regression coefficients used to construct the linear fit onto the maximum smoothed NAM$_{10}$ index in the ERA5 dataset (purple point).}
\label{special_events_30}
\end{center}
\end{figure}

\section{Sensitivity to NAM thresholds and filtering}
\label{sec:sensitivity}

The results showing interactions between the vortex and the AMOC presented so far (figures \ref{AMOC_comp_NAM} and \ref{NAO_AMOC_T_response}) are derived from a stratospheric index which is highly processed. The NAM$_{10}$ is both smoothed using a Gaussian filter and extreme values are subsequently selected to define intervals of anomalous vortex behaviour. Lagged responses of the AMOC are then obtained from these intervals. This processing relies on a number of parameters that need to be chosen: First, the filtering kernel width (the parameter $\sigma$, see equation \ref{Gaussian_filter}) which dictates the number of years which contribute to smoothed NAM$_{10}$ values. Second, the threshold value of the smoothed NAM$_{10}$ used to define an interval of persistent vortex behaviour. Finally, for regression analysis such as that presented in figure \ref{special_events}d, the lag of the AMOC response in years used to correlate the magnitude of NAM$_{10}$ extreme with the resulting AMOC anomaly also needs to be selected. 

So far, we have chosen these parameters in line with previous studies that utilise a similar smoothed NAM$_{10}$ metric \citep{reichlerStratospheric2012b}. Namely, we utilise $\sigma$ = 2 years to smooth all indices which allows contributions from approximately 6-8 years either side of the interval centre which is approximately the length of the interval of persistent strong vortex behaviour in the 1990s of ERA5 (see figure \ref{ERA5_series}). We also utilise 5th and 95th percentiles to define weak and strong extremes respectively in the smoothed NAM$_{10}$. This yields approximately the same average rate of persistent vortex intervals in the pi-cntrl of UKESM as appears in the modelling study of \cite{reichlerStratospheric2012b}. While values of these parameters may dictate some key physical features of the timeseries (for example interval width in years), there is little obvious justification for some parameter choices. In this section, we test the sensitivity of our results on vortex-AMOC interactions to choice of smoothing width ($\sigma$), NAM$_{10}$ threshold and AMOC lag values. 

First, we assess the impact of parameter choice on lagged response values of the AMOC at 50N, the latitude at which the most prominent interactions were shown (see figure \ref{AMOC_comp_NAM}, following persistent NAM$_{10}$ extremes (figure \ref{sensitivity}). Parameter grids of the response to strong intervals for different combinations of lag and $\sigma$ (figure \ref{sensitivity}a) show a similar response variation with lag (along the $x$ axis of figure \ref{sensitivity}a) as that shown in figure \ref{NAO_AMOC_response_individual_types} - positive responses peaking 2-3 years following stratospheric extremes followed by negative responses at 10-20 year lags. However, the largest negative response magnitude is observed when the NAM$_{10}$ index is smoothed using $\sigma = \sim$5 years at a lag of approximately 15 years instead of our original choice of $\sigma$ = 2 years lagged by 17 years (as was used in figure \ref{special_events}d). This suggests that the AMOC responds most prominently to intervals which consist of contributions from approximately 10-15 years either side of the the central year. Extremes intervals of this size are unlikely to consist of winters exhibiting all the same sign of vortex anomaly due to their length - the largest run of consecutive positive NAM$_{10}$ winters are shown in figure \ref{special_events} and are only 10 years. Instead, these are likely intervals containing a combination of elevated NAM$_{10}$ and neutral (near 0 smoothed NAM$_{10}$) winters - I.E. intervals that consist of winters with either a strong vortex or one that exhibits near climatological vortex winds and even those that may feature partial minor disruptions. This suggests that a key factor in the AMOC response to the vortex may not simply be the presence of $\sim$6-8 consecutive strong vortex seasons for but the absence of major SSWs over a longer interval ($\sim$10-15 years). While this result indicates a larger response using a wider smoothing window, applying this to observations like the analysis in section \ref{sec:AMOC_obs_contribution} may be difficult given the width of the window (including contributions from 10-15 years either side of the centre) in relation to the length of the observational record (approximately 40 years in ERA5). Furthermore the observational record does not exhibit such a large interval of persistent vortex behaviour and the most prominent feature of the ERA5 NAM$_{10}$ is the interval of approximately 8 years of strong NAM$_{10}$ winters in the 1990s which is well captured using $\sigma$ = 2 years. 

The AMOC response magnitude is also sensitive to varying the percentile threshold used to define persistent vortex intervals. Figures \ref{sensitivity}b and c show the variation in response for different combinations of threshold and $\sigma$ and threshold and lag. Both sub-figures indicate that the average AMOC response magnitude increases significantly with threshold percentile over values of approximately 95\%. This increase in magnitude is expected as increasing the threshold for events will include NAM$_{10}$ intervals which on average contain more years with an anomalously strong vortex and likely a more persistent source of atmospheric forcing to the NAO and the ocean. 

\begin{figure}[h!]
\begin{center}
\noindent\includegraphics[width =\linewidth]{Figures/Figures-surface/sensitivity_contours_strong.png} 
\caption[Response of the AMOC at 50N to strong NAM$_{10}$ intervals for combinations of $\sigma$, threshold and lag]{Response of the AMOC at 50N to strong NAM$_{10}$ intervals for combinations of \textbf{a):} smoothing width ($\sigma$) and lag both measured in years with percentile threshold held at 95\%, \textbf{b):} $\sigma$ and percentile threshold used to define persistent NAM$_{10}$ extremes with a 17 year lag and \textbf{c):} Threshold and lag with $\sigma$ = 2 years.}
\label{sensitivity}
\end{center}
\end{figure}

We also test the sensitivity of our key result outlined in section \ref{sec:AMOC_obs_contribution}, that the magnitude of NAM$_{10}$ extremes is directly proportional to the AMOC anomaly 17 years later, to parameter choices. Figure \ref{cors_stats_parameters_strong} shows the correlation between strong NAM$_{10}$ extremes and the lagged AMOC anomaly at 50N (black vertical lines) for a range of threshold percentile values used to define persistent vortex intervals (increasing in panels left to right on the figure) and smoothing widths ($\sigma$, increasing top to bottom). Each correlation's significance level is compared to a PDF of surrogate correlations constructed using the same procedure as described for figure \ref{cors_stat_sigs}. For values of $\sigma$ between 2 and 7 years, the real correlation lies on the edge of the surrogate correlation PDF for all values of threshold percentile. This suggests the relationship between positive extremes in the smoothed NAM$_{10}$ and the resulting AMOC response is robust to the definition of strong persistent vortex interval for these values of smoothing width. for $\sigma$ = 1 year, the correlation lies well within the distribution surrogate PDF for all sampled values of threshold percentile suggesting the vortex-AMOC interaction is significantly weakened for these parameter combinations. This weakening is accounted for by the fact that with a filtering kernel with $\sigma$ = 1 year, contributions from only approximately 3 years either side of the interval centre (with the majority of contributions from 1-2 years either side) are included in smoothed NAM$_{10}$ values. This may not be sufficient to provide atmospheric forcing persistent enough to induce significant correlations between NAM$_{10}$ extremes and AMOC anomalies. For $\sigma$ values larger than 7 years, the correlations are also deemed not significant against the bootstrapping test for the sampled values of percentile thresholds. This is because at such large values of $\sigma$, each index is smoothed over time intervals as larger or larger than the lags considered in this analysis (17 years). As a result, variations in the NAM$_{10}$ and AMOC on timescales up to and beyond 17 years are removed thus also removing significant correlations.

\newpage
\begin{landscape}
\begin{figure}[h!]
\begin{center}
\noindent\includegraphics[width =0.9\linewidth]{Figures/Figures-surface/cors_sigs_thresh_and_sigma.png} 
\caption[PDFs for correlations between the magnitude of strong NAM$_{10}$ extreme and lagged AMOC for different $\sigma$ and threshold]{like figure \ref{cors_stat_sigs}a for different combinations of smoothing width ($\sigma$) (increasing in panels top to bottom) and percentile used to define NAM$_{10}$ extremes (decreasing left to right). The title of each sub-figure indicates the pair of parameter values used.}%, blue bars indicate the PDF of surrogate correlations and black vertical lines indicate the real correlation between strong NAM$_{10}$ extremes and 17 year lagged AMOC anomalies.}
\label{cors_stats_parameters_strong}
\end{center}
\end{figure}
\end{landscape}

We finally carry out a similar analysis on correlations between weak NAM$_{10}$ extremes and the resulting AMOC anomaly (figure \ref{cors_stats_parameters_weak}) which yielded low correlation with our original choice of parameters ($\sigma =$2 years, lag = 17 years, percentile threshold = 5\%). In contrast to figure \ref{cors_stats_parameters_strong} for strong intervals, the majority of correlations for different parameter combinations lie well within the corresponding surrogate PDF. There are some exceptions however, with significant negative correlations for $\sigma$= 5 years and threshold = 10\% and 15\% suggesting a possible connection between weak intervals of approximately 10 years with the AMOC. However, as discussed above, few intervals of this length which exhibit exclusively negative NAM$_{10}$ values appear in the model (only 2) and instead these likely contain both weak and neutral vortex winters. 

As discussed in section \ref{sec:AMOC_obs_contribution}, the small correlations associated with weak intervals may be due to the different dynamical behaviour associated with weak persistent intervals which is highly dependant on SSW timing within each winter. Overall, the results presented from this sensitivity analysis once again suggests a key role for positive (strong) extremes in the in vortex-AMOC interactions which are generally robust to reasonable choices of parameter values (threshold percentile and $\sigma$) while interactions involving anomalously weak vortex intervals appear less pronounced. 

\newpage
\begin{landscape}
\begin{figure}[h!]
\begin{center}
\noindent\includegraphics[width =0.9\linewidth]{Figures/Figures-surface/cors_sigs_thresh_and_sigma_weak.png} 
\caption[PDFs for correlations between the magnitude of weak NAM$_{10}$ extreme and lagged AMOC for different $\sigma$ and threshold]{like figure \ref{cors_stats_parameters_strong} for correlations between NAM$_{10}$ values associated with persistent weak intervals and 17 year lagged AMOC anomalies at 50N.}
\label{cors_stats_parameters_weak}
\end{center}
\end{figure}
\end{landscape}

%%% Local Variables:
%%% mode: latex
%%% TeX-master: "thesis"
%%% End:

\section{Summary and Discussion}
In this chapter, we have analysed the influence of persistent polar vortex extremes on surface and ocean circulation in a 1000 year pi-control simulation of UKESM1. Persistent vortex anomalies are identified using a smoothed NAM$_{10}$ index which characterises intervals of approximately 6-8 years during which the NH winter vortex is anomalously strong (positive NAM$_{10}$ anomaly) or weak (negative NAM$_{10}$ anomaly). While the surface impacts of stratospheric extremes in individual winters has received much attention \citep{baldwinStratospheric2001a, domeisenEstimating2019d, charlton-perezInfluence2018e}, the surface impacts of consecutive sets of persistently anomalous winters has been less well studied and teleconnections between the stratospheric vortex and ocean variability are not well characterised or understood. 

We examine the AMOC response to long-term variations in the stratospheric polar vortex using composite analysis of the AMOC strength following persistent anomalous NAM$_{10}$ intervals. We find oscillatory responses in the NAO and the AMOC consistent with previous work \citep{reichlerStratospheric2012b}. Diagnosis of the model supports a mechanism in which a persistently strong vortex (positive NAM) perturbs a positive NAO anomaly which subsequently induces a positive AMOC response at 2-3 year lags via an increase in subpolar North Atlantic ocean-atmosphere heat flux. This, in turn feeds back onto the NAO to drive a reversal in  NAO phase (to negative) which leads to a subsequent negative AMOC anomaly at 15-20 year lags. The integrated effect of long-term oscillatory signals in the NAM$_{10}$ and the associated NAO variations may thus act as a metronome for the AMOC, which is a natural mode of oscillation in  ocean circulation that varies on similar timescales. Further diagnosis of the separate impacts from  intervals of persistent positive NAM (strong vortex) and persistent negative NAM (weak vortex with repeated SSW occurrences) showed that persistent strong vortex intervals had a much larger impact on the AMOC, perhaps not surprisingly because the vortex anomaly is consistently present throughout the whole winter.

We additionally find prominent non-stationary variations across multiple timescales in the AMOC, NAO and NAM$_{10}$. Wavelet analysis revealed extended intervals in  which 30, 50 and 90 year periodicities were dominant, so the composite response patterns were complicated by  the superposition of contributions from intervals exhibiting different timescale behaviour. 

Interestingly, while all three vortex (i.e. NAM$_{10}$), NAO and AMOC indices co-varied at the 30 and 50 yr periodicities, only the NAM$_{10}$ and AMOC co-varied at the 90yr periodicity and co-spectra analysis suggested that the AMOC leads the vortex signal. This suggests a feedback of AMOC variability onto the vortex that does not involve the NAO. Chapter 3 found long-term (90yr) co-variability between the vortex and the QBO (figure \ref{fig:QBO_SSW_subfig}) so suggesting the possibility that the 90yr AMOC-vortex relationship could act via an influence of the AMOC on equatorial wave forcing of the QBO. This would then influences the vortex through the Holton-Tan relationship.  This was explored through wavelet and co-spectra analysis of variations in tropical east Pacific deep convection and QBO amplitude. These showed similar 90yr co-variability, indicating this as a plausible mechanism for the source of long-term variability in the QBO and the vortex, but further analysis of individual steps in the process, such as amplitude modulation of the various tropical waves that give rise to the QBO, would be required to confirm this.

Finally, we have applied the model results, that link a lagged AMOC response to the presence of persistent vortex anomalies (sets of consecutive strong or weak vortex winters), to assess the possible contribution  of the interval of persistently strong vortex in the 1990s to the recent observed changes in the AMOC. Our analysis suggests a maximum AMOC response at 50N with a lag of approximately 17 years. The lagged vortex-AMOC relationship is statistically significant using the AMOC response at both 50N ($r = -0.908$) and also at 30N ($r = -0.652$), the latter being closer in latitude to the RAPID Array observations. Using a regression technique we estimate that -0.89Sv (50N) and -0.49Sv (30N) of the observed RAPID Array AMOC trend by 2012 can be associated with the interval of persistent strong vortex behaviour centred on 1995. These estimates represents nearly 30\% (50N) and 17\% (30N) of the total decrease in AMOC transport between 2005 and 2013 in the RAPID array data. The observed negative trend in the AMOC has been attributed to a range of factors, including the influence of anthropogenic forcings \citep{caesarObserved2018, caesarCurrent2021}, but to our knowledge the potential role of vortex variability has not previously been considered. The origin of the interval of persistently strong vortex in the 1990s is unknown, but it was most likely due to internal variability. There is currently no consensus amongst climate models on how the vortex will respond to anthropogenic climate change \citep{ayarzaguenaUncertainty2020b}, and the 1990s appear to have been an anomalous period with no clear long-term trend emerging \citep{domeisenEstimating2019d}. As a result, our findings may indicate a significant role for internally generated signals in the recent negative AMOC trend. 

There are a number of caveats to these results. Firstly, the results come from a single model, although there has also been limited analysis of vortex-AMOC interactions in CMIP5 models (\cite{reichlerStratospheric2012b}. On the other hand, there is evidence that GCMs under-represent the influence of SSW events on the mid-latitude tropospheric jet, the NAO and surface temperatures, part of the "signal to noise" problem identified by \cite{scaifeSignaltonoise2018b}. Nevertheless, the model showed a clear NAO signal up to 3 months following vortex anomalies, indicating a reasonable representation of stratosphere - surface interaction. Analysis in chapter 3 also noted an underestimation of SSW frequency compared to the ERA-Interim dataset (figure \ref{fig:SSW_histogram}), which indicates a positive bias in the mean vortex strength as well as the NAM$_{10}$, leading to the possible over-representation of positive NAM$_{10}$ intervals. The variability in the AMOC may also be under-represented in the simulation, as \cite{roberts20042014} suggests that the  models' decadal variability is smaller than that in the RAPID dataset. Additional analysis using a suite of CMIP6 models would clearly be useful to assess the robustness of the results presented here. 

Overall, results from this chapter suggest a novel, non-stationary interaction between the vortex and Atlantic circulation as well as suggest a possible key role for the stratosphere in recent AMOC observations. Key limitations to the analysis presented here as well as further work following on from this study is explored in section \ref{sec:limitations}.






