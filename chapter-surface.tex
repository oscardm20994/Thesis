\chapter{A geometrical description of vortex variability}
\begin{quotation}
  Much of the work contained in this chapter is based upon \citet{Seviour2013},
  published in \emph{Geophysical Research Letters}, although the analysis
  presented here has been significantly extended.
\end{quotation}

\label{cha:moments}


\section{Introduction}



\section{Data And Methods}

We also use the NAM as a continuous metric for vortex strength following the methodology of \cite{baldwinStratospheric2001}. the $NAM_{10}$ index is defined as the leading Principle Component of the deseasonalised GPH field evaluated over the winter months (Dec-Mar, as with SSWs) on the 10hPa level. Utilising this metric gives the advantage of encoding both anomalously weak vortex winters (which include SSWs) as well as those in which the vortex is anomalously strong. In contrast, the use of the use of the ZMZW definition of an SSW gives a discretised metric of weak events (or absence of them) which may influence its spectral characteristics. This discretisation compared to the NAM definition is discussed in section xxxx.



%%% Local Variables:
%%% mode: latex
%%% TeX-master: "thesis"
%%% End:







